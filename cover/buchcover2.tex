%
% buchcover2.tex -- Cover für das Buch Variationsprinzipien
%
% (c) 2018 Prof Dr Andreas Müller, Hochschule Rapperswil
%
\documentclass[11pt]{standalone}
\usepackage{tikz}
\usepackage{times}
\usepackage{geometry}
\usepackage[utf8]{inputenc}
\usepackage[T1]{fontenc}
\usepackage{times}
\usepackage{amsmath,amscd}
\usepackage{amssymb}
\usepackage{amsfonts}
\usepackage{german}
\usepackage{txfonts}
\usepackage{ifthen}
\usepackage{qrcode}
\usetikzlibrary{math}
\geometry{papersize={417mm,278mm},total={417mm,278mm},top=72.27pt, bottom=0pt, left=72.27pt, right=0pt}
\newboolean{guidelines}
\setboolean{guidelines}{true}
\setboolean{guidelines}{false}
\newboolean{teilnehmer}
\setboolean{teilnehmer}{false}
\setboolean{teilnehmer}{true}

\begin{document}
\begin{tikzpicture}[>=latex, scale=1]
\tikzmath{
	real \ruecken, \einschlag, \gelenk, \breite, \hoehe;
	\ruecken = 2.5;
	\einschlag = 1.6;
	\gelenk = 0.8;
	\breite = 16.7;
	\hoehe = 24.6;
	real \bogengreite, \bogenhoehe;
	\bogenbreite = 2 * (\breite + \einschlag + \gelenk) + \ruecken;
	\bogenhoehe = 2 * \einschlag + \hoehe;
}

%\clip (0,0) circle (6);

\draw[fill=blue](0,0) rectangle({\bogenbreite},{\bogenhoehe});
\hsize=13.6cm

\begin{scope}
\clip (0,0) rectangle({\bogenbreite},{0.50*\bogenhoehe});

\coordinate (A) at (0,0);
\coordinate (B) at ({0.9*\bogenbreite},0);
\coordinate (C) at (\bogenbreite,0);
\coordinate (D) at (\bogenbreite,{0.2*\bogenhoehe});
\coordinate (E) at (\bogenbreite,{0.55*\bogenhoehe});
\coordinate (F) at ({0.2*\bogenbreite},{0.55*\bogenhoehe});
\coordinate (G) at (0,{0.55*\bogenhoehe});
\coordinate (H) at (0,{0.2*\bogenhoehe});

%\begin{scope}
%\clip (A) -- (B) -- (C) -- (D) -- (H) -- cycle;
%\node at ({\bogenbreite/2},5.0)
%	{\includegraphics[width=42cm]{surface-alpha.png}};
%
%\end{scope}

\fill[color=red] (A) circle[radius=0.1];

\begin{scope}
%\clip (D) -- (E) -- (F) -- (G) -- (H) -- cycle;
\node at ({\bogenbreite/2-0},8.4)
	{\includegraphics[width=46cm]{Katenoid2.png}};
\end{scope}

\end{scope}

\node at ({\einschlag+2*\gelenk+\ruecken+1.5*\breite},24.3)
	[color=white,scale=1]
	{\hbox to\hsize{\hfill%
	\sf \fontsize{24}{24}\selectfont Mathematisches Seminar}};

\node at ({\einschlag+2*\gelenk+\ruecken+1.5*\breite},21.9)
	[color=white,scale=1]
	{\hbox to\hsize{\hfill%
	\sf \fontsize{41}{41}\selectfont Variationsprinzipien}};

\node at ({\einschlag+2*\gelenk+\ruecken+1.5*\breite},19.7)
	[color=white,scale=1]
	{\hbox to\hsize{\hfill%
	\sf \fontsize{13}{5}\selectfont Andreas Müller}};

\ifthenelse{\boolean{teilnehmer}}{
\node at ({\einschlag+2*\gelenk+\ruecken+1.5*\breite},18.4)
	[color=white,scale=1]
	{\hbox to\hsize{\hfill%
	\sf \fontsize{13}{5}\selectfont
	Sofia Aaltonen,		% B
	Ronja Allenfort,	% B
	Selvin Blöchlinger,	% MSE
	Flurin Brechbühler%,	% E
	}};

\node at ({\einschlag+2*\gelenk+\ruecken+1.5*\breite},17.75)
	[color=white,scale=1]
	{\hbox to\hsize{\hfill%
	\sf \fontsize{13}{5}\selectfont
	Baris Catan,		% E
	Gabriela da Costa Rodrigues,	% B
	Maurin Doswald%,		% E
	}};

\node at ({\einschlag+2*\gelenk+\ruecken+1.5*\breite},17.1)
	[color=white,scale=1]
	{\hbox to\hsize{\hfill%
	\sf \fontsize{13}{5}\selectfont
	Jakob Gierer,		% E
	Jannis Gull,		% E
	Andrin Kälin,		% E
	Shaarujan Kamalanathan%,	% E
	}};
 
\node at ({\einschlag+2*\gelenk+\ruecken+1.5*\breite},16.45)
	[color=white,scale=1]
	{\hbox to\hsize{\hfill%
	\sf \fontsize{13}{5}\selectfont
	Kevin Kempf,		% I
	Tobias Locher,		% I
	Matthias Meyer,		% MSE
	Ana Milivojevic%,		% B
	}};

\node at ({\einschlag+2*\gelenk+\ruecken+1.5*\breite},15.8)
	[color=white,scale=1]
	{\hbox to\hsize{\hfill%
	\sf \fontsize{13}{5}\selectfont
	Patrick Müller,	% MSE
	Stephan Oseghale,	% E
	David Peter,		% E
	Anna Pietak%,		% E
	}};

\node at ({\einschlag+2*\gelenk+\ruecken+1.5*\breite},15.15)
	[color=white,scale=1]
	{\hbox to\hsize{\hfill%
	\sf \fontsize{13}{5}\selectfont
	Marco Rouge,		% E
	Sven Schlömmer,		% MSE
	Lukas Schöpf,		% E
	Joel Stohler%,		% E
	}};

\node at ({\einschlag+2*\gelenk+\ruecken+1.5*\breite},14.5)
	[color=white,scale=1]
	{\hbox to\hsize{\hfill%
	\sf \fontsize{13}{5}\selectfont
	Nico Tuscano%,		% I
	%Damian Ulrich%,		% B
	}};

}{}
 
%\node at (0,3) [color=white] {\sf \LARGE Mathematisches Seminar 2017};

% Rücken
\node at ({\bogenbreite/2 + 0.00},18.5) [color=white,rotate=-90]
	{\sf\fontsize{35}{0}\selectfont Variationsprinzipien\strut};

% Buchrückseite
\node at ({\einschlag+0.5*\breite},18.6) [color=white] {\sf
\fontsize{13}{16}\selectfont
\vbox{%
\parindent=0pt
%\raggedright
Das Mathematische Seminar der Ostschweizer Fachhochschule
in Rapperswil hat sich im Frühjahrssemester 2024 dem Thema
Variationsprinzipien
zugewandt.
Ziel war, etwas Ordnung in die vielseitigen Methoden und Anwendungen
der Variatioinsrechnung zu bringen.
Dieses Buch bringt das Skript des Vorlesungsteils mit den von den
Seminarteilnehmern beigetragenen Seminararbeiten zusammen.

\medskip

Zum Umschlagbild: Das Katenoid ist die einzige Minimalfläche,
die auch eine Rotationsfläche ist.
Bei grossem Abstand zwischen den beiden Ringen wird das 
Katenoid instabil und platzt.
Es bleibt je eine Seifenhaut in jedem Ring sowie
eine einzelne kleine Blase dazwischen.
In Kapitel~17 werden Minimalflächen und insbesondere das
Katenoid untersucht.
}};

\def\qrbreite{3}
\def\qrrightoffset{0}
\def\qrbottomoffset{1.5}

\fill[color=white]
        ({\einschlag+(\breite+13.6)/2-\qrbreite-0.1},{\einschlag+\qrbottomoffset-0.1})
        rectangle
        ({\einschlag+(\breite+13.6)/2+0.1},{\einschlag+\qrbottomoffset+\qrbreite+0.1});

\node at ({\einschlag+(\breite+13.6)/2-\qrbreite/2},{\einschlag+\qrbottomoffset+\qrbreite/2}) {
\qrcode[height=3cm]{https://mathsem.ch/jahre/2024/SeminarVariation.pdf}
};
\node at ({\einschlag+(\breite+13.6)/2-\qrbreite/2},{\einschlag+\qrbottomoffset+\qrbreite/2}) {
\includegraphics[width=10mm]{mathman.png}
};

\ifthenelse{\boolean{guidelines}}{
\draw[white] (0,{\einschlag})--({\bogenbreite},{\einschlag});
\draw[white] (0,{\bogenhoehe-\einschlag})--({\bogenbreite},{\bogenhoehe-\einschlag});

\draw[white] ({\einschlag},0)--({\einschlag},{\bogenhoehe});
\draw[white] ({\einschlag+\breite},0)--({\einschlag+\breite},{\bogenhoehe});
\draw[white] ({\einschlag+\breite+\gelenk},0)--({\einschlag+\breite+\gelenk},{\bogenhoehe});
\draw[white] ({\bogenbreite-\einschlag-\breite-\gelenk},0)--({\bogenbreite-\einschlag-\breite-\gelenk},{\bogenhoehe});
\draw[white] ({\bogenbreite-\einschlag-\breite},0)--({\bogenbreite-\einschlag-\breite},{\bogenhoehe});
\draw[white] ({\bogenbreite-\einschlag},0)--({\bogenbreite-\einschlag},{\bogenhoehe});
}{}

\end{tikzpicture}
\end{document}
