%
% einleitung.tex -- Beispiel-File für die Einleitung
%
% (c) 2020 Prof Dr Andreas Müller, Hochschule Rapperswil
%
% !TEX root = ../../buch.tex
% !TEX encoding = UTF-8
%
\section{Theoretische Grundlagen und Beispiele
	\label{minimalflaechen:section:Theoretische Grundlagen und Beispiele}}
\rhead{Theoretische Grundlagen und Beispiele}
Minimalflächen sind durch die Eigenschaft definiert, dass ihre mittlere Krümmung an jedem Punkt null ist.
Die mittlere Krümmung H einer Fläche wird durch den Durchschnitt der beiden Hauptkrümmungen $k_1$ und $k_2$ an einem Punkt berechnet

\begin{equation}
	H=\frac{k_{1}+k_{2}}{2}.
\end{equation}

Für Minimalflächen gilt daher

\begin{equation}
	k_{1}=-k_{2}.
\end{equation}

Ein weiteres charakteristisches Merkmal von Minimalflächen ist, dass sie Lösungen der Euler-Lagrange-Gleichung für den Flächeninhalt darstellen.
Diese Gleichung ergibt sich aus der Bedingung, dass die erste Variation des Flächeninhalts gleich null ist.
Mathematisch wird dies durch die folgende partielle Differentialgleichung ausgedrückt

\begin{equation}
	\Delta f = 0,
\end{equation}

wobei $\Delta$ der Laplace-Operator ist und $f$ eine Funktion, die die Fläche beschreibt.

\subsection{Beispiele von Minimalflächen
	\label{minimalflaechen:subsection:Beispiele von Minimalflächen}}
Es gibt viele verschiedene Arten von Minimalflächen, von denen einige besonders bekannt und gut untersucht sind.
Hier sind einige der Beispiele:
\begin{itemize}
	\item
	Helikoid (Wendefläche)
	
	Diese Minimalfläche entsteht durch eine unstetige, aber isometrische Deformation des Katenoids.
	Ein Helikoid kann als eine verallgemeinerte Spirale betrachtet werden.
	Die Gleichung, die ein Helikoid beschreibt, ist $z=\theta$, wobei $\theta$ der Winkel ist.
	\begin{figure}
		\centering
		\includegraphics[width=0.7\linewidth]{../../../../../../../Downloads/Helikoid}
		\caption{Helikoid}
		\label{fig:helikoid}
	\end{figure}
	\item
	Katenoid
	
	Das Katenoid ist eine Rotationsfläche, die durch die Drehung einer Kettenlinie um die x-Achse entsteht.
	Sie ist die einzige Minimalfläche, die auch eine Rotationsfläche ist.
	Eine Kettenlinie ist die Kurve, die durch eine ideale flexible Kette oder ein Seil gebildet wird, welche unter dem Einfluss der Schwerkraft hängt.
	Das Katenoid ist eine der einfachsten und bekanntesten Minimalflächen.
	
	Unter diesem Link \href{https://www.geogebra.org/m/BtAdpcYM}{www.geogebra.org/m/BtAdpcYM} kann man sehen, wie durch die Rotation einer Kettenlinie ein Katenoid entsteht.
	\item
	Scherksche Minimalfläche
	
	Diese Minimalfläche wurde von Heinrich Ferdinand Scherk im Jahr 1834 entdeckt.
	Sie hat eine schachbrettartige Struktur und wird durch die Gleichung 
	\begin{equation}
		z=ln(\frac{cos(y)}{cos(x)}) 
	\end{equation}
	beschrieben.
	
	\begin{figure}
		\centering
		\includegraphics[width=0.7\linewidth]{"../../../../../../../Downloads/Schreksche Minimalfläche"}
		\caption{Scherksche Minimalfläche}
		\label{fig:schreksche-minimalflache}
	\end{figure}
	\item
	Henneberg-Fläche
	
	Diese nicht orientierbare Minimalfläche wurde von Ernst Lebrecht Henneberg entdeckt
	Eine nicht orientierbare Fläche ist eine Fläche, die keine eindeutige Normalenrichtung hat, was bedeutet, dass sie nicht in eine Richtung zeigt.
	
	\begin{figure}
		\centering
		\includegraphics[width=0.7\linewidth]{"../../../../../../../Downloads/Henneberg Fläche"}
		\caption{Henneberg-Fläche}
		\label{fig:henneberg-flache}
	\end{figure}
\end{itemize}

\subsection{Anwendungen von Minimalflächen
	\label{minimalflaechen:subsection:Anwendungen von Minimalflächen}}
Minimalflächen haben zahlreiche Anwendungen in der realen Welt.
In der Architektur und im Bauwesen werden sie genutzt, um stabile und ästhetisch ansprechende Strukturen zu entwerfen.
Ihre Eigenschaft, die Fläche zu minimieren, macht sie ideal für Konstruktionen, die leicht werden müssen und damit weniger Material brauchen.

Ein bekanntes Beispiel für die Anwendung von Minimalflächen in der Architektur ist das Münchner Olympiastadion.
Das Dach dieses Stadions basiert auf Prinzipien von Minimalflächen und zeigt, wie solche Flächen genutzt werden können, um beeindruckende und funktionale Bauwerke zu schaffen.
Durch die Minimierung der Fläche wird nicht nur Material gespart, sondern auch eine elegante und leichte Struktur geschaffen, die dennoch stabil und langlebig ist.

Ein alltägliches Beispiel für eine Minimalfläche ist eine Seifenblase.
Wenn man einen Draht in eine Seifenlösung taucht und dann herauszieht, bildet sich zwischen den Drähten eine dünne Seifenhaut, die eine Minimalfläche darstellt.
Diese Seifenhaut strebt danach, ihre Oberfläche zu minimieren und so die Fläche zwischen den Drähten so klein wie möglich zu halten.
Die Kräfte, die auf die Seifenhaut wirken, sorgen dafür, dass sie diese minimalen Flächen einnimmt.

