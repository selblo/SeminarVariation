%
% main.tex -- Paper zum Thema <minimalflaechen>
%
% (c) 2020 Autor, OST Ostschweizer Fachhochschule
%
% !TEX root = ../../buch.tex
% !TEX encoding = UTF-8
%
\chapter{Minimalflächen\label{chapter:minimalflaechen}}
\kopflinks{Minimalflächen}
\begin{refsection}
	\chapterauthor{Ronja Allenfort und Ana Milivojevic}
\index{Ronja Allenfort}%
\index{Allenfort, Ronja}%
\index{Ana Milivojevic}%
\index{Milivojevic, Ana}%

\noindent
Unter einer Minimalfläche versteht man eine beliebige Fläche im
\index{Minimalfläche}%
Raum mit der Spezialität, dass der Flächeninhalt jeweils dem lokalen
Minimum entspricht.
Gefunden werden solche Flächeninhalte über Variationsrechnungen
unter Berücksichtigung der gegebenen Randbedingungen.
Ihre Anwendungen finden sie in den verschiedensten Bereichen der
Mathematik oder Physik.
	
Die Erforschung der Minimalflächen begann im 18. Jahrhundert mit
den Arbeiten des französischen Mathematikers Joseph-Louis Lagrange.
\index{Lagrange, Joseph-Louis}%
\index{Joseph-Louis Lagrange}%
1760 formulierte Lagrange das Problem der Minimalflächen als ein
Variationsproblem.
Er war einer der ersten Mathematiker, der systematisch die Bedingungen
untersuchte, unter denen eine Fläche einen minimalen Flächeninhalt
aufweist.
Diese frühen Arbeiten legten den Grundstein für die Variationsrechnung.

Ein weiterer wichtiger Beitrag zur Theorie der Minimalflächen kam
von Jean-Baptiste Meusnier im Jahr 1776.
\index{Jean-Baptiste Meusnier}%
\index{Meusnier, Jean-Baptiste}%
Meusnier zeigte, dass Minimalflächen eine mittlere Krümmung von
null haben.
Er entdeckte, dass die mittlere Krümmung an jedem Punkt einer
Minimalfläche null ist, was bedeutet, dass sich die Fläche lokal
nicht weiter verkleinern lässt.
Diese Erkenntnis war ein bedeutender Fortschritt in der Theorie der
Minimalflächen und half, die mathematischen Grundlagen dieses Gebiets
weiter zu festigen.

Im 19. Jahrhundert führte der belgische Physiker Joseph Plateau
Experimente mit Seifenfilmen durch.
\index{Joseph Plateau}%
\index{Plateau, Joseph}%
Plateau spannte Seifenfilme in Drahtschlingen und beobachtete, dass
diese Filme immer eine Form annahmen, die eine Minimalfläche
darstellt.
Diese Experimente bestätigten viele theoretische Vorhersagen und
zeigten, dass die mathematischen Modelle tatsächlich die physikalische
Realität beschreiben.
Plateaus Arbeiten haben wesentlich dazu beigetragen, das Verständnis
von Minimalflächen zu vertiefen und ihre Bedeutung in der realen
Welt aufzuzeigen.

%
% einleitung.tex -- Beispiel-File für die Einleitung
%
% (c) 2020 Prof Dr Andreas Müller, Hochschule Rapperswil
%
% !TEX root = ../../buch.tex
% !TEX encoding = UTF-8
%
\section{Einleitung\label{kettenlinie:section:Einleitung}}
\rhead{Einleitung}
Die Kettenlinie, beschreibt die natürliche Form einer idealen, homogenen und flexiblen Kette, die unter ihrem eigenen Gewicht hängt und an beiden Enden befestigt ist.
Diese Kurve hat in der Mathematik und Physik eine besondere Bedeutung, da sie als Lösung eines Variationsproblems auftritt und die Minimalfläche zwischen zwei Punkten darstellt.
Die Kettenlinie stellt ein Variationsproblem dar, weil sie die Form einer Kurve beschreibt, die eine bestimmte physikalische Bedingung erfüllt: die Minimierung der potenziellen Energie. 


%
% teil1.tex -- Beispiel-File für das Paper
%
% (c) 2020 Prof Dr Andreas Müller, Hochschule Rapperswil
%
% !TEX root = ../../buch.tex
% !TEX encoding = UTF-8
%
\section{Problemstellung\label{kettenlinie:section:Problemstellung}}
\rhead{Problemstellung}
Die Problemstellung der Kettenlinie besteht darin, die Form einer flexiblen, homogenen und nicht dehnbaren Kette zu bestimmen, die an zwei festen Punkten aufgehängt ist und unter dem Einfluss der Schwerkraft hängt.
Mathematisch wird dies als die Suche nach einer Kurve \( y(x) \) beschrieben, die diese Bedingungen erfüllt.
Gegeben für die Problemstellung sind: 
\begin{itemize}
\item
Zwei Aufhängepunkte \( A(x_1, y_1) \) und \( B(x_2, y_2) \).
\item
Eine homogene Kette, d.h., ihre Dichte \( \rho \) ist konstant.
\item
Die Kette ist flexibel und kann sich frei bewegen.
\end{itemize}
Ziel ist es, die Form der Kurve \( y(x) \), welche die Kette zwischen den beiden Punkten annimmt, zu bestimmen.
Wie diese Formel hergeleitet wird, ist im Kapitel \ref{kettenlinie:subsection:Minimum der potenziellen Energie} genauer beschrieben.

\subsection{Mathematische Formulierung
\label{kettenlinie:subsection:Mathematische Formulierung}}
Die potenzielle Energie der Kette wird durch das Integral
\begin{equation}
	U = \int_{x_1}^{x_2} \rho g y \sqrt{1 + (y')^2} \, dx
\end{equation}
beschrieben, wobei \( y' \) die Ableitung von \( y \) nach \( x \) ist und \( g \) die Gravitationskonstante.
Dieses Integral soll minimiert werden, was zur entsprechenden Euler-Lagrange-Gleichung führt.




\printbibliography[heading=subbibliography]
\end{refsection}
