%
% teil3.tex -- Beispiel-File für Teil 3
%
% (c) 2020 Prof Dr Andreas Müller, Hochschule Rapperswil
%
% !TEX root = ../../buch.tex
% !TEX encoding = UTF-8
%
\subsection{Crossover 
\label{genetic_algorithm:crossover}}
In diesem Teil des Algorithmus werden die gewählten Elternpaare 
neu kombiniert, um Nachkommen zu erzeugen. Bei der Kreuzung 
werden Teile des genetischen String ausgetauscht.

\begin{figure} [h]
	\centering
	\includegraphics[width=0.8\textwidth]{
        papers/variationsprinzip_algorithmen/images/teil3/05_genetic_string_cross.png
        }
	\caption{einfaches Beispiel wie eine Einpunkt-Kreuzung stattfinden könnte}
	\label{fig:one_point_crossover}
\end{figure}

für dieses gibt es unterschiedliche Taktiken

- Einpunkt-Kreuzung: Ein zufälliger Punkt wird auf den 
Elternchromosomen ausgewählt. Die Gene vor diesem Punkt 
stammen vom ersten Elternteil, die Gene nach diesem Punkt 
vom zweiten Elternteil. Mathematische Formel:
Wähle einen zufälligen Punkt \( k \) (1 ≤ k < n).
Erzeuge die Nachkommen durch:
\[ O_1 = (P_1[1], P_1[2], \ldots, P_1[k], P_2[k+1], \ldots, P_2[n]) \]
\[ O_2 = (P_2[1], P_2[2], \ldots, P_2[k], P_1[k+1], \ldots, P_1[n]) \]
\\
- Zweipunkt-Kreuzung: Zwei Punkte werden ausgewählt, und 
der Genabschnitt zwischen diesen Punkten wird zwischen 
den Eltern getauscht.
Wähle zwei zufällige Punkte \( k_1 \) und \( k_2 \) (1 ≤ k_1 < k_2 < n).
Erzeuge die Nachkommen durch:
\[ O_1 = (P_1[1], \ldots, P_1[k_1], P_2[k_1+1], \ldots, P_2[k_2], P_1[k_2+1], \ldots, P_1[n]) \]
\[ O_2 = (P_2[1], \ldots, P_2[k_1], P_1[k_1+1], \ldots, P_1[k_2], P_2[k_2+1], \ldots, P_2[n]) \]
\\
- Uniforme Kreuzung: Jedes Gen wird mit einer bestimmten 
Wahrscheinlichkeit vom ersten oder zweiten Elternteil 
übernommen, was zu einer zufälligeren Kombination führt.
Mathematische Formel:
Für jedes Gen \( i \) (1 ≤ i ≤ n):
    - Wähle eine zufällige Zahl \( r_i \) im Intervall [0, 1].
    - Wenn \( r_i \) kleiner als die vordefinierte Wahrscheinlichkeit \( p \) ist, dann wird das Gen von \( P_1 \) übernommen, ansonsten von \( P_2 \).

Mathematisch:

\[ O_1[i] = \begin{cases} 
P_1[i] & \text{wenn } r_i < p \\
P_2[i] & \text{wenn } r_i \geq p 
\end{cases} \]

\[ O_2[i] = \begin{cases} 
P_2[i] & \text{wenn } r_i < p \\
P_1[i] & \text{wenn } r_i \geq p 
\end{cases} \]

Hier wird die Variation aufrechterhalten, aber gleichzeitig 
wird aus den besten Genen neue erstellt in der Hoffnung ein 
noch besseres Resultat zu bekommen.

Wie schon in der Initialisierung kann nicht ein normaler 
genetischer String \ref{fig:one_point_crossover} mit An und 
Aus verwendet werden. Da würde man ein Resultat, wie auf 
nachfolgendem Bild \ref{fig:one_point_crossover_cities} erhalten.

\begin{figure} [h]
	\centering
	\includegraphics[width=0.8\textwidth]{
        papers/variationsprinzip_algorithmen/images/teil3/07_genetic_string_cities_crossover_standard.png
        }
	\caption{Beispiel einer Einpunkt-Kreuzung mit Städten}
	\label{fig:one_point_crossover_cities}
\end{figure}

Für das Travelings Salesman Problem wird die Kreuzung mit andern Systemen 
angepasst, damit es keine Doppelten gibt.

Für den Script wurde Order Crossover \ref{fig:crossover_order_cities} genutzt.
Dieser Algorithmusentfernt einen Teil des Strings, wie bei Zweipunkt-Kreuzung. 
Dabei fügt es nach der Reihenfolge des Elternteil 2 alle nicht vorhanden 
Städte bis der herausgeschnittene String wieder vollständige ist.

\begin{figure} [h]
	\centering
	\includegraphics[width=0.8\textwidth]{
        papers/variationsprinzip_algorithmen/images/teil3/08_genetic_string_cities_crossover_simple.png
        }
	\caption{Beispiel einer Einpunkt-Kreuzung mit Städten}
	\label{fig:crossover_order_cities}
\end{figure}

