%
% teil3.tex -- Beispiel-File für Teil 3
%
% (c) 2020 Prof Dr Andreas Müller, Hochschule Rapperswil
%
% !TEX root = ../../buch.tex
% !TEX encoding = UTF-8
%
\section{Anwendung des genetische Algorithmus
\label{genetic_algorithm:section:process}}
\rhead{Teil 3}
Der Genetische Algorithmus nutzt nach ChatGPT das Variationsprinzip 
und dies wird mit nachfolgender Aussage so definiert.
\\
\\
Das Variationsprinzip ist ein grundlegendes Konzept in der 
evolutionären Computertechnik, insbesondere in genetischen 
Algorithmen. Es besagt, dass genetische Vielfalt in einer 
Population von Individuen aufrechterhalten werden muss, 
um eine effektive Suche im Suchraum zu ermöglichen und eine 
Lösung für das zugrunde liegende Problem zu finden.
\\
Der genetische Algorithmus nutzt dieses Variationsprinzip, um eine 
Population von möglichen Lösungen zu einem Problem zu entwickeln 
und zu verfeinern.\cite{chatgpt2024}

In den nächsten Abschnitt wird der Ablauf des genetischen Algorithmus 
auf das Travelling Salesman Problem angepasst und angeschaut, das Kapitel
konzentiert sich eher auf den logischen Ablauf und weniger, um die 
Mathematischen Berechnung.

%
% teil3.tex -- Beispiel-File für Teil 3
%
% (c) 2020 Prof Dr Andreas Müller, Hochschule Rapperswil
%
% !TEX root = ../../buch.tex
% !TEX encoding = UTF-8
%
\subsection{Initialisierung
\label{genetic_algorithm:initialization}}
Der Startpunkt des genetischen Algorithmus ist die Initialisierung.
Dabei wird eine zufällige Population von möglichen Lösungen erstellt.
Diese wird als ein genetischer String dargestellt.

\begin{figure} [h]
	\centering
	\includegraphics[width=0.8\textwidth]{
        papers/variationsprinzip_algorithmen/images/teil2/01_genetic_string.png
        }
	\caption{Beispiel von möglichen Genetic String}
	\label{fig:possible_genetic_string}
\end{figure}

Dabei wird in jeder Position das Gen aktiviert mit 1 oder deaktiviert mit 0.
Problematik für Stätte funktioniert dies nicht, da wir eine Stadt nicht
einfach aus oder anschalten können. Beim Gen wie oben ändert sich die funktioniert
an der Position nicht. Beispiel Feld 2 ist veranwortlich, dass die Farbe Grün
dargestellt wird. Das ein und auschalten der Städten ist nicht möglich, da sich
es auf die Reihenfolge ankommt. Daher wird wie im nachfolgendem Bild 
\cite{cities_genetic_string}, die jeweilige Nummer der Stadt verwendet.

\begin{figure} [h]
	\centering
	\includegraphics[width=0.8\textwidth]{
        papers/variationsprinzip_algorithmen/images/teil2/02_genetic_string_cities.png
        }
	\caption{Beispiel von Stätten in einem Genetic String dargestellt}
	\label{fig:cities_genetic_string}
\end{figure}


%
% teil3.tex -- Beispiel-File für Teil 3
%
% (c) 2020 Prof Dr Andreas Müller, Hochschule Rapperswil
%
% !TEX root = ../../buch.tex
% !TEX encoding = UTF-8
%
\subsection{Evaluation
\label{genetic_algorithm:evaluation}}
Dieser Schritt befasst sich mit der auswertung der einzelnen 
Kombinationen.

In der Informatik wird die Liste genommen und die einzelnen 
zusammenstellungen ausgerechnet

Dafür wird die gleiche Fromel\ref{eq:bruteforce_min_formula}, 
wie in der Bruteforce verwendet.

\begin{figure} [h]
	\centering
	\includegraphics[width=0.8\textwidth]{
        papers/variationsprinzip_algorithmen/images/teil3/03_genetic_string_cities_results.png
        }
	\caption{Beispiel von genetischen String mit Resultate}
	\label{fig:cities_genetic_string_results}
\end{figure}


%
% teil3.tex -- Beispiel-File für Teil 3
%
% (c) 2020 Prof Dr Andreas Müller, Hochschule Rapperswil
%
% !TEX root = ../../buch.tex
% !TEX encoding = UTF-8
%
\subsection{Selektion 
\label{genetic_algorithm:selection}}
In diesem Schritt werden Elternpaare ausgesucht, welche 
in einem späteren schritt neue Kinderpaare erzeugen. Die 
Selektion soll aber so seine, dass in der Regel nur aus 
den Fittesten neue Kinder erzeugen werden.

Die Idee dahinter ist, dass man eine Reihe von verschiedenen 
Teile und diese sind mal fitter und weniger. Dabei Paaren sich
die Fitteren eher und die schwächeren seltener. Die Paare werden 
im System zufällig gewählt, aber die Wahrscheinlichkeiten das 
sich die fitteren eher nachwuchs erzeugen.

TODO add bild mit verschieden und unterschiedlichen Wahrscheinlichkeiten,
dabei aufzeigen die 



Für die Selektion
gibt es verschiedene Möglichkeiten.

- Roulette-Rad-Selektion: Individuen werden zufällig und proportional 
zu ihrer Fitness ausgewählt. Die Wahrscheinlickeit wird Anhand der 
der Fitness definiert.Im Fall mit den Stätten haben kürzere 
Strecken eine höhere Chance ausgesucht zu werden.
Die Wahrscheinlickeit wird mit dieser Formel berechnet

\begin{equation}
    \label{eq:probability_fittest}
    P_i = \frac{f_i}{\sum_{j=1}^{N} f_j}
\end{equation}

- Rangselektion: Individuen werden nach ihrer Fitness sortiert 
und basierend auf ihrem Rang ausgewählt. Die Wahrscheinlickeit 
wird anahnd des Rangs definiert.

\begin{equation}
    \label{eq:probability_rating}
    P_i = \frac{r_i}{\sum_{j=1}^{N} r_j}
\end{equation}

- Turnierselektion: Eine Gruppe von Individuen wird zufällig 
ausgewählt, und das fitteste Individuum dieser Gruppe wird 
als Elternteil gewählt.

Grundsätzlich lässt sich ein eigenes System entwickeln, wobei
es ein Ausscheidungsverfahren braucht, aus welchem dann ein 
Elternpaare herauskommt.

Das System ist ein logisches Ablauf und mathematisch wird nur 
die Wahrscheinlickeit berechnet.


%
% teil3.tex -- Beispiel-File für Teil 3
%
% (c) 2020 Prof Dr Andreas Müller, Hochschule Rapperswil
%
% !TEX root = ../../buch.tex
% !TEX encoding = UTF-8
%
\subsection{Crossover 
\label{genetic_algorithm:crossover}}
In diesem Teil des Algorithmus werden die gewählten Elternpaare 
neu kombiniert, um Nachkommen zu erzeugen. Bei der Kreuzung 
werden Teile des genetischen String ausgetauscht.

\begin{figure} [h]
	\centering
	\includegraphics[width=0.8\textwidth]{
        papers/variationsprinzip_algorithmen/images/teil3/05_genetic_string_cross.png
        }
	\caption{einfaches Beispiel wie eine Einpunkt-Kreuzung stattfinden könnte}
	\label{fig:one_point_crossover}
\end{figure}

für dieses gibt es unterschiedliche Taktiken

- Einpunkt-Kreuzung: Ein zufälliger Punkt wird auf den 
Elternchromosomen ausgewählt. Die Gene vor diesem Punkt 
stammen vom ersten Elternteil, die Gene nach diesem Punkt 
vom zweiten Elternteil. Mathematische Formel:
Wähle einen zufälligen Punkt \( k \) (1 ≤ k < n).
Erzeuge die Nachkommen durch:
\[ O_1 = (P_1[1], P_1[2], \ldots, P_1[k], P_2[k+1], \ldots, P_2[n]) \]
\[ O_2 = (P_2[1], P_2[2], \ldots, P_2[k], P_1[k+1], \ldots, P_1[n]) \]
\\
- Zweipunkt-Kreuzung: Zwei Punkte werden ausgewählt, und 
der Genabschnitt zwischen diesen Punkten wird zwischen 
den Eltern getauscht.
Wähle zwei zufällige Punkte \( k_1 \) und \( k_2 \) (1 ≤ k_1 < k_2 < n).
Erzeuge die Nachkommen durch:
\[ O_1 = (P_1[1], \ldots, P_1[k_1], P_2[k_1+1], \ldots, P_2[k_2], P_1[k_2+1], \ldots, P_1[n]) \]
\[ O_2 = (P_2[1], \ldots, P_2[k_1], P_1[k_1+1], \ldots, P_1[k_2], P_2[k_2+1], \ldots, P_2[n]) \]
\\
- Uniforme Kreuzung: Jedes Gen wird mit einer bestimmten 
Wahrscheinlichkeit vom ersten oder zweiten Elternteil 
übernommen, was zu einer zufälligeren Kombination führt.
Mathematische Formel:
Für jedes Gen \( i \) (1 ≤ i ≤ n):
    - Wähle eine zufällige Zahl \( r_i \) im Intervall [0, 1].
    - Wenn \( r_i \) kleiner als die vordefinierte Wahrscheinlichkeit \( p \) ist, dann wird das Gen von \( P_1 \) übernommen, ansonsten von \( P_2 \).

Mathematisch:

\[ O_1[i] = \begin{cases} 
P_1[i] & \text{wenn } r_i < p \\
P_2[i] & \text{wenn } r_i \geq p 
\end{cases} \]

\[ O_2[i] = \begin{cases} 
P_2[i] & \text{wenn } r_i < p \\
P_1[i] & \text{wenn } r_i \geq p 
\end{cases} \]

Hier wird die Variation aufrechterhalten, aber gleichzeitig 
wird aus den besten Genen neue erstellt in der Hoffnung ein 
noch besseres Resultat zu bekommen.

Wie schon in der Initialisierung kann nicht ein normaler 
genetischer String \ref{fig:one_point_crossover} mit An und 
Aus verwendet werden. Da würde man ein Resultat, wie auf 
nachfolgendem Bild \ref{fig:one_point_crossover_cities} erhalten.

\begin{figure} [h]
	\centering
	\includegraphics[width=0.8\textwidth]{
        papers/variationsprinzip_algorithmen/images/teil3/07_genetic_string_cities_crossover_standard.png
        }
	\caption{Beispiel einer Einpunkt-Kreuzung mit Städten}
	\label{fig:one_point_crossover_cities}
\end{figure}

Für das Travelings Salesman Problem wird die Kreuzung mit andern Systemen 
angepasst, damit es keine Doppelten gibt.

Für den Script wurde Order Crossover \ref{fig:crossover_order_cities} genutzt.
Dieser Algorithmusentfernt einen Teil des Strings, wie bei Zweipunkt-Kreuzung. 
Dabei fügt es nach der Reihenfolge des Elternteil 2 alle nicht vorhanden 
Städte bis der herausgeschnittene String wieder vollständige ist.

\begin{figure} [h]
	\centering
	\includegraphics[width=0.8\textwidth]{
        papers/variationsprinzip_algorithmen/images/teil3/08_genetic_string_cities_crossover_simple.png
        }
	\caption{Beispiel einer Einpunkt-Kreuzung mit Städten}
	\label{fig:crossover_order_cities}
\end{figure}


%
% teil3.tex -- Beispiel-File für Teil 3
%
% (c) 2020 Prof Dr Andreas Müller, Hochschule Rapperswil
%
% !TEX root = ../../buch.tex
% !TEX encoding = UTF-8
%
\subsection{Mutation
\label{genetic_algorithm:mutation}}
Der Startpunkt des genetischen Algorithmus ist die Initialisierung.
Dabei wird eine zufällige Population von möglichen Lösungen erstellt.
Diese wird als ein genetischer String dargestellt.

\begin{figure} [h]
	\centering
	\includegraphics[width=0.8\textwidth]{
        papers/variationsprinzip_algorithmen/images/teil2/01_genetic_string.png
        }
	\caption{Beispiel von möglichen Genetic String}
	\label{fig:possible_genetic_string}
\end{figure}

Dabei wird in jeder Position das Gen aktiviert mit 1 oder deaktiviert mit 0.
Problematik für Stätte funktioniert dies nicht, da wir eine Stadt nicht
einfach aus oder anschalten können. Beim Gen wie oben ändert sich die funktioniert
an der Position nicht. Beispiel Feld 2 ist veranwortlich, dass die Farbe Grün
dargestellt wird. Bei den Städten ändert sich aber die Reihenfolge, da wird 
nicht einfach Ein oder ausgeschaltet. Zur einfachheit wird in den Stellen 
die Nummer der Stadt genommen.

\begin{figure} [h]
	\centering
	\includegraphics[width=0.8\textwidth]{
        papers/variationsprinzip_algorithmen/images/teil2/02_genetic_string_cities.png
        }
	\caption{Beispiel von Stätten in einem Genetic String dargestellt}
	\label{fig:cities_genetic_string}
\end{figure}


in Geld 2 wird immer die Farbe Grün angezeigt,  Ausserdem ändern nicht die Reihenfolge
der der Stätte, wodurch bei änderungen nicht nur die Position 
Bei den Stätten ist dies nicht einfach so umzusetzen da Bei einem Rucksack, welches einzelne Gegenstände hat 



%
% teil3.tex -- Beispiel-File für Teil 3
%
% (c) 2020 Prof Dr Andreas Müller, Hochschule Rapperswil
%
% !TEX root = ../../buch.tex
% !TEX encoding = UTF-8
%
\subsection{Ersetzen
\label{genetic_algorithm:replacement}}
Der Ersetz schritt macht was er aussagt. Meistens wird die ganze 
Population durch die neue Ersetzt. Je nach Strategie macht es Sinn, dass 
die besten Population behalten werden. Wichtig ist das man die gesammt 
grösse der Generation nicht vergrössert, also von den neuen nur die besten 
halten bis der Satz voll ist.

Der Ersetzungsschritt macht genau das, was der Name impliziert. In der 
Regel wird die gesamte Population durch die neue ersetzt. Je nach 
Strategie kann es jedoch sinnvoll sein, die besten Individuen der 
alten Population zu behalten. Wichtig ist, dass die Gesamtgröße der 
Generation nicht vergrößert oder verkleinert wird. Es werden  
nur die besten Individuen der neuen Generation behalten, bis die 
ursprüngliche Populationsgröße wieder erreicht ist.

%
% teil3.tex -- Beispiel-File für Teil 3
%
% (c) 2020 Prof Dr Andreas Müller, Hochschule Rapperswil
%
% !TEX root = ../../buch.tex
% !TEX encoding = UTF-8
%
\subsection{Abbruchkriteriums
\label{genetic_algorithm:termination}}
Was auch in Maschine Learning ein wichtiges Thema ist, ab welchem 
Zustand sollen die Berechnungen beendet werden. Beim generischen 
Algorithmus wird in der Regel eine Anzahl an Generationen definiert.
Möglich wäre auch, dass man ab einer gewissen Fittness Wert, die Lösung 
als genügend empfinden.



