%
% teil3.tex -- Beispiel-File für Teil 3
%
% (c) 2020 Prof Dr Andreas Müller, Hochschule Rapperswil
%
% !TEX root = ../../buch.tex
% !TEX encoding = UTF-8
%
\section{Vergleich mit dem Mathematischen Variationsprinzip
\label{beispiel:section:teil3}}
\rhead{Teil 3}

TODO vor text muss noch angepasst werden
Der Begriff "Variationsberechnung" und die "Variation" in einem genetischen Algorithmus beziehen sich auf unterschiedliche Konzepte und Anwendungen in der Mathematik und Informatik. Hier sind die Unterschiede im Detail:

### Variationsberechnung
Die Variationsberechnung, auch bekannt als Variationsrechnung (englisch "calculus of variations"), ist ein Bereich der mathematischen Analyse. Sie befasst sich mit der Optimierung von Funktionalen, die Abhängigkeiten von Funktionen haben. Das Ziel ist es, eine Funktion zu finden, die ein gegebenes Funktional minimiert oder maximiert.

#### Hauptmerkmale der Variationsberechnung:
- **Problemtyp**: Optimierung von Funktionalen, die oft Integrale sind, über Funktionen.
- **Typische Probleme**: Bestimmung von Kurven oder Oberflächen, die bestimmte physikalische oder geometrische Eigenschaften optimieren (z.B. die Brachistochrone Kurve, die Form einer Seifenblase).
- **Mathematische Techniken**: Euler-Lagrange-Gleichung, Lagrange-Multiplikatoren, Hamiltonsches Prinzip.
- **Anwendungsbereiche**: Physik (z.B. Prinzip der kleinsten Wirkung), Ingenieurwissenschaften, Wirtschaftswissenschaften.

### Variation im genetischen Algorithmus
Die Variation im Kontext eines genetischen Algorithmus bezieht sich auf die Mechanismen, die genetische Vielfalt innerhalb der Population erzeugen und aufrechterhalten. Diese Mechanismen, wie Kreuzung und Mutation, sind entscheidend, um eine effektive Suche und Optimierung im Lösungsraum zu gewährleisten.

#### Hauptmerkmale der Variation im genetischen Algorithmus:
- **Problemtyp**: Optimierung von diskreten oder kontinuierlichen Problemen durch Simulation der natürlichen Evolution.
- **Mechanismen**: 
  - **Kreuzung (Crossover)**: Kombiniert Gene von zwei Eltern, um neue Nachkommen zu erzeugen.
  - **Mutation**: Verändert zufällig Gene innerhalb eines Chromosoms, um neue genetische Informationen einzuführen.
- **Anwendungsbereiche**: Künstliche Intelligenz, maschinelles Lernen, Optimierungsprobleme in verschiedensten Bereichen (z.B. Designoptimierung, Routenplanung).

### Unterschiede im Detail:

1. **Ziel und Anwendungsbereich**:
   - **Variationsberechnung**: Zielt auf die Optimierung von Funktionalen durch Bestimmung optimaler Funktionen. Hauptsächlich in der Mathematik und theoretischen Physik angewendet.
   - **Genetische Algorithmen**: Nutzen Variation, um eine Population von Lösungen für Optimierungsprobleme in der Informatik und Ingenieurwissenschaften zu verbessern.

2. **Methoden und Techniken**:
   - **Variationsberechnung**: Verwendet analytische Methoden wie die Euler-Lagrange-Gleichung zur Lösung von Optimierungsproblemen.
   - **Genetische Algorithmen**: Setzen auf stochastische Methoden wie Kreuzung und Mutation zur Erzeugung und Aufrechterhaltung genetischer Vielfalt.

3. **Art der Optimierung**:
   - **Variationsberechnung**: Fokus auf kontinuierliche Probleme und Funktionale.
   - **Genetische Algorithmen**: Anwendbar auf diskrete und kontinuierliche Probleme durch evolutionäre Strategien.

### Zusammenfassung:
Die Variationsberechnung ist ein analytisches Werkzeug zur Optimierung von Funktionalen in der Mathematik, während die Variation im genetischen Algorithmus eine Methode zur Erzeugung und Aufrechterhaltung genetischer Vielfalt in evolutionären Optimierungsverfahren ist. Beide Konzepte dienen der Optimierung, jedoch in sehr unterschiedlichen Kontexten und mit unterschiedlichen Methoden.

