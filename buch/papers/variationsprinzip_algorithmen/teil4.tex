%
% teil3.tex -- Beispiel-File für Teil 3
%
% (c) 2020 Prof Dr Andreas Müller, Hochschule Rapperswil
%
% !TEX root = ../../buch.tex
% !TEX encoding = UTF-8
%
\section{Variationsprinzip Mathematik und Algorithmus
\label{beispiel:section:teil4}}
\rhead{Teil 5}
Aus dem Vorherigen Kapitel \ref*{genetic_algorithm:section:process} wurde 
erklärt wie der Algorithmus funktioniert, welcher das Variationsprinzip 
anwendet.

--removable--
Mit den Variationsrechnungen werden Funktionale optimiert. Dabei ist das 
Ziel eine Funktion zu finden, welche das Funktional minimiert oder maximiert.

TODO unterschiede aufzeigen
Woran unterscheiden sich die beiden Prinzipien?

\begin{table}[h]
   \centering
   \begin{tabular}{|c|c|c|}
   \hline
      & Mathematik  & Algorithmus   \\ \hline
   Ziele  
   & Wie im Kapitel Funktionale 
   \ref*{buch:variation:problem:subsection:funktionale} ist das Ziel mit 
   der Variationsrechnung die Optimierung von Funktionalen mit der Findung
   einer optimalen Funktion.
   & Im Algorithmus ist mit der Variationen gemeint, dass es eine Menge mit
   möglichen Lösungen gibt, aus welchen die besten Lösungen genommen werden 
   und diese dann weiter verarbeitet werden, in der Hoffnung, dass die 
   neuen Lösungen besser werden.
   \\ \hline
   Techniken  
   & Hier werden analytische Technicken wie die Euler-Lagrange-Gleichung 
   verwendet, um die Optimierungsprobleme zu lösen. (Verschiedenen Techniken 
   sind in den Kapiteln vom Kapitel 2-10 beschrieben)
   & Im Algorithmus werden Mechanismen verwendet, welche stochastische
   \footnote{
      Im Fall von Algorithmus ist mit stochastischen Methoden gemeint, dass
      eine Anzahl an Zufallsereignise oder -Kombinationen erstellt werden 
      und diese ausgewerted oder weiterverarbeitet werden. 
   }
 Methoden wie Kreuzung und Mutation sind, dadurch wird die Vielfalt zu 
 erzeugen und aufrechtzuerhalten. 
   \\ \hline
   \end{tabular}
   \caption{Vergleich auf den begriff Variationen}
   \label{tab:example_bruteforce_cities}
\end{table}



Die Hauptmerkmale davon sind:



haben das Ziel die Optimierung der Funktionalen. Durch 
die

In diesem Kapitel gehen wir genauer darauf ein. 
Jetzt das wir wissen wie die Mechanik funktioniert und das wir hier keine 
Mathematische Rechnungen haben. 

TODO vor text muss noch angepasst werden
Der Begriff "Variationsberechnung" und die "Variation" in einem genetischen Algorithmus beziehen sich auf unterschiedliche Konzepte und Anwendungen in der Mathematik und Informatik. Hier sind die Unterschiede im Detail:

### Variationsberechnung
Die Variationsberechnung, auch bekannt als Variationsrechnung (englisch "calculus of variations"), ist ein Bereich der mathematischen Analyse. Sie befasst sich mit der Optimierung von Funktionalen, die Abhängigkeiten von Funktionen haben. Das Ziel ist es, eine Funktion zu finden, die ein gegebenes Funktional minimiert oder maximiert.

#### Hauptmerkmale der Variationsberechnung:
- **Problemtyp**: Optimierung von Funktionalen, die oft Integrale sind, über Funktionen.
- **Typische Probleme**: Bestimmung von Kurven oder Oberflächen, die bestimmte physikalische oder geometrische Eigenschaften optimieren (z.B. die Brachistochrone Kurve, die Form einer Seifenblase).
- **Mathematische Techniken**: Euler-Lagrange-Gleichung, Lagrange-Multiplikatoren, Hamiltonsches Prinzip.
- **Anwendungsbereiche**: Physik (z.B. Prinzip der kleinsten Wirkung), Ingenieurwissenschaften, Wirtschaftswissenschaften.

### Variation im genetischen Algorithmus
Die Variation im Kontext eines genetischen Algorithmus bezieht sich auf die Mechanismen, die genetische Vielfalt innerhalb der Population erzeugen und aufrechterhalten. Diese Mechanismen, wie Kreuzung und Mutation, sind entscheidend, um eine effektive Suche und Optimierung im Lösungsraum zu gewährleisten.

#### Hauptmerkmale der Variation im genetischen Algorithmus:
- **Problemtyp**: Optimierung von diskreten oder kontinuierlichen Problemen durch Simulation der natürlichen Evolution.
- **Mechanismen**: 
  - **Kreuzung (Crossover)**: Kombiniert Gene von zwei Eltern, um neue Nachkommen zu erzeugen.
  - **Mutation**: Verändert zufällig Gene innerhalb eines Chromosoms, um neue genetische Informationen einzuführen.
- **Anwendungsbereiche**: Künstliche Intelligenz, maschinelles Lernen, Optimierungsprobleme in verschiedensten Bereichen (z.B. Designoptimierung, Routenplanung).

### Unterschiede im Detail:

1. **Ziel und Anwendungsbereich**:
   - **Variationsberechnung**: Zielt auf die Optimierung von Funktionalen durch Bestimmung optimaler Funktionen. Hauptsächlich in der Mathematik und theoretischen Physik angewendet.
   - **Genetische Algorithmen**: Nutzen Variation, um eine Population von Lösungen für Optimierungsprobleme in der Informatik und Ingenieurwissenschaften zu verbessern.

2. **Methoden und Techniken**:
   - **Variationsberechnung**: Verwendet analytische Methoden wie die Euler-Lagrange-Gleichung zur Lösung von Optimierungsproblemen.
   - **Genetische Algorithmen**: Setzen auf stochastische Methoden wie Kreuzung und Mutation zur Erzeugung und Aufrechterhaltung genetischer Vielfalt.

3. **Art der Optimierung**:
   - **Variationsberechnung**: Fokus auf kontinuierliche Probleme und Funktionale.
   - **Genetische Algorithmen**: Anwendbar auf diskrete und kontinuierliche Probleme durch evolutionäre Strategien.

### Zusammenfassung:
Die Variationsberechnung ist ein analytisches Werkzeug zur Optimierung von Funktionalen in der Mathematik, während die Variation im genetischen Algorithmus eine Methode zur Erzeugung und Aufrechterhaltung genetischer Vielfalt in evolutionären Optimierungsverfahren ist. Beide Konzepte dienen der Optimierung, jedoch in sehr unterschiedlichen Kontexten und mit unterschiedlichen Methoden.

