%
% teil3.tex -- Beispiel-File für Teil 3
%
% (c) 2020 Prof Dr Andreas Müller, Hochschule Rapperswil
%
% !TEX root = ../../buch.tex
% !TEX encoding = UTF-8
%
\subsection{Selektion 
\label{genetic_algorithm:selection}}
In diesem Schritt werden Elternpaare ausgesucht, welche 
in einem späteren schritt neue Kinderpaare erzeugen. Die 
Selektion soll aber so seine, dass in der Regel nur aus 
den Fittesten neue Kinder erzeugen werden.

Die Idee dahinter ist, dass man eine Reihe von verschiedenen 
Teile und diese sind mal fitter und weniger. Dabei Paaren sich
die Fitteren eher und die schwächeren seltener. Die Paare werden 
im System zufällig gewählt, aber die Wahrscheinlichkeiten das 
sich die fitteren eher nachwuchs erzeugen.

TODO add bild mit verschieden und unterschiedlichen Wahrscheinlichkeiten,
dabei aufzeigen die 



Für die Selektion
gibt es verschiedene Möglichkeiten.

- Roulette-Rad-Selektion: Individuen werden zufällig und proportional 
zu ihrer Fitness ausgewählt. Die Wahrscheinlickeit wird Anhand der 
der Fitness definiert.Im Fall mit den Stätten haben kürzere 
Strecken eine höhere Chance ausgesucht zu werden.
Die Wahrscheinlickeit wird mit dieser Formel berechnet

\begin{equation}
    \label{eq:probability_fittest}
    P_i = \frac{f_i}{\sum_{j=1}^{N} f_j}
\end{equation}

- Rangselektion: Individuen werden nach ihrer Fitness sortiert 
und basierend auf ihrem Rang ausgewählt. Die Wahrscheinlickeit 
wird anahnd des Rangs definiert.

\begin{equation}
    \label{eq:probability_rating}
    P_i = \frac{r_i}{\sum_{j=1}^{N} r_j}
\end{equation}

- Turnierselektion: Eine Gruppe von Individuen wird zufällig 
ausgewählt, und das fitteste Individuum dieser Gruppe wird 
als Elternteil gewählt.

Grundsätzlich lässt sich ein eigenes System entwickeln, wobei
es ein Ausscheidungsverfahren braucht, aus welchem dann ein 
Elternpaare herauskommt.

Das System ist ein logisches Ablauf und mathematisch wird nur 
die Wahrscheinlickeit berechnet.

