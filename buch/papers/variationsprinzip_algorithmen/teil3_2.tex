%
% teil3.tex -- Beispiel-File für Teil 3
%
% (c) 2020 Prof Dr Andreas Müller, Hochschule Rapperswil
%
% !TEX root = ../../buch.tex
% !TEX encoding = UTF-8
%
\subsection{Evaluation
\label{genetic_algorithm:evaluation}}
\rhead{Evaluation}
Dieser Schritt befasst sich mit der auswertung der einzelnen 
Kombinationen.

In der Informatik wird die Liste genommen und die einzelnen 
Zusammenstellung wird berechnet.

Dafür wird die gleiche Formel \ref{eq:bruteforce_min_formula}, 
verwendet, die auch im Bruteforce-Methode Anwendung findet.

\begin{figure} [h]
	\centering
	\includegraphics[width=0.8\textwidth]{
        papers/variationsprinzip_algorithmen/images/teil3/03_genetic_string_cities_results.png
        }
	\caption{Beispiel eines genetischen Strings mit Ergebnissen}
	\label{fig:cities_genetic_string_results}
\end{figure}

