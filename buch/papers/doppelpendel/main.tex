%
% main.tex -- Paper zum Thema <doppelpendel>
%
% (c) 2020 Autor, OST Ostschweizer Fachhochschule
%
% !TEX root = ../../buch.tex
% !TEX encoding = UTF-8
%
\documentclass{article}
\usepackage[utf8]{inputenc}
\usepackage{biblatex}

\addbibresource{bibliography.bib}

\chapter{Thema\label{chapter:doppelpendel}}
\kopflinks{Thema}
\begin{refsection}
\chapterauthor{Shaarujan Kamalanathan}

Ein paar Hinweise für die korrekte Formatierung des Textes
\begin{itemize}
\item
Absätze werden gebildet, indem man eine Leerzeile einfügt.
Die Verwendung von \verb+\\+ ist nur in Tabellen und Arrays gestattet.
\item
Die explizite Platzierung von Bildern ist nicht erlaubt, entsprechende
Optionen werden gelöscht. 
Verwenden Sie Labels und Verweise, um auf Bilder hinzuweisen.
\item
Beginnen Sie jeden Satz auf einer neuen Zeile. 
Damit ermöglichen Sie dem Versionsverwaltungssysteme, Änderungen
in verschiedenen Sätzen von verschiedenen Autoren ohne Konflikt 
anzuwenden.
\item 
Bilden Sie auch für Formeln kurze Zeilen, einerseits der besseren
Übersicht wegen, aber auch um GIT die Arbeit zu erleichtern.
\end{itemize}

%
% einleitung.tex -- Beispiel-File für die Einleitung
%
% (c) 2020 Prof Dr Andreas Müller, Hochschule Rapperswil
%
% !TEX root = ../../buch.tex
% !TEX encoding = UTF-8
%
\section{Einleitung\label{circuit:section:teil0}}
\rhead{Einleitung}
Das folgende kapitel beschäftig sich mit der Frage wie man elektronische DC Schaltungen ohne die zuhilfenahme von Kirchhoff's gesetzen lösen kannm mit hilfe von Variationsrechnungen.  \cite{circuit:bibtex}.





%
% teil1.tex -- Beispiel-File für das Paper
%
% (c) 2020 Prof Dr Andreas Müller, Hochschule Rapperswil
%
% !TEX root = ../../buch.tex
% !TEX encoding = UTF-8
%
\section{Kirchhoffs Gesetz
\label{circuit:section:teil1}}
\rhead{Problemstellung}
Bevor wir die Laplace-Gleichung mithilfe der Lagrange-Funktion herleiten, versuchen wir das auf konventionelle Weise, indem wir  das kirchhoffsche Gesetz benutzen. 
Dafür brauchen wir aber zuerst noch ein paar Erkenntnisse, die hier aufgezeigt werden. Die Stromdichte $\vec{J}$ wird durch
\begin{equation}
	\vec{J}=\frac{\vec{I}}{A}
	\label{circuit:current_density_3}
\end{equation}
definiert. Diese Gleichung besagt, dass die Stromdichte das Verhältnis des Stroms $\vec{I}$ zur Flächen $A$ ist. 
Das kirchhoffsche Stromgesetz postuliert nun, dass die Summe der in einen Knoten einfliessenden Ströme gleich der Summe der aus diesem Knoten ausfliessenden Ströme in einer Schaltung ist. Dies impliziert, dass es in einem Gleichstromkreis im stationären Zustand keine Ladungsakkumulation an irgendeinem Punkt geben kann. Wir betrachten nun einen dreidimensionalen Schaltkreis, in dem die Leitfähigkeit $\sigma$ im gesamten Bereich von Interesse konstant ist. Die Verallgemeinerung des kirchhoffschen Stromgesetzes im dreidimensionalen Fall besagt, dass die Divergenz der Stromdichte $\vec{J}$ gleich Null ist, es kann kein Strom aus dem Nichts erzeugt werden. Daher gilt 
%\eqref{circuit:current_density_1}.
\begin{equation}
	\nabla \cdot  \vec{J}=0.
	\label{circuit:current_density_1}
\end{equation}

Zudem lässt sich der Zusammenhang zwischen dem elektrischen Feld $\vec{E}$ an einem gegebenen Punkt im Raum als negativer Gradient 
\begin{equation}
	\vec{E}=-\nabla \phi
	\label{circuit:current_density_4}
\end{equation}
der Potentialfunktion $\phi$ ausdrücken.

Wenn das elektrische Feld durch $\vec{E}$ repräsentiert wird und wir ein ohmsches Material vorliegen haben, kann die Stromdichte auch als Produkt  
\begin{equation}
\vec{J}=\sigma \vec{E}
\label{circuit:current_density_2}
\end{equation}
beschrieben werden.
Mit dem elektrischen Feld als Gradient von $\phi$ erhalten wir aus der Quellenfreiheit des Stromes \eqref{circuit:current_density_1} und dem ohmschen Gesetz \eqref{circuit:current_density_2} die Differentialgleichung 
\begin{equation}
	\nabla \cdot (\sigma \nabla \phi)=0
	\label{circuit:current_density_5}
\end{equation}
für das Potential $\phi$. Dies kann auch als
\begin{equation}
\nabla^2 \phi=0
\label{circuit:current_density_6}
\end{equation}
geschrieben werden, solange $\sigma$ konstant ist.

\section{Variationsprinzip} 
In diesem Abschnitt wird das Ziel verfolgt, die Gleichung \eqref{circuit:current_density_6} mithilfe eines Variationsprinzips herzuleiten und seine Bedeutung aufzuzeigen. Zunächst werden jedoch die Grundlagen anhand eines einfachen Beispiels wiederholt.
\begin{figure}
	\centering
	\begin{circuitikz}
		\draw (0,3) to[V, v=$U_0$, i=$I_0$] (0,0);
		\draw (0,3) to[short,-*] (3,3)
		to[R, l=$R_1$, i=$I_1$] (3,0) -- (0,0);
		\draw (3,3) -- (6,3)
		to[R, l=$R_2$, i=$I_2$] (6,0) to[short,-*] (3,0);
	\end{circuitikz}
	\caption{Parallelschaltung von $R_1= \SI{2}{\kilo\ohm}$ und $R_2= \SI{1}{\kilo\ohm}$}
	\label{fig:circuit_stromzweig}
\end{figure}

\subsection{Berechnung der Stromverteilung mit klassischen Mitteln} 
Die klassische Methode zur Berechnung der Stromverteilung in einer Parallelschaltung von Widerständen basiert auf dem Ohmschen Gesetz und den Kirchhoffschen Regeln. Das Ohmsche Gesetz lautet:

\begin{equation}
	U=R \cdot I
	\label{circuit:ohmic_law}
\end{equation}

Die Kirchhoffschen Regeln besagen, dass in einem Knotenpunkt eines elektrischen Netzwerkes die Summe der zufliessenden Ströme gleich der Summe der abfliessenden Ströme ist und dass alle Teilspannungen eines Umlaufs bzw. einer Masche in einem elektrischen Netzwerk sich zu null addieren \cite{dewiki:244855415}. Mit diesen Regeln kann man direkt die Ströme in Abbildung \ref{fig:circuit_stromzweig} berechnen und erhält:

\begin{equation}
	I_1 = \frac{I_0 \cdot R_2}{R_1 + R_2} = \frac{\SI{1}{\ampere} \cdot \SI{1}{\kilo\ohm}}{\SI{2}{\kilo\ohm}+ \SI{1}{\kilo\ohm}}=\SI{0.333}{\ampere}
	\label{circuit:current_circuit_power_example3}
\end{equation}

\begin{equation}
	I_2 = I_0-I_1=\SI{1}{\ampere}-\SI{0.333}{\ampere}=\SI{0.667}{\ampere}
	\label{circuit:current_circuit_power4}
\end{equation}
unter der Bedingung das $I_0=\SI{1}{\ampere}$.

Diese Berechnung scheint auf den ersten Blick einfach zu sein, doch es gibt auch eine alternative Methode, das Minimalprinzip, das zum gleichen Ergebnis führt.

\subsection{Das Minimalprinzip als Alternative} 
Das Minimalprinzip bietet eine alternative Methode zur Berechnung der Stromverteilung. Es besagt, dass ein physikalisches System so arbeitet, dass eine bestimmte Größe minimiert wird. In unserem Fall ist diese Größe die Leistung der Schaltung, die durch die Formel
\begin{equation}
	P(I_1)=  I_1^2 \cdot R_1+  (I_0-I_1)^2 \cdot R_2
	\label{circuit:current_circuit_power}
\end{equation}
ausgedrückt wird. Um das Minimum zu finden, leiten wir die Leistung nach $I_1$ ab und setzen die Ableitung gleich Null. Dies führt zu:
\begin{equation}
	\frac{dP}{dI_1} = 2\cdot I_1\cdot R_1 - 2\cdot (I_0 - I_1) \cdot R_2.
	\label{circuit:current_circuit_power1}
\end{equation}
Die zweite Ableitung 
\begin{equation}
	\frac{d^2P}{dI_1^2} = 2\cdot R_1 + 2\cdot R_2
	\label{circuit:current_circuit_power2}
\end{equation}
zeigt eindeutig, dass es sich um ein Minimum handelt, da $R_1$ und $R_2$ positiv sind und daher die zweite Ableitung positiv ist. Wenn wir nun die erste Ableitung \eqref{circuit:current_circuit_power1} gleich Null setzen und auf $I_1$ auflösen bekommen wir
\begin{equation}
	I_1 = \frac{I_0 \cdot R_2}{R_1 + R_2} = \frac{\SI{1}{\ampere} \cdot \SI{1}{\kilo\ohm}}{\SI{2}{\kilo\ohm}+ \SI{1}{\kilo\ohm}}=\SI{0.333}{\ampere}
	\label{circuit:current_circuit_power_a}
\end{equation}
und analog dazu
\begin{equation}
	I_2 = I_0-I_1=\SI{1}{\ampere}-\SI{0.333}{\ampere}=\SI{0.667}{\ampere},
	\label{circuit:current_circuit_power_b}
\end{equation}
wie auch schon in in Gleichung \eqref{circuit:current_circuit_power_example3} und Gelichung \eqref{circuit:current_circuit_power4}.

Dies demonstriert erfolgreich, dass die Verteilung des Stroms in der Schaltung direkt mit der Minimierung der gesamten Leistung korrespondiert. Dies kann auch grafisch wie in Abbildung \ref{fig:circuit_power} dargestellt werden.

\begin{figure}
	\centering
	\includegraphics[width=0.7\textwidth]{papers/circuit/two_parrallel_resistors.png}
	\caption{Leistung ($z$-Achse) von zwei parallel geschalteten Widerständen 1 Kilo Ohm ($x$-Achse) und 2 Kilo Ohm ($y$-Achse), in grau dargestellt die Schnittfläche des Stromes von einem Ampere.}
	\label{fig:circuit_power}
\end{figure}

\subsection{Verallgemeinerung auf den stetigen Fall}
Die obige Analyse kann auf den stetigen Fall verallgemeinert werden, indem die Leistung $P$ in einem von einer Oberfläche $S$ umgebenen Volumen $V$ durch 
\begin{equation}
	P=\int_V \sigma(\nabla \phi)^2 d V
	\label{circuit:current_density_7}
\end{equation}
ausdrücken. Wenn wir nun die Euler-Ostrogradski-Differentialgleichung für \eqref{circuit:current_density_7} bestimmen und lösen, gibt uns dies anschliessend die Lösung für die minimale Leistung in einem zweidimensionalen oder dreidimensionalen Raum. D. h. um das Minimum zu finden muss das Integral von \eqref{circuit:current_density_7} minimiert werden. Auf den ersten Blick mag \eqref{circuit:current_density_7} nicht sehr intuitiv erscheinen. Daher könnten wir die Gleichung auch anders formulieren, wie in 
\begin{equation}
	P=\frac{U^2}{R}
	\label{circuit:current_density_8}
\end{equation}
gezeigt, wobei $U^2=\left( \nabla \phi \right)^2$ und $R=\frac{1}{\sigma}$.





%
%
%
%Analog zu dem oben aufgeführten Beispiel können wir die Leistung $P$ in einem von einer Oberfläche $S$ umgebenen Volumen $V$ durch 
%\begin{equation}
%	P=\int_V \sigma(\nabla \phi)^2 d V
%	\label{circuit:current_density_7}
%\end{equation}
%ausdrücken. Wenn wir nun die Euler-Ostrogradski-Differentialgleichung für \eqref{circuit:current_density_7} bestimmen und lösen gibt uns dies anschliessend die Lösung für die minimale Leistung in einem zweidimensionalen oder dreidimensionalen Raum. D.h. um das Minimum zu finden muss das Integral von \eqref{circuit:current_density_7} minimiert werden. Auf den ersten Blick mag \eqref{circuit:current_density_7} nicht sehr intuitiv erscheinen. Daher könnten wir die Gleichung auch anders formulieren, wie in 
%\begin{equation}
%	P=\frac{U^2}{R}
%	\label{circuit:current_density_8}
%\end{equation}
%gezeigt, wobei $U^2=\left( \nabla \phi \right)^2$ und $R=\frac{1}{\sigma}$.


\section{Herleitung der Laplace-Gleichung}
In diesem Kapitel leiten wir die Laplace-Gleichung her. Die Herleitung basiert auf der Euler-Ostrogradski-Differentialgleichung, die in Gleichung \eqref{buch:felder:ostrogradski:eqn:euler-ostrogradski} dargestellt ist.

Wir beginnen mit Gleichung \eqref{circuit:current_density_7} und wenden darauf die genannte Differentialgleichung an:
\begin{enumerate}
	\item Schritt: Lagrange-Funktion des Problems ohne $\sigma$ (da wir das Minimum suchen und $\sigma$ eine Konstante ist hat $\sigma$ keinen Einfluss auf die Lösung und kann daher weggelassen werden)
	\begin{equation}
		L(U, U_x)= U_x^2 = (U_{x_1}^2+U_{x_2}^2).
	\end{equation}
	\item Schritt: partielle Ableitungen
	\begin{equation}
		\begin{aligned}
			\frac{\partial L}{\partial U}&=0\\
			\frac{\partial L}{\partial U_{x_1}}&=2U_{x_1}\\
			\frac{\partial L}{\partial U_{x_2}}&=2U_{x_2}.\\
		\end{aligned}
	\end{equation}
	\item Schritt: Ableiten nach $x_1$ und $x_2$
	\begin{equation}
		\begin{aligned}
			\frac{\partial}{\partial x_1}\frac{\partial L}{\partial U_{x_1}}(x,\phi,\nabla \phi)=2\frac{\partial \phi}{\partial {x_1}}\cdot \frac{\partial}{\partial x_1},\\
			\frac{\partial}{\partial x_2}\frac{\partial L}{\partial U_{x_2}}(x,\phi,\nabla \phi)=2\frac{\partial \phi}{\partial {x_2}} \cdot \frac{\partial}{\partial x_1}.\\
		\end{aligned}
	\end{equation}
	\item Schritt: Euler-Ostrogradski Differentialgleichung
	\begin{equation}
		0=-\frac{\partial}{\partial x_1}\cdot 2\frac{\partial \phi}{\partial {x_1}}-\frac{\partial}{\partial x_2}\cdot 2\frac{\partial \phi}{\partial {x_2}}=-2\Delta\phi.
	\end{equation}
\end{enumerate}

Das Ergebnis der Anwendung der Theorie der Euler-Ostrogradski-Differentialgleichung ist:
	\begin{equation}
	\sigma \cdot 2\Delta\phi=0.
	\end{equation}
Wir können nun noch durch $2\sigma$ teilen und erhalten die Laplace-Gleichung aus \eqref{circuit:current_density_6}. Somit wurde gezeigt, dass die Laplace-Gleichung sowohl durch das Variationsprinzip als auch durch die Kirchhoffschen Regeln gefunden werden kann.



\section{Praktische Anwendungen}
In diesem Abschnitt werden wir die numerische Lösung der elliptischen partiellen Differentialgleichung \eqref{circuit:current_density_6} untersuchen und ihre Bedeutung erläutern.

\subsection{Diskretisierung der Ableitungen} Die Gleichung \eqref{circuit:current_density_6} kann durch eine Differenz zweiter Ordnung diskretisiert werden, wie in Gleichung \eqref{circuit:second-order-central} dargestellt:

\begin{equation}
	f^{\prime \prime}(x) \approx \frac{\delta_h^2[f](x)}{h^2}=\frac{\frac{f(x+h)-f(x)}{h}-\frac{f(x)-f(x-h)}{h}}{h}=\frac{f(x+h)-2 f(x)+f(x-h)}{h^2}.
	\label{circuit:second-order-central}
\end{equation}
\cite{enwiki:1220817436}

\subsection{Diskretisierung in zwei Dimensionen} Um diese Diskretisierung auf unseren zweidimensionalen Fall anzuwenden, führen wir ein Gitter ein, dessen Punkte durch die Koordinaten $(x_i, y_j)$ repräsentiert werden. Die Differenzen zwischen den Gitterpunkten in x- und y-Richtung werden durch $\Delta x$ und $\Delta y$ dargestellt. Dies führt zu folgender diskretisierter Gleichung:
\begin{equation}
	\frac{\phi(x_{i+1}, y_j) - 2\phi(x_i, y_j) + \phi(x_{i-1}, y_j)}{(\Delta x)^2} + \frac{\phi(x_i, y_{j+1}) - 2\phi(x_i, y_j) + \phi(x_i, y_{j-1})}{(\Delta y)^2} = 0.
	\label{circuit:discret_equation}
\end{equation}

\subsection{Auflösen nach $\phi(x_i, y_j)$} 
Wir können nun nach $\phi(x_i, y_j)$ auflösen und erhalten:
\begin{equation}
	\phi(x_i, y_j) = \frac{1}{4}(\phi(x_{i+1}, y_{j}) + \phi(x_{i-1}, y_{j}) + \phi(x_{i}, y_{j+1}) + \phi(x_{i}, y_{j-1})).
	\label{circuit:discret_equation2}
\end{equation}
Diese Gleichung ermöglicht es uns, das Potential an jedem Gitterpunkt zu berechnen, indem wir die Potentiale der umliegenden Punkte verwenden. Die Form von \eqref{circuit:discret_equation2} ist eine andere Darstellung von \eqref{circuit:discret_equation} für die numerische Implementierung.

\subsection{Numerisches Beispiel} 
\subsubsection{Quadratisches Kontakt}
Betrachten wir als Beispiel eine leitende Platte mit einer Leitfähigkeit von \SI{0.001}{\siemens\per\meter} und einer Größe von einem Quadratmeter. Ein spezifischer Bereich der Platte, Quadrat definiert durch $0.5 < x < 0.7$ und $0.5 < y < 0.7$, wird mit einem Potential von einem Volt belegt, während der Rand der Platte ein Potential von 0 Volt aufweist.

\begin{figure}
	\centering
	\includegraphics[width=0.99\textwidth]{papers/circuit/potential_distribution.png}
	\caption{Potential-Verteilung auf rechteckiger Platte mit rechteckigem Potential im oberen rechten Bereich und 0 Potential am Rand der Platte \cite{github:AndreasFMueller}}
	\label{fig:potential_distribution}
\end{figure}
Unter Verwendung von Gleichung \eqref{circuit:discret_equation2} zur Berechnung des Potentials und den gegebenen Randbedingungen erhalten wir die in Abbildung \ref{fig:potential_distribution} dargestellte Potentialverteilung. Dies ermöglicht uns, das gesamte Potential auf der Platte zu bestimmen.

Sobald das Potential bekannt ist, können wir mithilfe des Gradienten und Gleichung \eqref{circuit:current_density_7} die Leistungsdichte an jedem einzelnen Punkt berechnen. Dies führt zu den in Abbildung \ref{fig:power_2d} und \ref{fig:power_3d_rectangle} dargestellten Leistungsdichteverteilung.
\begin{figure}[h]
	\centering
	\includegraphics[width=0.99\textwidth]{papers/circuit/power_distribution.png}
	\caption{Leistungsdichte auf rechteckiger Platte mit rechteckigem Potential im oberen rechten Bereich und 0 Potential am Rand der Platte. (Code für die Generierung des Plots kann in \cite{github:AndreasFMueller} gefunden werden.)}
	\label{fig:power_2d}
\end{figure}
\begin{figure}[h]
	\centering
	\includegraphics[width=0.99\textwidth]{papers/circuit/power_distribution_circle.png}
	\caption{Leistungsdichte auf rechteckiger Platte mit kreisförmigen Potential im oberen rechten Bereich und 0 Potential am Rand der Platte. (Code für die Generierung des Plots kann in \cite{github:AndreasFMueller} gefunden werden.)}
	\label{fig:power_2d_circle}
\end{figure}
\begin{figure}[h]
	\centering
	\includegraphics[width=0.99\textwidth]{papers/circuit/3d.png}
	\caption{Leistungsdichte 3d auf rechteckiger Platte mit rechteckigem Potential und 0 Potential am Rand der Platte. (Code für die Generierung des Plots kann in \cite{github:AndreasFMueller} gefunden werden.)}
	\label{fig:power_3d_rectangle}
\end{figure}
\begin{figure}[h]
	\centering
	\includegraphics[width=0.99\textwidth]{papers/circuit/3d_circle.png}
	\caption{Leistungsdichte 3d auf rechteckiger Platte mit kreisförmigen Potential und 0 Potential am Rand der Platte. (Code für die Generierung des Plots kann in \cite{github:AndreasFMueller} gefunden werden.)}
	\label{fig:power_3d_circle}
\end{figure}
Abbildung \ref{fig:power_3d_rectangle} zeigt deutlich, dass die Leistungsdichte in den Ecken des zuvor definierten quadratischen Potentials am höchsten ist. Dies ist unter anderem ein Grund, warum auf Leiterplatten normalerweise keine 90°-Winkel für Leiterbahnen gezeichnet werden, sondern meistens 45°-Winkel verwendet werden, um die Leistungsdichte bzw. Stromdichte in den Ecken zu minimieren.

\subsubsection{Kreisförmiges Kontakt} Wenn wir anstelle eines rechteckigen Potentials ein kreisförmiges Potential verwenden, ist die Leistungsdichte deutlich geringer, wie in Abbildung \ref{fig:power_3d_circle} und Abbildung \ref{fig:power_2d_circle} dargestellt. Daher ist es vorteilhaft, abgerundete Ecken zu verwenden, wenn man die Leistungsdichte auf einer Leiterbahn minimieren möchte.

Diese Beispiele illustrieren die praktische Anwendung des Variationsprinzips und der numerischen Lösung von partiellen Differentialgleichungen in der Elektrotechnik. Sie zeigen, wie die Form von Leiterbahnen auf einer Leiterplatte optimiert werden können, um die Leistungsdichte zu minimieren und die Effizienz zu maximieren.
%
% teil2.tex -- Beispiel-File für teil2 
%
% (c) 2020 Prof Dr Andreas Müller, Hochschule Rapperswil
%
% !TEX root = ../../buch.tex
% !TEX encoding = UTF-8
%
\section{Herleitung der Balkengleichung
	\label{balken:section:teil2}}
\subsection{Variationsprinzip}
Die Variationsrechnung ist ein Teilgebiet der Analysis, das sich mit kleinen Änderungen in Funktionen und Funktionalen beschäftigt, um Minima und Maxima von Funktionen zu ermitteln. Dabei handelt es sich um mathematische Ausdrücke, die Integrale über eine unbekannte Funktion und ihre Ableitung darstellen können. Ziel ist es, ein Maximum, ein Minimum oder einen Sattelpunkt ausfindig zu machen.

In der Mechanik kommen Variationsrechnungen oft zum Einsatz, da sie die Grundlage aller physikalischen Extremalrechnungen bilden. \ref{balken:Variationsrechnung}

\subsection{Minimalprinzip}
Das Minimalprinzip ist ein Konzept, das besagt, dass ein physikalisches System einen Zustand annimmt, der mit dem geringsten Energieaufwand erreicht wird. In der Physik wird das Minimalprinzip oft formuliert, indem eine minimale Wirkung oder Energie angestrebt wird.

Ein Beispiel hierfür ist eine Feder, die an einem Ende an einer Wand befestigt ist und an ihrem anderen Ende eine Masse trägt. Zieht man die Masse nach unten und lässt sie los, nimmt sie eine Position ein, bei der die potenzielle Energie minimal ist.

Das Minimalprinzip in Bezug auf die Balkengleichung ist ein grundlegendes Konzept der Mechanik, auch bekannt als das Prinzip von Hamilton. Es besagt, dass ein System den Gleichgewichtszustand annimmt, bei dem die potenzielle Energie minimal ist. Für einen Balken tritt dieser Zustand ein, wenn alle äusseren Kräfte, Momente, inneren Beanspruchungen sowie Verformungen des Balkens im Gleichgewicht stehen. Die Anwendung dieses Minimalprinzips führt zur Balkengleichung, die die Gleichgewichtsbedingungen des Balkens beschreibt. 

\subsection{Herleitung der Balkengleichung aus dem Variationsprinzip}
Die Verformungen des Balkens aufgrund der auftretenden Biegespannungen $σ_x$ werden durch
\begin{equation}
	\sigma_x = \frac{E}{p} z
\end{equation}
und
\begin{equation}
	\sigma_x = \frac{M_y}{I_y} z
\end{equation}
beschrieben. Setzt man diese beiden Gleichungen gleich, erhält man
\begin{equation}
	\frac{E}{p} z = \frac{M_y}{I_y} z
\end{equation}
und kürzt anschliessend $z$ heraus, so bekommt man
\begin{equation}
	\frac{E}{p} = \frac{M_y}{I_y}
\end{equation}
Dividiert man diese Gleichung durch $E$, um $p$ zu isolieren, erhält man die Formel für den Krümmungsradius
\begin{equation}
	\frac{1}{p} = \frac{M_y}{E I_y} = \kappa
\end{equation}

Bei Biegungen, die aufgrund von Querkräften auftreten, ist das Moment veränderlich und hängt von der Position $x$ ab. Dies führt dazu, dass die Krümmung des Balkens bzw. der Biegelinie von der Position $x$ abhängt. Um die Krümmung zu bestimmen, benötigt man den kürzesten Weg zwischen beiden Auflagern des Balkens.
\begin{figure}[h]
	\centering
	\includegraphics[width=0.8\textwidth]{papers/balken/images/teil2/BiegungBalke2.jpg}
	\caption{Darstellung der Biegelinie $y(x)$ mit dem Balken (rot gekennzeichnet) und dessen Auflagern.}
	\label{fig:Darstellung_der_Biegelinie}
\end{figure}

Zuerst berechnet man die Länge einer Geraden auf der Kurve $y(x)$:
\begin{equation}
	\Delta s = \sqrt{\Delta x^2 + \Delta y^2} \approx \sqrt{1 + y'(x)^2} \cdot \Delta x
\end{equation}
und danach mit dem Variationsprinzip die Kurvenlänge $l(y)$:
\begin{equation}
	l(y) = \int_{x_1}^{x_2} \sqrt{1 + {y'(x)}^2} \, dx
\end{equation}

Um das Variationsprinzip auf die Balkengleichung anwenden zu können, betrachtet man die potenzielle Energie des Balkens. Diese setzt sich aus der Biegeenergie sowie der Energie der äueren Kräfte zusammen. Die potenzielle Energie im Balken wird minimiert, wenn sich das System im Gleichgewicht befindet.
\begin{figure} [h]
	\centering
	\includegraphics[width=0.8\textwidth]{papers/balken/images/teil2/federgesetz.pdf}
	\caption{Veranschaulichung zur Energie im Balken durch das Flächenträgheitsmoment}
	\label{fig:Veranschaulichung zur Energie im Balken durch das Flächenträgheitsmoment}
\end{figure}

Die Energiedichte des Balkens an einem Punkt $x$ ist gegeben durch
\begin{equation}
	\frac{1}{2} E I \left( \frac{\partial^2 w}{\partial x^2} \right)^2
\end{equation}

Hierbei ist $E I$ das Produkt aus dem Elastizitätsmodul und dem Flächenträgheitsmoment $I$, $w(x)$ beschreibt die Durchbiegung des Balkens.

Zusätzlich zur inneren Energie kommt noch die Last $q(x)$, sodass das Funktional der Euler-Bernoulli-Gleichung wie folgt aussieht:
\begin{equation}
	\int_0^L \left( -\frac{1}{2} \left( \frac{\partial^2 w}{\partial x^2} \right)^2 + q(x) w(x) \right) \, dx
\end{equation}

Für zeitabhängige Durchbiegungen $w(x,t)$ kommt noch ein kinetischer Energieterm hinzu:
\begin{equation}
	\frac{1}{2} \mu \left( \frac{\partial w}{\partial t} \right)^2
\end{equation}

Die Lagrange-Funktion lautet daher:
\begin{equation}
	L(x,w,w',w'') = -\frac{1}{2} E I (w'')^2 + q w
\end{equation}

Die Ableitungen der Lagrange-Funktion ergeben:
\begin{equation}
	\begin{align}
		\frac{\partial L}{\partial w} &= q \\
		\frac{\partial L}{\partial w'} &= 0 \\
		\frac{\partial L}{\partial w''} &= -E I w''
	\end{align}
\end{equation}

Da die Lagrange-Funktion eine höhere Ableitung enthält, erweitert sich die Euler-Lagrange-Differentialgleichung zu:
\begin{equation}
	\frac{\partial L}{\partial w} - \frac{d}{dx} \frac{\partial L}{\partial w'} + \frac{d^2}{dx^2} \frac{\partial L}{\partial w''} = 0
\end{equation}

Einsetzen der Ableitungen ergibt:
\begin{equation}
	\begin{align}
		q - \frac{d^2}{dx^2}(E I w'') &= 0 \\
		\Rightarrow w''''(x) &= \frac{q}{E I}
	\end{align}
\end{equation}

Unsere Lagrange-Funktion ist somit:
\begin{equation}
	L(x,y,y') = \sqrt{1 + {y'}^2}
\end{equation}

Mittels der partiellen Ableitung von $L$ erhalten wir die folgende Formel:
\begin{equation}
	\frac{\partial L}{\partial y} = 0
\end{equation}

Durch weiteres Rechnen erhalten wir:
\begin{equation}
	\frac{\partial L}{\partial y'} = \frac{y'}{\sqrt{1 + {y'}^2}}
\end{equation}

Setzt man $y(x)$ ein und leitet nach $x$ ab, ergibt sich:
\begin{equation}
	\frac{d}{dx} \frac{\partial L}{\partial y'}(x,y(x),y'(x)) = \frac{d}{dx} \frac{y'(x)}{\sqrt{1 + {y'}^2}} = \frac{(1 + {y'(x)}^2 - {y'(x)}^2) y''(x)}{(1 + {y'(x)}^2)^{\frac{3}{2}}}
\end{equation}

Dies führt zu:
\begin{equation}
	\frac{d}{dx} \frac{\partial L}{\partial y'}(x,y(x),y'(x)) = \frac{y''}{(1 + {y'}^2)^{\frac{3}{2}}}
\end{equation}

Um die folgende Gleichung zu erhalten:
\begin{equation}
	\kappa = \frac{1}{p} = \pm \frac{y''}{(1 + {y'}^2)^{\frac{3}{2}}}
\end{equation}

Da wir in unserem Fall $w(x)$ als Funktion für die Biegelinie verwenden, müssen wir das Vorzeichen bestimmen. Da ein positives Moment eine Biegung nach unten verursacht, müssen wir ein negatives Vorzeichen verwenden. Somit erhalten wir:
\begin{equation}
	\kappa=
	\frac{1}{p}=
	-\frac{w''}{\left(1+{w'}^2\right)^\frac{3}{2}}
\end{equation}
mit
\begin{equation}
	w'=
	\frac{dy}{dx} 
\end{equation}
und
\begin{equation}
	w''=
	\frac{d^2y}{dx^2}
\end{equation}
$w$ = Funktion der Durchbiegung

$w’$ = Neigung der Durchbiegung

$w’’$ = Krümmung

Da wir in der Baustatik dem Rechtssystem verwenden, kehren wir den $z$-Achse so, dass nach unten das positive Vorzeichen ist.
\begin{figure}
\centering
	\includegraphics[width=0.8\textwidth]{papers/balken/images/teil2/BiegungverdrehteAchsen.jpg}
\caption{Abbildung von den verdrehten $z$-Achse und die positive Momenten, welche auf der Biegelinie wirken.}
\label{fig:Abbildung von den verdrehten $z$-Achse und die positive Momenten, welche auf der Biegelinie wirken.}
\end{figure}

Unsere Funktion zeigt eine Krümmung nach links, jedoch durch die Spiegelung an der $z$-Achse ergibt sich eine Krümmung nach rechts.
Bei Rechtskrümmungen sind die 2. Ableitungen kleiner als 0. $w’’ < 0$.
Das ergibt
\begin{equation}
	\kappa=
	-\frac{w''}{\left(1+{w'}^2\right)^\frac{3}{2}}=
	\frac{M_y}{EI_y}
\end{equation}
Der Term $w’$ kann vernachlässigt werden, da im Betracht des Hookesche Gesetz nur kleine Verformungen vorliegen.
Daraus ergeben sich Tangentensteigungen von $w’ << 1$.
\begin{equation}
	\kappa=
	-\frac{w''}{\left(1+{w'}^2\right)^\frac{3}{2}}=
	-\frac{w''}{\left(1+0\right)^\frac{3}{2}}=
	-\frac{w''}{1}=-w''=
	\frac{M_y}{EI_y}
\end{equation}
Daraus ergibt sich.
\begin{equation}
	w''=
	-\frac{M_y}{EI_y}
\end{equation}
mit
\begin{equation}
	\kappa=
	-w''
\end{equation}.

Temperaturunterschiede verursachen Verformungen in der Balkenachse.
Daher ist es wichtig, die Krümmung der Biegelinie bei Temperaturänderungen zu berücksichtigen.
Daraus ergibt sich der Formel.
\begin{equation}
	w''=
	-\frac{M_y}{EI_y}-\alpha_{\text{th}}\frac{\Delta T}{h}
\end{equation}
$α_th$ = thermische Ausdehnungskoeffizient des Balkenmaterials

$ΔT$ = Temperaturunterschied

$h$ = Höhe des Balkens

$I_y$ = Flächenträgheitsmoment

Den oben genannten Formel kann auch folgend angegeben werden.
\begin{equation}
	w''''=
	\left(-\frac{M_y}{EI_y}-\alpha_{\text{th}}\frac{\Delta T}{h}\right)^{''}
\end{equation}
Bei einer linearen Temperaturverlauf ergibt sich bei der zweiten Ableitung 0.
\begin{equation}
	w''''=
	\left(-\frac{M_y}{EI_y}\right)^{''}
\end{equation}
Die erste Ableitung des Biegemoments ergibt die Querkraft.
\begin{equation}
	\frac{dM}{dx}=
	M'=
	Q
\end{equation}
Daraus ergibt sich.
\begin{equation}
	w'''=
	-\frac{Q}{(EI_y)'}
\end{equation}
Mit konstanter Biegesteifigkeit $EI = konst.$ erfolgt.
\begin{equation}
	EIw^{'''}=
	-Q\left(x\right)
\end{equation}

Die 1. Ableitung der Querkraft bzw. die 2, Ableitung des Biegemoments ergibt der Linienlast entlang der x-Achse.
\begin{equation}
	\frac{dQ}{dx}=
	Q'=
	-q(x)
\end{equation}
\begin{equation}
	w''''=
	\frac{-q(x)}{(EI_y)''}
\end{equation}
Mittels dieser Formel kann die Biegelinie $w(x)$ ermittelt werden
\begin{equation}
	EIw^{''''}=
	q\left(x\right) 
\end{equation}
Jetzt Integrieren wir die Formel viermal.

1. Integration (Querkraft)
\begin{equation}
	EIw^{'''}=
	\int q_0dx=
	q_0\cdot x+C_1
\end{equation}

2. Integration (Biegemoment)
\begin{equation}
	EIw''=
	\int{q_0\cdot x}dx+\int C_1dx=
	\frac{1}{2}q_0x^2+C_1x+C_2
\end{equation}

3. Integration
\begin{equation}
	EIw'=
	\int{\frac{1}{2}q_0x^2}dx+\int{C_1x}dx+\int C_2dx=
	\frac{1}{6}q_0x^3+\frac{1}{2}C_1x^2+C_2x+C_3
\end{equation}

4. Integration
\begin{equation}
	EIw=
	\int{\frac{1}{6}q_0x^3}dx+\int{\frac{1}{2}C_1x^2}dx+\int{C_2x}dx+\int C_3=
	\frac{1}{24}q_0x^4+\frac{1}{6}C_1x^3+\frac{1}{2}C_2x^2+C_3x+C_4
\end{equation}
Jetzt werden die Randbedingungen berücksichtigt.
In unserem Fall haben wir bei $(x_1, y_1)$ einen Festlager und bei $(x_2, y_2)$ ein Loslager.
Dabei gelten für den Fest- und Loslager $w = 0$ und $M = 0$, das bedeutet, dass an diese Stellen keine Verschiebung in $z$-Richtung stattfindet und dass keinen Momenten aufgenommen werden können.

Es gilt
\begin{equation}
	EIw'' =
	-M_y
\end{equation}
und
\begin{equation}
	EIw'''=
	-Q_z
\end{equation}.

Als erstes betrachten wir die Stelle $(x_1, y_1)$ mit den Festlager.
Dabei setzen wir den Ursprung unser Koordinatennetz bei $(x_1, y_1)$, damit ergibt sich $x = 0$.
Die Bedingungen sind: $w = 0$ und $M = 0$.
\begin{equation}
	EIw=
	\frac{1}{24}q_0x^4+\frac{1}{6}C_1x^3+\frac{1}{2}C_2x^2+C_3x+C_4
\end{equation}
$w = 0$ und $x = 0$ einsetzen:
\begin{equation}
	EI\cdot0=
	\frac{1}{24}q_00^4+\frac{1}{6}C_10^3+\frac{1}{2}C_20^2+C_30+C_4
\end{equation}
\begin{equation}
	C_4=
	0
\end{equation}
Für den Momentenberechnung nehmen wir die 2. Ableitung und setzen $M = 0$ ein
\begin{equation}
	EIw''=
	-M_y=
	\frac{1}{2}q_0x^2+C_1x+C_2
\end{equation}
\begin{equation}
	0=
	\frac{1}{2}q_00^2+C_10+C_2
\end{equation}
\begin{equation}
	C_2=
	0
\end{equation}
Jetzt machen wir die gleichen Berechnungen an der Stelle $(x_2, y_2)$ mit den Loslager, für $x_2$ setzen wir $L$ (für der Länge des Balkens) ein.
\begin{equation}
	EIw=
	\frac{1}{24}q_0x^4+\frac{1}{6}C_1x^3+\frac{1}{2}C_2x^2+C_3x+C_4
\end{equation}
$w = 0$ und $x = L$, sowie $C_2 = 0$ und $C_4 = 0$ werden eingesetzt.
\begin{equation}
	EI\cdot0=
	\frac{1}{24}q_0L^4+\frac{1}{6}C_1L^3+\frac{1}{2}0\cdot L^2+C_3L+0
\end{equation}
\begin{equation}
	0=
	\frac{1}{24}q_0L^4+\frac{1}{6}C_1L^3+C_3L
\end{equation}
\begin{equation}
	C_3=
	-\frac{1}{24}q_0L^3-\frac{1}{6}C_1L^2
\end{equation}
Für den Momentenberechnung nehmen wir die 2. Ableitung und setzen $M = 0$, $x = L$ und $C__2 = 0$ ein.
\begin{align}
		EIw''&=
		-M_y=\frac{1}{2}q_0x^2+C_1x+C_2
    \\
		0 &=
		\frac{1}{2}q_0L^2+C_1L+0
    \\
		C_1&=
		-\frac{1}{2}q_0L
\end{align}
$C_1$ in $C_3$ einsetzen:
\begin{equation}
	C_3=
	-\frac{1}{24}q_0L^3-\frac{1}{6}C_1L^2
	=	-\frac{1}{24}q_0L^3-\frac{1}{6}\left(-\frac{1}{2}q_0L\right)L^2
	=	-\frac{1}{24}q_0L^3+\frac{1}{12}q_0L^3
	=	\frac{1}{24}q_0L^3
\end{equation}
Die Konstanten werden in die Biegelinien-Gleichung eingesetzt.
\begin{align}
	EIw&=
	\frac{1}{24}q_0x^4+\frac{1}{6}\left(-\frac{1}{2}q_0L\right)x^3+\frac{1}{2}\cdot0\cdot x^2+\left(\frac{1}{24}q_0L^3\right)x+0
  \\
	EIw&=
	\frac{1}{24}q_0x^4+\frac{1}{6}\left(-\frac{1}{2}q_0L\right)x^3+\left(\frac{1}{24}q_0L^3\right)x
	\\
	EIw&=
	\frac{1}{24}q_0x^4-\frac{1}{12}q_0Lx^3+\frac{1}{24}q_0L^3x
	\\
	EIw&=
	\frac{1}{12}q_0\left(\frac{1}{2}x^4-Lx^3+\frac{1}{2}L^3x\right)
	\\
	w&=
	\frac{1}{12EI}q_0\left(\frac{1}{2}x^4-Lx^3+\frac{1}{2}L^3x\right)
\end{align}

\subsection{Herleitung der Balkengleichung aus dem Baustatik}
Die Herleitung der Balkengleichung lässt sich ebenso gut mit den konventionellen Methoden der Baustatik durchführen, indem man die Beziehung zwischen dem Biegemoment $M$ und der Biegung $w$ verwendet
\begin{figure}
\begin{center}
	\includegraphics[width=0.8\textwidth]{papers/balken/images/teil2/HerleitungBaustatik.jpg}
\end{center}
\caption{Darstellung unsere Balke mit den Auflagern $A$ und $B$ und der Linienlast $q_0$.}
\end{figure}
Gegeben ist:
\begin{equation}
	w''(x)=
	-\frac{M_y(x)}{EI_y}
	\rightarrow-M_y(x)=
	EI_y\cdot w''(x)
\end{equation}.
Die Auflagerkräfte $A$ und $B$
\begin{equation}
	A=
	B=
	\frac{q_0\cdot L}{2}
\end{equation}
benötigen wir um die Schnittmomente an der Stelle $x$ zu berechnen,
\begin{equation}
	M_y(x)=
	A\cdot x-\frac{q_0\cdot x^2}{2}=
	\frac{q_0\cdot L}{2}\cdot x-\frac{q_0\cdot x^2}{2}
\end{equation}
Moment ist gleich Kraft mal Hebelarm.
Jetzt setzen wir $My(x) = EIy \cdot w''(x)$ ein
\begin{align}
		EI_y\cdot w''(x)&=
		\frac{q_0\cdot x^2}{2}-\frac{q_0\cdot L}{2}\cdot x
	\\
		EI_y\cdot w'\left(x\right)&=
		\frac{q_0}{2}\cdot\frac{1}{3}\cdot x^3-\frac{q_0\cdot 	L}{2}\cdot\frac{1}{2}\cdot x^2+C_1
	\\
		EI_y\cdot w\left(x\right)&=
		\frac{q_0}{6}\cdot\frac{1}{4}\cdot x^4-\frac{q_0\cdot 	L}{4}\cdot\frac{1}{3}\cdot x^3+C_1\cdot x+C_2
	\\
		EI_y\cdot w\left(x\right)&=
		\frac{q_0}{24}\cdot x^4-\frac{q_0\cdot L}{12}\cdot x^3+C_1\cdot x+C_2
\end{align}
1. Randbedingung: Durchbiegung an der Stelle $x = 0$ ist 0.
\begin{align}
		EI_y\cdot w\left(x\right)&=
		\frac{q_0}{24}\cdot x^4-\frac{q_0\cdot L}{12}\cdot x^3+C_1\cdot x+C_2
	\\
		0&=
		\frac{q_0}{24}\cdot0^4-\frac{q_0\cdot L}{12}\cdot0^3+C_1\cdot0+C_2
	\\
		C_2=&0
\end{align}
2. Randbedingung: Durchbiegung an der Stelle $x = L$ ist auch 0.
\begin{align}
		EI_y\cdot w\left(x\right)&=
		\frac{q_0}{24}\cdot x^4-\frac{q_0\cdot L}{12}\cdot x^3+C_1\cdot x+C_2
	\\
		0&=
		\frac{q_0}{24}\cdot L^4-\frac{q_0\cdot L}{12}\cdot L^3+C_1\cdot L+0
	\\
		C_1&=
		\frac{q_0\cdot L^3}{24}
\end{align}
Daraus ergibt sich der Formel:
\begin{align}
		EI_y\cdot w\left(x\right)&=
		\frac{q_0}{24}\cdot x^4-\frac{q_0\cdot L}{12}\cdot x^3+\frac{q_0\cdot L^3}{24}\cdot x
	\\
		w&=
		\frac{1}{12EI_y}q_0\left(\frac{1}{2}x^4-Lx^3+\frac{1}{2}L^3x\right)
\end{align}
Das ist äquivalent zu dem, was wir bei der mathematischen Herleitung erhalten haben.

\subsection{Erläuterung der Annahmen und Randbedingungen}
Um die Berechnungen innerhalb der Plattentheorie pragmatischer zu gestalten, werden einige Vereinfachungen vorgenommen und die Randbedingungen festgelegt, unter denen sie gültig sind.
Dadurch können die Berechnungen vereinfacht und lösbar gemacht werden. \ref{balken:Differentialgleichung der Biegelinie}
Zu den Annahmen und Randbedingungen gehören folgende Aspekte:

\begin{enumerate}
	\item Die Balken werden als dünn angenommen, die hat zu bedeuten, dass die Dicke im Vergleich zur Länge des Balkens vernachlässigbar klein ist und deshalb für die Berechnung irrelevant ist.
	\item Die Balken sind eine konstante Linienlast ausgesetzt.
	\item Die Balken werden als ebene Struktur betrachtet, ohne signifikante Krümmungen und Verwindungen.
	\item Im unbelasteten Zustand bleiben Linienabschnitte, die senkrecht zur Mittelfläche stehen, auch im verformten Zustand gerade und senkrecht zur verformten Mittelfläche
	\item Randbedingungen werden festgelegt, wobei die Art der Auflagerung oder Einspannung des Balkens berücksichtigt wird.
	Diese Randbedingungen sind in der untenstehenden Tabelle aufgeführt.
	\item Die Verformungen und Spannungen innerhalb des Balkens werden als vernachlässigbar klein betrachtet und können deshalb ausgeschlossen werden
	\item Die Biegesteifigkeit des Balkens ist als konstant anzunehmen
	\item Es wird angenommen, dass die Temperaturdifferenz, die zu Verformungen der Balkenachse führt, konstant ist, und daher ergibt seine zweite Ableitung Null.
	\item Die Belastungen des Balkens wirken senkrecht zur Achse.
\end{enumerate}
\begin{figure}
\begin{center}
	\includegraphics[width=0.4\textwidth]{papers/balken/images/teil2/Randbedingungen.jpg}
\end{center}
\caption{Tabelle der unterschiedlichen Randbedingungen für verschiedene Auflagertypen.}
\end{figure}

%
% teil3.tex -- Beispiel-File für Teil 3
%
% (c) 2020 Prof Dr Andreas Müller, Hochschule Rapperswil
%
% !TEX root = ../../buch.tex
% !TEX encoding = UTF-8
%
\section{Anwendungen der Balkengleichung
\label{balken:section:teil3}}
\rhead{Anwendungen der Balkengleichung}

Die Differentialrechnung ist von grundlegender Bedeutung für die Untersuchung und Lösung von Problemen im Zusammenhang mit der Balkengleichung. 
In diesem Abschnitt werden einige konkrete Anwendungen der Differentialrechnung in Bezug auf die Balkengleichung erläutert. 
Anschliessend werden Fallstudien und Beispiele vorgestellt, um diese Anwendungen weiter zu veranschaulichen.

\subsection{Praktische Anwendungen im Ingenieurwesen und Physik
\label{Praktische Anwendungen im Ingenieurwissenschaften und Physik}}
\textbf{ Berechnung von Biegemomenten und Biegespannungen:}
Die Differentialrechnung wird verwendet, um das Biegemoment entlang eines Balkens zu bestimmen, der verschiedenen Belastungen ausgesetzt ist. 
Durch die Integration der Biegemomente entlang der Länge des Balkens kann die Biegelinie und somit die Krümmung des Balkens berechnet werden. 
Aus der Krümmung können dann die Biegespannungen mit Hilfe des Elastizitätsmoduls und des Trägheitsmoments ermittelt werden.

\textbf{ Optimierung von Balkenprofilen:}
Durch die Differentialrechnung können Ingenieure die optimale Geometrie von Balkenprofilen bestimmen, um bestimmte Anforderungen hinsichtlich Festigkeit, Steifigkeit und Gewicht zu erfüllen. 
Dies kann durch die Minimierung von Materialkosten oder das Maximieren der strukturellen Leistung erfolgen.

\textbf{ Analyse von statischen und dynamischen Verhalten:}
Die Differentialrechnung ermöglicht es, das statische und dynamische Verhalten von Balken unter verschiedenen Belastungen zu analysieren. 
Dies umfasst die Berechnung von Eigenfrequenzen, Schwingungsmoden und Schwingungsantworten, die für die Bewertung der strukturellen Stabilität und Leistung wichtig sind.

\textbf{ Entwurf von Tragstrukturen:}
Bei der Entwicklung von Tragstrukturen wie Brücken, Gebäuden oder Maschinenkomponenten ist die Differentialrechnung unerlässlich, um die strukturelle Integrität und Zuverlässigkeit zu gewährleisten. 
Sie ermöglicht es Ingenieuren, die Auswirkungen von Lasten und Belastungen auf die Struktur zu verstehen und entsprechende Designentscheidungen zu treffen.

\textbf{ Finite-Elemente-Analyse (FEA):}
Die Finite-Elemente-Methode, ein gängiges Werkzeug zur numerischen Lösung von Balkengleichungen, basiert auf der Differentialrechnung. 
Durch die Unterteilung des Balkens in kleine Elemente und die Anwendung von Differentialgleichungen auf jedes einzelne Element können Ingenieure komplexe strukturelle Probleme lösen und das Verhalten des Balkens unter verschiedenen Bedingungen simulieren.


\subsection{Fallstudien und Beispiele
\label{Fallstudien und Beispiele}}
x


\printbibliography[heading=subbibliography]
\end{refsection}
