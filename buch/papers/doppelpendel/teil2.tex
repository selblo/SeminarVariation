\section{Lagrange-Formalismus}
Das Prinzip der kleinsten Wirkung besagt, dass die Natur immer Trajektoren mit Minimaleigenschaften wählt.
Beispielsweise wählt das Licht durch jedes Medium den Weg mit der kürzesten Zeit, was zur Folge hat,
dass das Licht gebrochen wird wenn es ein Medium verlässt und in ein anderes übergeht.
Dieses Phänomen wird auch im Brechungsgesetz von Snellius beschrieben. (Zitat)

Die Wirkung \(S\) wird wie folgt definiert als
\begin{align}
    S = \int_{t_0}^{t_1} L(t,q_i,\dot{q}_i) \,dt 
\end{align}
Das Ziel ist es dieses sogenannte Funktional zu minimieren.
Dazu reicht es natürlich wenn der Integrand minimal wird.
Dieser wird auch die Lagrange-Funktion genannt.
Um die Lagrange-Funktion zu minimieren muss sie in die Euler-Lagrange Differentialgleichung eingesetzt werden.(Zitat 2.3)
\begin{equation}
    \frac{d}{dt} \left( \frac{\partial L}{\partial \dot{q}_i} \right) 
    - \frac{\partial L}{\partial q_i} = 0
\end{equation}

Der Lagrange-Formalismus verwendet die sogenannten generalisierten 
Koordinaten \(q_i\) und ihre Ableitungen \(\dot{q}_i\), um die Bewegung eines Systems zu beschreiben.
Der Vorteil darin ist man kann beliebigen Koordinaten verwenden, sofern die Energie im System in 
den Koordinaten ausgedrückt werden kann.
Die Lagrange-Funktion \(L\) ist definiert als die Differenz zwischen 
kinetischer Energie \(T\) und potenzieller Energie \(V\).
\begin{equation}
    L = T - V
    \label{eq:lagrange} 
\end{equation}
Dies gilt aber nur für Systeme mit generalisiertem Potential und holonomen Zwangsbedingungen. (Zitat)


