%
% einleitung.tex -- Beispiel-File für die Einleitung
%
% (c) 2020 Prof Dr Andreas Müller, Hochschule Rapperswil
%
% !TEX root = ../../buch.tex
% !TEX encoding = UTF-8
%
\section{Vorbetrachtung zur Balkengleichung\label{balken:section:teil0}}
\kopfrechts{Vorbetrachtung zur Balkengleichung}
\subsection{Was ist die Balkengleichung?}
Die Balkengleichung ist eine fundamentale Differentialgleichung in der Strukturmechanik, die das Verhalten eines Balkens unter mechanischer Belastung beschreibt \cite{balken:Balkentheorie}.
Sie wurde erstmals 1744 von Leonhard Euler mathematisch formuliert und später von anderen Wissenschaftlern wie Eytelwein und Navier weiterentwickelt. 
\index{Leonhard Euler}%
\index{Euler, Leonhard}%
\index{Claude-Louis Navier}%
\index{Navier, Claude-Louis}%
\index{Johann Albert Eytelwein}%
\index{Eytelwein, Johann Albert}%
Diese Gleichung ist ein wesentliches Werkzeug, um das Deformationsverhalten von Balken zu analysieren und zu verstehen \cite{balken:Biegelinie}.

\subsection{Wozu wird die Balkengleichung benötigt?}
Die Balkengleichung wird benötigt, um die Durchbiegung eines Balkens unter Belastung zu berechnen.
Dies ist von entscheidender Bedeutung für das Design und die Konstruktion von Tragwerken wie Brücken, Gebäuden und anderen Bauwerken. 
Durch die genaue Vorhersage der Durchbiegung können Ingenieure sicherstellen, dass die Struktur die erforderliche Festigkeit und Stabilität aufweist, um den auftretenden Belastungen standzuhalten.

\subsection{Zielsetzung der Arbeit}
Das Hauptziel dieser Arbeit widmet sich der umfassenden Untersuchung des Zusammenhangs zwischen der Balkengleichung und dem Variationsprinzip. 
Es werden verschiedene Aspekte beleuchtet, wie die Herleitung der Balkengleichung, ihre praktischen Anwendungen in den Ingenieurwissenschaften und der Physik sowie Fallstudien und Beispiele. 
Das Hauptziel besteht darin, ein tieferes Verständnis für das Verhalten von Strukturen unter Belastung zu erlangen und den Zusammenhang zwischen der Balkengleichung (Biegelinie) und dem Variationsprinzip (Differentialgleichung) zu klären.

Besonderes Augenmerk liegt auf der praktischen Anwendung der Balkengleichung, wobei verschiedene Szenarien betrachtet und analysiert werden, um wichtige Parameter für die Strukturanalyse zu bestimmen. Dies ermöglicht Ingenieuren, fundierte Entscheidungen während des Entwurfs- und Konstruktionsprozesses zu treffen und die Sicherheit sowie Effizienz von Strukturen zu verbessern.

Des Weiteren werden die mathematischen Grundlagen der Balkengleichung behandelt, um ein fundiertes Verständnis für ihre Herleitung und Anwendung zu vermitteln. Dies beinhaltet die Diskussion von Differentialgleichungen und anderen mathematischen Konzepten, die für die Analyse von Balken und anderen Tragwerken relevant sind.

Insgesamt strebt diese Arbeit danach, einen umfassenden Einblick in die Bedeutung und Anwendung der Balkengleichung zu geben. Durch die Untersuchung verschiedener Aspekte und die praktische Anwendung sollen Ingenieure und andere Fachleute auf diesem Gebiet wertvolle Einblicke und Erkenntnisse gewinnen, die zur Verbesserung der Strukturleistung und -sicherheit beitragen können.

