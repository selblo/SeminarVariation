%
% einleitung.tex -- Beispiel-File für die Einleitung
%
% (c) 2020 Prof Dr Andreas Müller, Hochschule Rapperswil
%
% !TEX root = ../../buch.tex
% !TEX encoding = UTF-8
%
\section{Vorbetrachtung zur Balkengleichung\label{balken:section:teil0}}
\subsection{Was ist die Balkengleichung?}
Die Balkengleichung ist eine fundamentale Differentialgleichung in der Strukturmechanik, die das Verhalten eines Balkens unter mechanischer Belastung beschreibt. 
Sie wurde erstmals 1744 von Leonhard Euler mathematisch formuliert und später von ande-ren Wissenschaftlern wie Eytelwein und Navier weiterentwickelt. 
Diese Gleichung ist ein wesentliches Werkzeug, um das Deformationsverhalten von Balken zu analysieren und zu verstehen
\subsection{Wozu wird die Balkengleichung benötigt?}
Die Balkengleichung wird benötigt, um die Durchbiegung eines Balkens unter Belastung zu berechnen.
Dies ist von entscheidender Bedeutung für das Design und die Konstruktion von Tragwerken wie Brücken, Gebäuden und anderen Bauwerken. 
Durch die genaue Vorhersage der Durchbiegung können Ingenieure sicherstellen, dass die Struktur die erforderliche Festigkeit und Stabilität aufweist, um den auftretenden Belastungen standzuhalten.
\subsection{Zielsetzung der Arbeit}
Das Hauptziel der Arbeit besteht darin, den Zusammenhang zwischen der Balkengleichung und der Variationsprinzip eingehend zu untersuchen.
Hierbei werden verschiedene Aspekte beleuchtet, darunter die Herleitung der Balkengleichung, Praktische Anwendungen im Ingenieurwissenschaft und Physik sowie Fallstudien und Beispiele.
Die Arbeit zielt darauf ab, den Zusammenhang zwischen der Balkengleichung (Biegelinie) und der Variationsprinzip (Differentialgleichung) besser zu verstehen und ein vertieftes Verständnis für das Verhalten von Strukturen unter Belastung zu gewinnen.

Ein wichtiger Schwerpunkt liegt auch auf der praktischen Anwendung der Balkengleichung. 
Hierbei werden verschiedene Szenarien betrachtet und analysiert, um die Durchbiegung und andere wichtige Parameter für die Strukturanalyse zu bestimmen.
Dies ermöglicht es Ingenieuren, fundierte Entscheidungen während des Entwurfs- und Konstruktionsprozesses zu treffen und die Sicherheit sowie die Effizienz von Strukturen zu verbessern.

Darüber hinaus wird in der Arbeit auch auf die mathematischen Grundlagen der Balkengleichung eingegangen, um ein tieferes Verständnis für ihre Herleitung und Anwendung zu vermitteln.
Dies umfasst auch die Diskussion von Differentialgleichungen und anderen mathematischen Konzepten, die für die Analyse von Balken und anderen Tragwerken relevant sind.

Insgesamt strebt die Arbeit danach, einen umfassenden Einblick in die Bedeutung und Anwendung der Balkengleichung zu geben. 
Durch die Untersuchung verschiedener Aspekte und die praktische Anwendung soll sie Ingenieuren und anderen Fachleuten auf diesem Gebiet wertvolle Einblicke und Erkenntnisse liefern, die zur Verbesserung der Strukturleistung und -sicherheit beitragen können.