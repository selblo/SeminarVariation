%
% teil3.tex -- Beispiel-File für Teil 3
%
% (c) 2020 Prof Dr Andreas Müller, Hochschule Rapperswil
%
% !TEX root = ../../buch.tex
% !TEX encoding = UTF-8
%
% TODO stimmt das?
\usepackage{tabularx} % Paket für automatische Tabellenbreite
\section{Variationsprinzip Mathematik und Algorithmus
\label{buch:paper:varalg:section:variations_math_algorithm_result}}
\rhead{Variationsprinzip Mathematik und Algorithmus}
Im vorherigen Kapitel \ref{buch:paper:varalg:section:process} wurde erklärt, 
wie der Algorithmus funktioniert, der das Variationsprinzip anwendet. Dies 
passt jedoch nicht genau zu dem Variationsprinzip, das in der Mathematik 
verwendet wird. In diesem Kapitel wird genauer darauf eingegangen.

Woran unterscheiden sich die beiden Prinzipien?

\begin{table}[h]
   \centering
   \begin{tabularx}{\textwidth}{|X|X|X|}
      \hline
       & Mathematik 
       & Algorithmus 
       \\ \hline
      Ziele  
       & Wie im Kapitel Funktionale \ref{buch:variation:problem:subsection:funktionale}

      beschrieben, ist das Ziel der Variationsrechnung die Optimierung von Funktionalen 
      durch die Findung einer optimalen Funktion.
       & Im Algorithmus bedeutet Variation, dass es eine Menge möglicher Lösungen gibt, 
      aus denen die besten Lösungen ausgewählt und weiterverarbeitet werden, in der 
      Hoffnung, dass die neuen Lösungen besser sind.
      \\ \hline
      Techniken  
       & Hier werden analytische Techniken wie die Euler-Lagrange-Gleichung verwendet, 
      um Optimierungsprobleme zu lösen. (Verschiedene Techniken sind in den Kapiteln 
      2-10 beschrieben)
       & Im Algorithmus werden Mechanismen verwendet, die stochastische 
      \footnote{
         Im Fall des Algorithmus sind stochastische Methoden gemeint, bei denen 
         eine Anzahl an Zufallsereignissen oder -kombinationen erstellt und 
         diese anschließend ausgewertet oder weiterverarbeitet werden.
      }
      Methoden wie Kreuzung und Mutation beinhalten, um Vielfalt zu erzeugen und aufrechtzuerhalten.
      \\ \hline
      Nachbarlösungen
       & Sind nicht direkt ein primäres Konzept und eher implizit \footnote{

         Untersuchung von Nachbarlösungen ist implizit, da die Variationen 
         von Funktionen und deren Einfluss auf das Funktional nicht direkt 
         als Nachbarlösung bezeichnet werden kann.
      }. Man untersucht die kleinen Variationen einer Funktion und wie sich 
      die Funktionale verändern, aber am Ende ist das Ziel die Optimierung 
      der Funktionale.
       & Die Nachbarlösungen sind ein primäres Konzept, da die Variationen
      von Lösungen direkt untersucht, gekreuzt und mutiert werden, um die 
      besten Lösungen zu finden.
      \\ \hline
      Lösung
       & Die Lösung ist eine Funktion, mit der das Optimum errechnet werden kann.
       & Die Lösung am Ende könnte das Optimum sein oder nur sehr nah dran.
      \\ \hline
   \end{tabularx}
   \caption{Vergleich auf den begriff Variationen}
   \label{tab:example_bruteforce_cities}
\end{table}

Die Aussage, dass das Variationsprinzip im Algorithmus das gleiche ist, 
stimmt nicht. Beide suchen zwar das Optimum, aber in der Mathematik findet 
man eine Funktion, mit der das Optimum berechnet werden kann, wie z.B. 
die Durchbiegung einer Kette. Im Algorithmus wird versucht, das Optimum 
durch Zufall und Variationen (neue Kombinationen) zu finden. Am Ende 
hat man eine Lösung, die das Optimum sein könnte, sehr nahe dran ist 
oder komplett daneben liegt.
