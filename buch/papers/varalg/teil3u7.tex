%
% teil3.tex -- Beispiel-File für Teil 3
%
% (c) 2020 Prof Dr Andreas Müller, Hochschule Rapperswil
%
% !TEX root = ../../buch.tex
% !TEX encoding = UTF-8
%
\subsection{Abbruchkriterium
\label{buch:paper:varalg:subsection:termination}}
\index{Abbruchkriterium}%
Wie auch im maschinellen Lernen ist es wichtig zu bestimmen, unter 
welchen Bedingungen der Vorgang beendet werden soll. Bei einem 
genetischen Algorithmus wird in der Regel eine maximale Anzahl an 
Generationen definiert. Eine weitere Möglichkeit besteht darin, 
die Berechnungen zu beenden, sobald ein gewisser Fitnesswert erreicht 
wird und die Lösung als ausreichend angesehen werden kann.

\subsubsection{Abbruchkriterium auf das TSP angepasst
\label{buch:paper:varalg:subsection:termination_tsp}}
Für das Travelling-Salesman-Problem sind beide Abbruchkriterien im 
Abschnitt \ref{buch:paper:varalg:subsection:termination} anwendbar.
