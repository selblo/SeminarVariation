%
% teil3.tex -- Beispiel-File für Teil 3
%
% (c) 2020 Prof Dr Andreas Müller, Hochschule Rapperswil
%
% !TEX root = ../../buch.tex
% !TEX encoding = UTF-8
%
\section{Variationsprinzip beim genetische Algorithmus
\label{buch:paper:varalg:section:genetic_algorithm_process}}
\rhead{Variationsprinzip beim genetische Algorithmus}
Der Genetische Algorithmus nutzt nach ChatGPT das Variationsprinzip, 
das wie folgt definiert wird:
\begin{quote}
Das Variationsprinzip ist ein grundlegendes Konzept in der 
evolutionären Computertechnik, insbesondere in genetischen 
Algorithmen. Es besagt, dass genetische Vielfalt in einer 
Population von Individuen aufrechterhalten werden muss, 
um eine effektive Suche im Suchraum zu ermöglichen und eine 
Lösung für das zugrunde liegende Problem zu finden.
\\
Der genetische Algorithmus nutzt dieses Variationsprinzip, um eine 
Population von möglichen Lösungen zu einem Problem zu entwickeln 
und zu verfeinern\cite{chatgpt2024}.
\end{quote}
In diesem Kapitel werden die einzelnen Schritte des genetischen Algorithmus 
erläutert. Da der Algorithmus nicht 1 zu 1 auf das Travelling Salesman
Problem angewendet werden kann, wird in den Unterkapiteln zuerst beschrieben,
was in jedem Schritt passiert und anschließend, wie diese auf das Travelling 
Salesman Problem angepasst werden. Der Ablauf besteht aus den folgenden Schritten:
%TODO beschreibung mit Lösung und Permutation beschreiben
\begin{enumerate}
    \item Initialisierung: Erstellung der Anfangspopulation
    \item Selektion: Auswahl der Individuen für die Weiterentwicklung
    \item Kreuzung: Erzeugung neuer Individuen
    \item Mutation: Veränderung der Individuen
    \item Evaluation: Bewertung der Individuen
    \item Abbruchkriterium: Festlegung, wann der Algorithmus beendet wird
\end{enumerate}

%
% teil3.tex -- Beispiel-File für Teil 3
%
% (c) 2020 Prof Dr Andreas Müller, Hochschule Rapperswil
%
% !TEX root = ../../buch.tex
% !TEX encoding = UTF-8
%
\subsection{Initialisierung
\label{buch:paper:varalg:subsection:initialization}}
\rhead{Initialisierung}
Der Startpunkt des genetischen Algorithmus ist die Initialisierung.
Dabei wird eine zufällige Population von möglichen Lösungen erstellt.
Diese wird als ein genetischer String dargestellt.

\begin{figure}
	\centering
	\includegraphics[width=0.8\textwidth]{
        papers/varalg/images/teil3/01GeneticString.png
        }
	\caption{Beispiel eines möglichen genetischen Strings}
	\label{fig:possible_genetic_string}
\end{figure}

In jeder Position kann ein Gen aktiviert (1) oder deaktiviert (0) sein.
Diese Darstellung eignet sich jedoch nicht für das Travelling Salesman 
Problem (TSP), da eine Stadt nicht einfach ein- oder ausgeschaltet werden kann.
Auf die Wege ist es auch nicht anwendbar, das dann immer nur 1 Weg der vielen Wege 
aktiviert werden kann. Im TSP ist die Reihenfolge der Städte entscheidend. 
Daher wird die jeweilige Nummer der Stadt verwendet, wie im folgenden Bild 
\cite{cities_genetic_string} dargestellt:

\begin{figure}
	\centering
	\includegraphics[width=0.8\textwidth]{
        papers/varalg/images/teil3/02GeneticStringCities.png
        }
	\caption{Beispiel von Städten in einem genetischen String dargestellt}
	\label{fig:cities_genetic_string}
\end{figure}

Anstatt alle möglichen Lösungen zu erstellen, wird nur ein kleiner Teil 
(Population) zufällig generiert und weiter bearbeitet. Diese Taktik 
spart Zeit und Ressourcen.

Die zufällige Erzeugung der Anfangspopulation stellt ebenfalls eine Form 
der Variation dar. Sie sorgt dafür, dass die Suche nicht von einem begrenzten 
Bereich des Lösungsraums startet, sondern eine breite Palette von möglichen 
Lösungen berücksichtigt. Dabei werden jedoch keine Berechnungen durchgeführt, 
sondern zufällige Lösungen erstellt, in der Hoffnung, dass eine davon nahe 
an das Optimum herankommt.


%
% teil3.tex -- Beispiel-File für Teil 3
%
% (c) 2020 Prof Dr Andreas Müller, Hochschule Rapperswil
%
% !TEX root = ../../buch.tex
% !TEX encoding = UTF-8
%
\subsection{Evaluation
\label{buch:paper:varalg:subsection:evaluation}}
\index{Evaluation}%
Dieser Schritt befasst sich mit der Auswertung der einzelnen 
Permutationen. Die Berechnung wird unter anderem auch als Fitnessfunktion benannt.
Was die Fitnessfunktion zurückgibt, ist abhängig vom Ziel der Lösung.

\subsubsection{Evaluation auf das TSP angepasst
\label{buch:paper:varalg:subsection:evaluation_tsp}}
Beim Travelling-Salesman-Problem kommt es auf die Streckenlänge an, 
daher wird zur Streckenberechnung die gleiche Formel \eqref{eq:bruteforce_min_formula},
wie in der Bruteforce-Methode, angewendet.
\begin{figure}
	\centering
	\includegraphics[width=0.8\textwidth]{
        papers/varalg/images/teil3/03GeneticStringCitiesResults.png
        }
	\caption{Beispiel eines genetischen Strings mit Ergebnissen}
	\label{fig:cities_genetic_string_results}
\end{figure}

%
% teil3.tex -- Beispiel-File für Teil 3
%
% (c) 2020 Prof Dr Andreas Müller, Hochschule Rapperswil
%
% !TEX root = ../../buch.tex
% !TEX encoding = UTF-8
%
\subsection{Selektion
    \label{buch:paper:varalg:subsection:selection}}
    \rhead{Selektion}
In diesem Schritt werden Elternpaare ausgesucht, die später 
neue Nachkommen erzeugen. Die Selektion erfolgt so, dass in der 
Regel nur die Fittesten neue Kinder erzeugen.

Die Idee dahinter ist, dass es verschiedene Individuen gibt, 
die unterschiedlich fit sind. Die fitteren Individuen paaren 
sich eher, während die schwächeren seltener Nachwuchs erzeugen. 
So ähnlich wie in der Natur. Die Paare werden im System zufällig 
gewählt, aber die Wahrscheinlichkeit, dass die fitteren 
Individuen Nachkommen zeugen, ist höher.

\begin{figure}
    \centering
    \includegraphics[width=0.8\textwidth]{
        papers/varalg/images/teil3/04OffspringProbability.png
    }
    \caption{Eltern die Ausgewählt werden für Nachkommen}
    \label{fig:selection_of_parents}
\end{figure}

Auf dem Bild \ref{fig:selection_of_parents} sind eltern mit unterschiedlichem
Weg länge. Dabei ist die Wahrscheinlichkeiten das sich die Blaue Linie ereignet 
viel grösser als die Rote Linie. Diese ist aber trotzdem möglich.
\\
Für die Selektion gibt es verschiedene Möglichkeiten.
\\
1. **Roulette-Rad-Selektion:** 
- Individuen werden zufällig und proportional zu ihrer 
Fitness ausgewählt. Die Wahrscheinlichkeit wird anhand ihrer 
Fitness definiert. Kürzere Strecken haben eine höhere Chance, 
ausgewählt zu werden. Die Wahrscheinlichkeit wird mit einer 
entsprechenden Formel berechnet.

\begin{equation}
    \label{eq:probability_fittest}
    P_i = \frac{f_i}{\sum_{j=1}^{N} f_j}
\end{equation}

2. **Rangselektion:**
- Individuen werden nach ihrer Fitness sortiert und basierend 
auf ihrem Rang ausgewählt. Die Wahrscheinlichkeit wird anhand 
des Rangs definiert.

\begin{equation}
    \label{eq:probability_rating}
    P_i = \frac{r_i}{\sum_{j=1}^{N} r_j}
\end{equation}

3. **Turnierselektion:**
- Eine Gruppe von Individuen wird zufällig ausgewählt, und 
das fitteste Individuum dieser Gruppe wird als Elternteil gewählt.

Es gibt auch die Möglichkeit, ein eigenes Selektionssystem zu entwickeln, 
das ein Ausscheidungsverfahren beinhaltet, aus dem schließlich ein 
Elternpaar hervorgeht. Das System folgt einem logischen Ablauf, wobei 
die Wahrscheinlichkeit mathematisch berechnet wird.

%
% teil3.tex -- Beispiel-File für Teil 3
%
% (c) 2020 Prof Dr Andreas Müller, Hochschule Rapperswil
%
% !TEX root = ../../buch.tex
% !TEX encoding = UTF-8
%
\subsection{Kreuzung
\label{buch:paper:varalg:subsection:crossover}}
\rhead{Kreuzung}
In diesem Teil des Algorithmus werden die gewählten Elternpaare 
neu kombiniert, um Nachkommen zu erzeugen. Bei der Kreuzung 
werden Teile des genetischen String ausgetauscht.

\begin{figure}
	\centering
	\includegraphics[width=0.8\textwidth]{
		papers/varalg/images/teil3/05GeneticStringCross.png
	}
	\caption{Ein einfaches Beispiel für eine Einpunkt-Kreuzung}
	\label{fig:one_point_crossover}
\end{figure}

Für die Kreuzung gibt es unterschiedliche Taktiken:

- **Einpunkt-Kreuzung:** Ein zufälliger Punkt wird auf den 
Elternchromosomen ausgewählt. Die Gene vor diesem Punkt 
stammen vom ersten Elternteil, die Gene nach diesem Punkt 
vom zweiten Elternteil. Mathematische:\\
Wähle einen zufälligen Punkt \( k \) \((1 \leq k < n)\).
Erzeuge die Nachkommen durch:
\begin{align*}
	O_1 = (P_1[1], P_1[2], \ldots, P_1[k], P_2[k+1], \ldots, P_2[n])
	O_2 = (P_2[1], P_2[2], \ldots, P_2[k], P_1[k+1], \ldots, P_1[n])
\end{align*}
- **Zweipunkt-Kreuzung:** Zwei Punkte werden ausgewählt und 
der Genabschnitt zwischen diesen Punkten wird zwischen 
den Eltern getauscht.\\
Wähle zwei zufällige Punkte \( k_1 \) und \( k_2 \) \((1 \leq k_1 < k_2 < n)\).
Erzeuge die Nachkommen durch:

\begin{align*}
	O_1 = (P_1[1], \ldots, P_1[k_1], P_2[k_1+1], \ldots, P_2[k_2], P_1[k_2+1], \ldots, P_1[n])
	O_2 = (P_2[1], \ldots, P_2[k_1], P_1[k_1+1], \ldots, P_1[k_2], P_2[k_2+1], \ldots, P_2[n])
\end{align*}


- Uniforme Kreuzung: Jedes Gen wird mit einer bestimmten 
Wahrscheinlichkeit vom ersten oder zweiten Elternteil 
übernommen, was zu einer zufälligeren Kombination führt.
Mathematische Formel:\\
Für jedes Gen \( i \) \((1 \leq i \leq n)\):
Wähle eine zufällige Zahl \( r_i \) im Intervall [0, 1].\\
Wenn \( r_i \) kleiner als die vordefinierte Wahrscheinlichkeit 
\( p \) ist, dann wird das Gen von \( P_1 \) übernommen, 
ansonsten von \( P_2 \).
\begin{align*}
O_1[i] =
	\begin{cases} 
		P_1[i] & \text{wenn } r_i < p      
		P_2[i] & \text{wenn } r_i \geq p 
	\end{cases}


O_2[i] =
	\begin{cases} 
		P_2[i] & \text{wenn } r_i < p      
		P_1[i] & \text{wenn } r_i \geq p 
	\end{cases}

\end{align*}
Hier wird die Variation aufrechterhalten (Grösse der Population
bleibt gleich gross) und gleichzeitig wird aus den besten Genen 
neue erstellt, mit dem Versuch noch besseres Resultat zu bekommen.

Wie schon in der Initialisierung erwähnt, kann nicht ein normaler 
genetischer String \ref{fig:one_point_crossover} mit An und 
Aus verwendet werden. Dies würde zu einem Resultat wie in der Abbildung 
\ref{fig:one_point_crossover_cities} führen.

\begin{figure}
	\centering
	\includegraphics[width=0.8\textwidth]{
		papers/varalg/images/teil3/07GeneticStringCitiesCrossoverStandard.png
	}
	\caption{Beispiel einer Einpunkt-Kreuzung mit Städten}
	\label{fig:one_point_crossover_cities}
\end{figure}

Für das Travelings Salesman Problem wird die Kreuzung mit andern Systemen 
angepasst, damit es keine Doppelten gibt.

Für den Script wurden die Logik angepasst. Die Algorithmen entfernen einen 
Teil des Strings, wie bei in den vorherigen Kreuzungen, aber es fügt nach 
der Reihenfolge des zweiten Elternteils alle nicht vorhandenen 
Städte hinzu, bis der herausgeschnittene String wieder vollständig ist.

\begin{figure}
	\centering
	\includegraphics[width=0.8\textwidth]{
		papers/varalg/images/teil3/08GeneticStringCitiesCrossoverSimple.png
	}
	\caption{Beispiel einer Einpunkt-Kreuzung mit Städten}
	\label{fig:crossover_order_cities}
\end{figure}


%
% teil3.tex -- Beispiel-File für Teil 3
%
% (c) 2020 Prof Dr Andreas Müller, Hochschule Rapperswil
%
% !TEX root = ../../buch.tex
% !TEX encoding = UTF-8
%
\subsection{Mutation
\label{buch:paper:varalg:subsection:mutation}}
\index{Mutation}%
Dieser Schritt sorgt dafür, dass zufällige Änderungen in den Genen 
stattfinden. Dadurch entstehen neue Variationen, welche in der 
Kreuzung nicht entstanden wären und der gesamte Lösungsraum wird 
vergrössert. Zusätzlich soll dieser Schritt verhindern, dass nach 
einer Anzahl von Generationen immer wieder die gleichen 
Genmuster entstehen und der Algorithmus in einem lokalen Extrempunkt 
stecken bleibt.
Die Mutation tritt nicht bei jedem Nachkommen auf, sondern wird, 
ähnlich wie in der Natur, zufällig ausgelöst. Dazu wird eine Bedingung 
der Form \( \text{Zufällige Zahl} < \text{Grenzzahl} \) definiert. 
Je kleiner die Grenzzahl, desto unwahrscheinlicher ist eine Mutation. 
Die Grenzzahl wird im Voraus festgelegt und liegt im Intervall \([0,1]\).
Nun kann über den String iteriert werden, wobei an jeder Position eine 
zufällige Zahl generiert wird, die ebenfalls im Intervall \([0,1]\) 
liegt. Wird die Bedingung erfüllt, wird die entsprechende Stelle, 
wie in Abbildung \ref{fig:mutation_genetic_string} dargestellt,
invertiert.
\begin{figure}
	\centering
	\includegraphics[width=0.8\textwidth]{
        papers/varalg/images/teil3/09GeneticStringMutation.png
        }
	\caption{
	Beispiel einer Mutation mit einem genetischen String aus 0 und 1. Die
	Mutation wird zufällig ausgelöst und invertiert die Stelle.
	}
	\label{fig:mutation_genetic_string}
\end{figure}

\subsubsection{Mutation auf das TSP angepasst
\label{buch:paper:varalg:subsection:mutation_tsp}}
Für das Travelling-Salesman-Problem wird die Mutation so angepasst,
dass, wenn eine Mutation stattfindet, zwei zufällige Stellen ausgewählt
und diese Städte vertauscht werden, 
wie in der Abbildung \ref{fig:mutation_genetic_string_cities}.
\begin{figure}
	\centering
	\includegraphics[width=0.8\textwidth]{
        papers/varalg/images/teil3/09GeneticStringCitiesMutation.png
        }
	\caption{Beispiel einer Mutation mit Städten}
	\label{fig:mutation_genetic_string_cities}
\end{figure}

%
% teil3.tex -- Beispiel-File für Teil 3
%
% (c) 2020 Prof Dr Andreas Müller, Hochschule Rapperswil
%
% !TEX root = ../../buch.tex
% !TEX encoding = UTF-8
%
\subsection{Ersetzen
\label{varalgbuch:paper:varalg:subsection:replacement}}
\rhead{Ersetzen}
Der Ersetzungsschritt macht was er aussagt. Meistens wird die ganze 
Population durch die neue Ersetzt, so dass eine neue Generation startet.
Es gibt aber auch die Möglichkeit, ein den besten Teil der alten Population 
zu erhalten. Dies wird gemacht in der Hoffnung, dass aus den besten Individuen
der alten Generation und die besten Individuen der neuen Generation viel 
bessere Lösungen entstehen. Wichtig ist, dass die Gesamtgrösse der Population
gleicht bleibt.

\subsection{Ersetzen auf das TSP angepasst
\label{buch:paper:varalg:subsection:replacement_tsp}}
\rhead{Ersetzen TSP}
Dieser Schritt kann auf das TSP angewendet werden, ohne weitere
Anpassungen vorzunehmen.

%
% teil3.tex -- Beispiel-File für Teil 3
%
% (c) 2020 Prof Dr Andreas Müller, Hochschule Rapperswil
%
% !TEX root = ../../buch.tex
% !TEX encoding = UTF-8
%
\subsection{Abbruchkriteriums
\label{buch:paper:varalg:subsection:termination}}
\rhead{Abbruchkriteriums}
Wie auch im maschinellen Lernen ist es wichtig zu bestimmen, unter 
welchen Bedingungen die Berechnungen beendet werden sollen. Beim 
genetischen Algorithmus wird in der Regel eine maximale Anzahl an 
Generationen definiert. Eine weitere Möglichkeit besteht darin, 
die Berechnungen zu beenden, sobald ein gewisser Fitnesswert erreicht 
wird und die Lösung als ausreichend angesehen werden kann.

\subsection{Abbruchkriteriums auf das TSP angepasst
\label{buch:paper:varalg:subsection:termination_tsp}}
\rhead{Abbruchkriteriums TSP}
Für das Travelling Salesman Problem sind beide Abbruchkriterien im 
Kapitel \ref{buch:paper:varalg:subsection:termination} anwendbar.


