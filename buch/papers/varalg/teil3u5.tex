%
% teil3.tex -- Beispiel-File für Teil 3
%
% (c) 2020 Prof Dr Andreas Müller, Hochschule Rapperswil
%
% !TEX root = ../../buch.tex
% !TEX encoding = UTF-8
%
\subsection{Mutation
\label{buch:paper:varalg:subsection:mutation}}
\rhead{Mutation}
Dieser Schritt sorgt dafür, dass zufällige Änderungen in 
den Genen stattfinden. Dadurch entstehen neue Gene, was den 
gesamten Lösungsraum vergrößert. Zusätzlich soll es verhindern,
dass nach einer Anzahl von Generationen immer wieder die
gleichen Genmuster entstehen. Mit dieser Methode wird
verhindert, dass der Algorithmus in einem lokalen Extrempunkt
stecken bleibt. Die Mutation findet auch nicht bei jedem Kind
statt, sondern wie in der Natur werden diese per Zufall
ausgelöst werden. Die Mutation läuft so ab, dass über den String
iteriert wird und jedesmal ein eine Randomzahl ensteht. Wird 
die Bedingung erfüllt, dann wird die Stelle invertiert, wie in der
Abbildung\ref{mutation_genetic_string}. Dieser Schritt lässt Variationen 
entstehen, die bei der Kreuzung nicht entstanden wären.  
\begin{figure}
	\centering
	\includegraphics[width=0.8\textwidth]{
        papers/varalg/images/teil3/09GeneticStringMutation.png
        }
	\caption{Beispiel einer Mutation mit einem genetischen String aus 0 und 1}
	\label{fig:mutation_genetic_string}
\end{figure}

\subsection{Mutation auf das TSP angepasst
\label{buch:paper:varalg:subsection:mutation_tsp}}
\rhead{Mutation TSP}
Für das Traveling Salesman Problem wird die Mutation so angepasst,
dass wenn eine Mutation stattfindet, dass zwei zufällige Stellen 
genommen werden und diese Städte dann mit einander getauscht werden, 
wie in der Abbildung \ref{fig:mutation_genetic_string_cities}.
\begin{figure}
	\centering
	\includegraphics[width=0.8\textwidth]{
        papers/varalg/images/teil3/09GeneticStringCitiesMutation.png
        }
	\caption{Beispiel einer Mutation mit Städten}
	\label{fig:mutation_genetic_string_cities}
\end{figure}
