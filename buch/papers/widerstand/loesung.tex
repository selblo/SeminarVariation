%
% loesung.tex -- Lösung des Variationsproblems
%
% (c) 2020 Prof Dr Andreas Müller, Hochschule Rapperswil
%
% !TEX root = ../../buch.tex
% !TEX encoding = UTF-8
%
\section{Lösung des Variationsproblems
\label{widerstand:section:loesung}}
\kopfrechts{Lösung}
Das Minimalproblem~\ref{widerstand:aufgabe} ist ein Variationsproblem
mit einem Randterm.
In diesem Abschnitt lösen wird das Problem mit den im ersten Teil
entwickelten Methoden.

%
% Die Euler-Lagrange-Differentialgleichung
%
\subsection{Die Euler-Lagrange-Differentialgleichung}
Die Lagrange-Funktion des Funktionals~\eqref{widerstand:eqn:Wy} ist 
\[
F(x,y')
=
\frac{2x}{1+y^{\prime 2}}
\]
mit den partiellen Ableitungen
\begin{align*}
\frac{\partial F}{\partial y}&=0
\\
\frac{\partial F}{\partial y'}
&=
\frac{2xy'}{(1+y^{\prime 2})^2}.
\end{align*}
Daraus ergibt sich die Euler-Lagrange-Differentialgleichung
\begin{align*}
0
=
\frac{d}{dx} \frac{\partial F}{\partial y'}
&=
\frac{d}{dx}
\frac{2xy'(x)}{(1+y'(x)^2)^2}.
\end{align*}
Statt diesen Ausdruck explizit abzuleiten schliessen wir, dass die rechte
Seite konstant sein muss, es gibt also eine Konstante $c$ derart, dass
\[
\frac{ 2xy'(x) }{ (1+y'(x)^2)^2 } = c.
\]


