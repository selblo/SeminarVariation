%
% koeper.tex -- Rotationskörper nach Newton
%
% (c) 2021 Prof Dr Andreas Müller, OST Ostschweizer Fachhochschule
%
\documentclass[tikz]{standalone}
\usepackage{amsmath}
\usepackage{times}
\usepackage{txfonts}
\usepackage{pgfplots}
\usepackage{csvsimple}
\usetikzlibrary{arrows,intersections,math,calc}
\definecolor{darkred}{rgb}{0.8,0,0}
\definecolor{darkgreen}{rgb}{0,0.6,0}
\begin{document}
\def\skala{1}
\begin{tikzpicture}[>=latex,thick,scale=\skala,
declare function={
	y(\t) = 5*(1-(\t/4)*(\t/4));
	yprime(\t) = -(5/8)*\t;
	nlength(\t) = sqrt(1 + yprime(\t)*yprime(\t));
	xn(\t) = -yprime(\t) / nlength(\t);
	yn(\t) = 1 / nlength(\t);
	rx(\t) = -2 * xn(\t) * yn(\t);
	ry(\t) = 1 - 2 * yn(\t) * yn(\t);
}]

\draw (-4,-0.05) -- (-4,0.05);
\draw (4,-0.05) -- (4,0.05);
\node at (4,-0.05) [below] {$R\mathstrut$};
\draw[line width=0.3pt] (1,0) -- (1,5);
\draw (1,-0.05) -- (1,0.05);
\node at (1,-0.05) [below] {$r\mathstrut$};
\node at (0,{y(1)}) [above left] {$y=L$};

\draw[color=darkred,line width=1.2pt]
	plot[domain=-4:-1] (\x,{y(\x)})
	--
	plot[domain=1:4] (\x,{y(\x)});

\def\x{2}
\draw[->,color=darkgreen,line width=1.4pt] (\x,{y(\x)}) -- +({xn(\x)},{yn(\x)});
\node[color=darkgreen] at ({\x+xn(\x)},{y(\x)+yn(\x)}) [above right] {$\vec{n}$};

\draw[color=blue,line width=0.3pt] (\x,0) -- (\x,5.5);
\draw[color=blue,line width=1pt]
	(\x,5.5)
	--
	(\x,{y(\x)})
	--
	+({-2*rx(\x)},{-2*ry(\x)})
	;

\draw[->,color=blue,line width=1.4pt] (\x,5.5) -- (\x,4.5);
\node[color=blue] at (\x,5) [left] {$\vec{e}$};
\fill[color=darkgreen] (\x,{y(\x)}) circle[radius=0.08];

\def\x{-2.5}
\draw[line width=0.3pt,color=darkred] (\x,0) -- (\x,{y(\x)});
\fill[color=darkred] (\x,{y(\x)}) circle[radius=0.08];
\node[color=darkred] at (\x,{y(\x)}) [above left] {$y(x)$};

\draw[->] (-4.1,0) -- (4.5,0) coordinate[label={$x$}];
\draw[->] (0,-0.1) -- (0,5.9) coordinate[label={right:$y$}];

\end{tikzpicture}
\end{document}

