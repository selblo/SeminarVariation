%
% einleitung.tex -- Beispiel-File für die Einleitung
%
% (c) 2020 Prof Dr Andreas Müller, Hochschule Rapperswil
%
% !TEX root = ../../buch.tex
% !TEX encoding = UTF-8
%
%Elektrodynmaik
\section{Elektrodynamik\label{section:maxwell:elektrodynmaik}}
\rhead{Elektrodynamik}
In der Elektrodynamik erlauben wir es, dass
\[
\frac{\partial f}{\partial t}
\neq
0
\]
für die Funktionen, die wir betrachten, gelten darf.
Das heisst Ladungen dürfen sich nun mit einer beliebigen Geschwindigkeit $\vec{v}$ bewegen und wir berücksichtigen zeitabhängige Skalar- und Vektorfelder.
Bei den statischen Maxwellgleichungen fällt auf, dass das elektrische und magnetische Feld keinerlei Abhängigkeiten voneinander haben.
Wie sich später herausstellen wird, ändert sich dies, denn die Zeitabhängigkeit führt dazu, dass sich das elektrische und magnetische Feld gegenseitig beeinflussen.
Daraus folgen einige spannende Tatsachen, auf die wir später noch zu Sprechen kommen.
 
Im folgenden werden wir bekannte Vektorfelder in ihrer Definition anpassen.

\subsubsection{Elektrisches Feld dynamisch}
Das elektrische Feld
\(
\vec{E}: \mathbb{R}^4 \rightarrow \mathbb{R}^3
\)
wird in der Elektrodynamik definiert als
\begin{equation}
	\vec{E}(t,x,y,z)
	=
	- \nabla\phi(t,x,y,z) - \frac{\partial \vec{A}}{\partial t}(t,x,y,z).
	\label{maxwell:section:definiton_dynamisch_elektrischesFeld}
\end{equation}
Es fällt auf, dass das elektrische Feld, das elektrische Potentialfeld und das magnetische Vektorpotential nun einen vierdimensionalen Inputvektor besitzen, wobei die zusätzliche Dimension die Zeit ist.

\subsubsection{Magnetisches Feld dynamisch}
Das magnetische Feld
\(
\vec{B}: \mathbb{R}^4 \rightarrow \mathbb{R}^3
\)
wird in der Elektrodynamik ähnlich definiert wie in der Magnetostatik. Nämlich ist
\begin{equation}
	\vec{B}(t,x,y,z)
	=
	\nabla \times \vec{A}(t,x,y,z).
\end{equation}
Der Unterschied liegt hier einzig in der zusätzlichen Zeitkomponente im Inputvektor.

\subsection{Dynamische Maxwell-Gleichungen}
Das Ziel dieses Abschnittes ist ein Variationsprinzip zu formulieren, welches uns die Verhaltensgleichungen der Elektrodynamik liefert. 
Erneut betrachten wir den selben dreidimensionalen, luftleeren Raum $V \subset \mathbb{R}^3$. In diesem Raum existieren nun das elektrische Feld $\vec{E}(t,x,y,z)$, das magnetische Feld $\vec{B}(t,x,y,z)$, eine Stromdichte $\vec{\jmath}(t,x,y,z)$ und eine statische Ladungsdichte $\varrho(x,y,z)$.

\subsubsection{Ansatz}
Wir wählen wieder den Ansatz die Energie im Raum zu minimieren. Für die Formulierung des Energieintegrals superponieren wir die bisher gefundenen Energien und erhalten
\begin{align*}
	W_{\text{tot}}
	&=
	W_{\text{e}} - W_{\text{q}} + W_{\text{p}} - W_{\text{m}}
				\\
				&= \iiint_V \frac{1}{2}\,\varepsilon_0\,\vec{E}\,^2 - \varrho \, \phi
				+ \vec{A}\cdot\vec{j} - \frac{1}{2\mu_0}\vec{B}^2 \, dV.
\end{align*}
Hiermit ist es uns noch nicht gelungen ein Energiefunktional zu formulieren, das von der Form
\[
I(f) = \int_{t_0}^{t_1} \int_{x_0}^{x_1} \int_{y_0}^{y_1} \int_{z_0}^{z_1} L(t,x,y,z,f,f_t,f_x,f_y,f_z)\,dt\,dx\,dy\,dz 
\]
ist.



