%erste Maxwell Gleichung mit Quelle
\section{Elektrostatik mit Quelle
\label{maxwell:section:elektrostatik_mit_quelle}}
\rhead{Problemstellung}
Nun betrachten wir einen luftleeren, dreidimensionalen Raum
\[
V\in\mathbb{R}^3
\]
, in dem ein elektrisches Potentialfeld $\phi(x,y,z)$ und eine Ladungsdichte $\rho(x,y,z)$ existieren.
Auch für diesen Raum möchten wir mittels Variationsrechnung eine Gleichung finden, die das Verhalten des elektrischen Potentialfeldes beschreibt.

\subsection{Ansatz}
\rhead{Ansatz}
Es ist naheliegend, dass auch in diesem Szenario die Energie im System minimiert werden muss.
Wie auch in Gleichung \eqref{maxwell:section:energieintegral_quellenfrei} ist die Energie im elektrischen Feld
\[
W_e
=
\iiint_V \frac{1}{2}\epsilon_0\vec{E}(x,y,z)^2\, dV.
\]
% TODO: referenz einfügen von definition elektrisches potential
Jedoch ist dies nicht die einzige Komponente der gesamt Energie des Systems.
Laut \ref{???} ist das elektrische Potential eine auf Ladung normierte potenzielle Energie.
Somit haben wir die fehlende Komponente gefunden.
Damit wir die, durch die Ladung
\[
q
=
\iiint_V \rho(x,y,z)\, dV
\]
verursachte, potenzielle Energie $W_q$ im System mit der Lasungsdichte $\rho(x,y,z)$ Ausdrücken können, müssen wir untersuchen, was eine infinitesimale potenzielle Energie verursacht.
Wen wir diese infinitesimal kleine potenzielle Energie
\[
dW_q
=
\phi(x,y,z)dq
\]
unter die Lupe nehmen und das Differential von $dq$ einsetzen, erhalten wir
\begin{align*}
dW_q
&=
\phi(x,y,z)\underbrace{\frac{dq}{dV}}_{\rho(x,y,z)}dV.
\\
dW_q
&=
\phi(x,y,z)\rho(x,y,z)dV.
\end{align*}
Jetzt müssen die infinitesimalen potenziellen Energien im Raum $V$ zusammengezählt werden und was resultiert ist
\begin{equation}
W_q
=
\iiint_V \phi(x,y,z)\rho(x,y,z)\, dV.
\label{maxwell:section:potenzielle_energie_ladung}
\end{equation}

