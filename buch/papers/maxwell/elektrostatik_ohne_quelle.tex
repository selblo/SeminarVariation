% erste Maxwellgleichung ohne Quelle
\section{Elektrostatik ohne Quelle
	\label{maxwell:section:elektrostatik_ohne_quelle}}
\rhead{Problemstellung}
Man stelle sich nun einen ladungsfreien, luftleeren , drei dimensionalen Raum
\[
V\in\mathbb{R}^3
\]
vor, indem ein elektrisches Potentialfeld $\phi(x,y,z)$ existiert.
Für diesen Raum möchten wir mittels der Varitaionsrechnung eine Gleichung entwickeln, die beschreibt, wie sich das Potentialfeld verhählt. 

\subsection{Ansatz}
Damit wir eine Gleichungen erhalten, die das Verhalten des elektrischen Potentialfeldes beschreibt, muss ganz allgemein die Energie im System minimiert werden. 
In diesem Fall ist die Energie im System
\[
W_e
=
\int_V \omega_e\, dV.
\]
Dieses Integral gilt es zu minimieren, was die Grundlage für unser Variationsproblem darstellt.
Die Energiedichte
% TODO: Bei definitionen erwähnen, dass ein vektor v^2 = v dot v ist!
\[
\omega_e\
=
\frac{1}{2}\epsilon_0\vec{E}(x,y,z)^2
\]
ist die Energiedichte im elektrischen Feld.
Diese Gleichung können wir nun mittels definition \ref{(???)} mit dem elektrischen Potentialfeld ausdrücken.
somit ist
\begin{align}
\omega_e
&=
\frac{1}{2}\epsilon_0\left(-\nabla\phi(x,y,z)\right)^2
\\
\omega_e
&=
\frac{1}{2}\epsilon_0
\begin{pmatrix}
-\phi_x(x,y,z)\\
-\phi_y(x,y,z)\\
-\phi_(x,y,z)z
\end{pmatrix}
\cdot
\begin{pmatrix}
-\phi_x(x,y,z)\\
-\phi_y(x,y,z)\\
-\phi_z(x,y,z)
\end{pmatrix}
\\
\omega_e
&=
\frac{1}{2}\epsilon_0\left(\phi_x(x,y,z)^2 + \phi_y(x,y,z)^2 + \phi_z(x,y,z)^2\right).
\label{maxwell:section:energiedichte}
\end{align}
Dies können wir in das zu minimerende Integral einsetzen und bekommen
%TODO: eventuell underbraces weg lassen.
\begin{equation}
	W_e
	=
	\int_V \underbrace{
		\frac{1}{2}\epsilon_0\left(\phi_x(x,y,z)^2 + \phi_y(x,y,z)^2 + \phi_z(x,y,z)^2\right)}_{L(x,y,z,\phi(x,y,z),\phi_x(x,y,z),\phi_y(x,y,z),\phi_z(x,y,z))}\, dV.
	\label{maxwell:section:energieintegral_quellenfrei}
\end{equation}
Aus dieser Gleichung können wir entnehmen, dass unsere Lagrangefunktion
\begin{equation}
	L(x,y,z,\phi(x,y,z),\phi_x(x,y,z),\phi_y(x,y,z),\phi_z(x,y,z))
	=
	\frac{1}{2}\epsilon_0\left(\phi_x(x,y,z)^2 + \phi_y(x,y,z)^2 + \phi_z(x,y,z)^2\right)
	\label{maxwell:section:lagrangefunktion_quellenfrei}
\end{equation}
ist.
Somit haben wir unsere Lagrangefunktion gefunden, die wir in einem nächsten Schritt in die Euler-Ostrogradski-Differentialgleichung einsetzen können.

\subsection{Einsetzen in die Euler-Ostrogradski-Differentialgleichung}
%TODO: label suchen von E-O-DGL von müller kapitel
Nun gilt es die in \eqref{maxwell:section:lagrangefunktion_quellenfrei} gefundene Gleichung in die E-O-DGL \ref{???} einzusetzen.
Nach Einsetzen wird die Differentialgleichung
\[
\frac{1}{2}\epsilon_0\left(\underbrace{\frac{\partial}{\partial\phi}\left(\phi_x^2 + \phi_y^2 + \phi_z^2\right)}_{=0} - \frac{\partial}{\partial x}\frac{\partial}{\partial \phi_x}\left(\phi_x^2 + \phi_y^2 + \phi_z^2\right) - 
\frac{\partial}{\partial y}\frac{\partial}{\partial \phi_y}\left(\phi_x^2 + \phi_y^2 + \phi_z^2\right) - 
\frac{\partial}{\partial z}\frac{\partial}{\partial \phi_z}\left(\phi_x^2 + \phi_y^2 + \phi_z^2\right)\right)
=
0.
\]
Man sieht, dass die partielle Ableitung nach $\phi$ verschwindet.
Nach den partiellen Ableitungen nach $\phi_x$, $\phi_y$ und $\phi_z$ wird die Differentialgleichung
\[
\frac{1}{2}\epsilon_0\left(-\frac{\partial}{\partial x}2\phi_x(x,y,z) - \frac{\partial}{\partial y}2\phi_y(x,y,z) - \frac{\partial}{\partial z}2\phi_z(x,y,z)\right)
=
0.
\]
Wenn man nun noch die letzten partiellen Ableitungen macht, wird die Differentialgleichung
\begin{equation}
	- \underbrace{\epsilon_0}_{\not{=}0}\underbrace{\left(\frac{\partial^2\phi(x,y,z)}{\partial x^2} + \frac{\partial^2\phi(x,y,z)}{\partial y^2} + \frac{\partial^2\phi(x,y,z)}{\partial z^2}\right)}_{=0}
	=
	0.
	\label{maxwell:section:laplace_gleichung_1}
\end{equation}
%definiton des laplace operators suchen
An dieser Gleichung sieht man, dass die Klammer mit den partiellen Ableitungen gleich null sein muss, da die Permittivität von Vakuum nicht null sein kann.
Zusätzlich wird nun ersichtlich, dass der Klammerterm nach definition \ref{???} mit dem Laplace-Operator angewendet auf das elektrische Potentialfeld $\phi$ ersetzt werden kann.
Somit wird unsere schluss Differentialgleichung
\begin{equation}
	\Delta\phi(x,y,z)
	=
	0.
	\label{maxwell:section:laplace_gleichung_2}
\end{equation}
Durch Anwendng der Definiton des Laplace-Operator \ref{???} und der Definiton des Elektrischenfeldes \ref{???} erhalten wird die Gleichung
\[
\nabla\cdot\underbrace{\nabla\phi(x,y,z)}_{-\vec{E}(x,y,z)}
=
0.
\]
Hiermit erhalten wir, dass
\begin{equation}
	\nabla\cdot\vec{E}(x,y,z)
	=
	0
	\label{maxwell:section:e_feld_quellenfrei}
\end{equation}
% TODO:Bild referenz einfügen und Bild erstellen
sein muss. Diese Differentialgleichung besagt, dass das Elektrische Feld quellenfrei ist.
Dies bedeutet, dass Felldlinien des Elektrischenfeldes an keinem Ort im Raum enstehen oder enden können \ref{???}.
Dies ist sehr naheliegend, da ohne Ladungen im Raum das Elektrische Feld quellenfrei sein muss.

%Darf auch weggelassen werden.
\subsubsection{Exkurs zur Laplace-Gleichung}
Ein Potentialfeld, das die Laplace-Gleichung
\[
\Delta\varphi
=
0
\]
erfüllt, führt zu einem Gradientenfeld $\nabla\varphi$, das rotationsfrei und quellenfrei ist.
Diese Gleichung findet nicht nur Anwendungen in der Elektrostatik, sondern auch in stationärer Fluiddynamik und stationärer Wärmeleitung.






