%
% main.tex -- Paper zum Thema <relativ>
%
% (c) 2020 Autor, OST Ostschweizer Fachhochschule
%
%  TEX root = ../../buch.tex
%  TEX encoding = UTF-8
%
\chapter{Relativistische Mechanik\label{chapter:relativ}}
\kopflinks{Thema}
\begin{refsection}
\chapterauthor{Selvin Blöchlinger}

Ein paar Hinweise für die korrekte Formatierung des Textes
\begin{itemize}
\item
Absätze werden gebildet, indem man eine Leerzeile einfügt.
Die Verwendung von \verb+\\+ ist nur in Tabellen und Arrays gestattet.
\item
Die explizite Platzierung von Bildern ist nicht erlaubt, entsprechende
Optionen werden gelöscht. 
Verwenden Sie Labels und Verweise, um auf Bilder hinzuweisen.
\item
Beginnen Sie jeden Satz auf einer neuen Zeile. 
Damit ermöglichen Sie dem Versionsverwaltungssysteme, Änderungen
in verschiedenen Sätzen von verschiedenen Autoren ohne Konflikt 
anzuwenden.
\item 
Bilden Sie auch für Formeln kurze Zeilen, einerseits der besseren
Übersicht wegen, aber auch um GIT die Arbeit zu erleichtern.
\end{itemize}


\section{Das Relativitätsprinzip 
\label{relativ:section:relativistik}}
\rhead{Das Relativitätsprinzip}

Um relativistische Mechanik verstehen zu können,
muss natürlich zuerst auf das Relativitätsprinzip eingegangen werden.
Dabei gibt es zwei wesentliche Unterschiede zur klassischen Mechanik.

Der erste wichtige Unterschied ist, dass die Zeit unter relativistischer Betrachtung keine absolute Grösse mehr ist.
Geschehenes kann also nicht einfach anhand eines starren Zeitstrahls erklärt werden, sondern die Zeit ist abhängig vom Betrachter.
Man geht über vom dreidimensionalen Raum in die vierdimensionale Raumzeit, in welcher die Welt relativistisch beschrieben wird.

Der zweite Unterschied ist,
dass die Geschwindigkeit der Wirkungsausbreitung begrenzt ist,
und zwar durch die Lichtgeschwindigkeit \(c=\qty[per-mode=fraction]{299792458}{\metre\per\second}\).
Diese ist zwar auch in der klassischen Mechanik eine Konstante,
jedoch ist die Geschwindigkeit der Wirkungsausbreitung dort unbegrenzt.
Ein einfaches Beispiel, um dies zu veranschaulichen,
ist die Betrachtung von starren Körpern in der klassischen Mechanik.
Wird ein starrer Körper beispielsweise an einem Punkt angestossen,
so muss sich, gemäss Definition eines starren Körpers,
jeder Teil dieses Körpers augenblicklich und zeitgleich in Bewegung setzen.
Dies bedeutet also, dass sich die Wirkung (Anstossen des Körpers)
vom Punkt aus, in dem dieser angestossen wurde,
mit unendlicher Geschwindigkeit in alle Teile des Körpers ausbreitet.


\subsection{Koordinatentransformationen
\label{relativ:section:koordtrafo}}




\section{Die relativistische Mechanik
\label{relativ:section:rel-mech}}
\kopfrechts{Die relativistische Mechanik}

Dieser Abschnitt soll eine kurze Einführung geben in die
Grundlagen der Berechnungen in der relativistischen Mechanik.


\subsection{Elementarteilchen in der Relativitätstheorie
\label{relativ:section:elementarteilchen}}
Wie bereits in der Einleitung zu Abschnitt~\ref{relativ:section:relativistik} erwähnt,
kann es gemäss der Relativitätstheorie keine starren Körper geben.
Angemessener ist daher die Betrachtung \emph{punktförmiger Elementarteilchen}.
\index{Elementarteilchen}%
Der Zustand eines solchen Elementarteilchens ist dabei vollständig definiert durch
die drei Raumkoordinaten und die drei zugehörigen Geschwindigkeitskomponenten.


\subsection{Das Prinzip der kleinsten Wirkung
\label{relativ:section:kleinste-wirkung}}

Für die Berechnung der Bewegung von materiellen Teilchen geht man
von dem bereits in Kapitel~\ref{buch:chapter:mechanik} eingeführten
Maupertuisschen Prinzip der kleinsten Wirkung aus.
Dieses besagt, dass es für jedes mechanische System ein sogenanntes
Wirkungsintegral \(S\) gibt,
das für diejenige Bewegung ein Minimum besitzt, die tatsächlich erfolgt.
Wie bereits zu erahnen ist, kann dieses Minimum durch die Variation \(\delta S\)
des Wirkungsintegrals ermittelt werden.

\subsubsection{Wirkungsintegral für ein freies Teilchen
\label{relativ:section:wirk-int-freies-teilchen}}

Nachfolgend werden das Wirkungsintegral und die Lagrange-Funktion
für ein freies Teilchen, das heisst ein Teilchen, welches nicht
unter dem Einfluss äusserer Kräfte oder Felder steht, hergeleitet.

Da dieses Integral unabhängig vom gewählten Koordinatensystem sein muss,
kann es nur von Skalaren und nicht von den Variablen \(x, y, z, t\) abhängen.
Das Wirkungsintegral hat daher die Form
\begin{equation}
    S = - \alpha \int_{a}^{b} ds,
\label{relativ:eqn:wirk-int-grundform}
\end{equation}
wobei \(\alpha\) eine vom Teilchen abhängige Konstante,
und \(a\) und \(b\) zwei Ereignisse in der Raumzeit sind.
Integriert wird über die Weltlinie des Teilchens,
d.h. entlang dessen Bahn durch die Raumzeit.

Das Wirkungsintegral lässt sich auch über die Zeit darstellen,
wobei es die Form
\begin{equation}
    S = \int_{t_1}^{t_2} L \, dt
\label{relativ:eqn:wirk-int-zeit}
\end{equation}
annimmt, mit der Lagrange-Funktion \(L\).
Ersetzt man \(ds\) in \eqref{relativ:eqn:wirk-int-grundform}
gemäss \eqref{relativ:eqn:differential-eigenzeit} und passt
die Integrationsgrenzen an, so erhält man
\begin{equation}
    S = -\int_{t_1}^{t_2} \alpha c \sqrt{1-\frac{v^2}{c^2}}\, dt
    \quad \text{und somit} \quad
    L = -\alpha c \sqrt{1-\frac{v^2}{c^2}}.
    \label{relativ:eq:wirk-int-lagrange}
\end{equation}

Zur Bestimmung der Teilchenkonstante \(\alpha\) beginnt man mit der Forderung,
dass die Lagrange-Funktion im Grenzfall \(c\rightarrow\infty\) in den
Ausdruck \(L=mv^2/2\) der klassischen Mechanik übergehen soll.
Mittels der Potenzreihe
\begin{equation*}
    \sqrt{1+w} = 1 + \frac{1}{2} w - \frac{1}{2\cdot4} w^2 +
    \frac{1\cdot3}{2\cdot4\cdot6} w^3 \mp \cdots, \quad
    -1\leq w \leq1
\end{equation*}
kann die Lagrange-Funktion aus
\eqref{relativ:eq:wirk-int-lagrange} beim Weglassen der
Glieder höherer Ordnung und mit der Substitution
\(w=-\frac{v^2}{c^2}\) entwickelt werden zu
\begin{equation}
    L = - \alpha c \sqrt{1-\frac{v^2}{c^2}}
    \approx -\alpha c \biggl(1 - \frac{v^2}{2c^2}\biggr)
    = -\alpha c + \frac{\alpha v^2}{2c}.
\end{equation}
Da Konstanten in der Lagrange-Funktion keinen Einfluss auf die Bewegungsgleichung haben,
kann der erste Term ignoriert werden. Somit bleibt die Forderung
\begin{equation}
    \frac{\alpha v^2}{2c} = \frac{mv^2}{2}
    \quad \text{und daraus ergibt sich} \quad
    \alpha = mc.
\end{equation}
Ein freies Teilchen besitzt folglich die Lagrange-Funktion
\begin{equation}
    L = -mc^2 \sqrt{1-\frac{v^2}{c^2}}.
\label{relativ:eqn:lagrange-freies-teilchen}
\end{equation}

\subsubsection{Wirkungsintegral für ein Teilchen im elektromagnetischen Feld
\label{relativ:section:wirk-int-em-feld}}

Das Wirkungsintegral für ein Elementarteilchen im elektromagnetischen Feld ist
\index{Wirkungsintegral für Elementarteilchen}%
\begin{equation}
    S = \int_a^b \biggl(-mc\,ds + \frac{e}{c} \bm{A}\,d\bm{r} - e\varphi\,dt\biggr)
    \label{relativ:eqn:wirk-int-em-feld}
\end{equation}
und somit eine Erweiterung von \eqref{relativ:eqn:wirk-int-grundform}.
Dabei ist \(\bm{A}\) das \emph{Vektorpotential des Feldes}\footnote{
\index{Vektorpotential}
    Auch bekannt als \emph{magnetisches Vektorpotential}.}
\index{magnetisches Vektorpotential}%
und \(\varphi\) das \emph{skalare Potential des Feldes}\footnote{
    Auch bekannt als \emph{elektrisches Potential}.}.
\index{elektrisches Potential}%
In der relativistischen Raumzeit schreibt man diese zwei Grössen
auch als Vierervektor \(A^i = (\varphi, \bm{A})\),
wobei \(\varphi\) zur Zeitkoordinate \(t\) gehört und
\(\bm{A}\) zu den räumlichen Koordinaten \(x, y, z\).
Als Integral über die Zeit geschrieben wird das Wirkungsintegral zu
\begin{equation}
    S = \int_{t_1}^{t_2} \biggl( -mc^2 \sqrt{1-\frac{v^2}{c^2}} + \frac{e}{c} \bm{A} \bm{v} - e \varphi \biggr) \, dt,
    \label{relativ:eqn:wirk-int-em-feld-zeit}
\end{equation}
wobei \(\displaystyle \bm{v} = \frac{d\bm{r}}{dt}\) der Geschwindigkeitsvektor der drei räumlichen Dimensionen ist.
Der Integrand in \eqref{relativ:eqn:wirk-int-em-feld-zeit} ist gerade die Lagrange-Funktion
\begin{equation}
    L = -mc^2 \sqrt{1-\frac{v^2}{c^2}} + \frac{e}{c} \bm{A} \bm{v} - e \varphi.
    \label{relativ:eqn:lagrange-em-feld}
\end{equation}


\section{Ladungen im Elektromagnetischen Feld 
\label{relativ:section:em_feld}}
\rhead{Ladungen im Elektromagnetischen Feld}



\printbibliography[heading=subbibliography]
\end{refsection}
