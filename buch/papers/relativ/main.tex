%
% main.tex -- Paper zum Thema <relativ>
%
% (c) 2020 Autor, OST Ostschweizer Fachhochschule
%
%  TEX root = ../../buch.tex
%  TEX encoding = UTF-8
%
\chapter{Relativistische Mechanik\label{chapter:relativ}}
\kopflinks{Thema}
\begin{refsection}
\chapterauthor{Selvin Blöchlinger}

Ein paar Hinweise für die korrekte Formatierung des Textes
\begin{itemize}
\item
Absätze werden gebildet, indem man eine Leerzeile einfügt.
Die Verwendung von \verb+\\+ ist nur in Tabellen und Arrays gestattet.
\item
Die explizite Platzierung von Bildern ist nicht erlaubt, entsprechende
Optionen werden gelöscht. 
Verwenden Sie Labels und Verweise, um auf Bilder hinzuweisen.
\item
Beginnen Sie jeden Satz auf einer neuen Zeile. 
Damit ermöglichen Sie dem Versionsverwaltungssysteme, Änderungen
in verschiedenen Sätzen von verschiedenen Autoren ohne Konflikt 
anzuwenden.
\item 
Bilden Sie auch für Formeln kurze Zeilen, einerseits der besseren
Übersicht wegen, aber auch um GIT die Arbeit zu erleichtern.
\end{itemize}


\section{Das Relativitätsprinzip 
\label{relativ:section:relativistik}}
\rhead{Das Relativitätsprinzip}


\section{Relativistische Mechanik 
\label{relativ:section:rel_mechanik}}
\rhead{Relativistische Mechanik}

\subsection{Das Prinzip der kleinsten Wirkung 
\label{relativ:section:kleinste-wirkung}}

Für die Berechnung der Bewegung von materiellen Teilchen geht man
vom Prinzip der kleinsten Wirkung aus.
Dieses besagt, dass es für jedes mechanische System ein sogenanntes
Wirkungsintegral \(S\) gibt,
dass für diejenige Bewegung ein Minimum besitzt, die tatsächlich erfolgt.
Wie bereits zu erahnen ist, kann dieses Minimum durch die Variation \(dS\)
des Wirkungsintegrals ermittelt werde.

Da dieses Integral unabhängig vom gewählten Koordinatensystem sein muss,
kann es nur von Skalaren und nicht von den Variablen \(x, y, z, t\) abhängen.
Das Wirkungsintegral hat daher die Form
\begin{equation}
    S = - \alpha \int_{a}^{b} ds,
\label{relativ:eqn:wirk-int-grundform}
\end{equation}
wobei \(\alpha\) eine vom Teilchen abhängige Konstante,
und \(a\) und \(b\) zwei Ereignisse in der Raumzeit sind.
Integriert wird über die Weltlinie des Teilchens,
d.h. entlang dessen Bahn durch die Raumzeit.

Das Wirkungsintegral lässt sich auch über die Zeit darstellen,
wobei es die Form
\begin{equation}
    S = \int_{t_1}^{t_2} L \, dt
\label{relativ:eqn:wirk-int-zeit}
\end{equation}
annimmt, mit der Lagrange-Funktion \(L\).

\todo{Saubere Überleitung erarbeiten.}

Für ein freies Teilchen ergibt sich schliesslich die Lagrange-Funktion
\begin{equation}
    L = -mc^2 \sqrt{1-\frac{v^2}{c^2}}.
\label{relativ:eqn:lagrange-freies-teilchen}
\end{equation}


\section{Ladungen im Elektromagnetischen Feld 
\label{relativ:section:em_feld}}
\rhead{Ladungen im Elektromagnetischen Feld}

\subsection{Elementarteilchen in der Relativitätstheorie 
\label{relativ:section:elementarteilchen}}

Wie bereits in der Einleitung zu Abschnitt~\ref{relativ:section:relativistik} erwähnt,
kann gemäss der Relativitätstheorie keine starren Körper geben.
Angemessener ist daher die Betrachtung \emph{punktförmiger Elementarteilchen}.
Der Zustand eines solchen Elementarteilchens ist dabei vollständig definiert durch
die drei Raumkooridnaten und die drei zugehörigen Geschwindigkeitskomponenten.

\subsection{ Wirkungsintegral 
\label{relativ:section:wirkungsintegral}}

Das Wirkungsintegral für ein Elementarteilchen im elektromagnetischen Feld ist
\begin{equation}
    S = - \int_{t_1}^{t_2} \left( -mc^2 \sqrt{1-\frac{v^2}{c^2}} + \frac{e}{c} \mathbf{A} \mathbf{v} - e \varphi \right) \, dt
\end{equation}



\printbibliography[heading=subbibliography]
\end{refsection}
