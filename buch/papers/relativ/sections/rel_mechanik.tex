
\section{Relativistische Mechanik 
\label{relativ:section:rel_mechanik}}
\rhead{Relativistische Mechanik}

\subsection{Das Prinzip der kleinsten Wirkung 
\label{relativ:section:kleinste-wirkung}}

Für die Berechnung der Bewegung von materiellen Teilchen geht man
vom Prinzip der kleinsten Wirkung aus.
Dieses besagt, dass es für jedes mechanische System ein sogenanntes
Wirkungsintegral \(S\) gibt,
dass für diejenige Bewegung ein Minimum besitzt, die tatsächlich erfolgt.
Wie bereits zu erahnen ist, kann dieses Minimum durch die Variation \(dS\)
des Wirkungsintegrals ermittelt werden.

Da dieses Integral unabhängig vom gewählten Koordinatensystem sein muss,
kann es nur von Skalaren und nicht von den Variablen \(x, y, z, t\) abhängen.
Das Wirkungsintegral hat daher die Form
\begin{equation}
    S = - \alpha \int_{a}^{b} ds,
\label{relativ:eqn:wirk-int-grundform}
\end{equation}
wobei \(\alpha\) eine vom Teilchen abhängige Konstante,
und \(a\) und \(b\) zwei Ereignisse in der Raumzeit sind.
Integriert wird über die Weltlinie des Teilchens,
d.h. entlang dessen Bahn durch die Raumzeit.

Das Wirkungsintegral lässt sich auch über die Zeit darstellen,
wobei es die Form
\begin{equation}
    S = \int_{t_1}^{t_2} L \, dt
\label{relativ:eqn:wirk-int-zeit}
\end{equation}
annimmt, mit der Lagrange-Funktion \(L\).
Ersetzt man \(ds\) in \eqref{relativ:eqn:wirk-int-grundform}
gemäss \eqref{relativ:eqn:differential-eigenzeit} und passt
die Integrationsgrenzen an, so erhält man
\begin{equation}
    S = -\int_{t_1}^{t_2} \alpha c \sqrt{1-\frac{v^2}{c^2}} dt.
\end{equation}

Zur Bestimmung der Teilchenkonstante \(\alpha\) beginnt man mit der Forderung,
dass die Lagrange-Funktion im Grenzfall \(c\rightarrow\inf\) in den
Ausdruck \(L=mv^2/2\) der klassischen Mechanik übergehen soll.
Die Taylorreihe von \(L\) ergibt beim Weglassen der Terme höherer Ordnung
\begin{equation}
    L = - \alpha c \sqrt{1-\frac{v^2}{c^2}}
    \approx -\alpha c + \frac{\alpha v^2}{2c}.
\end{equation}
Da Konstanten in der Lagrange-Funktion keinen Einfluss auf die Bewegungsgleichung haben,
kann der erste Term ignoriert werden. Somit bleibt die Forderung
\begin{equation}
    \frac{\alpha v^2}{2c} = \frac{mv^2}{2}\
    \text{und daraus ergibt sich}\
    \alpha = mc.
\end{equation}
Für ein freies Teilchen ergibt sich folglich die Lagrange-Funktion
\begin{equation}
    L = -mc^2 \sqrt{1-\frac{v^2}{c^2}}.
\label{relativ:eqn:lagrange-freies-teilchen}
\end{equation}
