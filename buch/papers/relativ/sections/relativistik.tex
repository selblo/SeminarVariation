
\section{Das Relativitätsprinzip 
\label{relativ:section:relativistik}}
\rhead{Das Relativitätsprinzip}

Um relativistische Mechanik verstehen zu können,
muss natürlich zuerst auf das Relativitätsprinzip eingegangen werden.
Dabei gibt es zwei wesentliche Unterschiede zur klassischen Mechanik.

Der erste wichtige Unterschied ist, dass die Zeit unter relativistischer Betrachtung keine absolute Grösse mehr ist.
Geschehenes kann also nicht einfach anhand eines starren Zeitstrahls erklärt werden, sondern die Zeit ist abhängig vom Betrachter.
Man geht über vom dreidimensionalen Raum in die vierdimensionale Raumzeit, in welcher die Welt relativistisch beschrieben wird.

Der zweite Unterschied ist,
dass die Geschwindigkeit der Wirkungsausbreitung begrenzt ist,
und zwar durch die Lichtgeschwindigkeit \(c=\qty[per-mode=fraction]{299792458}{\metre\per\second}\).
Diese ist zwar auch in der klassischen Mechanik eine Konstante,
jedoch ist die Geschwindigkeit der Wirkungsausbreitung dort unbegrenzt.
Ein einfaches Beispiel, um dies zu veranschaulichen,
ist die Betrachtung von starren Körpern in der klassischen Mechanik.
Wird ein starrer Körper beispielsweise an einem Punkt angestossen,
so muss sich, gemäss Definition eines starren Körpers,
jeder Teil dieses Körpers augenblicklich und zeitgleich in Bewegung setzen.
Dies bedeutet also, dass sich die Wirkung (Anstossen des Körpers)
vom Punkt aus, in dem dieser angestossen wurde,
mit unendlicher Geschwindigkeit in alle Teile des Körpers ausbreitet.


\subsection{Koordinatentransformationen
\label{relativ:section:koordtrafo}}


