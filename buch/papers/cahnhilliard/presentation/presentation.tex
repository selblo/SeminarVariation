% !TeX spellcheck = ge_De
% !TeX encoding = UTF-8

\documentclass[ngerman, aspectratio=169, xcolor={rgb}]{beamer}

% style
\mode<presentation>{
	\usetheme{Frankfurt}
}
%packages
\usepackage[utf8]{inputenc}\DeclareUnicodeCharacter{2212}{-}
\usepackage[ngerman]{babel}
\usepackage{graphicx}
\usepackage{array}

\newcolumntype{L}[1]{>{\raggedright\let\newline\\\arraybackslash\hspace{0pt}}m{#1}}
\usepackage{ragged2e}

\usepackage{bm} % bold math
\usepackage{amsfonts}
\usepackage{amssymb}
\usepackage{mathtools}
\usepackage{amsmath}
\usepackage{multirow} % multi row in tables
\usepackage{booktabs} %toprule midrule bottomrue in tables
\usepackage{scrextend}
\usepackage{textgreek}
\usepackage[rgb]{xcolor}

\usepackage{ marvosym } % \Lightning

\usepackage{multimedia} % embedded videos

\usepackage{tikz}
\usepackage{pgf}
\usepackage{pgfplots}

\usepackage{algorithmic}

%citations
%\usepackage[style=verbose,backend=biblatex]{biblatex}
%\addbibresource{references.bib}


%math font
\usefonttheme[onlymath]{serif}

%Beamer Template modifications
%\definecolor{mainColor}{HTML}{0065A3} % HSR blue
\definecolor{accentColor}{HTML}{D72864} % OST raspberry
\definecolor{mainColor}{HTML}{8C195F} % OST raspberry
\definecolor{invColor}{HTML}{28d79b} % OST pink
\definecolor{dgreen}{HTML}{38ad36} % Dark green

%\definecolor{mainColor}{HTML}{000000} % HSR blue
\setbeamercolor{palette primary}{bg=white,fg=mainColor}
\setbeamercolor{palette secondary}{bg=orange,fg=mainColor}
\setbeamercolor{palette tertiary}{bg=yellow,fg=red}
\setbeamercolor{palette quaternary}{bg=mainColor,fg=white} %bg = Top bar, fg = active top bar topic
\setbeamercolor{structure}{fg=mainColor} % itemize, enumerate, etc (bullet points)
\setbeamercolor{section in toc}{fg=black} % TOC sections
\setbeamertemplate{section in toc}[sections numbered]
\setbeamertemplate{subsection in toc}{%
	\hspace{1.2em}{$\bullet$}~\inserttocsubsection\par}

\setbeamertemplate{itemize items}[circle]
\setbeamertemplate{enumerate items}[circle]
\setbeamertemplate{description item}[circle]
\setbeamertemplate{title page}[default][colsep=-4bp,rounded=true]
% \setbeamertemplate{blocks}[rounded]{shadow=false}
% \setbeamertemplate{blocks}[default]{shadow=true}
\setbeamercolor{block title}{use=structure,fg=white,bg=accentColor}
\beamertemplatenavigationsymbolsempty

\setbeamercolor{footline}{fg=gray}
\setbeamertemplate{footline}{%
	\hfill\usebeamertemplate***{navigation symbols}
	\hspace{0.5cm}
	\insertframenumber{}\hspace{0.2cm}\vspace{0.2cm}
}

% \usepackage{caption}
\usepackage{subcaption}
% \captionsetup{labelformat=empty}

%Title Page
\title{Cahn-Hilliard-Gleichung}
\subtitle{Wieso trennt sich meine Salatsauce?}
\author{Patrik Müller}
% \institute{OST Ostschweizer Fachhochschule}
% \institute{\includegraphics[scale=0.3]{../img/ost_logo.png}}
\date{\today}

%
% packages.tex -- special packages needed by doppelpendel
%


\newcommand*{\QED}{\hfill\ensuremath{\blacksquare}}%

\newcommand*{\HL}{\textcolor{mainColor}}
\newcommand*{\RD}{\textcolor{red}}
\newcommand*{\BL}{\textcolor{blue}}
\newcommand*{\GN}{\textcolor{dgreen}}

\definecolor{darkgreen}{rgb}{0,0.6,0}


\makeatletter
\newcount\my@repeat@count
\newcommand{\myrepeat}[2]{%
	\begingroup
	\my@repeat@count=\z@
	\@whilenum\my@repeat@count<#1\do{#2\advance\my@repeat@count\@ne}%
	\endgroup
}
\makeatother

\usetikzlibrary{automata,arrows,positioning,calc,shapes.geometric, fadings}

\newcommand{\dd}[2][]{\mathrm{d}^{#1} #2}
\newcommand{\di}[2][]{\,\dd[#1]{#2}}
\newcommand{\deriv}[3][]{\frac{\dd[#1]{#2}}{\dd[]{#3^{#1}}}}
\newcommand{\pderiv}[3][]{\frac{\partial^{#1} #2}{\partial #3^{#1}}}
\newcommand{\abs}[1]{\left| #1 \right| }
\newcommand{\energy}{\mathcal{E}}
\newcommand{\flux}{\mathcal{J}}

\begin{document}

\begin{frame}
	\titlepage
\end{frame}

\begin{frame}{Rezept}
	\tableofcontents
\end{frame}
%
% \begin{frame}{}
% \textbf{Zutaten:}\newline
% \begin{enumerate}
% \item Einführung
% \item Definition der freien Energie
% \begin{enumerate}
% \item Für homogene Systeme
% \item Erweiterung auf heterogene Systeme
% \end{enumerate}
% \item Globale Betrachtung des Problems
% \item Cahn-Hilliard-Gleichungen
% \item Spinodale Entmischung
% \item Gegenmassnahmen
% \end{enumerate}
% \end{frame}

% !TeX spellcheck = de_CH
% !TeX encoding = UTF-8
% !TeX root = ../presentation.tex

\section{Einführung}

\begin{frame}{Um was geht es?}
\begin{itemize}
\item<1-> Verhalten von Mischungen zweier Materialien (Phasen) über die Zeit
\item<2-> Ursprüngliche Anwendung war ein theoretisches Modell für die Entmischung von binären Legierungen
\end{itemize}
\uncover<3->{
\begin{figure}
\centering
\begin{subfigure}{0.18\textwidth}
\centering
\includegraphics[width=\textwidth]{images/0.pdf}
\caption{$t = 0\,\tau$}
\end{subfigure}
\qquad
\begin{subfigure}{0.18\textwidth}
\centering
\includegraphics[width=\textwidth]{images/300.pdf}
\caption{$t = 300\,\tau$}
\end{subfigure}
\caption{Beispiel spinodaler Entmischung}
\end{figure}
}
\end{frame}

\begin{frame}{Definitionen}
\begin{itemize}
\item<+-> $N_A$ lokale Stoffmenge von Stoff A, $N_A > 0$
\item<+-> $N_B$ lokale Stoffmenge von Stoff B, $N_B > 0$
\item<+-> $c$ lokale Konzentration von Stoff B
\begin{align*}
c
=
\frac{N_B}{N_A + N_B}
\quad \Rightarrow \quad c \in [0, 1]
\end{align*}
\item<+-> $F(c)$ freie Energie pro Molekül eines homogenen Gemisches
(wir betrachten diese Funktion als gegeben)
\end{itemize}
\end{frame}

% \begin{frame}{Was}
%   frame
% \end{frame}

% !TeX spellcheck = de_CH
% !TeX encoding = UTF-8
% !TeX root = ../presentation.tex

\section{Freie Energie}

\begin{frame}{Annahmen}
\begin{itemize}
\item Die Temperatur ist konstant
\item $\nabla c$ ist sehr gross im Verhältnis zur intermolekularen Distanz
\item $c$ und dessen Ableitungen sind unabhängige Variablen
\end{itemize}
\end{frame}

\begin{frame}{Freie Energy für heterogene Gemische}
\uncover<+->{
\textbf{Vermutung:}
Die lokale freie Energie in einem heterogenen Gemisch ist abhängig
von der lokalen Konzentration und
von der Konzentration der anliegenden Umgebung.
}

$ $

\uncover<+->{
\textbf{Idee:} $f(c, \nabla c, \nabla^2 c, \ldots)$ multivariable Taylorreihe
um den Punkt $\mathbf{c_0} = (c, 0, 0, \ldots)$
\begin{align*}
f(c, \nabla c, \nabla^2 c, \ldots)
=
\uncover<+->{
& F(c)
}
\uncover<+->{
+ \sum_{i=1}^3 \pderiv{f(\mathbf{c_0})}{c_i} c_i
}
\uncover<+->{
+ \sum_{i,j=1}^3 \pderiv{f(\mathbf{c_0})}{c_{ij}} c_{ij}
}
\uncover<+->{
+ \frac{1}{2} \sum_{i,j=1}^3 \frac{\partial^2 f(\mathbf{c_0})}{\partial c_i \partial c_j} c_{i} c_{j}
% \\
% & + \frac{1}{2} \sum_{i,j,k=1}^3 \frac{\partial^2 f(c_0)}{\partial c_k \partial c_{ij}} c_k c_{ij}
% + \frac{1}{2} \sum_{i,j,k,l=1}^3 \frac{\partial^2 f(c_0)}{\partial c_{ij} \partial c_{kl}} c_{ij} c_{kl}
+ \ldots
}
\end{align*}
\begin{align*}
\text{wobei}
\quad
c_i
=
\pderiv{c}{x_i}
,\quad
c_{ij}
=
\frac{\partial^2 c}{\partial x_i \partial x_j}
\end{align*}
}
\end{frame}

\begin{frame}{Vereinfachung}
\uncover<+->{
Wir betrachten nur isotrope Medien
}
\uncover<+->{
\begin{block}{Isotropie}
Isotropie bezeichnet die Unabhängigeit einer Eigenschaft von der Richtung
\end{block}
}
\uncover<+->{
$\Rightarrow$ invariant in Bezug auf Spiegelung ($x_i \rightarrow -x_i$)  und Permutation ($x_i \rightarrow x_j$)
}
\begin{align*}
\uncover<+->{
\pderiv{f(\mathbf{c_0})}{c_i}
&=
0
,&
}
\uncover<+->{
\pderiv{f(\mathbf{c_0})}{c_{ij}}
&=
\frac{\partial^2 f(\mathbf{c_0})}{\partial c_i \partial c_j}
=
0
\quad \forall i \neq j
\\
}
\uncover<+->{
\pderiv{f(\mathbf{c_0})}{c_{ii}}
&=
\kappa_1
,&
}
\uncover<+->{
 \pderiv[2]{f(\mathbf{c_0})}{c_i}
&=
\kappa_2
}
\end{align*}
% \begin{align*}
% f(c, \nabla c, \nabla^2 c, \ldots)
% &=
% F(c) + \kappa_1 \Delta c + \frac{\kappa_2}{2} \abs{\nabla c}^2  + \ldots
% \end{align*}
\end{frame}

\begin{frame}{}
\begin{align*}
\mathcal{E}(c)
=
&N_V \int_\Omega f\di{x}
=
N_V \int_\Omega \left[
F(c) + \kappa_1 \Delta c + \frac{\kappa_2}{2} \abs{\nabla c}^2  + \ldots
\right]\di{x}
\\
&\text{wobei $N_V$ die Anzahl Moleküle im Gebiet $\Omega$ sind}
\end{align*}
\end{frame}

\begin{frame}
\uncover<+->{
$\Delta c$-Term in eine angenehmere Form bringen
% $\pderiv{c}{n}$ verschwindet am Rand
}
\begin{align*}
\int_\Omega \kappa_1 \Delta c \di{x}
&=
\uncover<+->{
\int_{\partial\Omega} \kappa_1 \pderiv{c}{n} \di{s}
- \int_\Omega \nabla \kappa_1 \cdot \nabla c \di{x}
\\
}
\uncover<+->{
&=-\int_\Omega \sum_{i=1}^3 \pderiv{\kappa_1}{x_i} \pderiv{c}{x_i} \di{x}
\\
}
\uncover<+->{
&=-\int_\Omega \sum_{i=1}^3 \pderiv{\kappa_1}{c} \pderiv{c}{x_i} \pderiv{c}{x_i} \di{x}
\\
}
\uncover<+->{
&=
-\int_\Omega \pderiv{\kappa_1}{c} \abs{\nabla c}^2 \di{x}
}
\end{align*}
\end{frame}

\begin{frame}{Totale freie Energie}
\begin{align}
\uncover<+->{
\mathcal{E}(c)
&=
N_V \int_\Omega \left[
F(c) + \kappa_1 \Delta c + \frac{\kappa_2}{2} \abs{\nabla c}^2  + \ldots
\right]\di{x}
\nonumber
\\
}
\uncover<+->{
&=
N_V \int_\Omega \left[
  F(c) + \underbrace{\left( \frac{\kappa_2}{2} - \pderiv{\kappa_1}{c} \right)}_{\frac{1}{2}\epsilon^2} \abs{\nabla c}^2  + \ldots
\right]\di{x}
\nonumber
\\
}
\uncover<+->{
&=
N_V \int_\Omega \left[
  F(c) + \frac{\epsilon^2}{2} \abs{\nabla c}^2  + \ldots
\right]\di{x}
,\quad
\text{wobei $\epsilon$ konstant}
\label{eq:energy}
}
\end{align}
\end{frame}


% \input{sections/global}
% !TeX spellcheck = de_CH
% !TeX encoding = UTF-8
% !TeX root = ../presentation.tex

\section{Cahn-Hilliard-Gleichung}

\begin{frame}{Plan für die Herleitung}
\begin{enumerate}
\item Aufstellen des Variationsproblems mit \eqref{eq:energy}
\item Anwenden Euler-Ostrogradski-Gleichung
\item Koppeln mit Massenerhaltung
\end{enumerate}
\end{frame}

\begin{frame}{Problemdefinition}
Aus \eqref{eq:energy} Skalierungsfaktor und höher gradige Terme entfernen
\begin{align}
I(c)
&=
\int_\Omega L(x, c, \nabla c) \di{x}
\nonumber
\\
&=
\int_\Omega F(c) + \frac{\epsilon^2}{2} \abs{\nabla c}^2 \di{x}
\end{align}
\end{frame}

\begin{frame}{Anwenden von Euler-Ostrogradski-Gleichung}
\begin{align*}
\pderiv{I}{c}
&=
\uncover<2->{
\deriv{F}{c}
}
\\
\pderiv{I}{(\nabla c)}
&=
% \pderiv{}{(\nabla c)} \frac{\epsilon^2}{2} \abs{\nabla c}^2
% &=
\uncover<3->{
\epsilon^2 \sum_{i=1}^3 \pderiv{c}{x_i}
}
\\
\pderiv{}{x_i}\pderiv{I}{(\nabla c)}
&=
\uncover<4->{
\epsilon^2 \sum_{i=1}^3 \pderiv[2]{c}{x_i}
}
\uncover<5->{
=
\epsilon^2 \Delta c
}
% \pderiv{}{(\nabla c)} \left[
% \sum_{i=1}^3 \left( \pderiv{I}{c_i} \right)
% \right]
% \\
% &&&=
% 2 \sum_{i=1}
\end{align*}
\uncover<6->{
Somit ergibt sich das Funktional
\begin{align*}
\frac{\delta I}{\delta c}
&=
\deriv{F}{c} -  \epsilon^2 \Delta c
\equiv
\mu
\end{align*}
}
\end{frame}


\begin{frame}{Koppeln mit Massenerhaltung}
\uncover<+->{
\begin{alignat*}{2}
\pderiv{c}{t}
&=
- \nabla \cdot \mathcal{J}
,\quad&
x &\in \Omega
\\
\nabla c \cdot n
&=
0
,&
x &\in \partial\Omega
\\
\flux \cdot n
&=
0
,&
x &\in \partial\Omega
\end{alignat*}
}
\uncover<+->{
Was könnte der Fluss in unserem Problem sein?
}
\uncover<+->{
\begin{align*}
\flux
=
- M \nabla \mu
,\quad \text{wobei } M > 0
\end{align*}
}
\end{frame}

\begin{frame}{Stabiles System}
Freie Energie im System kann nicht zunehmen (1. Gesetz der Thermodynamik)
\begin{align*}
\uncover<+->{
\deriv{}{t} I
&=
\int_\Omega \left[
\deriv{F}{c} \pderiv{c}{t} + \epsilon^2 \nabla c \cdot \nabla \pderiv{c}{t}
\right] \di{x}
\\
}
\uncover<+->{
&=
\int_\Omega \mu \pderiv{c}{t} \di{x}
\\
}
\uncover<+->{
&=
\int_\Omega \mu \nabla \cdot (M \nabla \mu) \di{x}
\\
}
\uncover<+->{
&=
\int_{\partial\Omega} \mu M \nabla \mu \cdot n \di{s} - \int_\Omega \nabla \mu \cdot (M \nabla \mu) \di{x}
\\
}
\uncover<+->{
&=
-\int_\Omega M \abs{\nabla \mu}^2 \di{x}
}
\end{align*}
\end{frame}

\begin{frame}{Massenerhaltung}
Die Masse im System kann sich nicht ändern
\begin{align*}
\uncover<+->{
0
&=
\deriv{}{t} \int_\Omega c \di{x}
\\
}
\uncover<+->{
&=
\int_\Omega \pderiv{c}{t} \di{x}
\\
}
\uncover<+->{
&=
\int_\Omega M \Delta \mu \di{x}
\\
}
\uncover<+->{
&=
\int_{\partial\Omega} \underbrace{M \nabla \mu}_\flux \cdot n \di{s}
}
\end{align*}
\end{frame}

\begin{frame}{Cahn-Hilliard-Gleichung}
\begin{alignat*}{2}
\pderiv{c}{t}
&=
\nabla \cdot (M \nabla \mu)
,\quad&
x &\in \Omega
\\
\mu
&=
\deriv{F}{c} -  \epsilon^2 \Delta c
,&
x &\in \Omega
\\
\nabla c \cdot n
&=
0
,&
x &\in \partial\Omega
\\
M \nabla \mu \cdot n
&=
0
,&
x &\in \partial\Omega
\end{alignat*}
\end{frame}

%
% spinodal.tex -- Spinodale Entmischung
%
% (c) 2024 Patrik Müller, Hochschule Rapperswil
%
% !TeX root = ../../buch.tex
% !TeX encoding = UTF-8
% !TeX spellcheck = de_CH
%

\section{Spinodale Entmischung\label{cahnhilliard:section:spinodal}}
\rhead{Spinodale Entmischung}

Nachdem wir die Cahn-Hilliard-Gleichung hergeleitet haben,
möchten wir nun überprüfen,
ob \eqref{cahnhilliard:cheq} tatsächlich das Verhalten von Essig und Öl widerspiegelt.
Zu diesem Zweck wurde eine Simulation mit der Finite-Elemente-Methode (FEM) durchgeführt.
In der Simulation wurden die folgenden Parameter,
Funktionen und Anfangsbedingungen verwendet:
\begin{align*}
\begin{aligned}
M
&=
1,
&
\epsilon
&=
0.01,
&
F(c)
&=
100 c^2 (c - 1)^2,
&
c(x,0)
&=
\frac{2}{3} + 0.01 X
,\; \text{wobei }
X
\sim
\mathcal{U}(-1,1)
\end{aligned}
\end{align*}
Dabei stellt $\mathcal{U}(-1,1)$ eine Gleichverteilung von $-1$ bis $1$ dar.
Für $c(x,0)$ wurde absichtlich das aus Kochbüchern typische Verhältnis von
2 Teile Öl zu 1 Teil Essig gewählt.
Wir nehmen außerdem an, dass die Sauce initial sehr gut gemischt ist.

In Abbildung \ref{cahnhilliard:fig:chsim} sind die Resultate der Simulation
zu verschiedenen Zeitpunkten dargestellt.
Man kann sehen,
wie sich das homogene Gemisch in seine einzelnen Komponenten zerlegt
und dabei eine Konfiguration
mit möglichst kleiner Grenzoberfläche zwischen den Phasen bildet.
Diese Simulation zeigt anschaulich die spinodale Entmischung,
bei der kleine Fluktuationen in der Konzentration zu einer Trennung der Phasen führen.

\begin{figure}
\centering
\foreach \n [count=\xi] \i in {0,5,10,15,25,50,80,130,300,1500,70000}{
\subfigure[$t = \i\,\tau$]{
\includegraphics[width=0.3\textwidth]{papers/cahnhilliard/presentation/images/ch_sim/\i.pdf}
}}
\subfigure[Farbskala]{\includegraphics[width=0.3\textwidth]{papers/cahnhilliard/presentation/images/colorbar.book.pdf}}
\caption[Simulation der Cahn-Hilliard-Gleichung]{%
Simulation der Cahn-Hilliard-Gleichung über einen langen Zeitraum.}
\label{cahnhilliard:fig:chsim}
\end{figure}

\subsection{Ursache für die Entmischung}

Die Phasentrennung in einem Gemisch wie Essig und Öl
ist ein faszinierendes physikalisches Phänomen,
das auf die grundlegenden Prinzipien der Thermodynamik zurückgeht.
Um zu verstehen,
warum sich zwei Flüssigkeiten wie Essig und Öl nicht mischen,
müssen wir die Konzepte der freien Energie und
der chemischen Potenziale näher betrachten.

In einem binären Gemisch wird die Gesamtenergie des Systems
durch die freie Energie $F(c)$ bestimmt,
die sowohl die inneren Wechselwirkungen der Moleküle
als auch die Entropie des Systems berücksichtigt.
Diese Funktion reflektiert die energetischen Kosten der Mischung der beiden Komponenten.
Die freie Energie für ein solches Gemisch kann laut \cite{cahnhilliard:deriv}
folgendermaßen ausgedrückt werden:
\begin{align*}
F(c)
&=
\omega c (1 - c) + R T \left[ (1-c) \log(1-c) + c \log(c) \right]
\end{align*}
Hierbei ist $\omega$ ein Parameter,
der die Wechselwirkungsenergie zwischen den Komponenten beschreibt,
$R$ die universelle Gaskonstante und
$T$ die Temperatur.

Ein wesentlicher Aspekt ist die unterschiedliche chemische Affinität
der Moleküle zueinander.
Essig,
der hauptsächlich aus Wasser besteht,
und Öl haben sehr unterschiedliche molekulare Strukturen und Wechselwirkungen.
Wasser ist ein polares Molekül und bildet starke Wasserstoffbrückenbindungen,
während Öl aus unpolaren Kohlenwasserstoffmolekülen besteht,
die keine solchen Bindungen eingehen.
Diese unterschiedlichen Wechselwirkungen führen dazu,
dass die Moleküle von Essig und Öl es vorziehen,
sich jeweils mit ihresgleichen zu umgeben,
anstatt sich miteinander zu vermischen.

\subsubsection{Kritische Temperatur}
Wenn wir die freie Energie $F(c)$ eines homogenen Gemisches analysieren,
stellen wir fest,
dass bei bestimmten Temperaturen die freie Energie eine Form annimmt,
die eine Mischung der Komponenten energetisch ungünstig macht.
Insbesondere bei Temperaturen unterhalb einer kritischen Temperatur $T_\text{krit}$
zeigt die freie Energie eine konkave Form im Bereich der mittleren Konzentrationen.
In Abbildung \ref{cahnhilliard:fig:fc}
ist diese Temperaturabhängigkeit von $F(c)$ dargestellt.
Dies bedeutet,
dass das System energetisch bevorzugt ist,
sich in zwei Phasen mit unterschiedlichen Konzentrationen zu trennen,
um die Gesamtenergie zu minimieren.
Die kritische Temperatur kann durch die Beziehung
\begin{align*}
T_\text{krit}
&=
\frac{\omega}{2 R}
\end{align*}
bestimmt werden.

\begin{figure}
\centering
\includegraphics[scale=0.8]{papers/cahnhilliard/presentation/images/energy.book.pdf}
\caption{Temperaturabhängigkeit von $F(c)$}
\label{cahnhilliard:fig:fc}
\end{figure}

Diese Trennung der Phasen wird durch kleine Fluktuationen in der Konzentration initiiert,
die durch thermische Bewegungen der Moleküle verursacht werden.
Sobald diese Fluktuationen auftreten,
führt die minimierte freie Energie dazu,
dass sich die Phasen weiter trennen und stabile Bereiche hoher
und niedriger Konzentration bilden.
Dieser Prozess wird als spinodale Entmischung bezeichnet
und beschreibt die spontane Bildung von Phasen,
die in der Cahn-Hilliard-Gleichung \eqref{cahnhilliard:cheq} modelliert wird.

\subsection{Gegenmassnahmen}
Nachdem wir nun die Ursachen der Phasentrennung verstanden haben,
stellt sich die Frage,
wie man diese Trennung verhindern oder verzögern kann.
Hier sind einige mögliche Gegenmaßnahmen:
\begin{enumerate}
\item \emph{Erhöhung der Temperatur:}
Da die Neigung zur Phasentrennung bei höheren Temperaturen abnimmt,
kann eine Erhöhung der Temperatur helfen,
die Mischung stabiler zu halten.
Allerdings ist dies für kalt servierte Salate eher ungeeignet.
\item \emph{Verwendung von Stabilisatoren:}
Stabilisatoren wie Xanthan oder Agar-Agar können der Mischung hinzugefügt werden,
um die Viskosität zu erhöhen und die Bewegung der Tröpfchen zu verlangsamen,
was die Phasentrennung erschwert.
\item \emph{Emulgatoren hinzufügen:}
Emulgatoren sind Moleküle,
die sowohl hydrophile (wasserliebende)
als auch lipophile (fettliebende) Eigenschaften besitzen.
Sie können sich an die Grenzfläche zwischen Essig und Öl anlagern
und die Oberflächenspannung reduzieren.
Dadurch wird die Bildung kleiner,
stabiler Tröpfchen erleichtert,
was die Entmischung verhindert.
Beispiele für Emulgatoren sind Lecithin (in Eigelb enthalten) oder Senf.
\item \emph{Intensives Rühren:}
Mechanisches Rühren oder Schütteln kann die Mischung homogen halten,
indem es die Bildung von Phasengrenzen stört und die Tröpfchen klein hält.
Je intensiver und länger gerührt wird,
desto feiner und stabiler wird die Emulsion.
Allerdings muss dann die Mischung ständig in Bewegung sein,
ansonsten beginnt sofort die spinodale Entmischung.
\end{enumerate}
Zum letzten Punkt möchten wir im folgenden Abschnitt noch zeigen,
dass die Cahn-Hilliard-Gleichung erweitert werden können,
so dass der Einfluss von Rührbewegungen auf die Mischung simuliert werden können.

\subsubsection{Rühren der Mischung}
Da die Cahn-Hilliard-Gleichung ursprünglich für feste Lösungen
und Legierungen entwickelt wurde,
müssen wir sie erweitern,
um auch den Einfluss des mechanischen Rührens in flüssigen Gemischen zu berücksichtigen.
Bis jetzt ist unser Fluss $\flux$ nur durch die Diffusion bestimmt,
die durch den Konzentrationsgradienten angetrieben wird.
Das mechanische Rühren erzeugt jedoch eine zusätzliche Flusskomponente,
die ebenfalls berücksichtigt werden muss.
Gemäss \cite{cahnhilliard:deriv-advective} ergibt sich dann
\begin{align}
\begin{aligned}
\pderiv{c}{t} + v \cdot \nabla c
&=
\nabla \cdot (M \nabla \mu)
\\
\mu
&=
\deriv{F}{c} -  \epsilon^2 \Delta c
\\
\nabla \cdot v
&=
0
.
\end{aligned}
\label{cahnhilliard:acheq}
\end{align}
Dabei ist $v$ das durch Rühren erzeugte Geschwindigkeitsfeld.
Nun müssen wir ein geeignetes Geschwindigkeitsfeld finden.
Gemäss \eqref{cahnhilliard:acheq} sollte dieses divergenzfrei sein.
Eine einfache Möglichkeit besteht darin,
die jeweiligen Komponenten des Geschwindigkeitsfeldes
unabhängig von der assoziierten räumlichen Komponente zu machen,
also
\begin{align*}
\begin{aligned}
v_x(x,y,t)
&=
v_x(y,t),
&
v_y(x,y,t)
&=
v_y(x,t)
.
\end{aligned}
\end{align*}
Eine einfache Lösung ist das folgende Geschwindigkeitsfeld:
\begin{alignat*}{2}
v_x(x, y, t)
&=
\alpha \sin(y + \phi_n)
,\quad&
& n \tau \leq t < (n+1) \tau
\\
v_y(x, y, t)
&=
\alpha \sin(x + \psi_n)
,&
& n \tau \leq t < (n+1) \tau
,
\end{alignat*}
wobei die Phasen $\phi_n$ und $\psi_n$ jede Periode zufällig gewählt werden.
Dieses Geschwindigkeitsfeld erzeugt kreisförmige Strömungslinien,
was Rührbewegungen gut abbildet.

\subsubsection{Simulation verschiedener Rührstärken}
Um das Verhalten der Mischung unter unterschiedlichen Bedingungen weiter zu untersuchen,
werden wir \eqref{cahnhilliard:acheq} nun für verschiedene Rührstärken simulieren.
Dazu verwenden wir erneut FEM als Simulationsmethode.
In diesen Simulationen variieren wir die Stärke des Rührens,
repräsentiert durch den Parameter $\alpha$ in unserem Geschwindigkeitsfeld.
So können wir den Einfluss verschiedener Rührintensitäten auf die Phasentrennung
und die Stabilität der Emulsion untersuchen.

In Abbildung~\ref{cahnhilliard:fig:achsim}
sind die Resultate der Simulation mit verschiedenen Rührstärken dargestellt.
Dabei wurden in der Simulation 1000 Zeitschritte berechnet.
In Abbildung~\ref{cahnhilliard:subfig:initial}
ist zudem die Anfangsbedingung $c(x,0)$ ersichtlich.
Diese Ergebnisse verdeutlichen,
wie wichtig die Rührintensität für die Homogenität der Mischung ist.
Eine unzureichende Rührstärke
(wie z.B. in Abbildung~\ref{cahnhilliard:subfig:vweak}
und Abbildung~\ref{cahnhilliard:subfig:weak})
führt dazu,
dass sich die Phasen nur minimal vermischen
und die gewünschte Emulsion nicht erreicht wird.
Bei mittlerer Rührintensität,
wie in Abbildung~\ref{cahnhilliard:subfig:nearly}
beginnt die Mischung sich zu homogenisieren,
aber es sind noch separate Phasen erkennbar.
Erst bei ausreichender Rührstärke wird eine nahezu perfekte Durchmischung erreicht,
die eine stabile Emulsion bildet.
Diese ist in Abbildung~\ref{cahnhilliard:subfig:strong} dargestellt.

Diese Simulationen zeigen deutlich,
dass die richtige Wahl der Rührintensität entscheidend ist,
um eine homogene und stabile Salatsauce herzustellen.
Dies bestätigt,
dass neben den chemischen Eigenschaften der Zutaten auch die mechanischen Einflüsse,
wie das Rühren, eine wesentliche Rolle bei der Herstellung von Emulsionen spielen.

\begin{figure}
\centering
\subfigure[Anfangsbedingung]{
\includegraphics[width=0.3\textwidth]{%
papers/cahnhilliard/presentation/images/ach_sim/initial.pdf}
\label{cahnhilliard:subfig:initial}}
%
\subfigure[$\alpha$ sehr klein]{
\includegraphics[width=0.3\textwidth]{%
papers/cahnhilliard/presentation/images/ach_sim/very_weak.pdf}
\label{cahnhilliard:subfig:vweak}}
%
\subfigure[$\alpha$ klein]{
\includegraphics[width=0.3\textwidth]{%
papers/cahnhilliard/presentation/images/ach_sim/weak.pdf}
\label{cahnhilliard:subfig:weak}}
%
\subfigure[$\alpha$ moderat]{
\includegraphics[width=0.3\textwidth]{%
papers/cahnhilliard/presentation/images/ach_sim/nearly.pdf}
\label{cahnhilliard:subfig:nearly}}
%
\subfigure[$\alpha$ gross]{
\includegraphics[width=0.3\textwidth]{%
papers/cahnhilliard/presentation/images/ach_sim/strong.pdf}
\label{cahnhilliard:subfig:strong}}
%
\subfigure[Farbskala]{\includegraphics[width=0.3\textwidth]{papers/cahnhilliard/presentation/images/colorbar.book.pdf}}
\caption[Simulation der angepassten Cahn-Hilliard-Gleichung]{%
Anfangsbedingung $c(x,0)$ und
Simulationsresultate von \eqref{cahnhilliard:acheq} nach $1000\,\tau$
mit unterschiedlichen Rührstärken $\alpha$.}
\label{cahnhilliard:fig:achsim}
\end{figure}

% !TeX spellcheck = de_CH
% !TeX encoding = UTF-8
% !TeX root = ../presentation.tex

\section{Gegenmassnahmen}

\begin{frame}{Was genau ist $F(c)$?}
\begin{itemize}
\item Ergibt sich aus thermodynamischen,
sowie chemischen Eigenschaften
\item Ändert sich \glqq sprunghaft\grqq{} um eine Temperatur $T_\text{krit}$
\end{itemize}
\begin{figure}
\centering
\includegraphics[scale=0.8]{images/energy}
\caption{Temperaturabhängigkeit von $F(c)$}
\label{fig:fc}
\end{figure}
\end{frame}

\begin{frame}{Gegenmassnahmen}
\begin{enumerate}
\item<+-> Erhitzen über kritische Temperatur $T_\text{krit}$
\uncover<+->{$\rightarrow$ eher ungeeignet für Salatsaucen}
\item<+-> Verwenden eines Emulgators (z. B. Senf, Eier, Honig, \ldots)
\item<+-> Rühren
\end{enumerate}
\end{frame}

\begin{frame}{Rühren}
\begin{itemize}
\item Hinzufügen eines extern angeregets inkrompessibles Strömungsfeld $v(x, t)$. (Koppeln mit Navier-Stokes-Gleichungen)
\begin{align*}
\pderiv{c}{t} + v \cdot \nabla c
&=
\nabla \cdot (M \nabla \mu)
\\
\mu
&=
\deriv{F}{c} -  \epsilon^2 \Delta c
\\
\nabla \cdot v
&=
0
\end{align*}
\item Hinzufügen von periodischen Randbedingungen
\begin{alignat*}{2}
v_x(x, y, t)
&=
\alpha \sin(y + \phi_n)
,\quad&
& n \tau \leq t < (n+1) \tau
\\
v_y(x, y, t)
&=
\alpha \sin(x + \psi_n)
,&
& n \tau \leq t < (n+1) \tau
\end{alignat*}
\end{itemize}
\end{frame}

\begin{frame}{}
\begin{figure}
\centering
\foreach \i in {0.1,0.3,0.5,0.7,1.0}{
\begin{subfigure}{0.19\textwidth}
\centering
\includegraphics[width=\textwidth]{images/\i.png}
\caption{$\alpha = \i$}
\end{subfigure}
}
\begin{subfigure}{0.19\textwidth}
\centering
\includegraphics[width=0.2\textwidth]{images/cb.png}
\caption{Farbskala}
\end{subfigure}
\caption{Resultate nach langem Rühren für unterschiedliche Amplituden $\alpha$ {\color{mainColor}[1]}}
\end{figure}
\end{frame}


\appendix
\begin{frame}
\centering
\Large
$ $

\textbf{Vielen Dank für die Aufmerksamkeit}

\textbf{und ich wünsche viel Spass beim Salatsauce machen}
\end{frame}

\end{document}
