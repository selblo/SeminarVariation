%
% resultat.tex 
%
% 
%
% !TEX root = ../../buch.tex
% !TEX encoding = UTF-8
%

\section{Resultat\label{antennen:resultat}}

Das Verfahren nach Ritz hat ergeben, dass die Antenne eine kreisförmige Abrundung aufweist. 
Es muss noch bestimmt werden, wie gross der Radius dieser Abrundungen ist. 

\subsection{Parametrisierung Dreieck\label{antennen:param3eck}}
Hierfür kann wiederum das Verhältnis REF verwendet werden. Die Länge l, hierbei der Umfang 
des abgerundeten Dreiecks, sowie dessen Fläche A kann mit den Formeln
\begin{equation}
	l=2\cdot{\pi}\cdot{r}+3\cdot{s}-6\cdot{\sqrt{3}}\cdot{r}
	\label{antennen:Länge}
\end{equation}
\begin{equation}
	A=r^2\cdot{\pi}+3\cdot{r}\cdot{(s-2\cdot{\sqrt{3}}\cdot{r})}+\frac{\sqrt{3}\cdot{(s-2\cdot{\sqrt{3}}\cdot{r})^2}}{4}
	\label{antennen:Fläche}
\end{equation}
berechnet werden.
Der Wirkungsgrad ist nun zu einem Problem geworden, welches nur noch abhängig von 
der Seitenlänge s des Dreiecks und des Radius r der Kreise ist. Bildlich ist dies 
in Abbildung \ref{antennen:tikzdreieckAufteilung} veranschaulicht.

%TODO Erklären der Formeln l und A mittels Grafik (Formelabschnitte einfärben??)
\begin{figure}[htpb]
	\centering
	\begin{tikzpicture}
		% Define the length of the sides of the triangles
		\def\sidelength{3.14}
		
		% Calculate the height of the equilateral triangle
		\pgfmathsetmacro{\triangleheight}{sqrt(3)/2*\sidelength}
		% Draw the large outer triangle (white background)
		\draw[fill=white] (0,0) -- (\sidelength,0) -- (0.5*\sidelength, \triangleheight) -- cycle;
		\coordinate (A) at (0,0);
		\coordinate (B) at (2.51,0);
		\coordinate (C) at (2.51/2,1.73/2*2.51);
		% Verschiebung der Koordinaten um 10pt nach rechts und 10pt nach oben
		\coordinate (As) at ($(A) + (9pt,5pt)$);
		\coordinate (Bs) at ($(B) + (9pt,5pt)$);
		\coordinate (Cs) at ($(C) + (9pt,5pt)$);
		% Nodes with blue border and green fill
		\node[circle, inner sep=0pt, minimum size=9pt, fill=green, draw=blue, line width=1pt] at (As) {};
		\node[circle, inner sep=0pt, minimum size=9pt, fill=green, draw=blue, line width=1pt] at (Bs) {};
		\node[circle, inner sep=0pt, minimum size=9pt, fill=green, draw=blue, line width=1pt] at (Cs) {};
		
		% Black lines with yellow outer lines
		\draw[line width=10pt, yellow] (As) -- (Bs) (Bs) -- (Cs) (Cs) -- (As);
		\draw[line width=8pt, black] (As) -- (Bs) (Bs) -- (Cs) (Cs) -- (As);
		
		% Red filled triangle
		\path[draw=black, fill=red] (As) -- (Bs) -- (Cs) -- (As);
		% Draw axes
		\draw[->] (0,0) -- (4,0) node[below right] {};
		\draw[->] (0,0) -- (0,4) node[left] {};
		% Add ticks and labels on axes
		\foreach \x in {0, 1,2,3}
		\draw (\x,1pt) -- (\x,-1pt) node[below] {\x};
		\foreach \y in {1, 2, 3}
		\draw (1pt,\y) -- (-1pt,\y) node[left] {\y};
	\end{tikzpicture}
	\caption{Aufteilung des Dreiecks}
	\label{antennen:tikzdreieckAufteilung}
\end{figure}

Durch ableiten und null-setzten
\begin{equation}
	\frac{d}{dr} \bigg(\frac{l}{A^2}\bigg)=0
	\label{antennen:Ableitung}
\end{equation}
wird für einen gegebenen Parameter s ein Radius r als Lösung erhalten. 
Die Ableitung ergibt die Lösung 
\begin{equation}
	\frac{\left(- 4 \pi r + 12 \sqrt{3} r\right) \left(- 6 \sqrt{3} r + 2 \pi r + 3 s\right)}{\left(\pi r^{2} + 3 r \left(- 2 \sqrt{3} r + s\right) + \frac{\sqrt{3} \left(- 2 \sqrt{3} r + s\right)^{2}}{4}\right)^{3}} + \frac{- 6 \sqrt{3} + 2 \pi}{\left(\pi r^{2} + 3 r \left(- 2 \sqrt{3} r + s\right) + \frac{\sqrt{3} \left(- 2 \sqrt{3} r + s\right)^{2}}{4}\right)^{2}}=0
	\label{antennen:Ableitunggelöst}
\end{equation}
in welcher s mit der gewünschten Seitenlänge parametrisiert werden kann. Nach Lösen des Gleichungssystems resultieren nun die möglichen Werte für den Radius r.

