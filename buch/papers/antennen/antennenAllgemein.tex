%
% antennenAllgemein.tex 
%
% 
%
% !TEX root = ../../buch.tex
% !TEX encoding = UTF-8
%

\section{Antennen\label{antennen:antennenAllgemein}}
\rhead{Antennen}

Antennen sind in der heutigen technologisch geprägten Welt nicht mehr weg zu denken. Sie befinden sich in vielen alltäglichen elektronischen Geräten. Eine Antenne wandelt die in einer Leitung geführte gebundene elektromagnetische Welle in eine freie Welle im Raum um. Da eine Antenne freie Wellen ebenso in gebundene leitungsgeführte Wellen umwandeln kann hat sie reziproke Eigenschaften.

Eine genauere Erklärung von elektromagnetischen Wechselwirkungen kann sich im \href{chapter:maxwell}{Kapitel Maxwell}, welches sich mit den Maxwell Gleichungen befasst, angeeignet werden. 
\subsection{Loop-Antennen\label{antennen:antennenAllgemein_loop}}
\rhead{Loop-Antennen}
Eine Loop-Antenne hat eine simple Funktionsweise. Sie besteht aus einem geschlossenen Stück Draht wie in Abbildung... gesehen werden kann. Die Loop-Antenne kann, nicht wie der Name impliziert, verschiedenste Formen annehmen. Durch den Leiter wird ein Signal geführt, welches in der aufgespannten Fläche ein magnetisches Feld induziert. Dieses Feld wird durch Anpassung der Signalamplitude verändert. Eine solche Änderung kann als Information angesehen werden, welche sich nun im Äter, also dem freien Raum, fortbewegt.

\subsection{Eigenschaften\label{antennen:antennenEigenschaften}}
\rhead{Eigenschaften}
Der Wirkungsgrad
\begin{equation}
	\eta=\frac{P\textsubscript{rad}}{P\textsubscript{tot}}
	\label{antennen:Wirkungsgrad}
\end{equation}
, auch Effizienz genannt, kann mittels der Gesamtleistung der Antenne P\textsubscript{tot} und der abgestrahlten Leistung P\textsubscript{rad} berechnet werden. Eine Leistung P entspricht dem elektrischen Widerstand multipliziert mit der Stromstärke im Quadrat. Nach Vereinfachung ergibt sich, dass
\begin{equation}
	\eta=\frac{P\textsubscript{rad}}{P\textsubscript{tot}}=\frac{R\textsubscript{rad}\cdot{I^2}}{R\textsubscript{tot}\cdot{I^2}}=\frac{R\textsubscript{rad}\cdot{I^2}}{(R\textsubscript{rad}+R\textsubscript{loss})\cdot{I^2}}=\frac{R\textsubscript{rad}}{R\textsubscript{rad}+R\textsubscript{loss}}
	\label{antennen:Wirkungsgradkomplett}
\end{equation}
die Effizienz \texteta\ nur noch von vom Verlustwiderstand und dem Strahlungswiderstand abhängig ist. R\textsubscript{rad} und R\textsubscript{loss} werden wie folgt definiert:
\begin{equation}
	R\textsubscript{rad}=31171\Omega\cdot{\bigg(\frac{A}{\lambda^2}\bigg)^2}
	\label{antennen:Rrad}
\end{equation}
\section{Anfang\label{antennen:antennenAllgemein2}}
\rhead{Anfang}

hallo