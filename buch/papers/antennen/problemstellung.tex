%
% problemstellung.tex 
%
% 
%
% !TEX root = ../../buch.tex
% !TEX encoding = UTF-8
%

\section{Problem\label{antennen:problemstellung}}
\rhead{Problem}
 Es wird versucht eine Antenne zu designen, welche in einem Gerät mit der Form eines Prismas mit Deckfläche eines gleichseitigen Dreiecks. Dies bedeutet, dass die Antenne die Maas eines gleichseitigen Dreiecks nicht überschreiten darf.
\subsection{Geometrie\label{antennen:Geom}}
\rhead{Geometrie}
Das Ziel ist nun, den Wirkungsgrad durch Formoptimierung zu erhöhen. Die Faktoren k\textsubscript{1} und k\textsubscript{2} sind Konstanten, was bedeutet, dass für eine Erhöhung des Wirkungsgrads (Formel referenzieren oder neu Einfügen) nur die Länge l und die Fläche A veränderbar sind. In einem nächsten Schritt wird das Verhältnis
\begin{equation}
	\frac{l}{A^2}
	\label{antennen:Verhältnis}
\end{equation}
% TODO von l/A^2 zu l/A, (beste Fläche auch gleich beste Fläche im Quadrat)
% TODO Diese Formel... danach Besprechung nötig
erhöht. Diese Formel muss nun auf eine implizite Funktion angewendet werden. Eine erste Idee besteht darin, die Ecken abzuflachen, da in den Ecken mit viel zusätzlicher Länge nur wenig Fläche gewonnen wird. Zuerst muss eine Funktion f(x,y) gefunden werden, welche die Ecken, wie in Abbildung... zu sehen, möglichst effizient abflacht. Danach muss bestimmt werden, in welcher Grössenordnung diese Funktion auf das Dreieck "angewendet" wird. Dies wird in Abbildung... veranschaulicht.
Eine Eigenschaft welche das Problem vereinfacht ist die Symmetrie des Dreiecks. Das Gesamtproblem kann vereinfacht werden indem nur eine Ecke betrachtet wird. Wie in Abbildung... dargestellt sind die drei Eckfunktionen Äquivalent. Die drei gestrichelten Geraden spannen ein weiteres gleichseitiges Dreieck auf, welches jedoch vernachlässigt werden kann, da dieses für alle gesuchten Funktionen f(x,y) gleich bleibt. Somit ist das implizite Problem f(x,y) zur expliziten Funktion f(x) geworden, welche wesentlich leichter zu optimieren ist.