%
% teil1.tex -- Beispiel-File für das Paper
%
% (c) 2020 Prof Dr Andreas Müller, Hochschule Rapperswil
%
% !TEX root = ../../buch.tex
% !TEX encoding = UTF-8
%
\section{Kirchhoff's Gesetz
\label{circuit:section:teil1}}
\rhead{Problemstellung}
Kirchhoffs Stromgesetz besagt, dass die in einen Knoten einfließenden Ströme gleich den aus einem Knoten ausfließenden Strömen in einer Schaltung sind. Dies bedeutet auch, dass es in einem Gleichstromkreis im stationären Betrieb keine Ansammlung von Ladungen an irgendeinem Punkt geben kann. Wir betrachten nun einen dreidimensionalen Schaltkreis, in dem die Leitfähigkeit $\sigma$ im gesamten Bereich von Interesse konstant ist. Die Verallgemeinerung von Kirchhoffs Stromgesetz im dreidimensionalen Fall besagt, dass die Divergenz der Stromdichte $\vec{J}$ gleich Null ist. Daher haben wir
\begin{equation}
	\nabla  \vec{J}=0
	\label{circuit:current_density_1}
\end{equation}




Wenn das elektrische Feld durch $\vec{E}$ dargestellt wird, können wir die Stromdichte als Produkt aus Leitfähigkeit und elektrischem Feld an einem Punkt im Raum schreiben. Daher haben wir
$$
\vec{J}=\sigma \vec{E}
$$

Das elektrische Feld kann als negativer Gradient der skalaren Potentialfunktion $\phi$ geschrieben werden, d.h.
$$
\vec{E}=-\nabla \phi
$$

Daher erhalten wir aus den Gleichungen (6), (7) und (8)
$$
\nabla .(\sigma \nabla \phi)=0
$$

Wenn wir annehmen, dass die Leitfähigkeit im gesamten betrachteten Raum konstant ist, erhalten wir
$$
\nabla^2 \phi=0
$$

Der stationäre Zustand in einem Gleichstromkreis wird erreicht, wenn Gleichung (10) an allen Punkten im Schaltkreis erfüllt ist. Als nächstes ist die Wärmeerzeugungsrate $P$ in einem von einer Grenze $S$ umgebenen Volumen $V$ gegeben durch
$$
P=\int_V \sigma(\nabla \phi)^2 d V
$$

Wenn wir den Divergenzsatz anwenden und Gleichung (10) verwenden, erhalten wir
$$
P=\int_V \nabla \cdot(\sigma \phi \nabla \phi) d V=\int_S(\sigma \phi \nabla \phi) \cdot d \vec{S}
$$

Wenn die Leitfähigkeit im gesamten Volumen $V$ konstant ist, wird die Variation der Wärmeerzeugungsrate $P$ in $V$ gegeben durch
$$
\delta P=\sigma \int_S(\delta \phi \nabla \phi) \cdot d \vec{S}+\sigma \int_S(\phi \nabla(\delta \phi)) \cdot d \vec{S}
$$

Aus Gleichung (13) geht hervor, dass wenn Variationen im Fluss, $\delta \phi$, an der Grenze von $V$ verschwinden, dann ist $P$ stationär, d.h. $\delta P=0$. Daher werden elektrische Ströme in der Region verteilt, mit einer angelegten Spannung an ihrer Grenze,
\begin{equation}
\int_a^b x^2\, dx
=
\left[ \frac13 x^3 \right]_a^b
=
\frac{b^3-a^3}3.
\label{circuit:equation1}
\end{equation}





\begin{align*}
	\Delta h(x)=0
\end{align*}



Neque porro quisquam est, qui dolorem ipsum quia dolor sit amet,
consectetur, adipisci velit, sed quia non numquam eius modi tempora
incidunt ut labore et dolore magnam aliquam quaerat voluptatem.

Ut enim ad minima veniam, quis nostrum exercitationem ullam corporis
suscipit laboriosam, nisi ut aliquid ex ea commodi consequatur?
Quis autem vel eum iure reprehenderit qui in ea voluptate velit
esse quam nihil molestiae consequatur, vel illum qui dolorem eum
fugiat quo voluptas nulla pariatur?

\subsection{De finibus bonorum et malorum
\label{circuit:subsection:finibus}}
At vero eos et accusamus et iusto odio dignissimos ducimus qui
blanditiis praesentium voluptatum deleniti atque corrupti quos
dolores et quas molestias excepturi sint occaecati cupiditate non
provident, similique sunt in culpa qui officia deserunt mollitia
animi, id est laborum et dolorum fuga \eqref{circuit:equation1}.

Et harum quidem rerum facilis est et expedita distinctio
\ref{circuit:section:teil2}.
Nam libero tempore, cum soluta nobis est eligendi optio cumque nihil
impedit quo minus id quod maxime placeat facere possimus, omnis
voluptas assumenda est, omnis dolor repellendus
\ref{circuit:section:teil3}.
Temporibus autem quibusdam et aut officiis debitis aut rerum
necessitatibus saepe eveniet ut et voluptates repudiandae sint et
molestiae non recusandae.
Itaque earum rerum hic tenetur a sapiente delectus, ut aut reiciendis
voluptatibus maiores alias consequatur aut perferendis doloribus
asperiores repellat.


