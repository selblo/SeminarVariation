%
% linearer_ansatz.tex -- Formfunktionen linearer Ansatz
%
% (c) 2024 Flurin Brechbühler
%
\documentclass[tikz]{standalone}
\usepackage{amsmath}
\usepackage{times}
\usepackage{txfonts}
\usepackage{pgfplots}
\usepackage{csvsimple}

\usetikzlibrary{arrows,intersections,math}
\definecolor{darkred}{rgb}{0.8,0,0}
\definecolor{darkpurple}{rgb}{0.6,0,0.6}
\definecolor{darkblue}{rgb}{0,0,0.8}
\definecolor{darkgreen}{rgb}{0,0.6,0}

\begin{document}
\def\skala{1}

\begin{tikzpicture}[>=latex,thick,scale=\skala]
\begin{scope}

% Plots
%     % a_{n-1}
% \draw[color=darkred,line width=1.4pt,dashed] plot[domain=-1.025:-1, scale=4, smooth]
% ({\x},{\x+2});
% \draw[color=darkred,line width=1.4pt,dashed] plot[domain=-1:0, scale=4, smooth]
% ({\x},{-\x});
% \draw[color=darkred,line width=1.4pt,dashed] plot[domain=0:1.075, scale=4, smooth]
% ({\x},{0});
    % a_n
\draw[color=darkred,line width=1.4pt] plot[domain=-1.025:-1, scale=4, smooth]
({\x},{0});
\draw[color=darkred,line width=1.4pt] plot[domain=-1:0, scale=4, smooth]
({\x},{\x+1});
\draw[color=darkred,line width=1.4pt] plot[domain=0:1, scale=4, smooth]
({\x},{1-\x});
\draw[color=darkred,line width=1.4pt] plot[domain=1:1.075, scale=4, smooth]
({\x},{0});
%     % a_{n+1}
% \draw[color=darkblue,line width=1.4pt,dashed] plot[domain=-1.025:0, scale=4, smooth]
% ({\x},{0});
% \draw[color=darkblue,line width=1.4pt,dashed] plot[domain=0:1, scale=4, smooth]
% ({\x},{\x});
% \draw[color=darkblue,line width=1.4pt,dashed] plot[domain=1:1.075, scale=4, smooth]
% ({\x},{2-\x});

% x-Achse
\draw[->] (-4.1,0) -- (4.3,0) coordinate[label={$x$}];
\draw (4,-0.1) -- (4,0.1);
\draw (-4,-0.1) -- (-4,0.1);
\node at (-4,0) [below] {$-\Delta x$};
\node at ( 0,0) [below] {$0$};
\node at ( 4,0) [below] {$\Delta x$};

% y-Achse
\draw[->] (0,{-0.1}) -- (0,{4.3})
coordinate[label={right:$y$}];
\node at (-0.1,0) [left, inner sep=1pt, fill=white] {$0$};
\draw (-0.1,4) -- (0.1,4);
\node at (0,4) [left] {$1$};

\end{scope}
\end{tikzpicture}
\end{document}
