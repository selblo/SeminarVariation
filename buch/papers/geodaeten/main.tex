%
% main.tex -- Paper zum Thema <geodaeten>
%
% (c) 2020 Autor, OST Ostschweizer Fachhochschule
%
% !TEX root = ../../buch.tex
% !TEX encoding = UTF-8
%
\chapter{Geodäten\label{chapter:geodaeten}}
\kopflinks{Geodäten}
\begin{refsection}
\chapterauthor{Andrin Kälin, Marco Rouge}

Geodätenlinien beschreiben den kürzesten Weg zwischen zwei Punkten auf einer beliebigen Oberfläche.
Dabei ist sowohl Form als auch Dimension beliebig.
Um diese Linien zu berechnen wird das Standardverfahren für Geodätenlinein verwendet, welches auf dem metrischen Tensor des zu untersuchenden Raumes basiert.
Um den Metrischen Tensor aufstellen zu können, müssen zunächst die Linienelemente bekannt sein.
Die folgenden zwei Kapitel sollen zunächst die Linienelemente (Abschnitt \ref{geodaeten:section:Linienelemente}) und den metrischen Tensor (Abschnitt [\ref{geodaeten:section:MetrischerTensor}]) erklären.
Erst dann kann die Berechnung der Geodätenlinien anhand einiger Beispiele(Abschnitt \ref{geodaeten:section:StandardverfahrenBeispiele}) veranschaulicht werden.

%
% einleitung.tex -- Beispiel-File für die Einleitung
%
% (c) 2020 Prof Dr Andreas Müller, Hochschule Rapperswil
%
% !TEX root = ../../buch.tex
% !TEX encoding = UTF-8
%
\section{Linienelemente\label{geodaeten:section:Linienelemente}}
\rhead{Linienelemente}

Ein Linienelement beschreibt, wie sich der Raum in die entsprechende Dimension verändert.
Es entspricht also der Ableitung einer Richtung nach der Zeit oder in anderen Worten um ein infinitesimales Wegelement.
In einem $n$-dimensionalen Raum entspricht ein Linienelement also einem $n$-dimensionalen Vektor, welcher die Änderung der Kurve in jeder Dimension beschreibt.
Um die Weglänge unsere Geodätenlinie minimieren zu können, müssen wir erst mal den Weg berechnen.
Jeder Weg kann in kleinere Wegstücke unterteilt werden, welche addiert wieder den ganzen Weg ergeben.
Da die Linienelemente den infinitesimalen Wegelementen entsprechen, müssen sie entlang jeder Dimension integriert werden.
Der Weg $l$ entspricht also der Summe von Wegstücken also
\begin{equation}
	l = \sum \Delta \text{Weg}
\end{equation}
und für infinitesimal große Wegstücke kann man diese ersetzen durch Linienelemente $ds$  als
\begin{equation}	
	d\text{Weg} = ds .
	\label{geodaeten:equation:Linienelemente:equation1}
\end{equation}
Um den Weg $l$ zu erhalten müssen schließlich diese Linienelemente in allen Dimensionen integrieren mit,
\begin{equation}
	l = 
	\sum^{n} \int_a^b ds .
	\label{geodaeten:equation:Linienelemente:equation2}
\end{equation}

\section{Beispiele zu Linienelementen}
Die Beispiele sind für alle Unterkapitel gleich aufgebaut.
Zuerst wird das zweidimensionales Beispiel des Kartesischen Raumes als einfacher Einstieg behandelt.
Danach wird der Zylinder als einstieg in den dreidimensionalen Raum aufgezeigt.
Zum Schluss wird das Beispiel an einer Kugel untersucht, welches in der Praxis große Anwendung erfährt, aufgrund der Ähnlichkeit zur Erdkugel.

	%
% einleitung.tex -- Beispiel-File für die Einleitung
%
% (c) 2020 Prof Dr Andreas Müller, Hochschule Rapperswil
%
% !TEX root = ../../buch.tex
% !TEX encoding = UTF-8
%
\subsection{Kartesisch\label{geodaeten:section:LinKartesisch}}
\rhead{Linienelemente Beispiele}

Wie in Abbildung [\ref{geodaeten:Linienelemente:figure1}] zu sehen ist kann ein Wegstück auf einer Kurve im zweidimensionalen Kartesischen Raum mit

\begin{equation}
	\Delta s \approx \sqrt{\Delta x^2 + \Delta y^2}
\end{equation}
approximiert werden.
Durch Verkleinerung der Wegstücke bis zum Infinitesimal 

\begin{equation}
	d s = \sqrt{d x^2 + d y^2}
	= \sqrt{\left(\frac{d x}{d t}\right)^2 \cdot d t^2 + \left(\frac{d y}{d t}\right)^2 \cdot d t^2} ,
\end{equation}
Kann das Linienelement aufgestellt werden als

\begin{equation}
 	ds^2 = \left(\dot{x}^2 +\dot{y}^2\right) \cdot dt^2 .
\end{equation}
Als Vektor dargestellt entspricht das Linienelement

\begin{equation}
	\mathbf{d\vec{s}}^2 = \begin{pmatrix} \dot{x}^2 \\ \dot{y}^2 \end{pmatrix} = \begin{pmatrix} 1 \\ 1 \end{pmatrix} \cdot \begin{pmatrix} \dot{x}^2 \\ \dot{y}^2 \end{pmatrix} \cdot dt^2 .
\end{equation}

\begin{figure}
	\centering
	\includegraphics[width=0.7\linewidth]{papers/geodaeten/Abbildungen/Linienelemente/LinKartes1}
	\caption{Linienelement im Kartesischen Raum}
	\label{geodaeten:Linienelemente:figure1}
	\cite{geodaeten:kartesisch}
\end{figure}

	%
% einleitung.tex -- Beispiel-File für die Einleitung
%
% (c) 2020 Prof Dr Andreas Müller, Hochschule Rapperswil
%
% !TEX root = ../../buch.tex
% !TEX encoding = UTF-8
%
\subsection{Zylinder\label{geodaeten:section:Linienelemente:Zylinder}}
\rhead{Linienelemente Beispiele}

Eine Kurve auf der Oberfläche eines Zylinders kann als zweidimensional betrachtet werden, wobei gilt

\begin{equation}
	\Delta s \approx \sqrt{(r \cdot \Delta \phi)^2 + \Delta z^2}
\end{equation}
und $r$ ist konstant.
Analog zu den Polarkoordinaten können die abstände infinitesimal werden und das Linienelement ergibt sich dadurch für die Oberfläche des Zylinders zu

\begin{equation}
	ds^2 = \left(r^2 \cdot \dot{\phi}^2 +\dot{z}^2\right) \cdot dt^2 .
	\label{geodaeten:equation:Linienelemente:Zylinder:equation2}
\end{equation}

%Als Vektor dargestellt entspricht das Linienelement
%
%\begin{equation}
%	\mathbf{d\vec{s}}^2 = \begin{pmatrix} r^2 \cdot \dot{\phi}^2 \\ \dot{z}^2 \end{pmatrix} = \begin{pmatrix} r^2 \\ 1 \end{pmatrix} \cdot \begin{pmatrix} \dot{\phi}^2 \\ \dot{z}^2 \end{pmatrix} \cdot dt^2 .
%\end{equation}

Den einstieg in dreidimensionale Kurven können wir machen, indem $r$ nicht als Konstant angenommen wird.
Der Weg kann so mit
\begin{equation}
	\Delta s \approx \sqrt{\Delta r^2 + (r \cdot \Delta \phi)^2 + \Delta z^2} \cdot dt^2
\end{equation}
berechnet werden und das Linienelement entspricht 
\begin{equation}
	ds^2 = \left(\dot{r}^2 + r^2 \cdot \dot{\phi}^2 +\dot{z}^2\right) \cdot dt^2 .
\end{equation}

%und als Vektor
%\begin{equation}
%	\mathbf{d\vec{s}}^2 = \begin{pmatrix} \dot{r}^2 \\ r^2 \cdot \dot{x}^2 \\ \dot{y}^2 \end{pmatrix} = \begin{pmatrix} 1 \\ r^2 \\ 1 \end{pmatrix} \cdot \begin{pmatrix} \dot{r}^2 \\ \dot{\phi}^2 \\ \dot{z}^2 \end{pmatrix} \cdot dt^2 .
%\end{equation}


\begin{figure}
	\centering
	\includegraphics[width=0.7\linewidth]{papers/geodaeten/Abbildungen/Linienelemente/LinZyl1}
	\caption{Linienelement im Kartesischen Raum}
	\label{geodaeten:figure:Linienelemente:Zylinder:figure2}
	\cite{geodaeten:polarkoordinaten}
\end{figure}


	%
% einleitung.tex -- Beispiel-File für die Einleitung
%
% (c) 2020 Prof Dr Andreas Müller, Hochschule Rapperswil
%
% !TEX root = ../../buch.tex
% !TEX encoding = UTF-8
%
\subsection{Kugel\label{geodaeten:section:Linienelement:Kugel}}
In ähnlicher Weise wie im Beispiel mit dem Zylinder lässt sich eine Kugel lokal wie eine flache Ebene darstellen.
Was auf den ersten Blick für sogenannte ``Flat-Earther'' ein schwieriges Konzept zu sein scheint, ist intuitiv verständlich:
\index{Flat-Earther}%
Zoomt man nahe genug an die Erdoberfläche heran, verschwinden die Krümmungen, und Abstände lassen sich nahezu wie in einem euklidischen Raum berechnen.

In Kugelkoordinaten beschreibt $r$ den Abstand eines Punktes vom Zentrum der Kugel (Radius), $\vartheta$ den Winkel zur $z$-Achse (Polarwinkel), und $\varphi$ den Winkel in der $x$-$y$-Ebene (Azimutwinkel).
\index{Radius}%
\index{Polarwinkel}%
\index{Azimutwinkel}%
Diese Parameter definieren die Position eines Punktes auf der Kugeloberfläche eindeutig, wie in der Abbildung \ref{geodaeten:figure:Linienelemente:Kugelkoordinaten:Kugelkoordinaten} dargestellt.

Da wir uns auf der Oberfläche der Kugel befinden, bleibt $r$ konstant und wir betrachten nur die Winkel $\vartheta$ und $\varphi$.
Wenn ein Punkt nun um einen infinitesimalen Abstand auf der Kugeloberfläche verschoben wird, entstehen Verschiebungen entlang der zugehörigen Basisvektoren. 

Die Verschiebung in der Längsrichtung ergibt sich aus
\begin{equation}
	r \, d\vartheta
\end{equation}
und für die Verschiebung in der Breitenrichtung erigbt sich
\begin{equation}
	r \sin\vartheta \, d\varphi.
\end{equation}

Weil lokal wie in einem euklidischen Raum gerechnet werden kann, ergibt sich der Abstand zwischen zwei Punkten auf der Kugeloberfläche durch die Anwendung des Pythagoras auf diese jeweiligen infinetsimalen Verschiebungen. 
Das Linienelement $ds$ für die Oberfläche einer Kugel lässt sich daher beschreiben als
\index{Oberfläche}%
\index{Kugel}%
\index{Linienelement!Kugel}%
\begin{equation}
ds^2 = r^2 \, d\vartheta^2 + r^2 \sin^2\vartheta \, d\varphi^2.
\end{equation}


\begin{figure}
	\centering
	\includegraphics[width=6cm]{papers/geodaeten/Abbildungen/Linienelemente/LinKugel1}
	\caption{Kugelkoordinaten mit Radius $r$, Polarwinkel $\vartheta$ und Azimutwinkel $\varphi$. Bildquelle: \cite{geodaeten:Kugelkoordinaten}}
	\label{geodaeten:figure:Linienelemente:Kugelkoordinaten:Kugelkoordinaten}
\end{figure}



 %
% teil1.tex -- Beispiel-File für das Paper
%
% (c) 2020 Prof Dr Andreas Müller, Hochschule Rapperswil
%
% !TEX root = ../../buch.tex
% !TEX encoding = UTF-8
%
\section{Metrischer Tensor
\label{geodaeten:section:MetrischerTensor}}
\rhead{Der Metrische Tensor}

Im vorherigen Kapitel wurde das Konzept des Linienelements erläutert.
Es wurde aufgezeigt, wie dieses mittels geometrischer Analyse für verschiedene Räume und Koordinatensysteme berechnet werden kann.
Aus der Vektorgeometrie ist uns jedoch ein anderes nützliches Werkzeug bekannt, welches uns erlaubt, Abstände in einem Raum zu berechnen: Das Skalarprodukt.

\subsection{Skalarprodukt im euklidischen Raum}

Im euklidischen Raum ist das Skalarprodukt zwischen zwei Vektoren $\vec{u}$ und $\vec{v}$ definiert als
\begin{equation}
	\vec{u} \cdot \vec{v} = \sum_{i=1}^n u_i v_i,
\end{equation}
wobei $\vec{u} = (u_1, u_2, \ldots, u_n)$ und $\vec{v} = (v_1, v_2, \ldots, v_n)$ die Komponenten der Vektoren in einem $n$-dimensionalen Raum sind.

Die Länge eines Vektors $\vec{u}$ ergibt sich aus dem Skalarprodukt des Vektors mit sich selbst:
\begin{equation}
	\|\vec{u}\| = \sqrt{\vec{u} \cdot \vec{u}} = \sqrt{\sum_{i=1}^n u_i^2}. 
\end{equation}

Betrachten wir nun einen infinitesimal kleinen Vektor in einer Ebene mit kartesischen Koordinaten $(x, y)$, dann ergibt das Skalarprodukt dieses Vektors mit sich selbst
\begin{equation}
	ds^2 = dx \cdot dx + dy \cdot dy = dx^2 + dy^2,
\end{equation}
was die quadratische infinitesimale Länge des Vektors beschreibt und dem bereits bekannten Linienelement entspricht.

Diese Definition des Skalarprodukts ist jedoch nicht allgemeingültig und gilt nur für die Berechnung von Abständen im euklidischen Raum.
Für komplexere Räume mit speziellen Koordinatensystemen wie Zylinder- oder Kugelkoordinaten müssen wir zuerst die Konzepte der Mannigfaltigkeit und des metrischen Tensors verstehen.

\subsection{Mannigfaltigkeit}

Eine Mannigfaltigkeit ist ein mathematisches Konstrukt, das dazu dient, komplizierte geometrische Objekte in eine einfachere, lokal verständliche Form zu bringen.
Im Wesentlichen ist eine Mannigfaltigkeit eine Sammlung von Punkten, die den Raum in lokal euklidische Bereiche unterteilt. 

Ein einfaches Beispiel hierfür ist die Erdoberfläche.
Steht ein ``Flat-Earther" auf der Erdoberfläche, so erscheint ihm die Erde flach, weil der Bereich, den er betrachtet, sehr klein ist im Vergleich zur Gesamtgröße der Erde.
Würde dieser ``Flat-Earther" jedoch weiter zurücktreten und die gesamte Erde betrachten, sollte auch er erkennen können, dass sie in Wirklichkeit eine Kugel ist.

Eine Mannigfaltigkeit ist also ein mathematisches Objekt, das lokal flach oder euklidisch erscheint, aber global eine kompliziertere Struktur haben kann, wie zum Beispiel die Oberfläche einer Kugel.

Eine differenzierbare Mannigfaltigkeit setzt zusätzlich voraus, dass sie in der Umgebung jedes Punktes lokal differenzierbar ist, wodurch die Mannigfaltigkeit eine glatte Struktur erhält.
Da sich die Mannigfaltigkeit lokal wie ein differenzierbarer euklidischer Raum verhält, ermöglicht uns das, für jeden Punkt ein Skalarprodukt zu definieren.
Die Funktion, die das Skalarprodukt für jeden Punkt einer solchen Mannigfaltigkeit definiert, wird als Metrik bezeichnet.

\begin{figure}
	\centering
	\includegraphics[width=1\linewidth]{papers/geodaeten/Abbildungen/MetrischerTensor/Tangentialebene}
	\caption{Tangentialebene eines Punktes der riemannschen Mannigfaltigkeit einer Kugeloberfläche}
	\label{geodaeten:figure:MetrischerTensor:Tangentialebene}
\end{figure}

\subsection{Metrik und metrischer Tensor}

Die Metrik ist das grundlegende Werkzeug, welches uns ermöglicht, Längen, Abstände und Winkel in einem Raum zu messen.
In einfachen geometrischen Räumen, wie dem euklidischen Raum, kennen wir die Metrik bereits als den Satz des Pythagoras.
Wir haben auch schon in Abschnitt \ref{geodaeten:section:Linienelemente:Beispiele} die Metrik in verschiedenen Koordinatensystemen angewendet, um Linienelemente auf unterschiedlichen Oberflächen zu berechnen.

Um die Metrik mathematisch darstellen zu können wird der metrische Tensor $g_{ij}$ eingeführt.
Dieser Tensor ist eine symmetrische $n \times n$-Matrix, wobei $n$ der Dimension des Raumes entspricht, die die Metrik an jedem Punkt der Mannigfaltigkeit kodiert.
Er enthält alle Informationen darüber, wie Abstände und Winkel in einem Raum berechnet werden.
Auf diese Weise beschreibt der metrische Tensor die geometrische Struktur des gesamten Raumes.

Eine differenzierbare Mannigfaltigkeit, die an jedem Punkt durch einen metrischen Tensor definiert ist, wird als Riemannsche Mannigfaltigkeit bezeichnet. Zusätzlich erfordert eine Riemannsche Mannigfaltigkeit, dass der metrische Tensor selbst stetig differenzierbar und ausserdem noch positiv definit ist. 
Dies bedeutet, dass in Riemannschen Mannigfaltigkeiten kein negativer Abstand existieren kann.

Mit dem metrischen Tensor lässt sich eine allgemeine Definition für das Skalarprodukt zweier Vektoren finden, die unabhängig vom Koordinatensystem ist:
\begin{equation}
	\vec{u} \cdot \vec{v} = g_{ij} \, u^i \, v^j ,
\end{equation}
wobei die Notation des metrischen Tensors aus der Tensoralgebra stammt.
Die einsteinsche Summenkonvention, die in der Tensoralgebra verwendet wird, impliziert hier eine Summierung über die Indizes $i$ und $j$, welche die Dimensionen des Raumes durchlaufen.
Die Komponenten der Vektoren $u^i$ und $v^j$ werden dabei durch den metrischen Tensor $g_{ij}$ gewichtet, gemäss der Formel
\begin{equation}
	g_{11} \, u_1 \, v_1 + g_{12} \, u_1 \, v_2 + \dots + g_{nn} \, u_n \, v_n.
\end{equation}

Berechnet man nun mit dieser allgemeinen Definition das Skalarprodukt eines infinitesimalen Vektors mit sich selbst, erhält man
\begin{equation}
	ds^2 = g_{ij} \, du^i \, du^j.
	\label{geodaeten:equation:MetrischerTensor:AllgemeinesLinienelement}
\end{equation}
Somit lässt sich das Linienelement in einer allgemeinen Form ausdrücken, die mithilfe des metrischen Tensors zu einer koordinatenunabhängigen Funktion wird.

Der metrische Tensor ist daher von zentraler Bedeutung für das Verständnis der Geometrie eines Raumes und ist ein fundamentales Werkzeug in der Differentialgeometrie und der allgemeinen Relativitätstheorie. 
Er ermöglicht es uns, die Metrik eines Raumes in eine kompakte Schreibweise zu überführen und liefert die Grundlage für die Berechnung von Abständen und Winkeln in komplexeren geometrischen Strukturen.

Im nächsten Abschnitt werden wir uns ansehen, wie die Metrik eines Raumes verwendet wird, um den metrischen Tensor für verschiedene Koordinatensysteme herzuleiten.
Anhand von Beispielen wird erläutert, wie diese Metrik, die wir bereits in verschiedenen Koordinatensystemen angewendet haben, in die kompakte Form des metrischen Tensors überführt werden kann.


\section{Beispiele zum metrischen Tensor}

%
% teil1.tex -- Beispiel-File für das Paper
%
% (c) 2020 Prof Dr Andreas Müller, Hochschule Rapperswil
%
% !TEX root = ../../buch.tex
% !TEX encoding = UTF-8
%
\subsection{Kartesisch\label{geodaeten:section:MetrischerTensor:Kartesisch}}
\rhead{Metrischer Tensor Beispiele}

Der Metrische Tensor für einen zweidimensionalen kartesischen Raum kann aus der Gleichung \eqref{geodaeten:equation:MetrischerTensor:AllgemeinesLinienelement} des allgemeinen Linienelements hergeleitet werden.
Schreiben wir die einsteinsche-Summe für zwei Dimensionen aus ergibt sich
\begin{equation}
	ds^2 = g_{11} \cdot du^1 \cdot du^1 + g_{12} \cdot du^1 \cdot du^2 + g_{21} \cdot du^2 \cdot du^1 + g_{22} \cdot du^2 \cdot du^2 .
	\label{geodaeten:equation:MetrischerTensor:Kartesisch:EinsteinSumme}
\end{equation}

In dem kartesischen Raum gilt, $du^1 = dx$ und $du^2 = dy$ wobei zu beachten ist, dass bei der Einsteinschen Summenkonvention die Hochstele keiner Potenz sondern eines Index entspricht.
Aus Abschnitt \ref{geodaeten:section:Linienelemente:Kartesisch} kennen wir das Linienelement des Kartesischen Raums als

\begin{equation}
	ds^2 = dx^2 + dy^2 .
\end{equation}
Aus dem Linienelement können wir die Koeffizienten von 

\begin{equation}
du^1 \cdot du^1 = dx^2 \quad \text{und} \quad du^2 \cdot du^2 = dy^2 
\end{equation}
als $1$ herauslesen.
Die Koeffizienten für

\begin{equation}
du^1 \cdot du^2 = dx \cdot dy \quad \text{und} \quad du^2 \cdot du^1 = dy \cdot dx
\end{equation}
sind beide $0$.
In Gleichung \ref{geodaeten:equation:MetrischerTensor:Kartesisch:EinsteinSumme} ist zu erkennen, dass diese Koeffizienten den Werten im metrischen Tensor $g_{ij}$ entsprechen.
An den richtigen Stellen eingesetzt ergibt sich der metrische Tensor des kartesischen Raums zu

\begin{equation}
	\begin{aligned}
		g_{11} &= \textcolor{red}{1} \\
		g_{12} &= \textcolor{blue}{0} \\
		g_{21} &= \textcolor{darkgreen}{0} \\
		g_{22} &= \textcolor{magenta}{1} \\
		g_{ij} &= \begin{pmatrix} \textcolor{red}{1} && \textcolor{blue}{0} \\ \textcolor{darkgreen}{0} && \textcolor{magenta}{1} \end{pmatrix} .
	\end{aligned}
\end{equation}


%
% teil1.tex -- Beispiel-File für das Paper
%
% (c) 2020 Prof Dr Andreas Müller, Hochschule Rapperswil
%
% !TEX root = ../../buch.tex
% !TEX encoding = UTF-8
%
\subsection{Zylinder\label{geodaeten:section:MetZylinder}}
\rhead{Metrischer Tensor Beispiele}

Das Linienelement für den zylindrischen Raum hergeleitet in [\ref{geodaeten:equation:Linienelemente:Kartesisch:equation2}], enthält einen Koeffizienten vor $\dot{\phi} ^2$.
Daher muss dieser Koeffizient im metrischen Tensor vorkommen.
Für den Fall das $r$ konstant und damit die Dimension 2 ist, gilt
\begin{equation}
	\begin{aligned}
	\mathbf{d\vec{s}}^2 &= \left| \begin{pmatrix} r^2 \cdot \dot{\phi}^2 \\ \dot{z}^2 \end{pmatrix} \right| \cdot dt^2 \\
	&= \begin{pmatrix} r^2 && 0 \\ 0 && 1 \end{pmatrix} \cdot \begin{pmatrix} \dot{\phi}^2 \\ \dot{z}^2 \end{pmatrix} \cdot dt^2 .
	\end{aligned}
\end{equation}

Damit ist der Metrische Tensor 
\begin{equation}
		T = \begin{pmatrix} r^2 && 0 \\ 0 && 1 \end{pmatrix} .
\end{equation}

Für den Fall das $r$ nicht konstant und damit die Dimension 3 ist, gilt

\begin{equation}
	\begin{aligned}
	\mathbf{d\vec{s}}^2 &= \left| \begin{pmatrix} \dot{r}^2 \\ r^2 \cdot \dot{x}^2 \\ \dot{y}^2 \end{pmatrix} \right| \cdot dt^2 \\
	&= \begin{pmatrix} 1 && 0 && 0 \\ 0 && r^2 && 0 \\ 0 && 0 && 1 \end{pmatrix} \cdot \begin{pmatrix} \dot{r}^2 \\ \dot{\phi}^2 \\ \dot{z}^2\end{pmatrix} \cdot dt^2 .
	\end{aligned}
\end{equation}

Damit ist der Metrische Tensor 
\begin{equation}
	T = \begin{pmatrix} 1 && 0 && 0 \\ 0 && r^2 && 0 \\ 0 && 0 && 1 \end{pmatrix} .
\end{equation}

Dieses Beispiel veranschaulicht, dass der metrische Tensor eine $n$x$n$ Matrix ist, wobei $n$ der Anzahl Dimensionen entspricht.


%
% teil1.tex -- Beispiel-File für das Paper
%
% (c) 2020 Prof Dr Andreas Müller, Hochschule Rapperswil
%
% !TEX root = ../../buch.tex
% !TEX encoding = UTF-8
%
\subsection{Kugel\label{geodaeten:section:MetrischerTensor:Kugel}}
\rhead{Metrischer Tensor Beispiele}

Das Linienelement für die Oberfläche einer Kugel mit Radius $r$ in Kugelkoordinaten $(\vartheta, \varphi)$ ist gegeben durch
\begin{equation}
	ds^2 = r^2 \left( d\vartheta^2 + \sin^2\vartheta \, d\varphi^2 \right).
\end{equation}

Um den metrischen Tensor $g_{i\!j}$ für die Kugeloberfläche zu bestimmen, drücken wir das Linienelement in der allgemeinen Form
\begin{equation}
	ds^2 = g_{i\!j} \, du^i \, du^j
\end{equation}
aus, wobei $u^1 = \vartheta$ und $u^2 = \varphi$ die Koordinaten auf der Kugeloberfläche darstellen.

Vergleichen wir nun für das Linienelement die beiden Ausdrücke 
\begin{equation}
	ds^2 = r^2 \, d\vartheta^2 + r^2 \sin^2\vartheta \, d\varphi^2
\end{equation}
und
\begin{equation}
	ds^2 = g_{11} \, (d\vartheta)^2 + g_{22} \, (d\varphi)^2 + 2g_{12} \, d\vartheta \, d\varphi,
\end{equation}
dann sehen wir durch den Vergleich der Terme, dass
\begin{equation}
	g_{11} = r^2, \quad g_{22} = r^2 \sin^2\vartheta, \quad \text{und} \quad g_{12} = g_{21} = 0.
\end{equation}
Daraus ergibt sich der metrische Tensor für die Kugeloberfläche in Matrixform als
\begin{equation}
	g_{i\!j} = r^2 \begin{pmatrix}
		1 & 0 \\
		0 & \sin^2\vartheta
	\end{pmatrix}.
\end{equation}

Hier ist ersichtlich, dass der Radius $r$ lediglich ein Skalierungsfaktor ist und für die geometrischen Eigenschaften der Kugeloberfläche keine entscheidende Rolle spielt. 
Die wesentliche Geometrie wird durch die Winkelabhängigkeit der Metrik bestimmt.
Dieser Tensor beschreibt somit die Geometrie der Kugeloberfläche und ermöglicht die Berechnung von Abständen, Winkeln und anderen geometrischen Größen.


%
% einleitung.tex -- Beispiel-File für die Einleitung
%
% (c) 2020 Prof Dr Andreas Müller, Hochschule Rapperswil
%
% !TEX root = ../../buch.tex
% !TEX encoding = UTF-8
%
\section{Differentialgleichung für Geodätenlinien
\label{geodaeten:section:Standardverfahren}}
\rhead{Differentialgleichung für Geodätenlinien}

Die Länge einer Kurve auf einer Ebene mit kartesischen Koordinaten lässt sich durch das Integral
\begin{equation}
	L = \int_{x_1}^{x_2} \sqrt{1 + \left(y'\right)^2} \, dx
\end{equation}
berechnen.
Um den kürzesten Abstand zwischen beiden Punkten zu finden, minimiert man dieses Längenintegral.
Man identifiziert zuerst die Lagrange-Funktion
\begin{equation}
	\mathcal{L}(y, y') = \sqrt{1 + \left(y'\right)^2},
\end{equation}
und bildet daraus die Euler-Lagrange-Differentialgleichung
\begin{equation}
	\frac{d}{dx} \left(\frac{\partial \mathcal{L}}{\partial y'}\right) - \frac{\partial \mathcal{L}}{\partial y} = 0.
\end{equation}
Durch Umformungen und Vereinfachungen ergibt sich letztendlich eine Differentialgleichung in der Form
\begin{equation}
	0 = \frac{y''}{\left(1 + \left(y'\right)^2\right)^{\frac{3}{2}}},
\end{equation}
deren Lösung eine Funktion beschreibt, die den kürzesten Abstand zwischen zwei Punkten auf einer Ebene darstellt. Wie wir später sehen werden, ist die Lösung dieser Gleichung eine Gerade, was intuitiv einleuchtend ist.

Nun stellen wir uns die Frage, ob wir mit unserem neu erlernten Wissen eine allgemeine Differentialgleichung finden können, welche die Funktion des kürzesten Abstands zwischen zwei Punkten für beliebige $n$-dimensionale geometrische Räume beschreibt.

\subsection{Formulierung des Variationsprinzips}
Wir beginnen mit dem allgemeinen Linienelement
\begin{equation}
	ds^2 = g_{ij} \, du^i \, du^j.
\end{equation}
Dieses parametrisieren wir nun nach einem gemeinsamen Parameter, wie zum Beispiel der Zeit, und erhalten
\begin{equation}
	ds^2 = g_{ij} \, \dot{u}^i \dot{u}^j \, dt^2,
\end{equation}
wobei $\dot{u}^i = \frac{du^i}{dt}$ die Ableitungen der Koordinaten nach der Zeit $t$ darstellen.

Durch die gemeinsame Parametrisierung lässt sich nun das Funktional für die Variation,
\begin{equation}
	L = \int_{t_1}^{t_2} \sqrt{g_{ij} \, \dot{u}^i \dot{u}^j} \, dt,
	\label{geodaeten:equation:StandardverfahrenGeodaeten:Funktional}
\end{equation}
als Integral des zu minimierenden Weges aufstellen.
Die Verwendung des raumspezifischen metrischen Tensors ermöglicht es, diese allgemeine Formel für jede Art von Raum anzuwenden.

Weil das Funktional nur von einem einzigen Parameter abhängt, können wir die klassische Euler-Lagrange-Differentialgleichung verwenden.
Als erstes identifizieren wir somit die Lagrange-Funktion als
\begin{equation}
	\mathcal{L}(u^n, \dot{u}^n) = \sqrt{g_{ij} \, \dot{u}^i \dot{u}^j}.
	\label{geodaeten:equation:StandardverfahrenGeodaeten:LagrangeFunktion}
\end{equation}

Da das Funktional eine $n$-dimensionale Anzahl an Koordinaten umfasst, die selbst Funktionen des Parameters $t$ sind, muss für jede Koordinate eine Euler-Lagrange-Differentialgleichung in der Form von
\begin{equation}
	\frac{d}{dt} \left(\frac{\partial \mathcal{L}}{\partial \dot{u}^n}\right) - \frac{\partial \mathcal{L}}{\partial u^n} = 0
	\label{geodaeten:equation:StandardverfahrenGeodaeten:DGL1}
\end{equation}
aufgestellt werden.
Für einen $n$-dimensionalen Raum ergeben sich somit auch $n$ Differentialgleichungen. 

\subsection{Vereinfachungen und Rahmenbedingungen}
Wir setzen voraus, dass der Radikand des Funktionals \eqref{geodaeten:equation:StandardverfahrenGeodaeten:Funktional} nicht negativ wird, die Koordinaten sowie ihre Ableitungen stetig sind, und dass ein differenzierbarer metrischer Tensor existiert. 
Diese Voraussetzungen charakterisieren eine Riemannsche Mannigfaltigkeit.
Daher gilt diese Variationsrechnung für beliebige geometrische Räume unter der Voraussetzung, dass sie als Riemannsche Mannigfaltigkeiten interpretiert werden können\footnote{Tatsächlich gelten die Geodätengleichungen, die sich aus dieser Variationsrechnung ergeben, auch für sogenannte Pseudo-Riemannsche Räume.
	In diesen Räumen kann die Metrik negative Werte annehmen, wie es für die Beschreibung gekrümmter Raumzeiten der Fall ist.
	Die Herleitung weicht zwar leicht ab, aber das Ergebnis, die allgemeine Geodätengleichung, bleibt unverändert.
}.

Um weitere Berechnungen zu vereinfachen, substituieren wir den Radikanden in \eqref{geodaeten:equation:StandardverfahrenGeodaeten:LagrangeFunktion} mit
\begin{equation}
	\varphi = g_{ij} \, \dot{u}^i \dot{u}^j.
	\label{geodaeten:equation:StandardverfahrenGeodaeten:Substitution}
\end{equation}
Damit ergibt sich die Euler-Lagrange-Differentialgleichung \eqref{geodaeten:equation:StandardverfahrenGeodaeten:DGL1} zu
\begin{equation}
	\frac{d}{dt} \left(\frac{1}{2 \sqrt{\varphi}} \, \varphi_{\dot{u}^n}\right) - \frac{1}{2 \sqrt{\varphi}} \, \varphi_{u^n} = 0,
	\label{geodaeten:equation:StandardverfahrenGeodaeten:DGL2}
\end{equation}
wobei $\varphi_{\dot{u}^n} = \frac{\partial \varphi}{\partial \dot{u}^n}$ und $\varphi_{u^n} = \frac{\partial \varphi}{\partial u^n}$ die partiellen Ableitungen von $\varphi$ nach den jeweiligen Koordinaten sind.

Wir nehmen zusätzlich an, dass $\varphi = 1$ ist, was bedeutet, dass der Parameter $t$ mit der Bogenlänge $s$ identifiziert werden kann und somit als natürlicher Parameter fungiert.
Dies vereinfacht die Berechnungen und führt zur konstanten Geschwindigkeit entlang der Geodäte.
Dadurch vereinfacht sich die Gleichung \eqref{geodaeten:equation:StandardverfahrenGeodaeten:DGL2} noch weiter zu
\begin{equation}
	\frac{d}{ds} \left( \varphi_{\dot{u}^n} \right) - \varphi_{u^n} = 0,
	\label{geodaeten:equation:StandardverfahrenGeodaeten:DGL3}
\end{equation}
wobei die Punktableitungen hier bedeuten, dass jetzt nach $s$ abgeleitet wird.

\subsection{Berechnung der Ableitungen}
Für die partielle Ableitung $\varphi_{u^n}$ erhalten wir
\begin{equation} 
	\frac{\partial \varphi}{\partial u^n} = \frac{\partial}{\partial u^n} \left(g_{ij} \, \dot{u}^i \, \dot{u}^j\right) = \frac{\partial \, g_{ij}}{\partial u^n} \, \dot{u}^i \, \dot{u}^j. 
	\label{geodaeten:equation:StandardverfahrenGeodaeten:PartialPhi1}
\end{equation}

Die partielle Ableitung $\varphi_{\dot{u}^n}$ ergibt sich aus
\begin{equation}
	\frac{\partial \varphi}{\partial \dot{u}^n} = g_{ij} \, \frac{\partial}{\partial \dot{u}^n} \left( \dot{u}^i \dot{u}^j \right),
\end{equation}
wobei es zwei Fälle zu betrachten gibt, abhängig davon, ob $i = n$ oder $j = n$:
\begin{itemize}
	\item Wenn $i = n$, dann ist $\frac{\partial \dot{u}^i}{\partial \dot{u}^n} = 1$ und $\frac{\partial \dot{u}^j}{\partial \dot{u}^n} = 0$, für $i \neq j$.
	\item Wenn $j = n$, dann ist $\frac{\partial \dot{u}^j}{\partial \dot{u}^n} = 1$ und $\frac{\partial \dot{u}^i}{\partial \dot{u}^n} = 0$, für $i \neq j$.
\end{itemize}

Aus der Produktregel ergibt sich somit
\begin{equation}
	\frac{\partial \varphi}{\partial \dot{u}^n} = g_{nj} \dot{u}^j + g_{in} \dot{u}^i,
\end{equation}
und weil $g_{ij}$ symmetrisch ist ($g_{ij} = g_{ji}$), lässt sich dies zusammenfassen zu
\begin{equation}
	\frac{\partial \varphi}{\partial \dot{u}^n} = 2g_{in} \dot{u}^i.
	\label{geodaeten:equation:StandardverfahrenGeodaeten:PartialPhi2}
\end{equation}

Da $\varphi_{\dot{u}^n}$ von den Koordinaten $u^k$ und deren Ableitungen $\dot{u}^k$ abhängt, berechnet sich die Ableitung von $\varphi_{\dot{u}^n}$ nach $s$ durch Anwendung der Kettenregel als
\begin{equation}
	\frac{d}{ds} \left( \varphi_{\dot{u}^n} \right) = \sum_{k = 1}^n \left( \frac{\partial \varphi_{\dot{u}^n}}{\partial u^k} \, \dot{u}^k + \frac{\partial \varphi_{\dot{u}^n}}{\partial \dot{u}^k} \, \ddot{u}^k \right),
	\label{geodaeten:equation:StandardverfahrenGeodaeten:Ableitung1s}
\end{equation}
wobei ab diesem Punkt die Summierung über den Index $k$ implizit erfolgt, wie es beim metrischen Tensor üblich ist.

Mithilfe von \eqref{geodaeten:equation:StandardverfahrenGeodaeten:PartialPhi2} erhalten wir für die partielle Ableitung nach $u^k$  
\begin{equation}
	\frac{\partial \varphi_{\dot{u}^n}}{\partial u^k} = 
	2 \frac{\partial g_{in}}{\partial u^k} \ \dot{u}^i
\end{equation}
und für die Ableitung nach $\dot{u}^k$
\begin{equation}
	\frac{\partial \varphi_{\dot{u}^n}}{\partial \dot{u}^k} = 
	2 \, g_{kn}.
\end{equation}

Daraus folgt schliesslich für \eqref{geodaeten:equation:StandardverfahrenGeodaeten:Ableitung1s}
\begin{equation}
	\frac{d}{ds} \left( \varphi_{\dot{u}^n} \right) = 2 \frac{\partial g_{in}}{\partial u^k} \ \dot{u}^i \dot{u}^k + 2 g_{kn} \, \ddot{u}^k.
	\label{geodaeten:equation:StandardverfahrenGeodaeten:Ableitung2s}
\end{equation}

\subsection{Formulierung der allgemeinen Geodätengleichung}
Fügt man nun alles zusammen und setzt \eqref{geodaeten:equation:StandardverfahrenGeodaeten:PartialPhi1} und \eqref{geodaeten:equation:StandardverfahrenGeodaeten:Ableitung2s} in die Gleichung \eqref{geodaeten:equation:StandardverfahrenGeodaeten:DGL3} ein, ergibt sich
\begin{equation}
	2 \frac{\partial g_{in}}{\partial u^k} \ \dot{u}^i \dot{u}^k + 2 g_{kn} \, \ddot{u}^k - \frac{\partial g_{ij}}{\partial u^n} \, \dot{u}^i \, \dot{u}^j = 0.
\end{equation}

Im ersten Summanden tritt jedoch ein Problem auf, da dieser Term nicht symmetrisch ist. 
Die Koeffizienten $\frac{\partial g_{in}}{\partial u^k}$ und $\frac{\partial g_{kn}}{\partial u^i}$ können unterschiedlich sein, was die symmetrische Eigenschaft des metrischen Tensors gefährdet. 
Um dies zu korrigieren, bildet man den Mittelwert der beiden Koeffizienten
\begin{equation}
	\frac{1}{2} \left( \frac{\partial g_{kn}}{\partial u^i} + \frac{\partial g_{in}}{\partial u^k} \right),
\end{equation}
wodurch sich die Symmetrie wiederherstellen lässt.

Der korrigierte Ausdruck lautet dann
\begin{equation}
	2 \cdot \frac{1}{2} \left( \frac{\partial g_{kn}}{\partial u^i} + \frac{\partial g_{in}}{\partial u^k} \right) \dot{u}^i \dot{u}^k + 2 g_{kn} \, \ddot{u}^k - \frac{\partial g_{ij}}{\partial u^n} \, \dot{u}^i \, \dot{u}^j = 0.
\end{equation}

Da der Index $k$ im ersten Summanden und der Index $j$ im letzten Summanden beide bis $n$ summiert werden, kann der Index $k$ durch $j$ ersetzt werden. 
Dadurch lässt sich der Term $g_{ij}$ aufgrund der wiederhergestellten Symmetrie teilweise ausklammern und es ergibt sich
\begin{equation}
	 g_{kn} \, \ddot{u}^k + \left( \frac{\partial g_{jn}}{\partial u^i} + \frac{\partial g_{in}}{\partial u^j} - \frac{\partial g_{ij}}{\partial u^n} \right) \, \dot{u}^i \, \dot{u}^j = 0.
\end{equation}

Teilt man den gesamten Ausdruck durch $2g_{ij}$ und benennt den Index $n$ in $l$ um, so erhält man die Gleichung
\begin{equation}
	\ddot{u}^k + \frac{1}{2} g^{kl} \left( \frac{\partial g_{jl}}{\partial u^i} + \frac{\partial g_{il}}{\partial u^j} - \frac{\partial g_{ij}}{\partial u^l} \right) \dot{u}^i \dot{u}^j = 0,
\end{equation}
wobei $g^{kl}$ die Inverse des metrischen Tensors darstellt.

Dieser Ausdruck kann durch die sogenannten Christoffel-Symbole $\Gamma^k_{ij}$ weiter zusammengefasst werden.
Diese sind definiert als
\begin{equation}
	\Gamma^k_{ij} = \frac{1}{2} g^{kl} \left( \frac{\partial g_{jl}}{\partial u^i} + \frac{\partial g_{il}}{\partial u^j} - \frac{\partial g_{ij}}{\partial u^l} \right),
\end{equation}
und sind aufgrund der wiederhergestellten Symmetrie des metrischen Tensors, ebenfalls in den Indizes $i$ und $j$ symmetrisch ($\Gamma^k_{ij} = \Gamma^k_{ji}$).

Durch die Einführung der Christoffel-Symbole erhält man so schliesslich die allgemeine Geodätengleichung
\begin{equation}
	\ddot{u}^k + \Gamma^k_{ij} \dot{u}^i \dot{u}^j = 0.
	\label{geodaeten:equation:StandardverfahrenGeodaeten:Geodaetengleichung}
\end{equation}

Diese allgemeine Geodätengleichung kann so interpretiert werden, dass der kürzeste Weg zwischen zwei Punkten prinzipiell immer eine Gerade ist. 
Allerdings wird die Vorstellung einer "Geraden" durch die Geometrie des Raumes verzerrt und nimmt je nach Raum eine andere Form an. 
Hier kommen die Christoffel-Symbole in der Geodätengleichung ins Spiel.

Die Christoffel-Symbole beschreiben, wie sich ein Raum in die Richtung einer Dimension ändert.
Sie wirken wie Wegweiser, welche die Geodäten so korrigieren, dass sie im jeweiligen Raum am geradlinigsten verlaufen. 
Da die Christoffel-Symbole von der Geometrie des Raumes abhängen, reflektieren sie auch die Krümmung des Raumes.
Wenn beispielsweise alle Christoffel-Symbole null sind, ist der Raum krümmungsfrei, und eine Gerade bleibt eine Gerade.

Mit dieser allgemeinen Geodätengleichung können nun die Geodätenlinien für beliebige $n$-dimensionale Räume berechnet werden. 
Anstatt das Variationsprinzip jedes Mal von Grund auf anzuwenden, kann die Geodätengleichung direkt genutzt werden, um daraus standardmässig die Funktion des kürzesten Weges zu ermitteln (Abschnitt \ref{geodaeten:section:StandardverfahrenBeispiele}).

\section{Beispiele zu den Differentialgleichungen für Geodätenlinien 
\label{geodaeten:section:StandardverfahrenBeispiele}}

%
% einleitung.tex -- Beispiel-File für die Einleitung
%
% (c) 2020 Prof Dr Andreas Müller, Hochschule Rapperswil
%
% !TEX root = ../../buch.tex
% !TEX encoding = UTF-8
%
\subsection{Kartesisch\label{geodaeten:section:Standardverfahren:Kartesisch}}
\rhead{Standardverfahren Beispiele}

Für den kartesischen Raum mit dem metrischen Tensor
 
\begin{equation}
g_{ij} = \begin{pmatrix} 
	1 & 0 \\ 
	0 & 1 
\end{pmatrix},
\end{equation}
wollen wir die Christoffel-Symbole berechnen.
Die Christoffel-Symbole sind gegeben durch,

\begin{equation}
\Gamma^i_{jk} = \frac{1}{2} g^{kl} \left( \frac{\partial g_{jl}}{\partial u^i} + \frac{\partial g_{il}}{\partial u^j} - \frac{\partial g_{ij}}{\partial u^l} \right),
\end{equation}
wobei $g^{ij}$ die Inverse des metrischen Tensors ist.
Da der metrische Tensor $g_{ij}$ konstant ist und keine direkte Abhängigkeit von den Koordinaten aufweist, verschwinden alle Ableitungen.
Ohne weitere Berechnungen kann man also schließen, dass

\begin{equation}
\frac{\partial g_{ij}}{\partial u^k} = 0 .
\end{equation}
Somit ergeben sich alle Christoffel-Symbole als null

\begin{equation}
\Gamma^i_{jk} = 0 .
\end{equation}

Denkt man an die Definition aus Abschnitt \ref{geodaeten:section:Standardverfahren}, macht dies durchaus Sinn.
Denn der Kartesische Raum ist nicht gekrümmt, weshalb keine Korrektur der Geraden notwendig ist.

Setzt man die Christoffel-Symbole in die allgemeine Geodätengleichung ein, erhält man mit $u^1 = x(t)$
\begin{equation}
\begin{aligned}
\ddot{u}^1 + \Gamma_{ij}^1 \dot{u}^i \dot{u}^j &= 0 \\
\ddot{u}^1 + 0_{ij} \cdot \dot{u}^i \dot{u}^j &= 0\\
\ddot{u}^1 &= 0 \\
\ddot{x}(t) &= 0
\end{aligned}
\label{geodaeten:equation:Standardverfahren:Kartesisch:x}
\end{equation}
und mit $u^2 = y(t)$
\begin{equation}
\begin{aligned}
\ddot{u}^2 + \Gamma_{ij}^2 \dot{u}^i \dot{u}^j &= 0 \\
\ddot{u}^2 + 0_{ij} \cdot \dot{u}^i \dot{u}^j &= 0 \\
\ddot{u}^2 &= 0 \\
\ddot{y}(t) &= 0 \qquad \qquad .
\end{aligned}
\label{geodaeten:equation:Standardverfahren:Kartesisch:y}
\end{equation}

Man sieht bereits, da die zweite Ableitungen in beide Dimensionen Null sind, handelt es sich bei dem kürzesten Weg um eine Gerade, was aus der Erfahrung durchaus Sinn ergibt.

Setzt man nun zwei Punkte als Start und Endpunkt, kann man durch diese Nebenbedingungen eine konkrete Lösung erhalten.
Beispielsweise wollen wir den kürzesten Weg zwischen $P_A = (1,1)$ und $P_B = (3,5)$ berechnen. Wir integrieren die zweite Ableitung von $x(t)$

\begin{equation}
	\frac{d^2x}{dt^2} = 0 \Rightarrow \frac{dx}{dt} = c_1 ,
\end{equation}
wobei $c_1$ eine Integrationskonstante ist. Durch erneutes Integrieren erhalten wir

\begin{equation}
\Rightarrow x(t) = c_1 \cdot t + c_2  .
\label{geodaeten:equation:Standardverfahren:Kartesisch:equation1}
\end{equation}
Wieder ist $c_2$ eine Integrationskonstante. 
Wir sehen, die Gleichung entspricht einer parametrierten Geradengleichung mit $c_2$ als Startwert für $x(0)$ und $c_1$ als Steigung von $x(t)$. 

Mit $P_A(x_1,y_1)$ und $P_B(x_2,y_2)$ haben diese Gleichungen die Form

\begin{align}
	x(t) &= (x_2 - x_1) \cdot t + x_1 \\
	y(t) &= (y_2 - y_1) \cdot t + y_1
\end{align}
Da die Linie durch den Startpunkt gehen muss ist der Startwert bei $t=0$ bekannt als
 
\begin{equation}
	0 \cdot c_1 + c_2 = 1 \Rightarrow c_2 = 1 .	
\end{equation}
Die Steigung $c_1$ kann mithilfe von Endpunkt und Startpunkt berechnet werden als

\begin{equation}
	c_1 = x_2 - x_1 \\ = 3-1 \\ = 2
\end{equation}
und damit ist die Lösung der Geodätengleichung in $x$ gleich

\begin{equation}
	x(t) = 2t + 1 .
\end{equation}
Analog für $y(t)$ ist \eqref{geodaeten:equation:Standardverfahren:Kartesisch:equation1}
  
\begin{equation}
	\Rightarrow y(t) = c_3 \cdot t + c_4  .
\end{equation}
Durch Einsetzen der Randwerte ergibt sich für $y(t)$ 

\begin{equation}
	0 \cdot c_3 + c_4 = 1 \Rightarrow c_4 = 1 
\end{equation}
und für die Steigung aus Sicht von $y$

\begin{equation}
	c_4 = y_2 - y_1 \\=5-1 = 4 .
\end{equation}
Damit ist die Lösung der Geodätengleichung in $y$ gleich

\begin{equation}
	y(t) = 4t + 1 .
\end{equation}

\begin{figure}
	\centering
	\includegraphics[width=10cm]{papers/geodaeten/Abbildungen/Standardverfahren/Kartesisch}
	\label{geodaeten:figure:Standardverfahren:Kartesisch:figure1}
	\caption{Darstellung der Kurve von x(t) und y(t) mit $t \in [0 , 1]$ Wie man sieht ist der kürzeste Weg von Punkt A zu Punkt B eine gerade.}
\end{figure}

%
% einleitung.tex -- Beispiel-File für die Einleitung
%
% (c) 2020 Prof Dr Andreas Müller, Hochschule Rapperswil
%
% !TEX root = ../../buch.tex
% !TEX encoding = UTF-8
%
\subsection{Zylinder\label{geodaeten:section:StaZylinder}}
\rhead{Standardverfahren Beispiele}

Lorem ipsum dolor sit amet, consetetur sadipscing elitr, sed diam
nonumy eirmod tempor invidunt ut labore et dolore magna aliquyam
erat, sed diam voluptua \cite{geodaeten:bibtex}.
At vero eos et accusam et justo duo dolores et ea rebum.
Stet clita kasd gubergren, no sea takimata sanctus est Lorem ipsum
dolor sit amet.

Lorem ipsum dolor sit amet, consetetur sadipscing elitr, sed diam
nonumy eirmod tempor invidunt ut labore et dolore magna aliquyam
erat, sed diam voluptua.
At vero eos et accusam et justo duo dolores et ea rebum.  Stet clita
kasd gubergren, no sea takimata sanctus est Lorem ipsum dolor sit
amet.


Auch in diesem Beispiel sind die Christophsymbole gleich Null.
Auf den ersten Blick könnte das verwirrend sein, da man bei einem Zylinder doch eindeutig eine Krümmung sieht.
Der Grund dafür ist, dass es sich bei dem Zylinder um eine extrinsische Krümmung handelt.
Die Zylinderoberfläche wird von außen zu einem Zylinder gekrümmt.
Abgerollt sieht man allerdings, dass die Oberfläche Flach ist.
Als weiteres Beispiel lässt sich berechnen, dass die Christophsymbole im Polarkoordinaten-Raum nicht gleich Null sind und daher eine Krümmung existiert, obwohl der Raum Flach erscheint.
Damit zeigt sich, dass die Intuition in diesem Fall täuschen kann.
%
% einleitung.tex -- Beispiel-File für die Einleitung
%
% (c) 2020 Prof Dr Andreas Müller, Hochschule Rapperswil
%
% !TEX root = ../../buch.tex
% !TEX encoding = UTF-8
%
\subsection{Kugel\label{geodaeten:section:Standardverfahren:Kugel}}
\rhead{Standardverfahren Beispiele}

Intuitiv erwarten wir, dass die Geodäten auf der Kugeloberfläche Grosskreise sind.
Ein Grosskreis ist der geradlinigste Weg zwischen zwei Punkten auf einer Kugel.
Dies kann man sich vorstellen, indem man zwei Nadeln in eine Kugeloberfläche steckt und einen Faden um die beiden Nadeln spannt.
Der Faden wird immer entlang eines Grosskreises verlaufen.
Tatsächlich folgen Flugzeuge auf Langstreckenflügen dieser Logik und fliegen entlang von Grosskreisen, um die kürzeste Strecke zwischen zwei Punkten auf der Erde zurückzulegen.
Wir möchten nun überprüfen, ob wir mit dem Standardverfahren tatsächlich zu dieser Lösung kommen.

Um dies zu tun, benötigen wir den bereits im vorherigen Abschnitt hergeleiteten metrischen Tensor für die Kugeloberfläche.
Dieser ist in Kugelkoordinaten $(\theta, \phi)$ gegeben durch
\begin{equation}
	g_{ij} = r^2 \begin{pmatrix}
		1 & 0 \\
		0 & \sin^2\theta
	\end{pmatrix}.
\end{equation}

Zuerst berechnen wir die Inverse des metrischen Tensors, da diese für die Berechnung der Christoffel-Symbol benötigt wird.
Der inverse Tensor $g^{ij}$ ist ergibt sich zu
\begin{equation}
	g^{ij} = \frac{1}{r^2} 
	\begin{pmatrix}
		1 & 0 \\
		0 & \frac{1}{\sin^2\theta}
	\end{pmatrix}.
	\label{geodaeten:equation:StaKugel:TensorInverse}
\end{equation}

Nun berechnen wir die partiellen Ableitungen des metrischen Tensors $g_{ij}$ in Bezug auf die Koordinaten $\theta$ und $\phi$.
Da $g_{11} = r^2$ und $g_{22} = r^2 \sin^2\theta$, erhalten wir
\begin{equation}
	\frac{\partial g_{11}}{\partial \theta} = 0, \quad \frac{\partial g_{11}}{\partial \phi} = 0, \quad \frac{\partial g_{22}}{\partial \theta} = 2r^2 \sin\theta \cos\theta, \quad \frac{\partial g_{22}}{\partial \phi} = 0.
	\label{geodaeten:equation:StaKugel:Ableitungen}
\end{equation}

\subsection{Christoffel-Symbole}
Mit den Ableitungen \eqref{geodaeten:equation:StaKugel:Ableitungen} und dem inversen Tensor $g^{ij}$ \eqref{geodaeten:equation:StaKugel:TensorInverse} können wir nun die Christoffel-Symbole berechnen durch
\begin{equation}
	\Gamma_{ij}^k = \frac{1}{2} g^{kl} \left( \frac{\partial g_{jl}}{\partial u^i} + \frac{\partial g_{il}}{\partial u^j} - \frac{\partial g_{ij}}{\partial u^l} \right),
\end{equation}
wobei $u^1 = \theta$ und $u^2 = \phi$.

Nach Einsetzen der Werte und Vereinfachung erhalten wir folgende nicht-verschwindende Christoffel-Symbole
\begin{equation}
	\Gamma_{12}^2 = \Gamma_{21}^2 = \cot\theta \quad \text{und} \quad \Gamma_{22}^1 = -\sin\theta \cos\theta.
\end{equation}

Anders als im Fall des Zylinders, wo der metrische Tensor konstant war, führt die Abhängigkeit des metrischen Tensors von der Koordinate $\theta$ zu nicht-verschwindenden Christoffel-Symbolen, welche die intrinsische Krümmung der Kugeloberfläche beschreiben.

Um diese Krümmung zu verstehen, betrachten wir die Bewegung eines Tangentialvektors auf der Oberfläche in Bezug auf die Basisvektoren.

Wird die Komponente des Vektors entlang eines Breitenkreises verändert, also in $\phi$-Richtung, führt das zu keiner Änderung der Vektorrichtung, da das Bogenmass von $\phi$ entlang eines Breitenkreises konstant bleibt.
Diese Bewegung beeinflusst die allgemeine Richtung des Vektors also nicht, was auch im metrischen Tensor reflektiert wird, der keinerlei Abhängigkeit von $\phi$ aufweist.

Andererseits führt eine Veränderung der Komponente entlang eines Längengrades, also in $\theta$-Richtung, zu einer Veränderung der Vektorrichtung.
Dies liegt daran, dass das Bogenmass von $\phi$ mit $\theta$ variiert: 
Je näher man den Polen kommt, desto kürzer wird der Bogen für eine gegebene Änderung von $\theta$, was die Gesamtrichtung des Vektors beeinflusst. 
Damit der Tangentialvektor seine ursprüngliche Richtung beibehält, müsste die Bewegung in der $\phi$-Richtung zunehmend verringert werden, je näher man den Polen kommt.

Diese Richtungsänderung des Vektors ist eine direkte Folge der intrinsischen Krümmung der Kugeloberfläche.
Die nicht-verschwindenden Christoffel-Symbole beschreiben genau diese Krümmung und geben an, wie sich die Richtung eines Vektors bei seiner Bewegung entlang der Oberfläche verändert.

\subsection{Geodätengleichung}
Mit den berechneten Christoffel-Symbolen können wir nun die Geodätengleichungen für die Kugeloberfläche aufstellen.

Die allgemeine Form der Geodätengleichung lautet
\begin{equation}
	\ddot{u}^k + \Gamma^k_{ij} \dot{u}^i \dot{u}^j = 0,
\end{equation}
wobei $u^k$ die Koordinaten darstellen, in diesem Fall $\theta$ und $\phi$.

Setzen wir die entsprechenden Christoffel-Symbole und Koordinaten in die Gleichungen ein, erhalten wir
\begin{equation}
	\begin{aligned} 
		0 &= \ddot{\theta} + \Gamma^1_{11} \dot{\theta} \dot{\theta} + 2\Gamma^1_{12} \dot{\theta}\dot{\phi} + \Gamma^1_{22} \dot{\phi} \dot{\phi} \\
		0 &= \ddot{\phi} + \Gamma^2_{11} \dot{\theta} \dot{\theta} + 2\Gamma^2_{12} \dot{\theta}\dot{\phi} + \Gamma^2_{22} \dot{\phi} \dot{\phi},
	\end{aligned}
\end{equation}
und weil einige der Christoffel-Symbole Null sind, vereinfachen sich die Geodätengleichungen schliesslich zu
\begin{align}
	0 &= \ddot{\theta} - \sin\theta \cos\theta \, \dot{\phi}^2 \\
	0 &= \ddot{\phi} + 2 \cot\theta \, \dot{\theta} \dot{\phi}.
\end{align}

Da $r$ lediglich ein Skalierungsfaktor ist, kürzt es sich in den Geodätengleichungen heraus. 
Die Geometrie wird allein durch die Winkelabhängigkeit bestimmt, sodass der Radius keinen Einfluss auf die Form der Geodäten hat.

Das Lösen der Geodätengleichungen ist oft aufwendig und schwierig, was numerische Ansätze erforderlich macht. 
Daher überprüfen wir durch Analyse, ob Grosskreise diese tatsächlich erfüllen und als Lösungen gelten.

\subsection{Äquator}
Betrachten wir die Geodätengleichungen für konstantes $\theta$, also entlang eines Breitengrads.
Für konstantes $\theta$ gilt $\dot{\theta} = 0$ und $\ddot{\theta} = 0$.
Setzen wir dies in die erste Geodätengleichung ein, so erhalten wir:
\begin{equation}
	0 = -\sin\theta \cos\theta \, \dot{\phi}^2.
\end{equation}
Die zweite Geodätengleichung vereinfacht sich zu $0 = \ddot{\phi}$, was zeigt, dass $\phi(t)$ linear in der Zeit variiert.
Diese erste Geodätengleichung wird nur für $\theta = \frac{\pi}{2}$ erfüllt, das heisst, nur für den Äquator.
Für alle anderen Breitengrade mit $\theta \neq \frac{\pi}{2}$ muss $\dot{\phi}$ gleich null werden, was bedeutet, dass es keine Bewegung in der $\phi$-Richtung gibt und somit keine Geodäte vorliegt.
Der Äquator, als einziger Breitengrad, erfüllt daher die Geodätengleichungen und ist ein Grosskreis.

\begin{figure}
	\centering
	\includegraphics[width=7cm]{papers/geodaeten/Abbildungen/Standardverfahren/StaKugelBreitengrade}
	\caption{Erdkugel mit Breitengrade und Äquator als Geodätenlinie.}
	\label{geodaeten:figure:Standardverfahren:Breitengrade}
\end{figure}

\subsection{Längengrade}
Nun analysieren wir die Geodätengleichungen für konstantes $\phi$, also entlang eines Längengrads.
Für konstantes $\phi$ gilt $\dot{\phi} = 0$ und $\ddot{\phi} = 0$.
Setzen wir dies in die erste Geodätengleichung ein, erhalten wir:
\begin{equation}
	\ddot{\theta} = 0,
\end{equation}
was bedeutet, dass $\theta(t)$ linear in der Zeit variiert.
Die zweite Geodätengleichung wird ebenfalls trivial erfüllt, da $\dot{\phi} = 0$ und somit keine Beschleunigung in der $\phi$-Richtung vorliegt.
Damit sind alle Längengrade tatsächlich Geodäten.

Zusammenfassend konnten wir durch unsere Analyse zeigen, dass nur Grosskreise, wie der Äquator und die Längengrade, die Geodätengleichungen vollständig erfüllen und somit die wahren Geodäten auf der Kugeloberfläche darstellen.

Wir haben damit erfolgreich nachgewiesen, dass die Lösungen der Geodätengleichungen $\theta(t)$ und $\phi(t)$ zwangsläufig die Form von Grosskreisen annehmen müssen und dass die allgemeine Geodätengleichung diese Grosskreise korrekt als die Geodäten auf der Kugeloberfläche identifiziert.

\begin{figure}
	\centering
	\includegraphics[width=7cm]{papers/geodaeten/Abbildungen/Standardverfahren/StaKugelLaengengrade}
	\caption{Erdkugel mit Längengrade als Geodätenlinen.}
	\label{geodaeten:figure:Standardverfahren:Laengengrade}
\end{figure}


\printbibliography[heading=subbibliography]
\end{refsection}
