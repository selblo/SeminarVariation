%
% teil1.tex -- Beispiel-File für das Paper
%
% (c) 2020 Prof Dr Andreas Müller, Hochschule Rapperswil
%
% !TEX root = ../../buch.tex
% !TEX encoding = UTF-8
%
\subsection{Zylinder\label{geodaeten:section:MetZylinder}}
\rhead{Metrischer Tensor Beispiele}

Das Linienelement für den zylindrischen Raum hergeleitet in [\ref{geodaeten:equation:Linienelemente:Kartesisch:equation2}], enthält einen Koeffizienten vor $\dot{\phi} ^2$.
Daher muss dieser Koeffizient im metrischen Tensor vorkommen.
Für den Fall das $r$ konstant und damit die Dimension 2 ist, gilt
\begin{equation}
	\begin{aligned}
	\mathbf{d\vec{s}}^2 &= \left| \begin{pmatrix} r^2 \cdot \dot{\phi}^2 \\ \dot{z}^2 \end{pmatrix} \right| \cdot dt^2 \\
	&= \begin{pmatrix} r^2 && 0 \\ 0 && 1 \end{pmatrix} \cdot \begin{pmatrix} \dot{\phi}^2 \\ \dot{z}^2 \end{pmatrix} \cdot dt^2 .
	\end{aligned}
\end{equation}

Damit ist der Metrische Tensor 
\begin{equation}
		T = \begin{pmatrix} r^2 && 0 \\ 0 && 1 \end{pmatrix} .
\end{equation}

Für den Fall das $r$ nicht konstant und damit die Dimension 3 ist, gilt

\begin{equation}
	\begin{aligned}
	\mathbf{d\vec{s}}^2 &= \left| \begin{pmatrix} \dot{r}^2 \\ r^2 \cdot \dot{x}^2 \\ \dot{y}^2 \end{pmatrix} \right| \cdot dt^2 \\
	&= \begin{pmatrix} 1 && 0 && 0 \\ 0 && r^2 && 0 \\ 0 && 0 && 1 \end{pmatrix} \cdot \begin{pmatrix} \dot{r}^2 \\ \dot{\phi}^2 \\ \dot{z}^2\end{pmatrix} \cdot dt^2 .
	\end{aligned}
\end{equation}

Damit ist der Metrische Tensor 
\begin{equation}
	T = \begin{pmatrix} 1 && 0 && 0 \\ 0 && r^2 && 0 \\ 0 && 0 && 1 \end{pmatrix} .
\end{equation}

Dieses Beispiel veranschaulicht, dass der metrische Tensor eine $n$x$n$ Matrix ist, wobei $n$ der Anzahl Dimensionen entspricht.

