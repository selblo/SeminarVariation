 %
% teil1.tex -- Beispiel-File für das Paper
%
% (c) 2020 Prof Dr Andreas Müller, Hochschule Rapperswil
%
% !TEX root = ../../buch.tex
% !TEX encoding = UTF-8
%
\section{Metrischer Tensor
\label{geodaeten:section:MetrischerTensor}}
\rhead{Metrischer Tensor}

Sed ut perspiciatis unde omnis iste natus error sit voluptatem
accusantium doloremque laudantium, totam rem aperiam, eaque ipsa
quae ab illo inventore veritatis et quasi architecto beatae vitae
dicta sunt explicabo.
Nemo enim ipsam voluptatem quia voluptas sit aspernatur aut odit
aut fugit, sed quia consequuntur magni dolores eos qui ratione
voluptatem sequi nesciunt

\begin{equation}
\int_a^b x^2\, dx
=
\left[ \frac13 x^3 \right]_a^b
=
\frac{b^3-a^3}3.
\label{geodaeten:equation1}
\end{equation}

Neque porro quisquam est, qui dolorem ipsum quia dolor sit amet,
consectetur, adipisci velit, sed quia non numquam eius modi tempora
incidunt ut labore et dolore magnam aliquam quaerat voluptatem.

Ut enim ad minima veniam, quis nostrum exercitationem ullam corporis
suscipit laboriosam, nisi ut aliquid ex ea commodi consequatur?
Quis autem vel eum iure reprehenderit qui in ea voluptate velit
esse quam nihil molestiae consequatur, vel illum qui dolorem eum
fugiat quo voluptas nulla pariatur?

\section{Beispiele zum metrischen Tensor}

%
% teil1.tex -- Beispiel-File für das Paper
%
% (c) 2020 Prof Dr Andreas Müller, Hochschule Rapperswil
%
% !TEX root = ../../buch.tex
% !TEX encoding = UTF-8
%
\subsection{Kartesisch\label{geodaeten:section:MetrischerTensor:Kartesisch}}
\rhead{Metrischer Tensor Beispiele}

Der Metrische Tensor für einen zweidimensionalen kartesischen Raum kann aus der Gleichung \eqref{geodaeten:equation:MetrischerTensor:AllgemeinesLinienelement} des allgemeinen Linienelements hergeleitet werden.
Schreiben wir die einsteinsche-Summe für zwei Dimensionen aus ergibt sich
\begin{equation}
	ds^2 = g_{11} \cdot du^1 \cdot du^1 + g_{12} \cdot du^1 \cdot du^2 + g_{21} \cdot du^2 \cdot du^1 + g_{22} \cdot du^2 \cdot du^2 .
	\label{geodaeten:equation:MetrischerTensor:Kartesisch:EinsteinSumme}
\end{equation}

In dem kartesischen Raum gilt, $du^1 = dx$ und $du^2 = dy$ wobei zu beachten ist, dass bei der Einsteinschen Summenkonvention die Hochstele keiner Potenz sondern eines Index entspricht.
Aus Abschnitt \ref{geodaeten:section:Linienelemente:Kartesisch} kennen wir das Linienelement des Kartesischen Raums als

\begin{equation}
	ds^2 = dx^2 + dy^2 .
\end{equation}
Aus dem Linienelement können wir die Koeffizienten von 

\begin{equation}
du^1 \cdot du^1 = dx^2 \quad \text{und} \quad du^2 \cdot du^2 = dy^2 
\end{equation}
als $1$ herauslesen.
Die Koeffizienten für

\begin{equation}
du^1 \cdot du^2 = dx \cdot dy \quad \text{und} \quad du^2 \cdot du^1 = dy \cdot dx
\end{equation}
sind beide $0$.
In Gleichung \ref{geodaeten:equation:MetrischerTensor:Kartesisch:EinsteinSumme} ist zu erkennen, dass diese Koeffizienten den Werten im metrischen Tensor $g_{ij}$ entsprechen.
An den richtigen Stellen eingesetzt ergibt sich der metrische Tensor des kartesischen Raums zu

\begin{equation}
	\begin{aligned}
		g_{11} &= \textcolor{red}{1} \\
		g_{12} &= \textcolor{blue}{0} \\
		g_{21} &= \textcolor{darkgreen}{0} \\
		g_{22} &= \textcolor{magenta}{1} \\
		g_{ij} &= \begin{pmatrix} \textcolor{red}{1} && \textcolor{blue}{0} \\ \textcolor{darkgreen}{0} && \textcolor{magenta}{1} \end{pmatrix} .
	\end{aligned}
\end{equation}


%
% teil1.tex -- Beispiel-File für das Paper
%
% (c) 2020 Prof Dr Andreas Müller, Hochschule Rapperswil
%
% !TEX root = ../../buch.tex
% !TEX encoding = UTF-8
%
\subsection{Zylinder\label{geodaeten:section:MetZylinder}}
\rhead{Metrischer Tensor Beispiele}

Das Linienelement für den zylindrischen Raum hergeleitet in [\ref{geodaeten:equation:Linienelemente:Kartesisch:equation2}], enthält einen Koeffizienten vor $\dot{\phi} ^2$.
Daher muss dieser Koeffizient im metrischen Tensor vorkommen.
Für den Fall das $r$ konstant und damit die Dimension 2 ist, gilt
\begin{equation}
	\begin{aligned}
	\mathbf{d\vec{s}}^2 &= \left| \begin{pmatrix} r^2 \cdot \dot{\phi}^2 \\ \dot{z}^2 \end{pmatrix} \right| \cdot dt^2 \\
	&= \begin{pmatrix} r^2 && 0 \\ 0 && 1 \end{pmatrix} \cdot \begin{pmatrix} \dot{\phi}^2 \\ \dot{z}^2 \end{pmatrix} \cdot dt^2 .
	\end{aligned}
\end{equation}

Damit ist der Metrische Tensor 
\begin{equation}
		T = \begin{pmatrix} r^2 && 0 \\ 0 && 1 \end{pmatrix} .
\end{equation}

Für den Fall das $r$ nicht konstant und damit die Dimension 3 ist, gilt

\begin{equation}
	\begin{aligned}
	\mathbf{d\vec{s}}^2 &= \left| \begin{pmatrix} \dot{r}^2 \\ r^2 \cdot \dot{x}^2 \\ \dot{y}^2 \end{pmatrix} \right| \cdot dt^2 \\
	&= \begin{pmatrix} 1 && 0 && 0 \\ 0 && r^2 && 0 \\ 0 && 0 && 1 \end{pmatrix} \cdot \begin{pmatrix} \dot{r}^2 \\ \dot{\phi}^2 \\ \dot{z}^2\end{pmatrix} \cdot dt^2 .
	\end{aligned}
\end{equation}

Damit ist der Metrische Tensor 
\begin{equation}
	T = \begin{pmatrix} 1 && 0 && 0 \\ 0 && r^2 && 0 \\ 0 && 0 && 1 \end{pmatrix} .
\end{equation}

Dieses Beispiel veranschaulicht, dass der metrische Tensor eine $n$x$n$ Matrix ist, wobei $n$ der Anzahl Dimensionen entspricht.


%
% teil1.tex -- Beispiel-File für das Paper
%
% (c) 2020 Prof Dr Andreas Müller, Hochschule Rapperswil
%
% !TEX root = ../../buch.tex
% !TEX encoding = UTF-8
%
\subsection{Kugel\label{geodaeten:section:MetrischerTensor:Kugel}}
\rhead{Metrischer Tensor Beispiele}

Das Linienelement für die Oberfläche einer Kugel mit Radius $r$ in Kugelkoordinaten $(\vartheta, \varphi)$ ist gegeben durch
\begin{equation}
	ds^2 = r^2 \left( d\vartheta^2 + \sin^2\vartheta \, d\varphi^2 \right).
\end{equation}

Um den metrischen Tensor $g_{i\!j}$ für die Kugeloberfläche zu bestimmen, drücken wir das Linienelement in der allgemeinen Form
\begin{equation}
	ds^2 = g_{i\!j} \, du^i \, du^j
\end{equation}
aus, wobei $u^1 = \vartheta$ und $u^2 = \varphi$ die Koordinaten auf der Kugeloberfläche darstellen.

Vergleichen wir nun für das Linienelement die beiden Ausdrücke 
\begin{equation}
	ds^2 = r^2 \, d\vartheta^2 + r^2 \sin^2\vartheta \, d\varphi^2
\end{equation}
und
\begin{equation}
	ds^2 = g_{11} \, (d\vartheta)^2 + g_{22} \, (d\varphi)^2 + 2g_{12} \, d\vartheta \, d\varphi,
\end{equation}
dann sehen wir durch den Vergleich der Terme, dass
\begin{equation}
	g_{11} = r^2, \quad g_{22} = r^2 \sin^2\vartheta, \quad \text{und} \quad g_{12} = g_{21} = 0.
\end{equation}
Daraus ergibt sich der metrische Tensor für die Kugeloberfläche in Matrixform als
\begin{equation}
	g_{i\!j} = r^2 \begin{pmatrix}
		1 & 0 \\
		0 & \sin^2\vartheta
	\end{pmatrix}.
\end{equation}

Hier ist ersichtlich, dass der Radius $r$ lediglich ein Skalierungsfaktor ist und für die geometrischen Eigenschaften der Kugeloberfläche keine entscheidende Rolle spielt. 
Die wesentliche Geometrie wird durch die Winkelabhängigkeit der Metrik bestimmt.
Dieser Tensor beschreibt somit die Geometrie der Kugeloberfläche und ermöglicht die Berechnung von Abständen, Winkeln und anderen geometrischen Größen.

