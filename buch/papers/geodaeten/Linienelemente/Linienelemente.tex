%
% einleitung.tex -- Beispiel-File für die Einleitung
%
% (c) 2020 Prof Dr Andreas Müller, Hochschule Rapperswil
%
% !TEX root = ../../buch.tex
% !TEX encoding = UTF-8
%
\section{Linienelemente\label{geodaeten:section:Linienelemente}}
\rhead{Linienelemente}

Um die Weglänge unserer Geodätenlinie minimieren zu können, müssen wir erst einmal den Weg berechnen.
Jeder Weg kann in kleinere Wegstücke $\Delta \text{Weg}$ unterteilt werden, welche addiert wieder den ganzen Weg ergeben.
Der Weg $l$ entspricht 
\begin{equation}
	l = \sum \Delta \text{Weg} .
\end{equation}

Linienelemente sind als infinitesimal kleine Wegstücke definiert, welche entlang jeder Dimension integriert die Weglänge ergeben.
Ein Linienelement entlang einer Dimensions-Achse beschreibt, wie sich der Raum in die entsprechende Dimension verändert.
In einem $n$-dimensionalen Raum entspricht ein Linienelement also einem $n$-dimensionalen Vektor, welcher die Kurve in jeder Dimension beschreibt.

Die Wegstücke werden infinitesimal und die Formel für die Wegstücklänge $\Delta \text{Weg}$ wird zur Formel für das Linienelement $ds$  als
\begin{equation}	
	d\text{Weg} = ds .
	\label{geodaeten:equation:Linienelemente:equation1}
\end{equation}

Um den Weg $l$ zu erhalten müssen schliesslich diese Linienelemente in allen Dimensionen integriert werden mit
\begin{equation}
	l = 
	\sum^{n} \int_a^b ds .
	\label{geodaeten:equation:Linienelemente:equation2}
\end{equation}

\section{Beispiele zu Linienelementen\label{geodaeten:section:Linienelemente:Beispiele}}
Die Beispiele sind für alle Unterkapitel gleich aufgebaut.
Zuerst wird das zweidimensionale Beispiel des kartesischen Raumes als einfacher Einstieg behandelt.
Danach wird der Zylinder als Übergang in etwas komplexere Strukturen aufgezeigt.
Zum Schluss werden Untersuchungen an einer Kugel durchgeführt, welche in der Praxis grosse Anwendung finden, aufgrund der Ähnlichkeit zur Erdkugel (Abbildung \ref{geodaeten:figure:Geodaeten:Erdkugel}).

	%
% einleitung.tex -- Beispiel-File für die Einleitung
%
% (c) 2020 Prof Dr Andreas Müller, Hochschule Rapperswil
%
% !TEX root = ../../buch.tex
% !TEX encoding = UTF-8
%
\subsection{Kartesisch\label{geodaeten:section:LinKartesisch}}
\rhead{Linienelemente Beispiele}

Wie in Abbildung [\ref{geodaeten:Linienelemente:figure1}] zu sehen ist kann ein Wegstück auf einer Kurve im zweidimensionalen Kartesischen Raum mit

\begin{equation}
	\Delta s \approx \sqrt{\Delta x^2 + \Delta y^2}
\end{equation}
approximiert werden.
Durch Verkleinerung der Wegstücke bis zum Infinitesimal 

\begin{equation}
	d s = \sqrt{d x^2 + d y^2}
	= \sqrt{\left(\frac{d x}{d t}\right)^2 \cdot d t^2 + \left(\frac{d y}{d t}\right)^2 \cdot d t^2} ,
\end{equation}
Kann das Linienelement aufgestellt werden als

\begin{equation}
 	ds^2 = \left(\dot{x}^2 +\dot{y}^2\right) \cdot dt^2 .
\end{equation}
Als Vektor dargestellt entspricht das Linienelement

\begin{equation}
	\mathbf{d\vec{s}}^2 = \begin{pmatrix} \dot{x}^2 \\ \dot{y}^2 \end{pmatrix} = \begin{pmatrix} 1 \\ 1 \end{pmatrix} \cdot \begin{pmatrix} \dot{x}^2 \\ \dot{y}^2 \end{pmatrix} \cdot dt^2 .
\end{equation}

\begin{figure}
	\centering
	\includegraphics[width=0.7\linewidth]{papers/geodaeten/Abbildungen/Linienelemente/LinKartes1}
	\caption{Linienelement im Kartesischen Raum}
	\label{geodaeten:Linienelemente:figure1}
	\cite{geodaeten:kartesisch}
\end{figure}

	%
% einleitung.tex -- Beispiel-File für die Einleitung
%
% (c) 2020 Prof Dr Andreas Müller, Hochschule Rapperswil
%
% !TEX root = ../../buch.tex
% !TEX encoding = UTF-8
%
\subsection{Zylinder\label{geodaeten:section:Linienelemente:Zylinder}}
\rhead{Linienelemente Beispiele}

Eine Kurve auf der Oberfläche eines Zylinders kann als zweidimensional betrachtet werden, wobei gilt

\begin{equation}
	\Delta s \approx \sqrt{(r \cdot \Delta \phi)^2 + \Delta z^2}
\end{equation}
und $r$ ist konstant.
Analog zu den Polarkoordinaten können die abstände infinitesimal werden und das Linienelement ergibt sich dadurch für die Oberfläche des Zylinders zu

\begin{equation}
	ds^2 = \left(r^2 \cdot \dot{\phi}^2 +\dot{z}^2\right) \cdot dt^2 .
	\label{geodaeten:equation:Linienelemente:Zylinder:equation2}
\end{equation}

%Als Vektor dargestellt entspricht das Linienelement
%
%\begin{equation}
%	\mathbf{d\vec{s}}^2 = \begin{pmatrix} r^2 \cdot \dot{\phi}^2 \\ \dot{z}^2 \end{pmatrix} = \begin{pmatrix} r^2 \\ 1 \end{pmatrix} \cdot \begin{pmatrix} \dot{\phi}^2 \\ \dot{z}^2 \end{pmatrix} \cdot dt^2 .
%\end{equation}

Den einstieg in dreidimensionale Kurven können wir machen, indem $r$ nicht als Konstant angenommen wird.
Der Weg kann so mit
\begin{equation}
	\Delta s \approx \sqrt{\Delta r^2 + (r \cdot \Delta \phi)^2 + \Delta z^2} \cdot dt^2
\end{equation}
berechnet werden und das Linienelement entspricht 
\begin{equation}
	ds^2 = \left(\dot{r}^2 + r^2 \cdot \dot{\phi}^2 +\dot{z}^2\right) \cdot dt^2 .
\end{equation}

%und als Vektor
%\begin{equation}
%	\mathbf{d\vec{s}}^2 = \begin{pmatrix} \dot{r}^2 \\ r^2 \cdot \dot{x}^2 \\ \dot{y}^2 \end{pmatrix} = \begin{pmatrix} 1 \\ r^2 \\ 1 \end{pmatrix} \cdot \begin{pmatrix} \dot{r}^2 \\ \dot{\phi}^2 \\ \dot{z}^2 \end{pmatrix} \cdot dt^2 .
%\end{equation}


\begin{figure}
	\centering
	\includegraphics[width=0.7\linewidth]{papers/geodaeten/Abbildungen/Linienelemente/LinZyl1}
	\caption{Linienelement im Kartesischen Raum}
	\label{geodaeten:figure:Linienelemente:Zylinder:figure2}
	\cite{geodaeten:polarkoordinaten}
\end{figure}


	%
% einleitung.tex -- Beispiel-File für die Einleitung
%
% (c) 2020 Prof Dr Andreas Müller, Hochschule Rapperswil
%
% !TEX root = ../../buch.tex
% !TEX encoding = UTF-8
%
\subsection{Kugel\label{geodaeten:section:Linienelement:Kugel}}
In ähnlicher Weise wie im Beispiel mit dem Zylinder lässt sich eine Kugel lokal wie eine flache Ebene darstellen.
Was auf den ersten Blick für sogenannte ``Flat-Earther'' ein schwieriges Konzept zu sein scheint, ist intuitiv verständlich:
\index{Flat-Earther}%
Zoomt man nahe genug an die Erdoberfläche heran, verschwinden die Krümmungen, und Abstände lassen sich nahezu wie in einem euklidischen Raum berechnen.

In Kugelkoordinaten beschreibt $r$ den Abstand eines Punktes vom Zentrum der Kugel (Radius), $\vartheta$ den Winkel zur $z$-Achse (Polarwinkel), und $\varphi$ den Winkel in der $x$-$y$-Ebene (Azimutwinkel).
\index{Radius}%
\index{Polarwinkel}%
\index{Azimutwinkel}%
Diese Parameter definieren die Position eines Punktes auf der Kugeloberfläche eindeutig, wie in der Abbildung \ref{geodaeten:figure:Linienelemente:Kugelkoordinaten:Kugelkoordinaten} dargestellt.

Da wir uns auf der Oberfläche der Kugel befinden, bleibt $r$ konstant und wir betrachten nur die Winkel $\vartheta$ und $\varphi$.
Wenn ein Punkt nun um einen infinitesimalen Abstand auf der Kugeloberfläche verschoben wird, entstehen Verschiebungen entlang der zugehörigen Basisvektoren. 

Die Verschiebung in der Längsrichtung ergibt sich aus
\begin{equation}
	r \, d\vartheta
\end{equation}
und für die Verschiebung in der Breitenrichtung erigbt sich
\begin{equation}
	r \sin\vartheta \, d\varphi.
\end{equation}

Weil lokal wie in einem euklidischen Raum gerechnet werden kann, ergibt sich der Abstand zwischen zwei Punkten auf der Kugeloberfläche durch die Anwendung des Pythagoras auf diese jeweiligen infinetsimalen Verschiebungen. 
Das Linienelement $ds$ für die Oberfläche einer Kugel lässt sich daher beschreiben als
\index{Oberfläche}%
\index{Kugel}%
\index{Linienelement!Kugel}%
\begin{equation}
ds^2 = r^2 \, d\vartheta^2 + r^2 \sin^2\vartheta \, d\varphi^2.
\end{equation}


\begin{figure}
	\centering
	\includegraphics[width=6cm]{papers/geodaeten/Abbildungen/Linienelemente/LinKugel1}
	\caption{Kugelkoordinaten mit Radius $r$, Polarwinkel $\vartheta$ und Azimutwinkel $\varphi$. Bildquelle: \cite{geodaeten:Kugelkoordinaten}}
	\label{geodaeten:figure:Linienelemente:Kugelkoordinaten:Kugelkoordinaten}
\end{figure}


