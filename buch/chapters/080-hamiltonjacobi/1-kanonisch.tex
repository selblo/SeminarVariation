%
% 1-kanonisch.tex
%
% (c) 2024 Prof Dr Andreas Müller
%
\section{Kanonische Variablen
\label{buch:hamiltonjacobi:section:kanonisch}}
\kopfrechts{Kanonische Variablen}
In den Transversalitätsbedingungen taucht immer wieder Kombination
\[
F - y\frac{\partial F}{\partial y'}.
\]
Dieser Abschnitt geht daher der Frage nach, ob die Ableitung nach
$y'$ eine besondere Bedeutung haben.
Tatsächlich können diese Ableitungen als neue Koordinaten verwendet
werden, mit denen sich die Euler-Lagrange-Differentialgleichung, die
von zweiter Ordnung ist, in eine Differentialgleichung erster Ordnung
umwandeln lässt.





