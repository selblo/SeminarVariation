%
% kreis.tex -- template for standalon tikz images
%
% (c) 2021 Prof Dr Andreas Müller, OST Ostschweizer Fachhochschule
%
\documentclass[tikz]{standalone}
\usepackage{amsmath}
\usepackage{times}
\usepackage{txfonts}
\usepackage{pgfplots}
\usepackage{csvsimple}
\usetikzlibrary{arrows,intersections,math}
\definecolor{darkred}{rgb}{0.8,0,0}
\begin{document}
\def\skala{1}
\def\dx{4.5}
\def\dy{4.5}
\input{kreispfad.tex}
\begin{tikzpicture}[>=latex,thick,scale=\skala]

\draw[color=blue!40] \pfada;
\draw[color=blue!40] \pfadb;
\draw[color=blue!40] \pfadc;
\draw[color=blue!40] \pfadd;
\draw[color=blue!40] \pfade;
\draw[color=blue!40] \pfadf;
\draw[color=blue!40] \pfadg;
\draw[color=blue!40] \pfadh;
\draw[color=blue!40] \pfadi;
\draw[color=blue!40] \pfadj;
\draw[color=blue!40] \pfadk;
\draw[color=blue!40] \pfadl;
\draw[color=blue!40] \pfadm;
\draw[color=blue!40] \pfadn;
\draw[color=blue!40] \pfado;
\draw[color=blue!40] \pfadp;
\draw[color=blue!40] \pfadq;

\faechera
\faecherb
\faecherc
\faecherd
\faechere
\faecherf
\faecherg
\faecherh
\faecheri
\faecherj
\faecherk
\faecherl
\faecherm
\faechern
\faechero
\faecherp
\faecherq

\draw[color=darkred!40,line width=1.2pt] \kreisfaechera;
\draw[color=darkred!40,line width=1.2pt] \kreisfaecherb;
\draw[color=darkred!40,line width=1.2pt] \kreisfaecherc;
\draw[color=darkred!40,line width=1.2pt] \kreisfaecherd;
\draw[color=darkred!40,line width=1.2pt] \kreisfaechere;
\draw[color=darkred!40,line width=1.2pt] \kreisfaecherf;
\draw[color=darkred!40,line width=1.2pt] \kreisfaecherg;
\draw[color=darkred!40,line width=1.2pt] \kreisfaecherh;
\draw[color=darkred!40,line width=1.2pt] \kreisfaecheri;
\draw[color=darkred!40,line width=1.2pt] \kreisfaecherj;
\draw[color=darkred!40,line width=1.2pt] \kreisfaecherk;
\draw[color=darkred!40,line width=1.2pt] \kreisfaecherl;
\draw[color=darkred!40,line width=1.2pt] \kreisfaecherm;
\draw[color=darkred!40,line width=1.2pt] \kreisfaechern;
\draw[color=darkred!40,line width=1.2pt] \kreisfaechero;
\draw[color=darkred!40,line width=1.2pt] \kreisfaecherp;
\draw[color=darkred!40,line width=1.2pt] \kreisfaecherq;

\draw[color=darkred,line width=1.4pt] \kreiseins;
\draw[color=darkred,line width=1.4pt] \kreiszwei;

\draw[->,color=black] (-0.1,0) -- ({2.65*\dx},0) coordinate[label={$x$}];
\draw[->,color=black] (0,{-0.8*\dy}) -- (0,{1.40*\dy}) coordinate[label={right:$y$}];

\end{tikzpicture}
\end{document}

