%
% chapter.tex
%
% (c) 2023 Prof Dr Andreas Müller
%
\chapter{Zweite Variation
\label{buch:chapter:variation2}}
\kopflinks{Zweite Variation}
Mit Hilfe der zweiten Ableitung ist es bei einem Extremalproblem
mit endlich vielen Variablen möglich, hinreichende Bedingungen dafür
zu formulieren, dass ein Maximum oder Minimum vorliegt.
Es liegt nahe, solche Kriterien auch für Variationsproblem
zu erwarten.
Schliesslich lässt sich auch die Variation als Funktion
$f(\varepsilon) = I(y+\varepsilon\eta)$ als Potenzreihe
\[
f(\varepsilon)
=
I(y) + \varepsilon \delta I(y) + \frac{f''(0)}{2!}\varepsilon^2 + \dots
\]
entwickeln.
Notwendig für ein Extremum ist das verschwinden der ersten Variation
$\delta I(y)=0$, wie das in früheren Kapiteln ausführlich verwendet
worden ist.
Falls ausserdem $f''(0)>0$ ist, kann man schliessen, dass die
Funktion $f(\varepsilon)$ ein Minimum an der Stelle $\varepsilon=0$
hat.
Dies garantiert aber noch nicht, dass das Funktional $I(y)$
ein Minimum hat, dazu müssen alle denkbaren Funktionen $\eta(x)$
untersucht werden, also eine unendliche Menge von ``Richtungen''.

Für Funktionen $\mathbb{R}^n\to\mathbb{R}: x\mapsto f(x)$ von
$n$ Variablen sind Richtungsableitungen in alle Richtungen zu
untersuchen.
Die Taylor-Entwicklung an der Stelle $x=0$
\[
f(x)
=
f(0)
+
\operatorname{grad}f(0)\cdot x
+
\frac{1}{2!}\transpose{x} H x
+
\dots,
\]
wobei $H$ die Hessische Matrix mit den Matrixelementen
\[
h_{ik}
=
\frac{\partial^2\! f(0)}{\partial x_i\,\partial x_k}
\]
ist.
Für ein Minimum liegt vor, wenn $\operatorname{grad}f(0)=0$ und ausserdem
$\transpose{x}Hx>0$ für alle Vektoren $x\in\mathbb{R}^n\setminus\{0\}$.
Eine solche Matrix $H$ heisst positiv definit.
Die Matrix $H$ ist symmetrisch und damit immer diagonalisierbar.
Dass sie positiv definit ist, äussert sich darin, dass alle Eigenwerte
positiv sind.
Der kleinste Eigenwert beschreibt, wie schnell $f(x)$ mindestens
anwächst, es gibt so etwas wie eine minimale Krümmung des Graphen
in der Stelle $x=0$ Nullpunkt.

Es stellt sich aber heraus, dass es für Variationsprobleme viel
schwieriger ist, hinreichende Bedingungen für ein Extremum zu finden.
Da es unendlich viele mögliche Richtungen gibt, kann zwar für alle
möglichen Funktion $\eta$ die zweite Ableitung $f''(0)>0$ sein, aber 
sie kann trotzdem beliebig klein werden.
Damit lässt sich nicht mehr schliessen, dass der Punkt tatsächlich
ein Minimum ist.

In diesem Abschnitt werden zunächst aus der sogenannten zweiten
Variation notwendige Bedingungen für ein Maximum oder Minimum
hergeleitet.
Nach einigen Vorbereitungen über die Eigenschaften von Nullstellen
von Lösungen von linearen Differentialgleichungen zweiter Ordnung
wird dann die Jacobi-Differentialgleichung und das daraus ableitbare
Jacobi-Kriterium für ein Maximum oder Minumum vorgestellt.

%
% 2-zweitevariation.tex
%
% (c) 2023 Prof Dr Andreas Müller
%
\section{Die zweite Ableitung
\label{buch:variation2:section:zweiteableitung}}
\kopfrechts{Zweite Ableitung}

\begin{verbatim}
- zweite Ableitung
\end{verbatim}

%
% 2-legendre.tex
%
% (c) 2024 Prof Dr Andreas Müller
%
\section{Die Legendre-Bedingung
\label{buch:variation2:section:legendre}}
\kopfrechts{Die Legendre-Bedingung}
Der Ausdruck \eqref{buch:variation2:zweitevariation:eqn:SRintegral}
für die zweite Variation ist sicher positiv, wenn sowohl $S(x)$
als auch $R(x)$ auf dem ganzen Intervall positiv sind.
In diesem Abschnitt soll untersucht werden, inwieweit die
Umkehrung auch gilt.

\begin{satz}[Legendre]
\label{buch:variation2:legendre:satz:legendre}
Dafür, dass $\delta^2I(y)\ge 0$ für jede Funktion $\eta(x)$ gilt, ist
notwendig, dass
\[
R(x)
=
\frac{\partial^2F}{\partial y^{\prime 2}}(x,y(x),y'(x))
\ge
0
\]
im ganzen Intervall $[x_0,x_1]$ ist.
\end{satz}

\begin{definition}[Legendre-Bedingung]
Man sagt, eine Lagrange-Funktion $F(x,y,y')$ erfüllt die
{\em Legendre-Bedingung}
\index{Legendre-Bedingung}%
für eine Funktion $y(x)$, wenn 
\[
\frac{\partial^2 F}{\partial y^{\prime 2}}(x,y(x),y'(x)) \ge 0.
\]
$F$ erfüllt sogar die {\em starke Legendre-Bedingung}, wenn
\[
\frac{\partial^2 F}{\partial y^{\prime 2}}(x,y(x),y'(x)) > 0
\]
gilt.
\end{definition}

Der Satz~\ref{buch:variation2:legendre:satz:legendre} besagt also,
dass dafür, dass eine Lösung $y(x)$ der
Euler-Lagrange-Diffe\-ren\-tial\-gleichung
ein lokales Minimum des Funktionals ist, die Lagrange-Funktion
die Legendre-Bedingung für die Funktion $y(x)$ erfüllt.
Da sie auch $R(x)=0$ erlaubt, kann man nicht erwarten, dass sie auch
hinreichend ist.
Die später in Abschnitt~\ref{buch:variation2:section:jacobi} gezeigt wird,
verlangt das hinreichende Jacobi-Kriterium, dass sogar die starke
Lagrange-Bedingung erfüllt ist.

\begin{proof}
%
% legendre.tex
%
% (c) 2024 Prof Dr Andreas Müller
%
\begin{figure}
\centering
\includegraphics{chapters/060-variation2/images/legendre.pdf}
\caption{Herleitung des Legendre-Kriteriums (Beweis von Satz~\ref{buch:variation2:legendre:satz:legendre})
\label{buch:variation2:legendre:fig:legendre}}
\end{figure}

Wir zeigen mit Hilfe eines Widerspruchs, dass es keine Stelle $x$
geben kann, für die $R(x)<0$ ist.
Nehmen wir also an, es gäbe eine Stelle $x_*$ derart, dass
$R(x_*)<0$.
Da die Funktion $R(x)$ stetig ist, wird $R(x)$ auch noch in einer
Umgebung $(x_*-d,x_*+d)$ negativ sind.
Wir dürfen daher annehmen, dass es eine positive Zahl $a$ gibt
derart, dass $R(x)<-a$ für $x$ mit $|x-x_*|<d$.

Wir müssen jetzt zeigen, dass sich immer eine Funktion $\eta(x)$ finden
lässt, die das Integral $\delta^2 I(y)$ negativ macht, selbst wenn
der Koeffizient $S(x)$ auf dem Teilintervall
$(x_*-d,x_*+d)$ positiv ist.
Wir müssen zeigen, dass $\eta(x)$ so gewählt werden kann, dass der
Term mit $\eta(x)^2$ beliebig klein, der Term mit $\eta'(x)^2$ aber
beliebig negativ gemacht werden kann.

Wir verwenden die Funktion
\[
\eta(x)
=
\begin{cases}
\displaystyle
b\cos \frac{(c+\frac12)\pi (x-x_*)}{d}&\qquad |x-x_*|<d\\[3pt]
0&\qquad \text{sonst},
\end{cases}
\]
wobei $c\in\mathbb{N}$ eine natürliche Zahl ist
(Abbildung~\ref{buch:variation2:legendre:fig:legendre}).
Dies stellt sicher, dass $\eta(x)$ stetig ist,
wenn auch nicht differenzierbar an den Stellen $x_*\pm d$.

Zunächst ist $|\eta(x)|\le b$ für alle $x$, der erste Summand
des Integrals kann daher wegen
\[
\biggl|
\int_{x_0}^{x_1} S(x)\eta(x)^2\,dx
\biggr|
=
\biggl|
\int_{x_*-d}^{x_*+d} S(x)\eta(x)^2\,dx
\biggr|
\le
\int_{x_*-d}^{x_*+d} |S(x)|\cdot |\eta(x)|^2\,dx
\le
b^2
\int_{x_*-d}^{x_*+d} |S(x)|\,dx
\]
beliebig klein gemacht werden, indem $a$ klein gewählt wird.
Für die Ableitung gilt
\[
\eta'(x)
=
\begin{cases}
\displaystyle
-\frac{b(c+\frac12)\pi}{d}\sin\frac{(c+\frac12)\pi(x-x_*)}{d}&\qquad |x-x_*|<d\\[3pt]
0&\qquad\text{sonst.}
\end{cases}
\]
Der zweite Summand des Integrals wird damit zu
\begin{align*}
\int_{x_0}^{x_1} R(x) \eta'(x)^2\,dx
&=
\int_{x_*-d}^{x_*+d} R(x)\eta'(x)^2\,dx
\\
&\le 
\int_{x_*-d}^{x_*+d} (-a) \eta'(x)^2\,dx
\\
&=
-ab^2\int_{x_*-d}^{x_*+d} \frac{(c+\frac12)^2\pi^2}{d^2}
\sin^2 \frac{(c+\frac12)\pi(x-x_*)}{d}\,dx.
\intertext{Das Integral von $\sin^2 t$ über eine ganz Anzahl von
Perioden ist die halbe Intervalllänge, somit wird das Integral zu}
&=
-ab^2
\frac{(c+\frac12)^2\pi^2}{d^2}
2d
=
-\frac{ab^2(c+\frac12)^2\pi^2}{d}
\end{align*}
(siehe Lemma~\ref{buch:variation2:legendre:lemma:sin} weiter unten).
Die rechte Seite kann beliebig gross gemacht werden, indem ein
geeignet grosses $c$ gewählt wird.
So wird die zweite Variation
\begin{align}
\delta^2I(y)
=
\int_{x_0}^{x_1} S(x)\eta(x)^2 + R(x)\eta'(x)^2\,dx
&=
\int_{x_0}^{x_1} S(x)\eta(x)^2\,dx
+
\int_{x_0}^{x_1} R(x)\eta'(x)^2\,dx
\notag
\\
&\le
\int_{x_*-d}^{x_*+d}|S(x)|\eta(x)\,dx
+
\int_{x_*-d}^{x_*+d} R(x) \eta(x)\,dx
\notag
\\
&\le
b^2\int_{x_*-d}^{x_*+d} |S(x)|\,dx
-
\frac{ab^2(c+\frac12)^2\pi^2}{d}
\notag
\\
&=
\frac{ab^2\pi^2}{d}
\biggl(
\frac{d}{a\pi^2}
\int_{x_*-d}^{x_*+d}|S(x)|\,dx
-
(c+{\textstyle\frac12})^2
\biggr).
\label{buch:variation2:legendre:eqn:bgl}
\end{align}
Indem man 
\[
(c+{\textstyle\frac12})^2
>
\frac{d}{a\pi^2}\int_{x_*-d}^{x_*+d} |S(x)|\,dx
\]
wählt, wird die Klammer in \eqref{buch:variation2:legendre:eqn:bgl}
negativ, so dass die zweite Variation negativ wird.
Dies widerspricht der Annahme über die zweite Variation.
Der Widerspruch zeigt, dass $R(x)\ge 0$ sein muss.
\end{proof}

\begin{lemma}
\label{buch:variation2:legendre:lemma:sin}
Für $c\in\mathbb{N}$ gilt
\[
\int_{-d}^d
\sin^2 \frac{(c+\frac12)\pi t}{d}
\,dt
=
d.
\]
\end{lemma}

\begin{proof}
%
% sin2.tex -- Oszillationen von sin^2
%
% (c) 2021 Prof Dr Andreas Müller, OST Ostschweizer Fachhochschule
%
\documentclass[tikz]{standalone}
\usepackage{amsmath}
\usepackage{times}
\usepackage{txfonts}
\usepackage{pgfplots}
\usepackage{csvsimple}
\usetikzlibrary{arrows,intersections,math}
\definecolor{darkred}{rgb}{0.8,0,0}
\begin{document}
\def\skala{1}
\def\d{4.8}
\def\a{4}
\def\b{3}
\begin{tikzpicture}[>=latex,thick,scale=\skala,
declare function={
	f(\x) = \a*sin((\b+0.5)*180*\x/\d);
	g(\x) = f(\x)*f(\x)/\a;
}]

\fill[color=gray!20] (-\d,0) rectangle (\d,\a);
\fill[color=darkred!20]
	(-\d,0)
	--
	plot[domain=-\d:\d,samples=100]
	({\x},{g(\x)})
	--
	(\d,0)
	--
	cycle;

\draw[color=blue,smooth,line width=1.2pt]
	plot[domain=-\d:\d,samples=100] ({\x},{f(\x)});
\draw[color=darkred,smooth,line width=1.2pt]
	plot[domain=-\d:\d,samples=100] ({\x},{g(\x)});

\draw[->] (-5,0) -- (7.3,0) coordinate[label={$t$}];
\draw[->] (0,-4.1) -- (0,4.5) coordinate[label={right:$y$}];

\draw (-0.05,4) -- (0.05,4);
\node at (-0.05,4) [above left] {$1$};
\draw (-0.05,-4) -- (0.05,-4);
\node at (-0.05,-4) [left] {$-1$};

\draw (-\d,-0.05) -- (-\d,0.05);
\node at (-\d,-0.05) [below] {$-d\mathstrut$};
\draw (\d,-0.05) -- (\d,0.05);
\node at (\d,-0.05) [below] {$d\mathstrut$};

\coordinate (A) at ({-3*\d/7},{-6*\a/7});
\node[color=blue] at (A) {$\displaystyle\sin\frac{(c+\frac12)\pi t}{d}$};
\draw[color=blue,line width=0.2pt,shorten <= 0.6cm]
	(A) -- ({-9*\d/14},{f(-9*\d/14)});

\coordinate (B) at (6.0,{3*\a/4});
\node[color=darkred] at (B) {$\displaystyle\sin^2\frac{(c+\frac12)\pi t}{d}$};
\draw[color=darkred,line width=0.2pt,shorten <= 0.65cm]
	(B) -- ({13*\d/14},{g(13*\d/14)});

\end{tikzpicture}
\end{document}


Wegen
\[
\sin^2 t
=
\frac12 - \frac12\cos 2t
\]
ist das Integral die Fläche unter einer um $\frac12$ nach oben
verschobenen Cosinus-Kurve mit doppelter Frequenz.
Diese Kurve halbiert das Rechteck mit den Ecken $(-d,0)$ und $(d,1)$
mit dem Flächeninhalt $2d$.
Daher ist der Wert des Integrals $2d/2=d$.
\end{proof}

\begin{beispiel}
\label{buch:variation2:legendre:bsp:ebene}
Das Variationsproblem mit der Legendre-Funktion
$L(x,y,y')=\sqrt{1+y^{\prime 2}}$ findet kürzeste Verbindungen in
der Ebene.
Zur Überprüfung der Legendre-Bedingung muss die zweite Ableitung von
$L$ nach $y'$ berechnet werden.
Die erste Ableitung ist
\begin{align}
\frac{\partial L}{\partial y'}
&=
\frac{y'}{\sqrt{1+y^{\prime 2}}}
\notag
\intertext{und die zweite mit der Quotientenregel}
\frac{\partial^2 L}{\partial y^{\prime 2}}
&=
\frac{1\cdot \sqrt{1+y^{\prime 2}}}{1+y^{\prime 2}}
-
\frac{y'\cdot y'}{(1+y^{\prime 2})^{\frac32}}
=
\frac{
(1+y^{\prime 2})  - y^{\prime 2}
}{
(1+y^{\prime 2})^{\frac32}
}
=
\frac{1}{
(1+y^{\prime 2})^{\frac32}
}
> 0.
\label{buch:variation2:legendre:bsp:ebene:R}
\end{align}
Die starke Legrendre-Bedingung ist also immer erfüllt.
\end{beispiel}

%
% 3-diffgl.tex
%
% (c) 2024 Prof Dr Andreas Müller
%
\section{Nullstellen von Lösungen linearer Differentialgleichungen
\label{buch:variation2:section:diffgl}}
\kopfrechts{Nullstellen von Lösungen}
In Abschnitt~\ref{buch:variation2:section:jacobi} wird das
Jacobi-Kriterium für die Existenz eines Extremus formuliert.
Es basiert auf Eigenschaften der Lösungen der Jacobi-Differentialgleichung,
einer linearen Differentialgleichung zweiter Ordnung.
Diese Eigenschaften sind nicht spezifisch für das Variationsproblem
und werden daher in diesem Abschnitt allgemein dargestellt.

Wir betrachten die lineare Differentialgleichung zweiter Ordnung
\begin{equation}
y'' + p(x) y' + q(x) = 0.
\label{buch:variation2:diffgl:eqn:dgl}
\end{equation}
Die Koeffizienten $p(x)$ und $q(x)$ sind stetige Funktionen auf dem
abgeschlossenen Intervall $[x_0,x_1]$.
In expliziter Form kann sie auch als
\begin{equation}
y'' = - p(x) y' - q(x)
\label{buch:variation2:diffgl:eqn:explizit}
\end{equation}
geschrieben werden.
Für die numerische Rechnung und für Fragen über Existenz und Eindeutigkeit
der Lösung ist die Vektorschreibeise
\begin{equation}
\frac{d}{dx}
\begin{pmatrix} y\\ y' \end{pmatrix}
=
\underbrace{
\begin{pmatrix}
    0&    1\\
-q(x)&-p(x)
\end{pmatrix}
}_{\displaystyle =A(x)}
\begin{pmatrix} y\\ y' \end{pmatrix}
\qquad\Rightarrow\qquad
\frac{d}{dx}v = A(x) v
\label{buch:variation2:diffgl:eqn:vektordgl}
\end{equation}
der Differentialgleichung nützlicher.
Der Satz von Picard und Lindelöf besagt, dass die Differentialgleichung
\eqref{buch:variation2:diffgl:eqn:vektordgl} eine eindeutige
Lösung hat, wenn die rechte Seite eine Lipschitz-Bedingung
erfüllt.
Die Lipschitz-Bedingung verlangt im vorliegenden Fall, dass
\[
|A(x)v_1 - A(x)v_2|
\le
L |v_1-v_2|
\]
für eine geeignete Lipschitz-Konstante $L$ gilt.
Wegen
\[
|A(x)v_1-A(x)v_2|
=
|A(x)(v_1-v_2)|
\le
\|A(x)\|\cdot |v_1-v_2|
\]
ist die Lipschitz-Bedingung mit der Lipschitz-Konstanten
$L=\|A(x)\|$ immer erfüllt.
Zu vorgegebenen Anfangswerten $y_0$ und $y'_0$ gibt es also
immer eine eindeutige Lösung $y(x)$ mit $y(x_0)=y_0$ und
$y'(x_0)=y_0$.

%
% Keine mehrfachen Nullstellen
%
\subsection{Keine mehrfachen Nullstellen
\label{buch:variation2:diffgl:subsection:mehrfachenullstellen}}
Sei $y(x)$ eine nichttriviale Lösung der Differentialgleichung
\eqref{buch:variation2:diffgl:eqn:dgl} und $x_0$ eine
Nullstelle $y(x_0)=0$.
Die Stelle $x=x_0$ kann keine mehrfache Nullstelle sein, denn dann
wäre auch $y'(x_0)=0$, $y(x)$ wäre also eine Lösung der Differentialgleichung
mit Anfangsbedingungen $y(x_0)=0$ und $y'(x_0)=0$. 
Da die Nullfunktion eine Lösung mit den selben Anfangsbedingungen ist,
folgt aus dem Satz von Picard-Lindelöf, dass $y(x)=0$ sein müsste, im
Widerspruch zur Annahme, dass $y(x)$ eine nichttriviale Lösung ist.

Für jede Nullstelle $y_0$ von $y(x)$ ist also $y(x_0)\ne 0$.
Insbesondere wechselt $y(x)$ bei jeder Nullstelle das Vorzeichen.


%
% Linear unabhängige Lösungen
%
\subsection{Linear unabhängige Lösungen
\label{buch:variation2:diffgl:subsetion:linunabh}}
Seien jetzt $y_1(x)$ und $y_2(x)$ zwei Lösungen
der Differentialgleichung~\eqref{buch:variation2:diffgl:eqn:dgl}.
Falls an einer Stelle $x_0$ die Vektoren
\[
v_1(x_0)
=
\begin{pmatrix}
y_1(x_0)\\
y'_1(x_0)
\end{pmatrix}
\qquad\text{und}\qquad
v_2(x_0)
=
\begin{pmatrix}
y_2(x_0)\\
y'_2(x_0)
\end{pmatrix}
\]
linear abhängig sind, dann gibt es eine Zahl $a$ mit $av_1(x_0)=v_2(x_0)$.
Dies bedeutet, dass die Funktion $ay_1(x)$ an der Stelle $x_0$ 
die gleichen Anfangsbedingungen erfüllt wie $y_2(x)$, nach dem
Satz von Picard-Lindelöf folgt daher, dass $y_2(x)=ay_1(x)$ für 
\index{Picard-Lindelof@Picard-Lindelöf, Satz von}%
alle $x$.
Die Vektoren $v_1(x)$ und $v_2(x)$ sind daher immer für alle
$x$ linear abhängig und die Funktionen $y_1(x)$ und $y_2(x)$ haben
die gleichen Nullstellen.

Wir betrachten jetzt zwei linear unabhängige Lösungen
$y_1(x)$ und $y_2(x)$.
Die Vektoren $v_1(x)$ und $v_2(x)$ müssen dann
für alle $x$ linear unabhängig sein, denn wären sie an einer Stelle
linear abhängig und müssten damit nach dem vorangegangenen Absatz
überall linear abhängig sein.
Insbesondere ist die Wronski-Determinante
\index{Wronski-Determinante}%
\[
W(y_1,y_2)
=
\det(v_1,v_2)
=
\left|
\begin{matrix}
 y_1(x) &  y_2(x) \\
y'_1(x) & y'_2(x)
\end{matrix}
\right|
\ne
0.
\]
Insbesondere wechselt die Wronski-Determinante das Vorzeichen nicht.

Die Wronski-Determinante hat die Ableitung
\begin{align*}
\frac{d}{dx} W(y_1(x),y_2(x))
&=
\frac{d}{dx} \bigl(y_1(x)y'_2(x) - y_2(x)y'_1(x)\bigr)
\\
&=
y'_1(x)y'_2(x) + y_1(x)y''_2(x) - y'_2(x)y'_1(x) - y_2(x)y''_1(x)
\\
&=
y_1(x)y''_2(x) - y_2(x)y''_1(x).
\intertext{Durch Einsetzen der explizienten Form
\eqref{buch:variation2:diffgl:eqn:explizit} der
Differentialgleichung für $y''(x)$ wird daraus}
&=
y_1(x)(-p(x)y'_2(x)-q(x)) - y_2(x)(-p(x)y'_1(x)-q(x))
\\
&=
-p(x)
\bigl(
y_1(x)y'_2(x) - y_2(x)y'_1(x)
\bigr)
\\
&=
-p(x) W(y_1(x),y_2(x))
\end{align*}
Die Wronski-Determinante erfüllt also die Differentialgleichung
\[
u' = -p(x) u,
\]
die sich separieren lässt, indem man sie
\[
\frac{du}{u}
=
-p(x) \,dx
\qquad\Rightarrow\qquad
\int \frac{du}{u}
=
-\int p(x)\,dx
\]
schreibt.
Ihre Lösung
\[
\log |u(x)|
=
-\int p(x)\,dx
\qquad\Rightarrow\qquad
u(x) = u_0 e^{-\int_{x_0}^s p(\xi)\,d\xi}
\]
erfüllt die Anfangsbedingungen $u(x_0)=u_0$.
Es folgt, dass die Wronski-Determinante
\[
W(y_1(x),y_2(x))
=
W(y_1(x_0),y_2(x_0))
e^{-\int_{x_0}^x p(\xi)\,d\xi}
\]
erfüllt.
Da die Exponentialfunktion keine Nullstellen hat, ergibt sich erneut
die Eigenschaft, dass die Wronski-Determinante das Vorzeichen nicht
wechselt.

%
% Alternierende Nullstellen
%
\subsection{Alternierende Nullstellen
\label{buch:variation2:diffgl:subsection:alternierendenullstellen}}
Seien wieder $y_1(x)$ und $y_2(x)$ zwei linear unabhängige Lösungen
der Differentialgleichung \eqref{buch:variation2:diffgl:eqn:dgl}.
Da die Funktionen linear unab hängig sind, kann der
Quotient $y_2(x)/y_1(x)$ nicht konstant sein.
Wir berechnen daher
\[
\frac{d}{dx}\biggl(
\frac{y_2(x)}{y_1(x)}
\biggr)
=
\frac{y_2'(x)y_1(x)-y_1'(x)y_2(x)}{y_1(x)^2}
=
-
\frac{W(y_1(x),y_2(x))}{y_1(x)^2}
=
-
\frac{\Delta_0}{y_1(x)^2}
e^{-\int_{x_0}^x p(\xi)\,d\xi}
\]
mit $\Delta_0 = W(y_1(x_0),y_2(x_0))$.
Die Ableitung des Quotienten hat daher im ganzen Definitionsbereich
keine Vorzeichenwechsel.
Der Quotient ist daher monoton wachsend oder monoton fallend im 
ganzen Definitionsbereich.

Bei den Nullstellen der Funktion $y_1(x)$ hat der Quotient
$y_2(x)/y_1(x)$ Pole, auf beiden Seiten einer solchen Nullstelle
divergiert der Quotient mit verschiedenem Vorzeichen.
Da der Quotient zwischen zwei solchen Nullstellen monoton, kann es
nur eine einzige Nullstelle geben.
Folglich hat $y_2(x)$ zwischen zwei Nullstellen von $y_1(x)$ genau
eine Nullstelle.
Dasselbe gilt mit vertauschten Rollen von $y_1(x)$ und $y_2(x)$
auch für $y_1(x)$, diese Funktion hat zwischen zwei Nullstellen
von $y_2(x)$ genau eine Nullstelle.
Die Funktionen $y_1(x)$ und $y_2(x)$ haben also alternierende
Nullstellen.




%
% 4-jacobi.tex
%
% (c) 2024 Prof Dr Andreas Müller
%
\section{Jacobi-Kriterium
\label{buch:variation2:section:jacobi}}
Die zweite Variation in der Form
\eqref{buch:variation2:zweitevariation:eqn:SRintegral}
zeigt, dass es für die Beurteilung, ob tatsächlich ein Extremum 
vorliegt, auf das Verhalten der Funktion $S(x)$ ankommt.
In einem Minimum garantiert die Legendre-Bedingung bereits,
dass $R(x)\ge 0$ ist.
Es bleibt noch zu untersuchen, ob sich Bedinungen an die Funktion
$S(x)$ finden lassen, die garantieren können, dass ein Minimum
vorliegt.
Zu diesem Zweck betrachten wir das Funktional
\begin{equation}
K(u)
=
\int_{x_0}^{x_1}
S(x) u(x)^2 + R(x) u'(x)^2
\,dx
\label{buch:variation2:jacobi:eqn:K}
\end{equation}
mit Randbedingungen $u(x_0)=u(x_1)=0$ und versuchen das Minimum
zu bestimmen.
Für $u(x)=0$ ist $K(0)=0$.
Sollte es aber Funktionen $u(x)$ geben, die $K(u)$ negativ machen,
dann kann die Funktion $y(x)$ von der ausgehen die Funktionen $S(x)$
und $R(x)$ konstruiert wurden, kein Minimum sein.

Die Euler-Lagrange-Differentialgleichung des Funktionals
\eqref{buch:variation2:jacobi:eqn:K}
mit der Lagrange-Funk\-tion 
\[
G(x,u,u') = S(x) u^2 + R(x) u^{\prime 2}
\]
können wir nach trivialen Vereinfachungen
\[
L(u)
=
\frac{d}{dx}\frac{\partial G}{\partial y'}
-
\frac{\partial G}{\partial y}
=
\frac{d}{dx}(Ru') 
-
Su
=
0
\]
schreiben,
oder auch
\begin{equation}
\frac{d}{dx} R(x)\frac{d}{dx} u(x) - S(x) u(x) = 0,
\label{buch:variation2:jacobi:eqn:jacobisl}
\end{equation}
eine lineare Differentialgleichung zweiter Ordnung für die Funktion $u(x)$.
Sie ist ausserdem von der Form einer Sturm-Liouville-Differentialgleichung,
wie sie in Abschnitt~\ref{buch:variation2:section:diffgl} genauer
untersucht wurden.

Ist die Funktion $u(x)$ eine Lösung der
Differentialgleichung~\eqref{buch:variation2:jacobi:eqn:jacobisl}
mit $u(x_0)=u(x_1)=0$, dann ist auch jedes Vielfache von $u(x)$
eine Lösung mit den gleichen Randbedingungen, und die zweite Variation
des ursprünglichen Funktionals verschwindet für alle Werte von $\varepsilon$.
Falls es keine solche Funktion $u(x)$ gibt, dann ist die zweite
Variation $\ne 0$.
Die Eigenschaft, ob ein Minimum oder Maximum vorliegt, wird also
dadurch entschieden, ob es eine Lösung der Differentialgleichung
\eqref{buch:variation2:jacobi:eqn:jacobisl} mit
Randwerten $u(x_0)=u(x_1)=1$ gibt.

Ist $u(x)$ eine nichttriviale Lösung der Differentialgleichung
\eqref{buch:variation2:jacobi:eqn:jacobisl} mit $u(x_0)=0$,
dann muss $u'(x_0)\ne 0$ sein.
Durch Division druch $u'(x_0)$ erhalten wir eine Lösung, die
sogar $u'(x_0)=1$ hat.
Durch diese Anfangswerte ist aber die Lösung der Differentialgleichung
vollständig bestimmt.
Die Frage ob ein Minimum oder Maximum vorliegt, wird also
durch die Lage der Nullstellen von Lösungen der Differentialgleichung
\eqref{buch:variation2:jacobi:eqn:jacobisl} bestimmt.

\begin{definition}[Jacobi-Differentialgleichung]
Die Differentialgleichung
\eqref{buch:variation2:jacobi:eqn:jacobisl}
heisst die {\em Jacobi-Differentialgleichung}.
\end{definition}

%Aus der Definition des Funktionals $K(u)$ erhält man durch
%partielle Integration für Funktionen mit den Randbedingungen
%$u(x_0)=u(x_1)=0$ 
%\begin{align*}
%K(u)
%&=
%\int_{x_0}^{x_1} S(x)u(x)^2 + R(x) u'(x)^2 \,dx
%\\
%&=
%\int_{x_0}^{x_1} S(x)u(x)^2 \,dx
%+
%\int_{x_0}^{x_1} R(x) u'(x)^2 \,dx
%\\
%&=
%\int_{x_0}^{x_1} S(x)u(x)^2 \,dx
%+
%\int_{x_0}^{x_1} (R(x) u'(x)) u'(x) \,dx
%\\
%&=
%\int_{x_0}^{x_1} S(x)u(x)^2 \,dx
%+
%\biggl[R(x)u'(x)u(x)\biggr]_{x_0}^{x_1}
%-
%\int_{x_0}^{x_1} \frac{d}{dx}(R(x) u'(x)) u(x) \,dx
%\\
%&=
%\int_{x_0}^{x_1}
%\biggl(S(x) u(x) - \frac{d}{dx}\bigl(R(x)u'(x)\bigr)\biggr) u(x)
%\,dx
%\\
%&=
%-\int_{x_0}^{x_1} L(u) u(x) \,dx.
%\end{align*}







