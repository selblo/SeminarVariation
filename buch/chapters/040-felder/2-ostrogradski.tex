%
% 2-ostrogradski.tex
%
% (c) 2023 Prof Dr Andreas Müller
%
\section{Die Euler-Ostrogradski-Differentialgleichung
\label{buch:felder:section:euler-ostrogradski}}
\kopfrechts{Die Euler-Ostrogradski-Differentialgleichung}
Die in den vorangegangenen Abschnitten entwickelte Theorie der
Integration von Funktionen und Vektorfeldern für mehrere Variablen
soll jetzt dazu verwendet werden, Variationsprinzipien für solche
Felder zu formulieren und mit Hilfe einer der
Euler-Lagrange-Differentialgleichung ähnlichen partiellen
Differentialgleichung zu lösen.

% 
% Variationsprinzipien für Felder
%
\subsection{Variationsprinzipien für Felder}
In diesem Abschnitt werden drei Beispiele von Variationsprinzipien aus
der Physik vorgestellt.
Dabei geht es vor allem um die Illustration der Gemeinsamkeiten,
die anschliessend als Basis für eine nützliche Definition der
Funktionale dienen können, für die das Variationsproblem gelöst
werden soll.

%
% Gravitationsfeld
%
\subsubsection{Graviationsfeld}
Das newtonsche Gravitaionsgesetz berechnet die Kraft
\[
F = G \frac{m_1m_2}{r^2}
\]
zwischen zwei Massen $m_1$ und $m_2$ im Abstand $r$.
Befindet sich die Masse $m_1$ im Nullpunkt, dann erfährt die Masse im
Punkt $P$ im Abstand $r$ vom Nullpunkt die Kraft 
\begin{equation}
\vec{F}
=
m_2
\cdot
\underbrace{\frac{Gm_1}{r^2} \overrightarrow{PO}}_{\displaystyle =\vec{a}}.
\label{buch:felder:ostrogradski:eqn:gravitationsfeld}
\end{equation}
Alle Massen erfahren also die gleiche Beschleunigung, die durch das
Feld $\vec{a}$ gegeben ist.
Auf der Erdoberfläche ist dies die Erdbeschleunigung $\vec{g}$, die jeden
Körper mit ungefähr $9.81 \text{m}/\text{s}^2$ zum Mittelpunkt der
Erde hin beschleunigt.

Aus der Gleichung \eqref{buch:felder:ostrogradski:eqn:gravitationsfeld}
kann man ableiten, dass das Gravitationsfeld ein Potentialfeld ist.
Dies bedeutet, dass es eine Funktion $\varphi$ der Koordinaten gibt, 
so dass der Gradient von $\varphi$ die Gravitationsbeschleunigung
\[
\frac{Gm_1}{r^2}\overrightarrow{PO}
=
\grad\varphi
\]
ist.
Das Feld einer ausgedehnten Masseverteilung mit Dichte $\varrho$
kann daraus durch Überlagerung gewonnen werden.
Man kann auch zeigen, dass das Potential einer solchen Verteilung
die Laplace-Gleichung $\Delta \varphi = 4\pi G \varrho$ erfüllt.

Die Gleichung des Gravitationsfeldes lässt sich aber auch aus einem
Variationsprinzip gewinnen.
Die Lagrange-Mechanik eines Massepunktes kann helfen, die richtige
Lagrange-Funktion
\begin{equation}
L(x,\varrho,\varphi)
=
-\frac{1}{8\pi G}
\sum_{k=1}^3
\biggl(
\frac{\partial \varphi}{\partial x_k}(x)
\biggr)^2
-
\varrho(x)\varphi(x)
\label{buch:felder:ostrogradski:eqn:gravitationslagrange}
\end{equation}
zu finden.
Der potentiellen Energie $mgh$ in der Mechanik des Massepunktes
entspricht die potentielle Energiedichte $\varrho(x)\varphi(x)$
in der
Lagrangefunktion~\eqref{buch:felder:ostrogradski:eqn:gravitationslagrange}.
Wir werden später sehen, dass sich die Feldgleichungen für das
Gravitationsfeld dadurch finden lassen, dass man das Minimum des
Funktionals
\[
I(x,\varrho,\varphi)
=
\int_{\mathbb{R}^3} L(\varphi)
\,dx
\]
sucht.

%
% Elektromagnetisches Feld
%
\subsubsection{Elektromagnetisches Feld}
Das elektromagnetische Feld ist viel komplizierter als das Gravitationsfeld.
Das elektrostatische Feld einer Punktladung kann zwar ganz analog zum
Gravitationsfeld modelliert werden, aber die Bewegung einer Ladung
verursacht zusätzlich ein Magnetfeld, welches sich nur als ein
Vektorfeld beschreiben lässt.
Die magnetische Induktion $\vec{B}$ kann nicht als Gradient einer Funktion
geschrieben werden.
Da sie aber quellenfrei ist, kann man ein Vektorfeld $\vec{A}$,
das sogenannte Vektorpotential finden, aus welchem sich das $\vec{B}$-Feld
durch Bilden der Rotation $\operatorname{rot}\vec{A}=\vec{B}$ bilden
lässt.

Das elektrische Feld wird durch Ladungen erzeugt, die mit der
Ladungsdichte $\varrho(x,t)$ im Raum verteilt sind.
Die Bewegung der Ladung, die für die Magnetfelder verantwortlich ist,
äussert sich in Strömen, die durch den Stromdichtevektor $\vec{\jmath}(x,t)$
beschrieben werden.

Der Zusammenhänge zwischen $\varphi$ und $\vec{\jmath}$ und den Feldern
$\vec{E}$ und $\vec{B}$ lässt sich jetzt mit einer Lagrange-Funktion
\begin{equation}
L(x,t,\varrho,\vec{j},\varphi,\vec{A})
=
-\varrho(x,t) \varphi(x,t)
+
\vec{j}(x,t)\vec{A}(x,t)
+
\frac{\varepsilon_0}{2}\vec{E}(x,t)^2
-
\frac{1}{2\mu_0}\vec{B}(x,t)^2
\label{buch:felder:ostrogradski:eqn:emlagrange}
\end{equation}
erfassen.
\eqref{buch:felder:ostrogradski:eqn:emlagrange} ist tatsächlich nur
eine eine Funktion von der angegebenen Argumente, denn die Felder
$\vec{E}$ und $\vec{B}$ lassen sich aus $\varphi$ und $\vec{\jmath}$ 
berechnen.
Wir erwarten, dass sich die Maxwell-Gleichungen wieder durch
Lösung eines Variationsproblems finden lassen.

%
% Elastizitätstheorie
%
\subsubsection{Elastizitätstheorie}
In der Elastizitätstheorie wird die Deformation eines Körpers unter
dem Einfluss äusserer Kräfte durch die Verschiebungen
$u_{i}(x), i=1,\dots,3$ beschrieben.
Der Vektor mit den Komponenten $u_i(x)$ beschreibt, wie weit der Punkt
$x$ durch die Deformation bewegt wird.
Nach der Deformation befindet sich der Punkt $x$ also an der Stelle $x+u(x)$.
Die Grundaufgabe der Elastizitätstheorie ist, die Deformation $u_k(x)$
eines Körpers in Abhängigkeit von den äusseren Kräften zu berechnen.

Wenn alle Punkte mit dem gleichen Vektor verschoben werden, entstehen
im Material keine Spannungen.
Spannungen im Material manifestieren sich daher nur, wenn die Ableitungen
der Komponenten $u_i(x)$ nach den Koordinaten nicht verschwinden.
Daher wird der {\em Verzerrungstensor}
\index{Verzerrungstensor}%
durch
\begin{equation}
u_{ik}
=
\frac12\biggl(
\frac{\partial u_i}{\partial x_k}
+
\frac{\partial u_k}{\partial x_i}
\biggr)
\label{buch:felder:ostrogradski:eqn:verzerrungstensor}
\end{equation}
definiert.

Die im Körper vorhandenen Spannungen werden durch den sogenannten
Spannungstensor $\sigma_{ik}(x)$ beschrieben.
Die Diagonalkomponent $\sigma_{ii}$ beschreibt den Druck auf die 
Oberfläche eines kleinen Volumens in Richtung der $i$-ten Koordinate.
Die ausserdiagonalen Elemente $\sigma_{ik}$ beschreiben die
Scherkräfte in Richtung der $k$-ten Koordinate auf eine Oberfläche
mit Normalen in Richtung der $i$-ten Koordinate.
Eine weitere Deformation des Körpers ändert die Verzerrungen, so dass
sich zusätzliche Spannungen aufbauen.
Die bei der Deformation geleistete Arbeit wird in Form von Spannungsenergie
gespeichert.

In einer Flüssigkeit kann es keine Scherkräfte geben.
Der Spannungstensor ist diagonal und die Spannungen sind durch
den hydrostatischen Druck $p$ in der Flüssigkeit gegeben, also
durch $\sigma_{ik}=-p\delta_{ik}$.
Eine Änderung des Volumens bedeutet, dass der Verzerrungstensor
verändert wird.
Die diagonalen Komponenten geben den Streckungsfaktor in Richtung der
Koordinate $i$ an.
Die mit der Volumenänderung verbundene Arbeit gegen den Druck $p$ ist
daher durch
\[
dE
=
-p\sum_{i=1}^3 du_{ii}
\]
gegeben.

Im allgemeinen Fall eines elastischen Körpers muss bei einer
Deformationsänderung auch Arbeit gegen die Scherkräfte geleistet
werden.
Bei der Verdrehung eines Balkens ist dies der hauptsächliche Anteil
der nötigen Arbeit.
Diese Arbeit ist proportional zu den Scherspannungen und den ausserdiagonalen
Elementen des Verzerrungstensors.
Die freie Enthalpie
\[
\Phi
=
E - TS - \sum_{i,k=1}^n \sigma_{ik} u_{ik}
\]
des Körpers kann als Lagrange-Funktion für ein Minimalprinzip
verwendet werden, aus dem sich die Grundgleichungen der
Elastizitätstheorie ableiten lassen.

%
% Lagrange-Funktion
%
\subsection{Lagrange-Funktion}
Die Beispiele in
Abschnitt~\ref{buch:felder:ostrogradski:subsection:variationsprinzipien}
haben gezeigt, dass sich in der Physik Theorien über Felder verschiedenster
Art auf Variationsprinzipien zurückführen lassen.
Den Beispielen ist gemeinsam, dass die Lagrange-Funktion eines zu
variierenden Integrals nicht mehr nur von einer Funktion $y(x)$ von
einer Variablen und ihrer Ableitung $y'(x)$ abhängt, sondern von mindestens
einer Funktion $\varphi(x)$ von mehreren Variablen und allen ihren
ersten Ableitungen.

Die einfachste Form eines Variationsproblems für Funktionen mehrere
Variablen ist daher die folgende.
Gegeben ist ein Gebiet $\Omega\in\mathbb{R}$ mit Rand $\partial\Omega$.

\begin{definition}[Lagrange-Funktion]
Eine Funktion
\[
L
\colon
\mathbb{R}^n\times\mathbb{R}\times{R}^n
\to
\mathbb{R}
:
(x,u,u_x)
\mapsto
L(x,u,u_x),
\]
die in jeder Variable $n$-mal stetig diferenzierbar ist, heisst eine
{\em Lagrange-Funktion}.
Das zugehörige Funktional ist
\begin{equation}
\begin{aligned}
I(\varphi)
&=
\int_{\Omega}
L(x,\varphi,\grad\varphi(x))
\,dx
\\
&=
\int_{\Omega} 
L\biggl(
x_1,\dots,x_n,
\varphi(x_1,\dots,x_n),
\frac{\partial \varphi}{\partial x_1}(x_1,\dots,x_n),
\dots,
\frac{\partial \varphi}{\partial x_n}(x_1,\dots,x_n)
\biggr)
\,dx_1\dots\,dx_n.
\end{aligned}
\label{buch:felder:astrogradski:eqn:Lfunktional}
\end{equation}
\end{definition}

Zudem sei auf dem Rand $\partial\Omega$ auf $\Omega$
eine Randwertfunktion
$f(x)$ gegeben.
Betrachtet werden Funktionen
$\varphi\colon\overline{\Omega}\to \mathbb{R}$,
die vorgegebene Randwerte
\(
\varphi(x) = f(x)
\)
für $x\in\partial\Omega$ erfüllen.
Gesucht wird jetzt diejenige solche Funktion $\varphi(x)$, die das Funktional
$I(\varphi)$ zu einem Extremum macht.

In zwei Dimensionen für die unabhängigen Variablen $x$ und $y$
und eine gesuchte Funktion $\varphi(x,y)$ ausgeschrieben ist also
$\varphi(x,y)$ derart zu bestimmen, dass das Integral
\[
\int_{\Omega}
L\biggl(
x,y,\varphi(x,y),
\frac{\partial\varphi}{\partial x}(x,y),
\frac{\partial\varphi}{\partial y}(x,y)
\biggr)
\,dx
\]
einen Extremwert erreicht für alle Funktionen, die ausserdem die 
Randbedingungen $\varphi(x,y)=f(x,y)$ erfüllt für $(x,y)\in\partial\Omega$.

\begin{definition}[1.~Variation]
Die {\em erste Variation} des Funktionals $I(\varphi)$ von
\index{erste Variation!von mehreren Variablen}
\eqref{buch:felder:astrogradski:eqn:Lfunktional}
für eine Funktion $\eta(x)$, die auf dem Rand $\partial\Omega$ verschwindet,
ist die Richtungsableitung
\[
\delta I(\varphi)
=
\frac{\partial I(\varphi+\varepsilon \eta)}{\partial \varepsilon}
\bigg|_{\varepsilon=0}.
\]
\end{definition}

%
% Euler-Ostrogradski-Differentialgleichung
%
\subsection{Euler-Ostrogradski-Differentialgleichung}
Die Hauptzutaten für die Euler-Lagrange-Differentialgleichung für
die Extremale eines Funktionals von einer Funktion $y(x)$ einer
Variable waren das Fundamentallemma und die partielle Integration.
Beide stehen auch für Funktionen mehrere Variablen zur Verfügung.
Es ist daher nicht überraschend, dass sich eine Differentialgleichung
ähnlicher Art auch für Variationsprinzipien für Funktionen mehrere
Variablen finden lässt.

\begin{satz}[Euler-Ostrogradski]
Sei $L(x,u,u_x)$ eine Lagrange-Funktion auf dem Gebiet $\Omega$ mit
rektifizierbarem Rand gegeben.
Ist $\varphi$ eine zweimal stetig differenzierbare Funktion auf
dem Gebiet $\Omega$, die dem Funktional $I(\varphi)$ 
\eqref{buch:felder:astrogradski:eqn:Lfunktional}
ein Extremum erteilt, dann erfüllt $\varphi$ die Differentialgleichung
\begin{equation}
\begin{aligned}
0
&=
\frac{\partial L}{\partial u} 
-
\frac{\partial}{\partial x_1}
\frac{\partial L}{\partial u_{x_1}}(x,\varphi(x),\grad\varphi(x))
-
\dots
-
\frac{\partial}{\partial x_n}
\frac{\partial L}{\partial u_{x_n}}(x,\varphi(x),\grad\varphi(x)).
\\
&=
\frac{\partial L}{\partial u}(x,\varphi(x),\grad\varphi(x))
-
\operatorname{div}
(\grad_{u_x} L)(x,\varphi(x),\grad\varphi(x))
\end{aligned}
\label{buch:felder:ostrogradski:eqn:euler-ostrogradski}
\end{equation}
Sie heisst die {\em Euler-Ostrogradski-Differentialgleichung}.
\index{Euler-Ostrogradski-Differentialgleichung}%
\end{satz}

\begin{proof}
Da $\varphi$ dem Funktional ein Extremum erteilt, muss die erste
Variation für jede differenzierbare Funktion $\eta(x)$ mit Nullwerten
auf dem Rand verschwinden.
Es gilt also
\begin{align}
0
&=
\delta I(\varphi)
=
\frac{d}{d\varepsilon}
\int_{\Omega}
L\bigl(x,\varphi(x)+\varepsilon\eta(x),
\grad(\varphi(x) + \varepsilon\eta(x))
\bigr)
\,dx
\bigg|_{\varepsilon=0}
\notag
\intertext{für jede zulässige Funktion $\eta(x)$.
Entwicklung der Funktion $L$ nach Potenzen von $\varepsilon$ ergbibt
}
&=
\int_{\Omega} \frac{\partial L}{\partial u}(x,\varphi(x),\grad\varphi(x))\,dx
+
\int_{\Omega}
\sum_{k=1}^n
\frac{\partial L}{\partial u_{x_k}}
L\bigl(x,\varphi(x),\grad\varphi(x))
\frac{\partial\eta}{\partial x_k}
\,dx.
\notag
\intertext{Die Summe im zweiten Integral lässt sich etwas kompakter
als ein Skalarprodukt von der Form}
&=
\int_{\Omega} \frac{\partial L}{\partial u}(x,\varphi(x),\grad\varphi(x))\,dx
+
\int_{\Omega}
(\grad_{u_x}L)(x,\varphi(x),\grad\varphi(x))
\cdot
\grad\eta(x)
\,dx.
\label{buch:felder:ostrogradski:eqn:abgeleitet}
\end{align}
schreiben.

Der Integrand des zweiten Integrals in
\eqref{buch:felder:ostrogradski:eqn:abgeleitet}
ist von der Form $\vec{u}(x)\cdot \grad v(x)$, die auch in der
Produktformel
\[
\operatorname{div}\bigl(v(x)\vec{u}(x)\bigr)
=
\grad v(x)
\cdot
\vec{u}(x)
+
v(x) \, \operatorname{div} u(x)
\]
vorkommt, wenn man $v(x) = \eta(x)$ und $\vec{u}=\grad_{u_x}L$ setzt.
Die zugehörige partielle Integrationsformel wendelt das zweite Integral
\eqref{buch:felder:ostrogradski:eqn:abgeleitet}
in 
\[
\int_{\partial \Omega}
v(x) \eta(x) \grad_{u_x}(L(x,\varphi(x),\grad\varphi(x)))
\,d\vec{\omega}
-
\int_{\Omega}
(\grad_{u_x} L)(x,\varphi(x),\grad \varphi(x))
\,dx
\]
um.
Die Art des Integrals über den Rand spielt dabei überhaupt keine Rolle,
denn da $\eta(x)=0$ ist auf dem Rand, verschwindet es unabhängig davon.
Aus der Gleichung
\eqref{buch:felder:ostrogradski:eqn:abgeleitet}
wird dann
\begin{equation}
0
=
\int_{\Omega}
\biggl(
\frac{\partial L}{\partial u}(x,\varphi(x),\grad\varphi(x))
-
\operatorname{div}
(\grad_{u_x}L)(x,\varphi(x),\grad\varphi(x))
\biggr)
\eta(x)
\,dx.
\end{equation}
Dies muss gelten für jede stetig differenzierbare Funktion $\eta(x)$
mit verschwindenden Randwerten.
Nach dem
Fundamentallemma~\ref{buch:felder:fundamentallemma:satz:fundamentallemma}
für Funktionen mehrere Variablen folgt jetzt die Differentialgleichung
\eqref{buch:felder:ostrogradski:eqn:euler-ostrogradski}.
\end{proof}

%
% Gravitationsfeld
%
\subsubsection{Gravitationsfeld}
Die Lagrange-Funktion
\[
L(x,u,u_x)
=
-\frac{1}{8\pi G}
u_x\cdot u_x 
-\varrho u
=
-\frac{1}{8\pi G}
(u_{x_1}^2 + \dots + u_{x_2}^2)
\]
für das Gravitationsfeld wurde in
\eqref{buch:felder:ostrogradski:eqn:gravitationslagrange}
vorgestellt.
Wir können jetzt die zugehörige Euler-Lagrange-Differentialgleichung
ableiten.
Dazu muss die Ableitungen nach $u$ und $u_{x_k}$ ermittelt werden.
Sie sind
\begin{align*}
\frac{\partial L}{\partial u}(x,\varphi(x),\grad\varphi(x))
&=
-\varrho(x)
\\
\frac{\partial L}{\partial u_{x_k}}(x,\varphi(x),\grad \varphi(x))
&=
-\frac{1}{4\pi G}\frac{\partial \varphi}{\partial x_k}(x)
&&\Rightarrow&
\frac{\partial L}{\partial u_x}(x,\varphi,\grad\varphi)
&=
-\frac{1}{4\pi G}\grad \varphi(x).
\end{align*}
Eingesetzt in die Euler-Ostrogradski-Differentialgleichung ergibt sich
\begin{align*}
0
&=
-\varrho
-
\operatorname{div} \biggl(-\frac{1}{4\pi G}\grad \varphi\biggr)
=
-\varrho + \frac{1}{4\pi G} \operatorname{div}\grad \varphi
\qquad\Rightarrow\qquad
\Delta \varphi = 4\pi G\varrho.
\end{align*}
Dies ist die Laplace-Gleichung für das Gravitationspotential.

%
% Schrödingergleichung
%
\subsubsection{Die Schrödingergleichung}
%
% garbageman.tex
%
% (c) 2024 Prof Dr Andreas Müller
%
\begin{figure}
\centering
\includegraphics[width=\textwidth]{chapters/040-felder/images/Garbageman.png}
\caption{Dilbert-Comic über den Müllmann, der im Dilbert-Universum die
intelligenteste Person ist.
Er zeigt, dass Ingenieure unbedingt Quantenmechanik verstehen müssen.
\label{buch:felder:fig:garbageman}}
\end{figure}

Im Dilbert-Comic in Abbildung~\ref{buch:felder:fig:garbageman} beweist
der Müllmann, dass Ingenieure Quantenmechanik verstehen müssen.
Daher soll in diesem Abschnitt gezeigt werden, wie sich die
Schrödinger-Gleichung aus einem Variationsprinzip ableiten lässt.

Die Schrödinger-Gleichung ist eine partielle Differentialgleichung
für die sogenannte Wellenfunktion $\psi(x,t)$ eines Teilchens.
Die Funktion $\psi$ hat komplexe Werte, ihre physikalische
Interpretation ist, dass $|\psi(x,t)|^2$ die Wahrscheinlichkeitsdichte
dafür ist, das Teilchen an der Stelle $x$ zu finden.
Die Lagrange-Funktion muss eine Funktion der Orts- und Zeitkoordinaten,
der Funktion $\psi$ und den partiellen Ableitungen von $\psi$ nach
den Ortskoordinaten und der Zeit sein.
Wir schreiben sie als
\[
L
\colon
\mathbb{R}\times \mathbb{R}^3\times
\mathbb{R}\times
\mathbb{R}\times\mathbb{R}^3
\to\mathbb{R}
:
(t,x,u, u_t, u_x)
\mapsto
L(t,x,u, u_t, u_x).
\]
Die Schrödinger-Gleichung sollte sich dann als die Euler-Ostrogradski-Gleichung
zu La\-grange-Funktion $L$ ergeben.

Die Theorie schlägt die folgende Lagrange-Funktion vor:
\begin{equation}
L(t,x,u,u_t,u_x)
=
-\frac{\hbar}{i}\bar{u}u_t- \frac{\hbar^2}{2m} \bar{u}_x\cdot u_x - V \bar{u}u,
\label{buch:felder:ostrogradski:eqn:schroedingerlagrange}
\end{equation}
wobei $V(x)$ das Potential ist, in dem sich das Teilchen bewegt.
Zur Herleitung der Euler-Ostrogradski-Differentialgleichung zur
Lagrange-Funktion
\eqref{buch:felder:ostrogradski:eqn:schroedingerlagrange}
bestimmen wir zu nächst die Ableitung nach den Feldbariablen:
\begin{equation}
\begin{aligned}
\frac{\partial L}{\partial u}
&=
-V\bar{u}
\\
\frac{\partial L}{\partial u_t}
&=
-
\frac{\hbar}{i}
\bar{u}
\\
\frac{\partial L}{\partial u_{x_k}}
&=
-\frac{\hbar^2}{2m}\bar{u}_{x_k}
\end{aligned}
\label{buch:felder:ostrogradski:eqn:schroedingerableitungen}
\end{equation}
Diese Ableitungen sind rein formal zu verstehen, da uns im Moment die
notwendige komplexe Analysis für eine konsistente Herleitung fehlt.
Sie wird gerechtfertigt dadurch, dass in der komplexen Analysis die
Variablen $u$ und $\bar{u}$ als unabhängig betrachtet werden können,
insbesondere verschwindet die Ableitung von $\bar{u}$ nach $u$.
Eigentlich gibt es also zwei Felder, die unabhängig voneinander
variert werden können, nämlich $u$ und $\bar{u}$.
Indem wir die Ableitungen $\bar{u}$ ignorieren, führen wir die Variation
nur für die Funktionen $u$ durch.

Aus den Ableitungen
\eqref{buch:felder:ostrogradski:eqn:schroedingerableitungen}
lässt sich formal die zugehörige Euler-Ostrogradski-Glei\-chung entwickeln.
Sie wird
\begin{align*}
0
&=
- V\bar{\psi}(t,x)
+
\frac{\partial}{\partial t}\frac{\hbar}{i}\bar{\psi}(t,x)
+
\sum_{k=1}^{3}
\frac{\partial}{\partial x_k}
\frac{\hbar^2}{2m} \frac{\partial\bar{\psi}}{\partial x_k}(t,x)
\end{align*}
Durch Umordnen der Terme erhält man
\begin{align}
\frac{\hbar}{i}
\frac{\partial}{\partial t}\bar{\psi}(t,x)
&=
-\frac{\hbar^2}{2m} \Delta \bar{\psi}(t,x)
+
V\bar{\psi}(t,x)
\notag
\intertext{und nach komplexer Konjugation die übliche Form}
-\frac{\hbar}{i}\psi(t,x)
&=
-\frac{\hbar^2}{2m}\Delta \psi(t,x) + V(x)\psi(t,x)
\label{buch:felder:ostrogradski:eqn:schroedingergleichung}
\end{align}
der Schrödinger-Gleichung.
\index{Schrödinger-Gleichung}%


