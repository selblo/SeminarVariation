%
% 1-fundamtenallemma.tex
%
% (c) 2023 Prof Dr Andreas Müller
%
\section{Das Fundamentallemma für Funktionen mehrere Variablen
\label{buch:felder:section:fundamentallemma}}
\kopfrechts{Das Fundamentallemma}
Der Schlüssel für die Umwandlung eines eindimensionalen Variationsproblems
in eine Differentialgleichung war das Fundamentallemma.

%
% Gebiete
%
\subsection{Gebiete}
Variationsprobleme für Funktionen von nur einer Variablen sind auf 
Funktionen auf einem Intervall beschränkt.
Besteht der Definitionsbereich aus mehreren Intervallen, kann jedes
einzelne Intervall unabhängig von allen anderen betrachtet werden.
Für ein Intervall $[x_0,x_1]$ ist auch das Integral einfach zu definieren,
wie es zur Konstruktion des zu minimierenden Funktionals nötig ist.

Für Funktionen von zwei oder mehr Variablen ist bereits die
Festlegung des Definitionsgebietes viel komplizierter.
Sie muss so erfolgen, dass auch die partielle Integration eines
Produktes darauf übertragbar ist.
Diese wurde im eindimensionalen Fall dazu aus dem Integral
über $F\cdot\eta'$ eine Integral über $dF/dx\cdot \eta$ zu
machen, auf welches das Fundamentallemma anwendbar war.

%
% Offene Mengen
%
\subsubsection{Offene Mengen}

%
% Der Rand eines Gebietes
%
\subsubsection{Der Rand eines Gebiets}

%
% Integration über ein Gebiet
%
\subsubsection{Integration über ein Gebiet}

%
% Integration über eine Kurve
%
\subsubsection{Integration über eine Kurve}

%
% Integration
%
\subsection{Partielle Integration}
Die partielle Integration für ein Integral einer Variablen
transformiert das Integral
\[
\int_{a}^{b} f(x) g'(x)\,dx
\]
in die Summe
\[
\biggl[ f(x) g(x) \biggr]_a^b
-
\int_a^b f'(x) g(x)\,dx.
\]
Der erste Term hängt nur von Informationen auf dem Rand des 
Definitionsbereichs ab.
Eine Verallgemeinerung dieser Regel muss vor allem auch klären,
wie dieser erste Term aussehen soll.
Er darf nur von den Funktionswerten der Funktionen auf dem Rand
$\partial\Omega$ abhängen.
Die Abhängigkeit muss ausserdem linear sein.
In diesem Abschnitt sollen in einer Folge von Beispielen die
wichtigsten Fälle systematisch entwickelt werden.

%
% Ableitung von Produkten
%
\subsubsection{Ableitung von Produkten}
Die Regel für die partielle Integration von Produkten von Funktionen
einer Variable ist die Integralform der Produktregel
\[
\frac{d}{dx}(f(x)g(x)) = f'(x)g(x) + f(x)g'(x).
\]
Für eine Erweiterung auf ist daher als erstes zu klären, welche Art
von Differentialoperatoren überhaupt in Frage kommen.

Die partielle Ableitung nach einer der Variablen bei Funktionen
$f$ und $g$ von $n$ Variablen erfüllt eine Produktregel:
\[
\frac{\partial}{\partial x_k}
(f(x)g(x))
=
\frac{\partial f}{\partial x_k}(x)\,g(x)
+
f(x)
\,
\frac{\partial g}{\partial x_k}(x)
\]
für jedes $k=1,\dots,n$.
Die partiellen Ableitungen sind aber nur von untergeordnetem Interesse,
da sie für sich allein von der Wahl des Koordinatensystems abhängig 
sind und daher zum Beispiel keine koordinatenunabhängige physikalische
Bedeutung haben können.

Der Gradient wurde aus der Richtungsableitung konstruiert und es wurde
gezeigt, dass 
\[
\operatorname{grad} f(x)
=
\begin{pmatrix}
\frac{\partial f}{\partial x_1}\\
\vdots\\
\frac{\partial f}{\partial x_n}
\end{pmatrix}
\]
eine koordinatenunabhängige Bedeutung als der Vektor hat, in dessen
Richtung die schnellste Zunahme der Funktion $f$ erfolgt.
Und tatsächlich gilt auch für den Gradienten eine Produktformel:
\begin{align*}
\operatorname{grad}(f(x)g(x))
&=
\begin{pmatrix}
\frac{\partial}{\partial x_1}(f(x)g(x))\\
\vdots\\
\frac{\partial}{\partial x_n}(f(x)g(x))
\end{pmatrix}
=
\begin{pmatrix}
\frac{\partial f}{\partial x_1} g(x)
+
f(x) \frac{\partial g}{\partial x_1}\\
\vdots\\
\frac{\partial f}{\partial x_n} g(x)
+
f(x) \frac{\partial g}{\partial x_n}
\end{pmatrix}
\\
&=
(\operatorname{grad}f(x)) g(x)
+
f(x)(\operatorname{grad}g(x))
\end{align*}

%
% Der Satz von Green
%
\subsubsection{Der Satz von Green}

%
% Der Satz von Gauss
%
\subsubsection{Der Satz von Gauss}

%
% Der Satz von Stokes
%
\subsubsection{Der Satz von Stokes}

%
% Glatte Funktionen mit kompaktem Träger
%
\subsection{Glatte Funktionen mit kompaktem Träger}

%
% Das Fundamentallemma für mehrere Variablen
%
\subsection{Das Fundamentallemma für mehrere Variablen}

\begin{satz}[Fundamentallemma]
Sei $\Omega\subset\mathbb{R}^n$ ein Gebiet und $f\colon \Omega\to\mathbb{F}$
eine stetige Funktion.
Ausserdem gelte für jede glatte Funktion $g\colon\Omega\to\mathbb{R}$ 
\[
\int_{\Omega} f(x)g(x)\,dx = 0.
\]
Dann folgt $f(x)=0$.
\end{satz}
