%
% 2drand.tex -- Rand eines zweidmensionalen Gebietes
%
% (c) 2021 Prof Dr Andreas Müller, OST Ostschweizer Fachhochschule
%
\documentclass[tikz]{standalone}
\usepackage{amsmath}
\usepackage{times}
\usepackage{txfonts}
\usepackage{pgfplots}
\usepackage{csvsimple}
\usetikzlibrary{arrows,intersections,math,calc}
\definecolor{darkred}{rgb}{0.8,0,0}
\definecolor{darkgreen}{rgb}{0,0.6,0}
\begin{document}
\def\skala{1}
\begin{tikzpicture}[>=latex,thick,scale=\skala]

\coordinate (P1) at (1,4);
\coordinate (P2) at (7,1);
\coordinate (P3) at (10,2);
\coordinate (P4) at (6,5);
\coordinate (P5) at (9,7);
\coordinate (P6) at (3,7);

\node at (P1) [left] {$P_1$};
%\node at (P2) [below] {$P_2$};
\node at (P3) [right] {$P_2$};
\node at (P4) [right] {$P_3$};
\node at (P5) [right] {$P_4$};
%\node at (P6) [above] {$P_6$};

\fill[color=gray!20]
	(P1)
	to[out=-90,in=180]
	(P2)
	to[out=0,in=-90]
	(P3)
	to[out=90,in=-90]
	(P4)
	to[out=90,in=-90]
	(P5)
	to[out=90,in=0]
	(P6)
	to[out=180,in=90]
	(P1) -- cycle;

\fill[color=blue!10]
	(P1)
	to[out=-90,in=180]
	(P2)
	to[out=0,in=-90]
	(P3)
	-- cycle;
\fill[color=darkred!10]
	(P1)
	--
	(P3)
	to[out=90,in=-90]
	(P4)
	-- cycle;
\fill[color=darkgreen!10]
	(P1)
	--
	(P4)
	to[out=90,in=-90]
	(P5)
	-- cycle;
\fill[color=orange!10]
	(P1)
	--
	(P5)
	to[out=90,in=0]
	(P6)
	to[out=180,in=90]
	(P1) -- cycle;

\draw[line width=0.3pt] (P1) -- (P3);
\draw[line width=0.3pt] (P1) -- (P4);
\draw[line width=0.3pt] (P1) -- (P5);

\draw[color=blue,line width=1.4pt]
	(P1)
	to[out=-90,in=180]
	(P2)
	to[out=0,in=-90]
	(P3);
\draw[color=darkred,line width=1.4pt]
	(P3)
	to[out=90,in=-90]
	(P4);
\draw[color=darkgreen,line width=1.4pt]
	(P4)
	to[out=90,in=-90]
	(P5);
\draw[color=orange,line width=1.4pt]
	(P5)
	to[out=90,in=0]
	(P6)
	to[out=180,in=90]
	(P1);

\fill[color=white] (P1) circle[radius=0.05];
%\fill[color=white] (P2) circle[radius=0.05];
\fill[color=white] (P3) circle[radius=0.05];
\fill[color=white] (P4) circle[radius=0.05];
\fill[color=white] (P5) circle[radius=0.05];
%\fill[color=white] (P6) circle[radius=0.05];

\draw (P1) circle[radius=0.05];
%\draw (P2) circle[radius=0.05];
\draw (P3) circle[radius=0.05];
\draw (P4) circle[radius=0.05];
\draw (P5) circle[radius=0.05];
%\draw (P6) circle[radius=0.05];

\node[color=blue]      at (P2) [below]                  {$y_{12}(x)$};
\node[color=darkred]   at ($0.5*(P3)+0.5*(P4)$) [above] {$y_{23}(x)$};
\node[color=darkgreen] at ($0.5*(P4)+0.5*(P5)$) [below] {$y_{34}(x)$};
\node[color=orange]    at (P6) [above]                  {$y_{41}(x)$};

\draw[->] (0,-0.1) -- (0,8.3) coordinate[label={right:$y$}];
\draw[->] (-0.1,0) -- (11.3,0) coordinate[label={$x$}];

\end{tikzpicture}
\end{document}

