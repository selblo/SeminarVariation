%
% greeenbeweis.tex -- Illustration zum Beiweis des Satzes von Green
%
% (c) 2021 Prof Dr Andreas Müller, OST Ostschweizer Fachhochschule
%
\documentclass[tikz]{standalone}
\usepackage{amsmath}
\usepackage{times}
\usepackage{txfonts}
\usepackage{pgfplots}
\usepackage{csvsimple}
\usetikzlibrary{arrows,intersections,math}
\begin{document}
\definecolor{darkred}{rgb}{1.0,0,0}
\def\skala{1}
\begin{tikzpicture}[>=latex,thick,scale=\skala]

\def\gebiet{
	\fill[color=gray!20]
		(1.0,2)
		to[out=-90,in=180]
		(2.5,0.5) 
		to[out=0,in=-90]
		(4.8,3)
		to[out=90,in=0]
		(3.5,4)
		to[out=180,in=90]
		(1.0,2);
	\node at (2.5,2) {$\Omega$};
}

\def\koordinaten{
	\draw[->] (-0.1,0) -- (5.7,0) coordinate[label={$x$}];
	\draw[->] (0,-0.1) -- (0,4.5) coordinate[label={right:$y$}];
}

\begin{scope}
\gebiet
\draw[color=gray!40] (1.0,0) -- (1.0,2);
\draw[color=gray!40] (4.8,0) -- (4.8,3);
\node at (1.0,0) [below] {$x_1\mathstrut$};
\node at (4.8,0) [below] {$x_2\mathstrut$};
\draw[color=darkred,line width=1.2pt]
		(1.0,2)
		to[out=-90,in=180]
		(2.5,0.5) 
		to[out=0,in=-90]
		(4.8,3);
\draw[color=blue,line width=1.2pt]
		(4.8,3)
		to[out=90,in=0]
		(3.5,4)
		to[out=180,in=90]
		(1.0,2);
\fill[color=white] (1.0,2) circle[radius=0.05];
\draw (1.0,2) circle[radius=0.05];
\fill[color=white] (4.8,3) circle[radius=0.05];
\draw (4.8,3) circle[radius=0.05];
\node[color=darkred] at (2.5,0.5) [below] {$y_-(x)$};
\node[color=blue] at (3.5,4) [above] {$y_+(x)$};
\koordinaten
\end{scope}

\begin{scope}[xshift=6.5cm]
\gebiet
\draw[color=gray!40] (0,4) -- (3.5,4);
\node at (0,4) [left] {$y_2\mathstrut$};
\node at (0,0.5) [left] {$y_1\mathstrut$};
\draw[color=gray!40] (0,0.5) -- (2.5,0.5);
\draw[color=darkred,line width=1.2pt]
		(3.5,4)
		to[out=180,in=90]
		(1.0,2)
		to[out=-90,in=180]
		(2.5,0.5);
\draw[color=blue,line width=1.2pt]
		(2.5,0.5) 
		to[out=0,in=-90]
		(4.8,3)
		to[out=90,in=0]
		(3.5,4);
\fill[color=white] (3.5,4) circle[radius=0.05];
\draw (3.5,4) circle[radius=0.05];
\fill[color=white] (2.5,0.5) circle[radius=0.05];
\draw (2.5,0.5) circle[radius=0.05];
\node[color=darkred] at (1.0,2) [left] {$x_-(x)$};
\node[color=blue] at (4.8,3) [right] {$x_+(x)$};
\koordinaten
\end{scope}

\end{tikzpicture}
\end{document}

