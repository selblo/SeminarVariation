%
% chapter.tex
%
% (c) 2023 Prof Dr Andreas Müller
%
\chapter{Extremalprobleme für Funktionen mehrere Variablen
\label{buch:chapter:fuvar}}
\kopflinks{Extremalprobleme}
Die Beobachtung, dass in einem Extremum einer Funktion die Ableitung
verschwindet, gehört zu den frühesten Anwendungen, die Studierende
der Analysis praktisch umzusetzen lernen.
Genauere Aussagen darüber, ob ein Minimum oder Maximum 
vorliegt, lassen sich aus den höheren Ableitungen gewinnen.

In einzelnen Fällen lassen sich sogar Aufgaben lösen, die auf
den ersten Blick mehrere Variablen zu involvieren scheinen.
Um das Rechteck mit Seiten $l$ und $b$ hat den grössten Flächeninhalt
$F=lb$ bei gegebenem Umfang zu finden, nutzt man aus, dass $2l+2b=U$ ist
und damit $l=U/2-b$ ist.
Der Flächeninhalt wird damit zur Funktion $f(b)=lb=(U/2-b)b$, die nur
noch von der einen Variable $b$ abhängt. 
Die quadratische Funktion
\[
f(b) = \frac{U}2b-b^2
\]
hat die Ableitung
\[
f'(b) = \frac{U}2-2b
\]
mit der einzigen Nullstelle $b_{\text{max}}=U/4$.
Das Rechteck mit maximalem Inhalt ist also ein Quadrat mit Seitenlänge
$U/4$.

Die im obigen Beispiel mögliche Reduktion auf ein Problem mit nur einer
Variablen ist nur in sehr speziellen Situationen möglich.
Ziel dieses Kapitels ist daher, die Theorie der Bestimmung der
Extremwerte für Funktionen mehrerer Variablen zu entwickeln.
Die Variationsrechnung, die ab Kapitel~\ref{buch:chapter:variation}
dargestellt wird, verwendet sowohl die im Fall endlich vieler Variablen
entwickelten Methoden wie auch die Intuition, um analoge Techniken
für den unendlichdimensionalen Fall der Funktionale zu finden.

%
% 1-definitionen.tex
%
% (c) 2023 Prof Dr Andreas Müller
%
\section{Richtungsableitung und Gradient
\label{buch:fuvar:section:richtungsableitung}}
\kopfrechts{Richtungsableitung und Gradient}
Im Folgenden werden Funktionen betrachtet, die von mehreren unabhängigen
Variablen abhängen.
Solche Funktionen werden $f(x_1,\dots,x_n)$ für die Variablen
$x_1,\dots,x_n$ gesschrieben.
Der Definitionsbereich ist eine Teilmenge $U$ von $\mathbb{R}^n$.
Die reellwertige Funktion
\[
f\colon U\to\mathbb{R} : (x_1,\dots,x_n) \mapsto f(x_1,\dots, x_n)
\]
wird auch abgekürzt $f(x)$ mit $x\in\mathbb{R}^n$ geschrieben.

%
% Partielle Ableitungen
%
\subsection{Partielle Ableitungen}
Die Lösung von Extremalproblemen bei Funktionen einer Variablen
mit Mitteln der Analysis basiert auf der Beobachtung, dass die Ableitung
nach der unabhängigen Variablen verschwinden muss.
Bei einer Funktion mehrere Variablen muss aber erst geklärt werden,
das die Ableitung bei einer Funktion mehrerer Variablen bedeuten soll.

%
% partielle Funktion
%
\subsubsection{Partielle Funktion}
Hält man von einer Funktion $f(x_1,\dots,x_n)$ von den Variablen
$x_1,\dots,x_n$ alle Variablen ausser einer fest, bleibt von $f$ eine
Funktion übrig, die nur noch von dieser einen Variablen abhängt.
Seien also alle Variablen $x_i$ ausser der Variablen $x_k$ fest.
Aus der Funktion $f$ wird damit eine neue Funktion
\[
f_k
\colon
\mathbb{R} \to \mathbb{R}
:
x_k \mapsto f(x_1,\dots,x_k,\dots,x_n).
\]
Die Funktion $f_k$ ist eigentlich eine Funktion, die die festgehaltenen
Variablen als Parameter enthält.
Dies sind die Variablen $x_1,\dots,\widehat{x_k},\dots,x_n$, wobei
wir mit dem Hut anzeigen, dass dieses Element weggelassen werden soll.
Diese Notation wird für alle folgenden Kapitel beibehalten.
Die Funktion $f_k$ wird manchmal auch die {\em partielle Funktion}
\index{partielle Funktion}%
\index{Funktion!partiell}%
genannt.

Unglücklicherweise ist der Begriff der partiellen Funktion auch 
üblich für eine Funktion $A\to B$, die aber nur auf einem Teil
der Menge $A$ definiert ist, deren Definitionsbereich $D(f)$ also
eine echte Teilmenge von $A$ ist.
In diesem Buch ist immer der obengenannte Begriff einer nur von einer
Variablen abhängigen Funktion gemeint.

%
% Ableitungen
%
\subsubsection{Ableitungen}
Da die Funktion $f_k$ nur noch eine Funktion einer einzigen Variablen
ist, ist auch klar, wie sie abgeleitet werden muss.
Die Ableitung von $f_k$ ist
\begin{align*}
\frac{d}{dx_k} f_k(x_k)
&=
\lim_{h\to 0} \frac{f_k(x_k+h)-f_k(x_k)}{h}
\\
&=
\lim_{h\to 0}
\frac{f_k(x_1,\dots,x_k+h,\dots,x_n)-f_k(x_1,\dots,x_k,\dots,x_n)}{h}.
\end{align*}

\begin{definition}
\label{buch:fuvar:richtungsbleitung:def:partielleableitung}
Die {\em partielle Ableitung} einer Funktion $f(x_1,\dots,x_n)$ nach der
\index{partielle Ableitung}%
\index{Ableitung!partiell}%
Variablen $x_k$ ist
\[
\frac{\partial f}{\partial x_k}(x_1,\dots,x_n)
=
\lim_{h\to 0}
\frac{f(x_1,\dots,x_k+h,\dots,x_n)-f(x_1,\dots,x_k,\dots,x_n)}{h}.
\]
Eine Funktion $f(x_1,\dots,x_n)$ heisst {\em stetig differenzierbar}
nach der Variablen $x_k$, wenn die Ableitung der partiellen Funktion 
$f_k$ stetig ist.
\end{definition}

Die partiellen Ableitungen nach einer Variablen entstehen also dadurch,
dass man alle anderen Variablen einer Funktion als konstant betrachtet
und dann die gewöhnliche Ableitung nach der einen verbleibenden Variablen
bildet.

\begin{beispiel}
Die Funktion
\[
f(x,y)
=
\sin(xy)\cos(x+y)
\]
hängt von den Variablen $x$ und $y$ ab.
Die partiellen Funktionen sind
\begin{align*}
x&\mapsto f(x,y)
&&\text{und}&
y&\mapsto f(x,y).
\end{align*}
Zur Berechnung der Ableitungen muss die Produktregel verwendet
werden, sie ergibt die Ableitungen
\begin{align*}
\frac{\partial f}{\partial x}
&=
y\cos(xy)\cos(x+y) -\sin(xy)\sin(x+y)
\\
\frac{\partial f}{\partial x}
&=
x\cos(xy)\cos(x+y) -\sin(xy)\sin(x+y).
\end{align*}
Die beiden Ableitungen unterschieden sich nur in dem Faktor $y$ bzw.~$x$,
der von der inneren Ableitung bei der Ableitung des ersten Faktors
$x\mapsto\sin(xy)$ bzw.~$y\mapsto\sin(xy)$ herrührt.
\end{beispiel}

%
% Partielle Ableitungen als Steigung der Koordinatenlinien
%
\subsubsection{Partielle Ableitungen als Steigungen der Koordinatenlinien}
Der Graph einer Funktion $(x,y)\mapsto f(x,y)$ von zwei Variablen ist 
eine zweidimensionale Fläche über der $x$-$y$-Ebene
(Abbildung~\ref{buch:fuvar:richtungsableitung:fig:partabl}).
%
% partabl.tex
%
% (c) 2024 Prof Dr Andreas Müller
%
\begin{figure}
\centering
\vspace*{2cm}
XXX partielle Ableitungen als Steigung der Koordinatenlinien
\vspace*{2cm}
\caption{Die partiellen Ableitungen einer Funktion $f(x,y)$ von
zwei Variablen sind die Steigungen der Koordinationenlinien im
Punkt $(x,y)$.
\label{buch:fuvar:richtungsableitung:fig:partabl}}
\end{figure}

Die partielle Funktion $x\mapsto f(x,y_0)$ hat als Graph die Schnittkurve
der Fläche mit der vertikalen Ebene $y=y_0$, entsprechend ist der Graph
der partiellen Funktion $y\mapsto f(x_0,y)$ die Schnittkurve der Fläche
mit der vertikalen Ebene $x=x_0$, beide in
Abbildung~\ref{buch:fuvar:richtungsableitung:fig:partabl} blau hervorgehoben.

%
% Notation
%
\subsubsection{Notation}
Die in Definition~\ref{buch:fuvar:richtungsbleitung:def:partielleableitung}
eingeführte Notation ist intuitiv und erinnert an die Notation für den
Differentialquotienten einer Funktion einer Variablen.
Sie hat aber auch einen gravierenden Nachteil, der am Beispiel der
der Funktion $f(x,y)$ illustriert werden soll.
Die partiellen Ableitungen der Funktion $f$ nach $x$ und $y$ sind
\[
\frac{\partial f}{\partial x}
\quad\text{und}\qquad
\frac{\partial f}{\partial y}.
\]
Setzt man die Argumentwerte $x^2+y^2$ und $x^2-y^2$ in die Funktion
$f$ ein, entsteht dann die Funktion
\begin{equation}
(x,y) \mapsto f(x^2+y^2,x^2-y^2).
\label{buch:fuvar:richtungsableitung:eqn:feingesetzt}
\end{equation}
Wie soll man die Ableitung dieser Funktion nach $x$ schreiben?
Und was soll die Notation
\[
\frac{\partial f}{\partial x}(x^2+y^2,x^2-y^2)
\]
bedeuten?
Ist dies die partielle Ableitung nach dem ersten Argument von $f$
oder ist es die Ableitung der zusammengesetzten Funktion nach $x$?

Offenbar wurden hier die Symbole $x$ und $y$ auf zwei verschiedene
Arten verwendet.
In der Schreibweise $f(x,y)$ ist das $x$ ein Platzhalter für die
erste Variable, das $y$ ist Platzhalter für die zweite Variable.
Die Symbole haben also vor allem syntaktische Eigenschaften.
Die Aussage, dass $f$ eine Funktion $\mathbb{R}^2\to\mathbb{R}$
ist, sagt genauso viel.
Die Variablen $x$ und $y$ werden nur gebraucht, wenn man zum
Beispiel mit einer Formel wie $f(x,y)=\sin x\cdot\cos y$ 
anzeigen will, wie ein Funktionswert berechnet werden soll.

In der Form~\eqref{buch:fuvar:richtungsableitung:eqn:feingesetzt} ist
stehen aber $x$ und $y$ für beliebige reelle Zahlen.
Daraus müssen erst die Werte $x^2+y^2$ und $x^2-y^2$ berechnet werden,
die dann als erstes bzw.~zweites Argument der Funktion eingesetzt
werden müssen.
Die Notation~\eqref{buch:fuvar:richtungsableitung:eqn:feingesetzt} ist also
eigentlich eine abgekürzte Schreibweise für eine zusammengesetzte
Funktion, die sich aus
\[
g
\colon
\mathbb{R}^2 \to \mathbb{R}^2
:
(x,y) \mapsto (x^2+y^2,x^2-y^2)
\]
und der Funktion $f$ zusammensetzt.

\begin{beispiel}
\label{buch:fuvar:richtungsableitung:bsp:L}
In Abschnitt~\ref{buch:variation:section:eulerlagrange} werden wir
Extremalprobleme für Funktionen konstruieren, die als Integrale
einer Funktion $L$ von drei Variablen entstehen.
Zu jeder Funktion $y\colon \mathbb{R}\to\mathbb{R}:x\mapsto y(x)$
wird das Integral
\begin{equation*}
I
=
\int_a^b L(x,y(x),y'(x))\,dx
\end{equation*}
berechnet.
Die Funktion $L$ wird dabei meistens als $L(x,y,y')$ geschrieben.
Es wird dann die  Euler-Lagrange-Differentialgleichung abgeleitet,
in der die Ausdrücke
\begin{equation}
\frac{\partial L}{\partial y}
\qquad\text{und}\qquad
\frac{\partial L}{\partial y'}
\label{buch:fuvar:richtungsableitungen:eqn:Lderiv}
\end{equation}
auftreten.
Obwohl $y$ und $y'$ Funktionen von $x$ sind, sind mit den 
partiellen Ableitungen~\eqref{buch:fuvar:richtungsableitungen:eqn:Lderiv}
die Ableitungen nach der zweiten und dritten unabhängigen Variable
von $L$ gemeint.
\end{beispiel}

\begin{definition}
\label{buch:fuvar:richtungsableitung:def:D}
Sei $f\colon \mathbb{R}^n \to V$ eine Funktion von $n$ Variablen
mit Werten in einem endlichdimensionalen Vektorraum.
Die partielle Ableitung nach der $k$-ten Variable wird auch als
\[
D_kf(x_1,\dots,x_n)
=
\lim_{h\to 0}
\frac{f(x_1,\dots,x_k+h,\dots,x_n)-f(x_1,\dots,x_k,\dots,x_n)}{h}
\]
bezeichnet.
\end{definition}

\begin{beispiel}
Mit der Notation von Definition~\ref{buch:fuvar:richtungsableitung:def:D}
werden die partiellen Ableitungen der Funktion $L(x,y,y')$ von
Beispiel~\ref{buch:fuvar:richtungsableitung:bsp:L} zu
\begin{align*}
\frac{\partial L}{\partial x}(x,y,y')
&=
D_1L(x,y,y'),
\\
\frac{\partial L}{\partial y}(x,y,y')
&=
D_2L(x,y,y')
\intertext{und}
\frac{\partial L}{\partial y'}(x,y,y')
&=
D_3L(x,y,y').
\end{align*}
Die neue Notation eliminiert die Zweideutigkeit.
$D_3L(x,y(x),y'(x))$ bedeutet, dass die Funktion $L$ zuerst partiell
nach der dritten Variable abgeleitet werden soll, dann sollen die
Werte $x$, $y(x)$ und $y'(x)$ als Argumente eingesetzt werden.
\end{beispiel}

%
% Lineare Ersatzfunktion
%
\subsubsection{Lineare Ersatzfunktion}
%
% tangentialebene.tex
%
% (c) 2024 Prof Dr Andreas Müller
%
\begin{figure}
\centering
\includegraphics{chapters/010-fuvar/images/tangential.pdf}
%\includegraphics{chapters/010-fuvar/images/tangentialebene.pdf}
\caption{Mit den partiellen Ableitungen $\frac{\partial f}{\partial x}$
und $\frac{\partial f}{\partial y}$ einer Funktion lässt sich die rote
Tangentialebene in einem Punkt als lineare Ersatzfunktion konstruieren.
\label{buch:fuvar:richtungsableitung:fig:tangentialebene}}
\end{figure}

Die Ableitung einer Funktion wird manchmal auch als eine lineare
Ersatzfunktion definiert.

\begin{definition}
Eine Funktion $f\colon\mathbb{R}^n\to\mathbb{R}^m$ heisst differenzierbar
im Punkt $x_0\in\mathbb{R}^n$, wenn es eine lineare Funktion
$l\colon \mathbb{R}^n\to\mathbb{R}^m$ gibt mit der Eigenschaft, dass
\[
|f(x_0+x) - f(x_0) - l(x)| = o(x)h
\]
ist.
Diese Notation bedeutet, dass
\[
\lim_{x\to 0}
\frac{|f(x_0+x)-f(x_0)-l(x)|}{|x|}
=
0.
\]
Man sagt auch, eine $o(x)$-Funktion geht schneller gegen $0$ als $x$.
\end{definition}

%
% Kettenregel
%
\subsubsection{Kettenregel}
Die Kettenregel ermöglich, die Ableitungen zusammengestzter Funktionen
zu berechnen.
Im vorliegenden Kontext der Funktionen mehrerer Variablen suchen
wir nach einer Regel, die Ableitungen einer zusammengesetzten Funktion
$f\circ g$ berechnet für zwei Funktionen
\[
g\colon \mathbb{R}^r\to\mathbb{R}^n
\qquad\text{und}\qquad
f\colon \mathbb{R}^n\to\mathbb{R}^m.
\]
Die lineare Ersatzfunktion der Funktion $g$ ist jetzt eine Matrix.

\begin{definition}
\label{buch:fuvar:richtungsableitung:def:ableitungsmatrix}
Die partiellen Ableitungen einer Funktion
$g\colon\mathbb{R}^n\to\mathbb{R}^m$
kann als $m\times n$-Matrix
\[
Dg
=
\begin{pmatrix}
D_1g_1 & D_2g_1 & \dots  & D_ng_1 \\
D_1g_2 & D_2g_2 & \dots  & D_ng_2 \\
\vdots & \vdots & \ddots & \vdots \\
D_1g_m & D_2g_m & \dots  & D_ng_m
\end{pmatrix}
=
J_g
=
{\renewcommand{\arraystretch}{1.3}
\begin{pmatrix}
\frac{\partial g_1}{\partial x_1}
	& \frac{\partial g_1}{\partial x_2}
		&\dots
			&\frac{\partial g_1}{\partial x_n}
\\
\frac{\partial g_2}{\partial x_1}
	& \frac{\partial g_2}{\partial x_2}
		&\dots
			&\frac{\partial g_2}{\partial x_n}
\\
\vdots	&\vdots	&\ddots	&\vdots \\
\frac{\partial g_m}{\partial x_1}
	& \frac{\partial g_m}{\partial x_2}
		&\dots
			&\frac{\partial g_m}{\partial x_n}
\end{pmatrix}}
\]
gesschrieben werden.
Sie heisst auch die {\em Jacobi-Matrix}.
\index{Jacobi-Matrix}%
\end{definition}

Die Ableitung einer zusammengesetzten Funktion $f(g(x))$ einer 
Variable ist durch die Kettenregel
\[
(f\circ g)'(x)
=
f'(g(x)) g'(x)
\]
gegeben.
Für Funktionen mehrerer Variablen wird die Formel etwas komplizierter.

\begin{satz}
Seien $f\colon \mathbb{R}^m\to\mathbb{R}^n$ und
$g\colon\mathbb{R}^r\to\mathbb{R}^m$ Funktionen und sei zudem
$g$ im Punkt $x_0$ differenzierbar und $f$ im Punkt $g(x_0)$.
Dann ist die Zusammensetzung $f\circ g$ differenzierbar im Punkt $x_0$
und die Ableitungsmatrix ist
\[
D(f\circ g) = Df(g(x_0)) Dg(x_0).
\]
\end{satz}

Schreibt man $g(x)=y\in\mathbb{R}^m$ und ist $f(y)$ eine reellwertige Funktion
mit den partiellen Ableitungen
\[
D_k(f\circ g)(x)
=
D_kf(g(x_0)) Dg(x_0)
=
\sum_{l=1}^n
\frac{\partial f}{\partial y_l}(g(x_0))
\frac{\partial y_l}{\partial x_k}(x_0).
\]

%
% Höhere Ableitungen
%
\subsubsection{Höhere Ableitungen}
Ist $f\colon U\to\mathbb{R}$ eine differenzierbare Funktion, dann ist
die Ableitung
\[
\partial_if\colon x\mapsto \frac{\partial f}{\partial x_i}(x)
\]
wieder eine stetige Funktion auf $U$, und man kann erneut die Frage
stellen, ob sie differenzierbar ist.
Die folgende rekursive Definition schafft einen handlichen Begriff
dafür.

\begin{definition}[Höhere partielle Ableitungen]
Eine Funktion $f\colon U\to\mathbb{R}$ mit $U\subset\mathbb{R}^n$ heisst
{\em $n$ mal differenzierbar}, wenn die partiellen Ableitungen
\[
\partial_i f
\colon
U\to\mathbb{R}
:
x\mapsto
\frac{\partial f}{\partial x_i}(x)
\]
für alle $i=1,\dots,n$ $n-1$-mal differenzierbar sind.
Sie heisst {\em $n$-mal stetig} differenzierbar, wenn alle partiellen
ersten Ableitungen $n-1$-mal stetig differenzierbar sind.
\end{definition}

Wenn eine Funktion zweimal differenzierbar ist, existieren alle
partiellen Ableitungen
\[
\frac{\partial^2 f}{\partial x_i\,\partial x_k},
\quad
i,k=1,\dots,n,
\]
es ist aber a priori alles andere als klar, ob die Reihenfolge der
Ableitungen eine Rolle spielt.
Der folgende Satz von Schwarz klärt diese Frage.

\begin{satz}[Schwarz]
Ist $f\colon U\to\mathbb{R}$ für eine offene Umgebung $U\subset\mathbb{R}^n$
eine zweimal stetig differenzierbare Funktion, dann kommt es bei
gemischten Ableitungen nicht auf die Reihenfolge an, es ist
\[
\frac{\partial^2 f}{\partial x_i\,\partial x_k}(x)
=
\frac{\partial^2 f}{\partial x_k\,\partial x_i}(x)
\]
für alle $x\in U$.
\end{satz}

Man beachte, dass der Satz nicht gilt, wenn die Ableitungen nur
existieren, aber nicht notwendigerweise stetig sind.
In diesem Fall lassen sich Gegenbeispiele konstruieren, wie
das folgende Beispiel von Schwarz zeigt.

\begin{beispiel}
\label{buch:fuvar:richtungsableitung:beispiel:schwarz}
%
% schwarz.tex -- Gegenbeispiel zum Satz von Schwarz
%
% (c) 2024 Prof Dr Andreas Müller
%
\begin{figure}
\centering
\includegraphics{chapters/010-fuvar/images/schwarz.pdf}
\caption{Gegenbeispiel~\ref{buch:fuvar:richtungsableitung:beispiel:schwarz}
zum Satz von Schwarz.
Oben der Graph der Funktion $f(x,y)$, in der Mitte die beiden 
ersten partiellen Ableitungen.
Die gemischten Ableitungen in der untersten Abbildung ist an der
Stelle $(x,y)=(0,0)$ nicht mehr stetig.
\label{buch:fuvar:richtungsableitung:fig:schwarz}}
\end{figure}

Wir betrachten die Funktion
\[
f
\colon
\mathbb{R}^2\to\mathbb{R}
:
(x,y)
\mapsto
f(x,y)
=
\begin{cases}
\displaystyle \frac{x^3y-xy^3}{x^2+y^2}&\qquad (x,y)\ne (0,0)\\
(0,0)&\qquad\text{sonst},
\end{cases}
\]
die auch in Abbildung~\ref{buch:fuvar:richtungsableitung:fig:schwarz}
zusammen mit den partiellen ersten Ableitungen und den gemischten
zweiten Ableitungen dargestellt ist.
Die Ableitungen im Nullpunkt sind
\begin{align*}
\frac{\partial f}{\partial x}(x,y)
&=
\frac{x^4y+4x^2y^3-y^5}{(x^2+y^2)^2}
&&\Rightarrow&
\frac{\partial f}{\partial x}(0,y)
&=
\frac{-y^5}{y^4} = -y
\\
\frac{\partial f}{\partial y}(x,y)
&=
\frac{x^5-4x^3y^2-xy^4}{(x^2+y^2)^2}
&&\Rightarrow&
\frac{\partial f}{\partial y}(x,0)
&=
\frac{x^5}{x^4}
=
x.
\intertext{Sie sind in Abbildung~\ref{buch:fuvar:richtungsableitung:fig:schwarz}
in der Mitte rot hervorgehoben.
Daraus ergeben sich die zweiten Ableitungen}
\frac{\partial^2 f}{\partial y\,\partial x}(0,y)
&=
\frac{\partial}{\partial y}(-y)
=
-1
&&\Rightarrow&
\frac{\partial^2 f}{\partial y\,\partial x}(0,0)
&=
-1
\\
\frac{\partial^2 f}{\partial x\,\partial y}(x,0)
&=
\frac{\partial}{\partial x}(x)
=
1
&&\Rightarrow&
\frac{\partial^2 f}{\partial x\,\partial y}(0,0)
&=
1.
\end{align*}
Die erste Gleichung entspricht den roten Geraden über der $y$-Achse
in Abbildung~\ref{buch:fuvar:richtungsableitung:fig:schwarz},
die zweite entspricht den roten Geraden über der $x$-Achse.
Die roten Geraden zeigen daher gemischte zweite Ableitungen in
verschiedenen Differentiationsreihenfolgen mit verschiedenen
Werten.
Somit sind die Grenzwerte an der Stelle $(0,0)$ verschieden und
die Funktion der gemischten Ableitungen ist nicht stetig an der Stelle
$(0,0)$.

Die Abbildung~\ref{buch:fuvar:richtungsableitung:fig:schwarz}
zeigt auch, dass der Funktionsgraph sowohl von $\frac{\partial f}{\partial x}$
und $\frac{\partial f}{\partial y}$ im Nullpunkt keine lineare
Ersatzfunktion hat.
\end{beispiel}

%
% Richtungsableitung
%
\subsection{Richtungsableitung}
Die partiellen Ableitungen einer Funktion $f(x_1,\dots,x_n)$
nach der Variablen $x_k$ entstehen dadurch, dass alle Variablen
ausser $x_k$ konstant gehalten werden und die so entstehende
Funktion der einzigen Variablen $x_k$ abgeleitet wird.
%
% richtungsableitung.tex -- Richtungsableitung einer Funktion f(x,y)
%
% (c) 2024 Prof Dr Andreas Müller
%
\begin{figure}
\centering
XXX Richtungsableitung einer Funktion $f(x,y)$
%\includegraphics{chapters/010-fuvar/images/richtungsableitung.pdf}
\caption{Definition der Richtungsableitung einer Funktion $f(x,y)$
zweier Variablen.
\label{buch:fuvar:richtungsableitung:fig:richtungsableitung}}
\end{figure}

%
% partrichtung.tex
%
% (c) 2024 Prof Dr Andreas Müller
%
\begin{figure}
\centering
\includegraphics{chapters/010-fuvar/images/fxy.pdf}
\caption{Die partiellen Ableitungen sind Richtungsableitungen der
Funktion in Richtung der Koordinatenachsen.
\label{buch:fuvar:richtungsableitung:fig:partrichtung}}
\end{figure}

Die Funktion $f$ kann aber noch auf eine andere Art zu einer
Funktion nur einer Variablen gemacht werden.
Dazu verwendet man die Parameterdarstellung $x(t) = x + vt$ einer
Geraden durch den Punkt $x$ mit Richtungsvektor $v$
(Abbildung~\ref{buch:fuvar:richtungsableitung:fig:richtungsableitung}).
Die Funktion $f(x+vt)$ hängt nur noch von der Variablen $t$ ab
und kann wie gewohnt nach $t$ abgeleitet.

\begin{definition}
\label{buch:fuvar:richtungsableitung:def:richtungsableitung}
Die {\em Richtungsableitung} der Funktion $f\colon\mathbb{R}^n\to\mathbb{R}^m$
\index{Richtungsableitung}%
im Punkt $x\in\mathbb{R}^n$ in Richtung $v\in\mathbb{R}^n$ ist 
\[
D_vf(x)
=
\frac{d}{dt}f(x+tv)\bigg|_{t=0}.
\]
\end{definition}

Die Notation für die Richtungsableitung von
Definition~\ref{buch:fuvar:richtungsableitung:def:richtungsableitung}
konkurriert mit der Notation für die partiellen Ableitungen von
Definition~\ref{buch:fuvar:richtungsableitung:def:D}.
Tatsächlich entsteht aber kein Widerspruch, denn die 
Richtungsableitung in Richtung des Standardbasisvektors $e_k$ ist
\begin{align*}
D_{e_k}f(x)
&=
\frac{d}{dt} f(x+te_k)\bigg|_{t=0}
\\
&=
\lim_{h\to 0}
\frac{f(x_1,\dots,x_k+t,\dots,x_n)-f(x_1,\dots,x_k,\dots,x_n)}{h}
\\
&=
D_kf(x)
\end{align*}
oder kurz $D_{e_k}=D_k$
(siehe auch Abbildung~\ref{buch:fuvar:richtungsableitung:fig:partrichtung}).

Die Ableitung der Funktion $t\mapsto x+vt$ für $x\in\mathbb{R}^n$
und $v\in\mathbb{R}^n$ nach der einen Variablen $t$ ist
\[
\frac{d}{dt}
\begin{pmatrix}
x_1+v_1t\\
x_2+v_2t\\
\vdots  \\
x_n+v_nt
\end{pmatrix}
=
\begin{pmatrix}
v_1\\
v_2\\
\vdots\\
v_n
\end{pmatrix}.
\]
Damit kann die Richtungsableitung einer Funktion
$f\colon\mathbb{R}^n\to\mathbb{R}$
im Punkt $x\in\mathbb{R}^n$ kann mit der Kettenregel als
\begin{equation}
D_vf(x)
=
\frac{\partial f}{\partial x_1}(x) v_1
+
\dots
+
\frac{\partial f}{\partial x_n}(x) v_n
=
\sum_{k=1}^n \frac{\partial f}{\partial x_k} v_k
\label{buch:fuvar:richtungsableitung:eqn:richtungsableitungkette}
\end{equation}
berechnet werden.

%
% Gradient
%
\subsection{Gradient}
Die Schreibweise
\eqref{buch:fuvar:richtungsableitung:eqn:richtungsableitungkette}
für die Richtungsableitung der Funktion $f\colon\mathbb{R}^n\to\mathbb{R}$
lässt sich auch als Skalarprodukt des Vektors $v$ mit einem Vektor
bestehend aus den partiellen Ableitungen schreiben.

\begin{definition}
\label{buch:fuvar:richtungsableitung:def:gradient}
Der {\em Gradient} der Funktion $f\colon\mathbb{R}^n\to\mathbb{R}$ ist der
Vektor
\[
\operatorname{grad}f(x)
=
{
\renewcommand{\arraystretch}{1.9}
\begin{pmatrix}
\displaystyle
\frac{\partial f}{\partial x_1}(x)\\
\displaystyle
\frac{\partial f}{\partial x_2}(x)\\
\vdots \\
\displaystyle
\frac{\partial f}{\partial x_n}(x)
\end{pmatrix}
}
\]
\index{Gradient}
\end{definition}

Der Gradient ist auch die transponierte Matrix der Jacobi-Matrix:
$\operatorname{grad}f(x) = \transpose{Df(x)} = \transpose{J_f(x)}$.
Als weitere Notation ist ausserdem der Nabla-Operator gemäss der folgenden
Definition verbreitet.

\begin{definition}[Nabla-Operator]
\label{buch:fuvar:richtungsableitung:def:nabla}
Der {\em Nabla-Operator} ist der Differentialoperator
\index{Nabla-Operator}%
\index{Operator!Nabla-}%
\[
\nabla 
=
{
\renewcommand{\arraystretch}{1.8}
\begin{pmatrix}
\displaystyle
\frac{\partial}{\partial x_1}\\
\displaystyle
\frac{\partial}{\partial x_2}\\
\vdots\\
\displaystyle
\frac{\partial}{\partial x_n}
\end{pmatrix}
}
\]
auf Funktionen
$\mathbb{R}^n\to\mathbb{R}$, der durch
\[
\nabla f
=
{
\renewcommand{\arraystretch}{1.9}
\begin{pmatrix}
\displaystyle
\frac{\partial f}{\partial x_1}(x)\\
\displaystyle
\frac{\partial f}{\partial x_2}(x)\\
\vdots\\
\displaystyle
\frac{\partial f}{\partial x_n}(x)
\end{pmatrix}
}
=
\operatorname{grad}f(x)
\]
definiert ist.
\end{definition}




%
% 2-kritisch.tex
%
% (c) 2023 Prof Dr Andreas Müller
%
\section{Kritische Punkte
\label{buch:fuvar:section:kritisch}}
\kopfrechts{Kritische Punkte}

\begin{verbatim}
- verschwindende erste Ableitungen
- notwendige Bedingung für Extremum
- Orthogonalität
\end{verbatim}

%
% 3-nebenbedingungen.tex
%
% (c) 2023 Prof Dr Andreas Müller
%
\section{Nebenbedingungen
\label{buch:fuvar:section:nebenbedingungen}}
\kopfrechts{Nebenbedingungen}
Die Nullstellen des Gradienten sind Kandidaten für die Lösung
des Extremalproblems einer Funktion $f\colon\mathbb{R}^n\to\mathbb{R}$.
Oft trifft man jedoch Situation, in denen sich das Argument
$x\in\mathbb{R}^n$ der Funktion $f(x)$ nur auf einer Teilmenge
bewegen darf.
In diesem Abschnitt soll ein Verfahren beschrieben werden, wie sich
ein solches Problem exakt formulieren und lösen lässt.

%
% Nebenbedingungen
%
\subsection{Nebenbedingungen}
Wir betrachten das folgende Problem als Beispiel, wie sich 
Nebenbedingungen formulieren lassen.

%
% Ein Extremalproblem auf einer Kugeloberfläche
%
\subsubsection{Ein Extremalproblem auf einer Kugeloberfläche}
Auf der Oberfläche der Einheitskugel mit Mittelpunkt im Nullpunkt
des $(x_1,x_2,x_3)$-Koor\-di\-na\-ten\-sys\-tems sind die Extrema
der Funktion 
\begin{equation}
f(x)
=
x_2^2
-
x_1x_2
-
x_1x_3
-
x_2x_3
\label{buch:fuvar:nebenbedingungen:eqn:beispielf}
\end{equation}
zu finden.

%
% Lösung durch Umparametrisierung
%
\subsubsection{Lösung durch Umparametrisierung}
Eine erster Lösungsansatz ist, die Kugeloberfläche mit Kugelkoordinaten
\[
\begin{pmatrix}
x_1\\
x_2\\
x_3
\end{pmatrix}
=
\begin{pmatrix}
\sin\vartheta\cos\varphi\\
\sin\vartheta\sin\varphi\\
\cos\vartheta
\end{pmatrix}
\]
zu parametrisieren.
Durch Einsetzen der Parametrisierung in
\eqref{buch:fuvar:nebenbedingungen:eqn:beispielf}
entsteht die Funktion
\begin{align*}
f(\varphi,\vartheta)
&=
\sin^2\vartheta
\sin\varphi
(\sin\varphi
-
\cos\varphi)
-
(
\cos\varphi
+
\sin\varphi
)
\sin\vartheta
\cos\vartheta.
\end{align*}
Das Problem ist damit zu einem Extremalproblem in zwei Dimensionen
geworden und kann mit den bereits behandelten Methoden gelöst werden.

%
% Zusätzliche Bedingungen
%
\subsubsection{Zusätzliche Bedingungen}
Der Lösungsansatz durch Umparametrisierung wird weiter erschwert,
wenn zusätzliche Bedingungen erfüllt werden muss.
Wird zusätzlich Verlangt, dass der Punkt $x\in\mathbb{R}^3$ auf
dem Ellipsoid mit der Gleichung
\begin{equation}
g(x)
=
\begin{pmatrix}
x_1\\x_2\\x_3
\end{pmatrix}^t
\begin{pmatrix}
2&1&1\\
1&3&1\\
1&1&4
\end{pmatrix}
\begin{pmatrix}
x_1\\x_2\\x_3
\end{pmatrix}
=
1.
\label{buch:fuvar:nebenbedingungen:eqn:beispielg},
\end{equation}
liegt, dann werden die zulässigen Punkte durch eine Kurve in der
$(\varphi,\vartheta)$-Ebene beschrieben, für die eine Parametrisierung
gefunden werden muss.
Eine solche Lösung ist offenbar sehr kompliziert.
Gesucht wird ein Lösungsansatz, der ohne Umparametrisierung das folgende
allgemeine Problem löst.

\begin{aufgabe}
\label{buch:fuvar:nebenbedingungen:aufgabe:grund}
Gegeben sind stetig differenzierbare Funktionen
$f\colon\mathbb{R}^n\to\mathbb{R}$ und
$g_i\colon\mathbb{R}^n\to\mathbb{R}$ für $i=1,\dots,k$.
Man finde ein Extremum der Funktion $f(x)$ unter der Bedingungen,
dass
\[
g_1(x) = 0,\quad g_2(x)=0,\quad\dots,\quad g_k(x)=0
\]
gilt.
\end{aufgabe}

%
% Extremalaufgaben mit einer Nebenbedingung
%
\subsection{Extremalaufgaben mit einer Nebenbedingung}
In diesem Abschnitt untersuchen wir das Extremum einer stetig
differenzierbaren Funktion $f\colon\mathbb{R}^n\to\mathbb{R}$
unter der Bedinung $g(x)=0$ mit einer stetig differenzierbaren
Funktion $g\colon\mathbb{R}^n\to\mathbb{R}$.
Sei also $x_0\in\mathbb{R}^n$ so, dass $f(x_0)$ minimal ist unter
allen Punkten $x\in \mathbb{R}^n$, die die Nebenbedingungen $g(x)=0$
erfüllen.
Da $x_0$ ein Minimum ist, müssen die Werte $f(x)$ für Punkte $x$
in der Nähe von $x_0$ grösser sein, allerdings nur, wenn $x$ auch
die Bedingungen $g(x)=0$ erfüllt.

Für ein Extremalproblem ohne Nebenbedingung wurde das Verschwinden
aller Richtungsableitung als notwendige Bedingung für das
Extremum gefunden.
Tatsächlich nimmt die Funktion auch in jeder Richtung ausgehend
von $x_0$ zu.
Unter der Nebenbedingung gilt dies nicht mehr.
Nur noch Änderungen von $x$, die auch die Nebenbedingung
erfüllen, durfen in Betracht gezogen werden.
Die von $x_0$ ausgehdenden Geraden $t\mapsto x_0+vt$ mit dem
Richtungsvektor $v$ erfüllen die Nebenbedingungen nur dann, wenn
die Menge $M=\{x\in\mathbb{R}^n \mid g(x)=0\}$ Geraden enthält, also
typischerweise gar nicht.

Es müssen also Kurven $t\mapsto x(t)\in\mathbb{R}^n$ mit $x(0)=x_0$
betrachtet werden, die $g(x(t))=0$ für alle $t$ in einer kleinen Umgebung
von $0$ erfüllen.
Für ein Minimum ist dann notwendig, dass die Ableitung
\begin{equation}
0
=
\frac{d}{dt} f(x(t))\bigg|_{t=0}
=
Df(x_0)\cdot \frac{dx(t)}{dt}\bigg|_{t=0}
=
Df(x_0)\cdot \dot{x}(0)
\label{buch:fuvar:nebenbedingungen:eqn:gradf}
\end{equation}
ist.
Nicht jeder Vektor $v=\dot{x}$ kann vorkommen, da $\dot{x}(0)$ ein
Tangentialvektor der Menge $M$ ist.
Da $g(x(t))=0$ ist, sind diese Vektoren durch
\begin{equation}
0
=
\frac{d}{dt}g(x(t))\bigg|_{t=0}
=
Dg(x_0)\cdot \dot{x}(0)
=
\grad g(x_0)\cdot \dot{x}(0)
\label{buch:fuvar:nebenbedingungen:eqn:gradg}
\end{equation}
charakterisiert.
Nur Vektoren $\dot{x}(0)$, die auf dem Gradienten $\grad g(x_0)$ 
senkrecht stehen, sind zulässig.
Wir halten dieses Resultat im folgenden Lemma fest.

\begin{lemma}
\label{buch:fuvar:nebenbedingungen:lemma:nebenbedingungen}
Ist $x_0$ ein Extremum der Funktion $f(x)$ unter der Nebenbedingung
$g(x)=0$, dann verschwindet die Richtungsableitung
\(
D_vf(x_0) = \grad f(x_0)\cdot v = 0
\)
von $f$ an der Stelle $x_0$ in jeder Richtung $v$, die tangential
an die Menge $\{x\in\mathbb{R}^n \mid g(x)=0\}$ ist, für die also
$\grad g(x_0)\cdot v=0$ gilt.
\end{lemma}

Damit haben wir die folgende notwendige Bedingung gefunden.

\begin{satz}
Sei $f\colon\mathbb{R}^n\to\mathbb{R}$ eine stetig differenzierbare
Funktion, die im Punkt $x_0\in\mathbb{R}^n$ unter allen Punkten
$x\in\mathbb{R}^n$ ein Extremum annimmt unter der Nebenbedingung
$g(x)=0$ für eine stetig differenzierbare Funktion
$g\colon\mathbb{R}^n\to\mathbb{R}$ mit nicht verschwindendem Gradienten.
Dann ist $\grad{f}(x_0)$ ein Vielfaches von $\grad{g}(x_0)$.
\end{satz}

\begin{proof}
Nach
\eqref{buch:fuvar:nebenbedingungen:eqn:gradf}
und
\eqref{buch:fuvar:nebenbedingungen:eqn:gradg}
muss 
\[
\grad f(x_0)\cdot v = 0
\]
sein für alle Vektoren $v$ mit $\grad g(x_0)\cdot v= 0$.
Der Vektor $\grad f(x_0)$ steht also auf allen Vektoren senkrecht, die
auf $\grad g(x_0)$ senkrecht stehen.
Dies trifft wegen $\grad g(x_0)\ne 0$ genau dann zu, wenn $\grad f(x_0)$
ein Vielfaches von $\grad g(x_0)$ ist.
\end{proof}

\begin{beispiel}
Wir lösen die eingangs gestellte Aufgabe, die Extrema der Funktion
$f(x)$ von \eqref{buch:fuvar:nebenbedingungen:eqn:beispielf}
unter der Nebenbedingung zu finden, dass $|x|=1$ ist.
Die Funktion $g(x)$ ist
\[
g(x)
=
x_1^2 + x_2^2 + x_3^2 - 1.
\]
Die Gradienten von $f$ und $g$ sind
\[
\grad f(x)
=
\begin{pmatrix*}[r]
-x_2-x_3\\
-x_1+2x_2-x_3\\
-x_1-\phantom{2}x_2\phantom{\mathstrut-x_3}
\end{pmatrix*}
\qquad\text{und}\qquad
\grad g(x)
=
\begin{pmatrix}
2x_1\\
2x_2\\
2x_3
\end{pmatrix}
=
2x.
\]
Kandidaten für Extremale sind Punkte, in denen die beiden Gradienten
proportional sind, für die also das Gleichungssystem
\begin{equation}
\left.
\begin{aligned}
\grad f({\color{darkred}x})&=\lambda \grad g({\color{darkred}x})\\
      g({\color{darkred}x})&=0
\end{aligned}
\right\}
\quad\Rightarrow\quad
\left\{
\renewcommand\arraycolsep{2pt}
\begin{array}{rcrcrcr}
	&-& {\color{darkred}x_2}
		&-& {\color{darkred}x_3}
			&=& 2{\color{darkred}\lambda x_1}\\
-{\color{darkred}x_1}
	&+& 2{\color{darkred}x_2}
		&-& {\color{darkred}x_3}
			&=& 2{\color{darkred}\lambda x_2}\\
-{\color{darkred}x_1}
	&-& {\color{darkred}x_2}
		& &
			&=& 2{\color{darkred}\lambda x_3}\\
{\color{darkred}x}^2_{\color{darkred}1}
	&+& {\color{darkred}x}^2_{\color{darkred}2}
		&+& {\color{darkred}x}^2_{\color{darkred}3}
			&=& 1
\end{array}
\right.
\label{buch:fuvar:nebenbedingungen:eqn:glsystem}
\end{equation}
Dies ist ein System von vier nichtlinearen Gleichungen für die vier
Unbekannten ${\color{darkred}x_1}$, ${\color{darkred}x_2}$,
${\color{darkred}x_3}$ und ${\color{darkred}\lambda}$.
In Matrixform lassen sich die ersten drei Gleichungen als
\begin{equation}
\underbrace{
\begin{pmatrix*}[r]
 0&-1&-1\\
-1& 2&-1\\
-1&-1& 0
\end{pmatrix*}
}_{\displaystyle = A}
{\color{darkred}x}
=
A
{\color{darkred}x}
=
2{\color{darkred}\lambda x}
\label{buch:fuvar:nebenbedingungen:eqn:evgl}
\end{equation}
schreiben.
Da die vierte Gleichung von~\eqref{buch:fuvar:nebenbedingungen:eqn:glsystem}
verlangt, dass $\color{darkred}x$ nicht der Nullvektor ist, besagt
\eqref{buch:fuvar:nebenbedingungen:eqn:evgl}, dass ${\color{darkred}}$
ein Eigenvektor von $A$ mit Eigenwert $2{\color{darkred}\lambda}$ sein muss.
Das charakteristische Polynom
\[
\det(A-tI)
=
-t^3+2t^2+3t-4
=
-(t-1)(t^2+t-4)
\]
hat die Nullstellen $t_0=1$ und
\[
t_{\pm}
=
\frac{-1\pm\sqrt{17}}{2},
\]
was auf die zulässigen Werte
\[
{\color{darkred}\lambda}_0=-\frac12,
\quad\text{und}\quad
{\color{darkred}\lambda}_\pm = \frac{1\mp\sqrt{17}}{4}
\]
für ${\color{darkred}\lambda}$ führt.
Mit etwas Geduld oder einem Computeralgebrasystem kann man auch die
zugehörigen Eigenvektoren
\[
{\color{darkred}x}_0
=
\begin{pmatrix*}[r]
 1\\
 0\\
-1
\end{pmatrix*}
\quad\text{und}\quad
{\color{darkred}x}_\pm
=
\begin{pmatrix*}
1\\
\frac{-3\pm\sqrt{17}}{2}\\
1
\end{pmatrix*}
\]
finden.
Durch Normierung lassen sich jetzt Punkte finden, die die Nebenbedingung
erfüllen und damit Kandidaten für Extrema sind.
\end{beispiel}

%
% Lagrange-Multiplikatoren
%
\subsection{Lagrange-Multiplikatoren}
Wir kehren zum allgemeinen Problem der
Aufgaben~\ref{buch:fuvar:nebenbedingungen:aufgabe:grund}
zurück.
Sei also wieder $f\colon \mathbb{R}^n\to\mathbb{R}$
und
$g_i\colon\mathbb{R}^n\to\mathbb{R}$, $i=1,\dots,k$,
stetig differenzierbare Funktionen und $x_0$ ein Extremum
von $f$ unter allen $x\in\mathbb{R}^n$ mit $g_i(x)=0$ für $i=1,\dots,k$.
Die Richtungsableitung von $f$ an der Stelle $x_0$ muss für alle
Richtungen $v$ verschwinden, die tangential an die Menge
\[
M
=
\{
x\in\mathbb{R}^n
\mid
g_1(x)=\ldots=g_k(x)=0
\}
\]
sind.
Die zulässigen Richtungen $v$ erfüllen die Bedingung
\[
\grad g_i(x_0)\cdot v  = 0,\quad i=1,\dots,k,
\]
In Matrixform sind dies die Lösungen des homogenen Gleichungssystems
\[
{\renewcommand{\arraystretch}{1.2}
\begin{pmatrix}
\displaystyle\frac{\partial g_1}{\partial x_1}(x_0)
&\dots&
\displaystyle\frac{\partial g_1}{\partial x_n}(x_0)
\\
\vdots&\ddots&\vdots\\
\displaystyle\frac{\partial g_k}{\partial x_1}(x_0)
&\dots&
\displaystyle\frac{\partial g_k}{\partial x_n}(x_0)
\end{pmatrix}}
\begin{pmatrix}
v_1\\
\vdots\\
v_n
\end{pmatrix}
=
Dg(x_0) v
=
0.
\]

\begin{satz}
\label{buch:fuvar:nebenbedingungen:satz:lm}
Seien $f\colon\mathbb{R}^n\to\mathbb{R}$ und
$g_i\colon\mathbb{R}^n\to\mathbb{R}$, $i=1,\dots,k$, stetig
differenzierbare Funktionen.
Für ein Extremum $x_0$ von $f(x)$ unter allen $x\in\mathbb{R}^n$,
die die Nebenbedingungen $g_i(x)=0$, $i=1,\dots,k$ erfüllen, ist notwendig,
dass es reelle Zahlen $\lambda_1,\dots,\lambda_k\in\mathbb{R}$ gibt derart,
dass
\[
\grad f(x_0)
=
\lambda_1\grad g_1(x_0)
+\ldots+
\lambda_k\grad g_k(x_0)
=
\sum_{i=1}^k \lambda_i \grad g_i(x_0)
\]
gilt.
Kandidaten für Extrema von $f$ unter den Nebenbedingungen $g_i(x)=0$ sind
daher Lösungen des Gleichungssystems
\begin{equation}
\begin{aligned}
\grad f({\color{darkred}x})
&=
\sum_{i=1}^k {\color{darkred}\lambda_i} g_i({\color{darkred}x})
\\
g_i({\color{darkred}x})
&= 
0\qquad i=1,\dots k
\end{aligned}
\label{buch:fuvar:nebenbedingungen:eqn:lm}
\end{equation}
für ${\color{darkred}x}$ und ${\color{darkred}\lambda_i}$, $i=1,\dots,k$.
\end{satz}

Die erste Gleichung des Gleichungssytems
\eqref{buch:fuvar:nebenbedingungen:eqn:lm}
ist eine $n$-dimensionale Vektorgleichung, entspricht also $n$
Komponentengleichungen.
Sie kann auch geschrieben werden als ein lineares Gleichungssystem
mit $n$ Gleichungen für die Variablen
${\color{darkred}\lambda_1},\dots,{\color{darkred}\lambda_k}$
mit der Koeffizientenmatrix $Dg(x_0)^t$.
Die Nebenbedingungen steuern $k$ weitere Gleichungen bei.
Insgesamt ist \eqref{buch:fuvar:nebenbedingungen:eqn:lm}
daher ein im Allgemeinen nichtlineares Gleichungssystem von $n+k$
Gleichungen für die $n+k$ Unbekannten
\(
{\color{darkred}x_1},\dots,{\color{darkred}x_n},
{\color{darkred}\lambda_1},\dots,{\color{darkred}\lambda_k}
\).
Leider kann man keine allgemeinen Lösungsverfahren für solche
Gleichungen erwarten.
Eine interessante Ausnahme liegt vor, wenn die Funktionen $f$ und $g_i$
quadratische Funktionen sind, dies wird im
Abschnitt~\ref{buch:fuvar:section:quadratisch}
untersucht.

Alternativ kann man den Satz
\ref{buch:fuvar:nebenbedingungen:satz:lm}
auch so formulieren:

\begin{satz}
\label{buch:fuvar:nebenbedingungen:satz:lm2}
Unter den Voraussetzungen von Satz~\eqref{buch:fuvar:nebenbedingungen:satz:lm}
ist für ein Extremum von $f$ unter den Nebenbedingungen $g_i$, $i=1,\dots,k$,
notwendig, dass es
es relle Zahlen $\lambda_1,\dots,\lambda_k\in\mathbb{R}$ derart gibt,
dass 
\[
\grad\biggl(f-\sum_{i=1}^k \lambda_ig_i\biggr)(x_0) = 0
\qquad\text{und}\qquad
g_i(x_0)=0\quad \forall i=1,\dots,k
\]
gilt.
\end{satz}


%
% 4-hessische.tex
%
% (c) 2023 Prof Dr Andreas Müller
%
\section{Hessische Matrix
\label{buch:fuvar:section:hessische}}
\kopfrechts{Hessische Matrix}
Das Verschwinden der Ableitung einer differenzierbaren Funktion
$f\colon\mathbb{R}\to\mathbb{R}$ ist eine notwendige Bedingung
für ein Extremum, sie ist aber nicht hinreichend.
Im Grundlagenunterricht in der Analysis lernt man, dass man mit
Hilfe der zweiten Ableitung zu einem hinreichenden Kriterium
kommen kann.
Ist die zweite Ableitung $f''(x_0)$ an einer Stelle $x_0$ grösser
als $0$, dann liegt ein Minimum vor, falls sie negativ ist, liegt
ein Maximum vor.
In diesem Abschnitt soll gezeigt werden, wie dieses Kriterium auf
Funktionen mehrere Variablen verallemeinert werden kann.

%
% Die zweite Ableitung
%
\subsection{Die zweite Ableitung}
In diesem Abschnitt ist $f\colon\mathbb{R}^n\to\mathbb{R}$
eine zweimal stetig differenzierbare Funktionen, derer Gradient
an der Stelle $x_0$ verschwindet.
Durch eine Translationb $x\mapsto x-x_0$ kann man immer erreichen,
dass $x_0=0$ der Nullpunkt des Koordinatensystems ist.
In der folgenden Diskussion wird diese Konvention wo zweckmässig
stillschweigend verwendet.

%
% Die Hessische Matrix
%
\subsubsection{Die Hessische Matrix}
Damit der Punkt $x_0$ ein Minimum ist, muss jede auf die Richtung
$v\in\mathbb{R}^n$ eingeschränkte Funktion eine positive zweite
Ableitung haben.
Es muss also für alle $v\in\mathbb{R}^n$
\[
\frac{d^2}{dt^2} f(x_0+vt)\bigg|_{t=0}
> 0
\]
sein.
Wenn das umgekehrte Zeichen gilt, dann liegt ein Maximum vor.
Ausgeschrieben in Komponenten gilt
\[
\frac{d^2}{dt^2} f(x_0+vt)
=
\sum_{i,k=1}^n
\frac{\partial^2 f}{\partial x_i\,\partial x_k}(x_0+vt)
\,
v_iv_k,
\]
wobei die $v_i$ die Komponenten von $v$ sind.
An der Stelle $t=0$ bleibt die Eigenschaft
\[
\frac{d^2}{dt^2}f(x_0+vt)_{t=0}
=
\sum_{i,k=1}^n \frac{\partial^2 f}{\partial x_i\,\partial x_k}(x_0) \,v_iv_k
>
0.
\]
Die Summe kann mit Hilfe der Matrix $H$ mit den Einträgen
\[
h_{ik}
=
\frac{\partial^2 f}{\partial x_i\,\partial x_k}(x_0)
\]
kompakter als
\[
\frac{d^2}{dt^2}f(x_0+vt)_{t=0}
=
\sum_{i,k=1}^n \frac{\partial^2 f}{\partial x_i\,\partial x_k}(x_0) \,v_iv_k
=
v^tHv
>
0
\]
geschrieben werden.
Ein hinreichendes Kriterium für ein Minimum wird also zu einer
Eigenschaft der Matrix $H$.

\begin{definition}[Hessische Matrix]
Ist $f\colon\mathbb{R}^n\to\mathbb{R}$ eine zweimal stetig differenzierbare
Funktion, dann heisst die Matrix
\[
H(x_0)
=
\begin{pmatrix}
h_{11}(x_0)&\dots &h_{1n}(x_0)\\
\vdots&\ddots&\vdots\\
h_{n1}(x_0)&\dots &v_{nn}(x_0)
\end{pmatrix}
\qquad\text{mit}\qquad
h_{ik}(x_0)
=
\frac{\partial^2 f}{\partial x_i\,\partial x_k}(x_0)
\]
die {\em hessische Matrix}.
\end{definition}

Da die Funktion $f$ zweimal stetig differenzierbar angenommen wurde,
sind die partiellen Ableitungen nach dem Satz von Schwarz
\[
\frac{\partial^2 f}{\partial x_i\,\partial x_k}(x_0)
=
\frac{\partial^2 f}{\partial x_k\,\partial x_i}(x_0)
\]
vertauschbar.
Die hessische Matrix $H(x_0)$ ist daher immer symmetrisch: $H(x_0)^t=H(x_0)$.

%
% Notwendiges Kriterium für ein Extremum
%
\subsubsection{Notwendiges Kriterium für ein Extremum}
Das Vorzeichen von $v^tH(x_0)v$ entscheidet darüber, ob ein Extremum
vorliegt.
Dies führt auf die folgende Definition.

\begin{definition}
Eine symmetrische Matrix $A$ heisst {\em positiv definit}, wenn 
\index{positiv definit}%
\index{definit, positiv}%
$v^tH(x_0)v>0$ für alle Vektoren $v\in\mathbb{R}^n\setminus\{0\}$ ist.
Sie heisst {\em positiv semidefinit}, wenn $v^tH(x_0)v\ge 0$ für alle
Vektoren
\index{positiv semidefinit}%
\index{semidefinit, positiv}%
$v\in\mathbb{R}^n\setminus\{0\}$ ist.
\end{definition}

\begin{satz}[Hinreichendes Kriterium für Extremum]
Wenn die hessische Matrix einer zweimal stetig differenzierbaren
Funktion $f\colon\mathbb{R}^n\to\mathbb{R}$ mit $\grad f(x_0)=0$ 
an der Stelle $x_0$ positiv definit ist, dann hat $f$ an dieser
Stelle ein Minimum.
Wenn die hessische Matrix negativ definit ist, dann liegt ein Maximum
vor.
\end{satz}

Im Falle einer Funktion einer Variablen lässt sich im Fall $f'(x_0)=0$
und $f''(x_0)=0$ keine Aussage darüber machen lässt, ob ein Maximum 
oder Minimum vorliegt.
Bei einer Funktion mehrerer Variablen verunmöglicht bereits Semidefinitheit
der hessischen Matrix, dass an der Stelle $x_0$ ein Extremum vorliegt.
Zum Beispiel hat die Funktion $f(x,y)=x^2+y^3$ die hessische Matrix
\[
H(0)
=
\begin{pmatrix}
2&0\\
0&0
\end{pmatrix}
\qquad\text{die wegen}\qquad
v^tH(0)v = v_1^2 \ge 0
\]
positiv semidefinit ist.
Trotzdem hat $f$ im Nullpunkt kein Extremum.
Andererseits hat die Funktion $f(x,y)=x^2+y^4$ die gleiche hessische
Matrix im Nullpunkt, es liegt dort aber ein Minimum vor.

%
% Extremalkriterien
%
\subsubsection{Extremalkriterien}
Das definierende Kriterium der positiven Definitheit, dass $v^tH(x_0)v>0$
sein muss für alle Vektoren $v\in\mathbb{R}^n\setminus\{0\}$, ist
relativ schwer nachzuprüfen.
Daher sind alternative Kriterien wünschenswert, die leichter zu überprüfen
sind.

\begin{satz}[positive Eigenwerte]
\label{buch:fuvar:hessische:satz:positiveeigenwerte}
Ist $f\colon\mathbb{R}^n\to\mathbb{R}$ eine zweimal stetig differenzierbare
Funktion mit $\grad f(x_0)=0$, dann liegt an der Stelle $x_0$ ein Minimum
vor, wenn alle Eigenwerte der hessischen Matrix positiv sind.
\end{satz}

\begin{proof}
Da die hessische Matrix symmetrisch ist, lässt sie sich orthogonal
diagonalisieren.
Seien $u_i$ orthonormierte Eigenvektoren mit $Hu_i=\lambda_iu_i$ und
den Eigenwerten $\lambda_i>0$.
Jeder Vektor $v$ kann also als Linearkombination
\[
v = \sum_{i=1}^n v_iu_i
\]
der Eigenvektoren $u_i$ geschrieben werden und 
\[
v^tHv
=
\sum_{i,k=1}^n
v_ku_k^t
Hv_iu_i
=
\sum_{i,k=1}^n v_iv_k\lambda_i \underbrace{u_k^tu_i}_{\displaystyle=\delta_{ik}}
=
\sum_{i=1}^{n} v_i^2\lambda_i > 0,
\]
somit ist die hessische Matrix positiv definit und es liegt ein
Minimum vor.
\end{proof}

Die Anwendung von Satz~\ref{buch:fuvar:hessische:satz:positiveeigenwerte}
verlangt die Berechnung aller Eigenwerte der hessischen Matrix, was für
$n>2$ eher beschwerlich ist.
Für eine symmetrische Matrix gibt es aber ein numerisch sehr einfach
nachzuprüfendes Kriterium für positive Definitheit.

\begin{satz}[Cholesky-Zerlegung]
Sei $f\colon\mathbb{R}^n\to\mathbb{R}$ eine zweimal stetig differenzierbare
Funktion mit $\grad f(x_0)=0$.
Falls es eine untere Dreiecksmatrix $L$ mit $\det L\ne 0$ gibt derart,
dass $H(x_0)=LL^t$ ist, dann liegt an der Stelle $x_0$ ein Minimum vor.
\end{satz}

\begin{proof}
Für jeden beliebigen Vektor $v\in\mathbb{R}\setminus\{0\}$ folgt
\[
v^tHv
=
v^tLL^tv
=
(L^tv)^t L^tv
=
|L^tv|^2
\ge
0.
\]
Da die Matrix $L$ ausserdem invertierbar ist, gilt sogar $v^tHv>0$.
Damit folgt, dass die hessische Matrix positiv definit ist und dass
ein Minimum vorliegt.
\end{proof}

Die Zerlegung $H=LL^t$ heisst auch die Cholesky-Zerlegung, für die
\label{Cholesky-Zerlegung}%
es einen numerischen Algorithmus gibt, der sehr viel effizienter
durchzuführen ist als die Bestimmung der Eigenwerte
\cite[Abschnitt 12.3]{buch:linalg}.


%
% Die Tayloer-Reihe für Funktionen mehrere Variablen
%
\subsection{Die Taylor-Reihe für Funktionen mehrere Variablen}

\begin{verbatim}
- Taylorreihe in n Variablen
\end{verbatim}

%
% 5-quadratische.tex
%
% (c) 2023 Prof Dr Andreas Müller
%
\section{Quadratische Minimalprobleme
\label{buch:fuvar:section:quadratisch}}
\kopfrechts{Quadratische Minimalproblem}

\begin{verbatim}
- Scheitelpunkt einer Parabel
- Minimalproblem für eine positiv definite Matrix
- Gradientabstieg
\end{verbatim}


