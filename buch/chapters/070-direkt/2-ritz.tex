%
% ritz.tex -- Das Verfahren von Ritz
%
% (c) 2024 Prof Dr Andreas Müller
%
\section{Das Verfahren von Ritz
\label{buch:direkt:section:ritz}}
Zu einem gegebenen Funktional
\begin{equation}
I(y)
=
\int_{x_1}^{x_2}
L(x,y(x),y'(x))
\,dx
\label{buch:direkt:ritz:eqn:funktional}
\end{equation}
soll eine Funktion $y(x)$ gefunden werden, die das Funktional
extremal macht.

%
% Idee der Methode
%
\subsection{Idee der Methode}

%
% Reduktion auf eine endlichdimensionales Problem
%
\subsubsection{Reduktion auf ein endlichdimensionales Problem}
In Abschnitt~\ref{buch:direkt:section:gradient} wurde gezeigt, wie 
Extermalprobleme für endlich viele Variablen zum Beispiel mit der
Gradientabstiegsmethode gelöst werden können.
Diese Methoden sind in dieser Form nicht auf das Variationsproblem
\eqref{buch:direkt:ritz:eqn:funktional} anwendbar, da die Funktion
$y(x)$ nicht nur die Information enthält, die in endlich vielen
Koordinaten $x_1,\dots,x_n$ stecken kann.
Es ist daher nötig, die in Frage kommenden Funktionen durch eine
endlich Zahl von Parametern zu beschreiben.
Man wählt daher Funktionen $\psi_1(x),\dots,\psi_n(x)$ und beschränkt
sich auf Funktion $y(x)$ der Form
\[
y(x)
=
\sum_{k=1}^n
a_k \psi_k(x).
\]
Der Funktionenraum 
\[
B_n
=
\biggl\{
y(x)
=
\sum_{k=1}^n a_k\psi_k(x)
\bigg|
a_1,\dots,a_k\in\mathbb{R}
\biggr\}
\]
ist endlichdimensional und kann durch die Koeffizienten $a_1,\dots,a_n$
parametrisiert werden.
Die Funktionen $\psi_k$ heissen auch die Koordinatenfunktionen und die
$a_k$ die Koordinaten.
Das Funktional $I(y)$ wird jetzt ebenfalls eine Funktion
\[
f(a_1,\dots,a_n)
=
\int_{x_1}^{x_2}
L\biggl(x,
\sum_{k=1}^n a_k\psi_k(x),
\sum_{k=1}^n a_k\psi_k'(x)
\biggr)
\,dx
\]
der Variablen $a_1,\dots,a_n$.
Die Koordinatenfunktionen ermöglichen also, das unendlichdimensionale
Variationsproblem auf ein endlichdimensionales Problem zu reduzieren.

%
% Konvergenz
%
\subsubsection{Konvergenz}
Da der Funktionenraum $B_n$ nur endlichdimensional ist, wird sich darin
im Allgemeinen nur eine Approximation der Lösung des ursprünglichen
Extremalproblems für das Funktional finden lassen.
Je genauer die Koordinatenfunktionen die Lösung zu approximieren gestatten,
desto genauer kann auch der Gradientabstieg die Lösung in $B_n$ zu finden.
Um die Genauigkeit zu steigern, müssen weitere Funktionen zu $B_n$ hinzukommen.
Wir gehen daher im Folgenden von einer Folge von Koordinatenfunktionen
$\psi_k(x)$ mit $k\in \mathbb{N}$ und den Funktionenräumen
\[
B_n
=
\langle \psi_1(x),\dots,\psi_n(x)\rangle
=
\biggl\{
\sum_{k=1}^n a_k^{(n)} \psi_k(x)
\;
\bigg|
\;
a_1^{(n)},\dots,a_n^{(n)}\in\mathbb{R}
\biggr\}
\]
für alle $n\in\mathbb{N}$ aus.
In jedem $B_n$ gibt es eine optimale Lösung
\[
y_n(x)
=
\sum_{k=1}^n a_k^{(n)} \psi_k(x)
\]
Im besten Fall konvergiert nicht nur die Folge $y_n(x)$ gegen die
Funktion $y(x)$ die das Funktional extremal macht, sondern auch
die Folge
\[
a_k^{(k)},
a_k^{(k+1)},
a_k^{(k+2)},
\dots,
a_k^{(k+n)},
\dots
\]
Koeffizienten $a_k^{(k)}$ gegen den Grenzwert $a_k$.
Durch Vergrössern von $n$ kann so eine beliebig genaue Lösung des
ursprünglichen Problems gewonnen werden.
Allerdings ist die Entscheidung, ob die genannten Folgen tatsächlich
konvergieren werden, nicht immer einfach.

%
% Refraktion mit der Methode von Ritz
%
\subsection{Refraktion mit der Methode von Ritz
\label{buch:direkt:ritz:subsection:refraktion}}
In der Übungsaufgabe~\ref{201} wurde das Problem der Refraktion eines
Lichtstrahls in der inhomogenen Atmosphäre untersucht und es wurde als
Euler-Lagrange-Differentialgleichung eine Gleichung gefunden, die nicht
direkt lösbar war.
In diesem Abschnitt versuchen wir das Problem mit dem Verfahren von Ritz
zu lösen.

\begin{aufgabe}
Man finde den Weg eines Lichtstrahls durch ein inhomogenes Medium
in der $x$-$y$-Ebene,
dessen Brechungsindex $n(y)=1+\nu y$ ist, der die Zeit zwischen
den Punkten $(0,0)$ und $(\pi,0)$ minimiert.
\end{aufgabe}

Die Lichtgeschwindigkeit ist $c/n(y)$.
Die Zeit $t$, die der Lichtstrahl entlang des Pfades $y(x)$ braucht, ist
daher
\[
t
=
\frac{1}{c}
\int_0^\pi n(y) \sqrt{1+y'(x)^2}.
\]
Dieses Funktional ist also zu minimieren.
Als approximierende Funktionen auf dem Intervall $[0,\pi]$ verwenden 
wir die Funktionen $\psi_n(x) = \sin nx$, die Funktion $y(x)$ wird also
in eine Fourier-Sinus-Reihe entwickelt.
Die Randbedingung $y(0)=y(\pi)=0$ ist automatisch immer erfüllt.

Für $n=1$ wird das Extremalproblem zu der Aufgabe, den Koeffizienten $a_1$
im Integral
\[
f(a_1)
=
\int_0^\pi (1+\nu a_1\sin x)\sqrt{1+a_1^2\sin^2 x}\,dx
\]
so zu wählen, dass das Minimum erreicht wird.
Leider ist das Integral nicht elementar auswertbar, so dass die Aufgabe
nur numerisch gelöst werden kann.
Dasselbe gilt für die Approximation $n$-ter Ordnung
\[
f_n(a_1,\dots,a_n)
=
\int_0^\pi
\biggl(1+\nu \sum_{k=1}^n a_k\sin kx\biggr)
\sqrt{1+\sum_{k=1}^n a_k^2\sin^2 x}
\,dx.
\]

%
% Die Methode von Galerkin
%
\subsection{Die Methode von Galerkin
\label{buch:direkt:ritz:subsection:galerkin}}



