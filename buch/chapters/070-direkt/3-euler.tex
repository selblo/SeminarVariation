%
% fem.tex -- Das FEM-Verfahren
%
% (c) 2024 Prof Dr Andreas Müller
%
\section{Diskretisation nach Euler
\label{buch:direkt:section:euler}}
\kopfrechts{Diskretisation nach Euler}
Leonhard Euler
\index{Euler, Leonhard}%
\index{Leonhard Euler}%
hat die Euler-Lagrange auf eine Weise hergeleitet, die eher an die 
Vorgehensweise direkter Methoden erinnert.
In diesem Abschnitt soll das Integral
\begin{equation}
I(y)
=
\int_a^b
F(x,y(x),y'(x))
\,dx
\label{buch:direkt:euler:eqn:aufgabe}
\end{equation}
mit den Randbedingungen $y(a)=y_a$ und $y(b)=y_b$ minimiert werden.

%
% Polygon-Approximation
%
\subsection{Polygon-Approximation
\label{buch:direkt:euler:subsection:polygon}}
%
% euler.tex
%
% (c) 2024 Prof Dr Andreas Müller
%
\begin{figure}
\centering
\includegraphics{chapters/070-direkt/images/euler.pdf}
\caption{Diskretisation der Lösungsfunktion für das Funktional
$I(y)$ von \eqref{buch:direkt:euler:eqn:aufgabe}, mit der Leonhard
Euler die Euler-Lagrange-Differentialgleichung hergeleitet hat.
\label{buch:direkt:euler:fig:euler}}
\end{figure}

Die gesuchte Funktion $y\colon[a,b]\mathbb{R}$ wird durch eine stückweise
lineare Funktion approximiert (Abbildung~\ref{buch:direkt:euler:fig:euler}).
Sie ist definiert durch die Werte an den $n-1$ Zwischenpunkten
$x_k=a+kh$ mit $h=(b-a)/n$.
Die Endpunkte des Intervalls sind $x_0=a$ und $x_n=b$.
Zu bestimmen sind die Funktionswerte $y_k = y(x_k)$ für $k=1,\dots,n-1$.
Die Funktionswerte $y_0=y(x_0)=y(a)=y_a$ und $y_n=y(x_n)=y(b)=y_b$ 
sind durch die Randbedingung vorgegeben.

Das Integral $I(y)$ von \eqref{buch:direkt:euler:eqn:aufgabe} muss
jetzt durch die unbekannten Werte $y_1,\dots,y_{n-1}$ approximiert
werden.
Für die Ableitung an der Stelle $x_k$ kann der Differenzenquotient
\[
y'(x_k) 
\approx
y'_k
=
\frac{y_{k+1\mathstrut}-y_{k\mathstrut}}{h}
\]
verwendet werden.
Der Wert des Integrals 
\begin{equation}
I(y)
\approx
I(y_1,\dots,y_{n-1})
=
\sum_{k=1}^{n-1}
F(x_k, y_k, y'_k)\cdot h
\label{buch:direkt:euler:eqn:summe}
\end{equation}
Die Aufgabenstellung wird damit zum folgenden endlichdimensionalen
Problem.

\begin{aufgabe}
Finde die Werte $(y_1,\dots,y_{n-1})\in\mathbb{R}^{n-1}$, für die
die Summe~\eqref{buch:direkt:euler:eqn:summe} extremal wird.
\end{aufgabe}

%
% Notwendig Bedingung
%
\subsection{Euler-Lagrange-Differentialgleichung
\label{buch:direkt:euler:subsection:eldgl}}
Eine notwendige Bedingung dafür, dass die Funktion $I(y_1,\dots,y_{n-1})$
ein Extremum annimmt, ist das Verschwinden der partiellen Ableitungen
\begin{equation}
\frac{\partial I}{\partial y_i} (y_1,\dots,y_{n-1})
=
0
\end{equation}
für $i=1,\dots,n-1$.
In der Summe~\eqref{buch:direkt:euler:eqn:summe} kommt die Variable
$y_i$ zunächst im Term mit $k=i$ vor.
Da aber $y'_k$ aus $y_{k+1}$ und $y_k$ berechnet wird, kommt $y_i$
in der Summe auch im Term mit $k=i-1$ vor.
Die Ableitung der Summe nach $y_k$ ist daher
\begin{align*}
\frac{\partial I}{\partial y_i}(y_1,\dots,y_{n-1})
&=
\frac{\partial}{\partial y_i}
\sum_{k=1}^{n-1}
F(x_k, y_k, y'_k)\cdot h
\\
&=
\frac{\partial}{\partial y_i}
F(x_i, {\color{darkred}y_i}, {\color{darkred}y'_i})\cdot h
+
\frac{\partial}{\partial y_i}
F(x_{i-1}, y_{i-1}, {\color{darkred}y'_{i-1}})\cdot h
\\
&=
\frac{\partial}{\partial y_i}
F\biggl(x_i, {\color{darkred}y_i}, \frac{y_{i+1}-{\color{darkred}y_i}}{h}\biggr)\cdot h
+
\frac{\partial}{\partial y_i}
F\biggl(x_{i-1}, y_{i-1}, \frac{{\color{darkred}y_i}-y_{i-1}}{h}\biggr)\cdot h.
\intertext{Die Ableitung kann mit der Kettenregel berechnet werden
und ergibt}
&=
\frac{\partial F}{\partial y}\biggl(x_i, y_i,\frac{y_{i+1}-y_i}{h}\biggr)
\cdot h
\\
&\qquad\qquad
-
\frac{\partial F}{\partial y'}\biggl(x_i,y_i,\frac{y_{i+1}-y_i}{h}\biggr)
+
\frac{\partial F}{\partial y'}\biggl(x_{i-1},y_{i-1},\frac{y_i-y_{i-1}}h\biggr).
\intertext{Im zweiten und dritten Term sind die Faktoren $h$ verschwunden,
weil sie gegen einen von der inneren Ableitung herstammenden Faktor $1/h$ 
gekürzt wurden.
Die letzten beiden Terme können noch zusammengefasst werden:
}
&=
h\cdot \biggl(
\frac{\partial F}{\partial y}(x_i,y_i,y'_i)
-
\frac1h\biggl(
\frac{\partial F}{\partial y'}(x_i,y_i,y'_i)
-
\frac{\partial F}{\partial y'}(x_{i-1},y_{i-1},y'_{i-1})
\biggr)\biggr).
\intertext{Die innere Klammer ist ein Differenzenquotient, der die
Ableitung nach $x$ approximiert.
Für genügend feine Diskretisation ist daher }
\frac{\partial I}{\partial y_i}(y_1,\dots,y_{n-1})
&\approx
h\cdot\biggl(
\frac{\partial F}{\partial y}(x_i,y_i,y'_i)
-
\frac{d}{dx}\frac{\partial F}{\partial y'}(x_{i-1}, y_{i-1}, y'_{i-1})
\biggr)
=
0.
\end{align*}
Die Ableitungen $\partial I/\partial y_i$ verschwinden also, wenn 
die grosse Klammer den Wert $0$ annimmt.
Eine notwendige Bedingung dafür, dass die Lösungsfunktion $y(x)$, die in den
Stützstellen $x_i$ die Werte $y(x_i) = y_i$ annimmt, das
Integral $I(y)$ von \eqref{buch:direkt:euler:eqn:aufgabe} extremal macht,
ist also, dass $y(x)$ die Euler-Lagrange-Differentialgleichung
\[
\frac{\partial F}{\partial y}(x,y(x),y'(x))
-
\frac{d}{dx}
\frac{\partial F}{\partial y'}(x,y(x),y'(x))
=
0
\]
erfüllt.





