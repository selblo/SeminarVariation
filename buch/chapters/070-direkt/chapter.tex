%
% chapter.tex -- direkte Methoden
%
% (c) 2023 Prof Dr Andreas Müller
%
\chapter{Direkte Methoden
\label{buch:chapter:direkt}}
\kopflinks{Direkte Methoden}
Alle bisher dargestellten Lösungsverfahren haben das globale Extremalproblem
in ein lokales Problem in der Form einer aus der Idee der Richtungsableitung
entwickelten Differentialgleichung umgewandelt.
Das Extremum wurde dann als Lösung der Differentialgleichung gefunden.
Dieser Umweg ist vor allem dann problematisch, wenn die resultierenden
Differentialgleichungen sehr schwer zu lösen sind.

Endlichdimensionale Optimierungsprobleme können aber auch direkt gelöst
werden, indem man zum Beispiel dem Gradienten folgt, kann man schrittweise
zu immer grösseren Werten der Funktion folgen.
Dieses Verfahren wird zum Beispiel erfolgreich beim Trainieren
künstlicher neuronaler Netzwerke verwendet.
Es wird in Abschnitt~\ref{buch:direkt:section:gradientabstieg}
kurz gezeigt.

Das Gradient-Verfahren kann auch auf Variationsprobleme angewendet werden.
Dazu muss das Variationsproblem allerdings erst auf ein endlichdimensionales
Problem reduziert werden.
Das Verfahren von Ritz von Abschnitt~\ref{buch:direkt:section:ritz}
realisiert dies durch Verwendung eines endlichdimensionalen Vektorraumes
von Funktionen anstelle des Raumes aller Funktionen.
Der Nachteil dieses Verfahrens ist allerdings, dass sich Änderungen an
den Koeffizienten typischerweise auf alle Funktionswerte auswirken.
Die Finite-Element-Methode, kurz angedeutet in
Abschnitt~\ref{buch:direkt:section:fem}, verwendet eine alternative Basis
von Funktionen, die besser lokalisiert sind.

%
% gradient.tex -- slide template
%
% (c) 2021 Prof Dr Andreas Müller, OST Ostschweizer Fachhochschule
%
\bgroup
\begin{frame}[t]
\setlength{\abovedisplayskip}{5pt}
\setlength{\belowdisplayskip}{5pt}
\frametitle{Gradient}
\vspace{-20pt}
\begin{columns}[t,onlytextwidth]
\begin{column}{0.44\textwidth}
\begin{block}{$n$ Variablen}
Richtungsableitung
\begin{align*}
D_{\vec{v}}f(x)
&=
\vec{v}\cdot
\operatorname{grad} f(x) 
\\
&=
v_1\frac{\partial f}{\partial x_1}(x)
+
v_n\frac{\partial f}{\partial x_n}(x)
\end{align*}
\end{block}
\begin{block}{``Gradient'' für Variationsprobleme}
\begin{gather*}
\frac{\partial L}{\partial y}(x,y(x),y'(x))\qquad\qquad
\\
-
\frac{d}{dx}
\frac{\partial L}{\partial y'}(x,y(x),y'(x))
=
0
\end{gather*}
\end{block}
\end{column}
\begin{column}{0.52\textwidth}
\begin{block}{Variationsproblem}
\begin{align*}
\delta I(y)
&=
\int_{x_1}^{x_2}
\biggl(
\frac{\partial L}{\partial y}(x,y(x),y'(x))
\\
&\quad
-
\frac{d}{dx}
\frac{\partial L}{\partial y'}(x,y(x),y'(x))
\biggr)\eta(x)\,dx
\end{align*}
\end{block}
\begin{block}{$L^2$-Skalarprodukt}
Skalarprodukt von Funktionen
\begin{align*}
\langle f,g\rangle
&=
\int_{x_1}^{x_2}
f(x)\,g(x)\,dx
\\
\delta I(y)
&=
\biggl\langle
\frac{\partial L}{\partial y}(x,y(x),y'(x))
\\
&\qquad
-
\frac{d}{dx}
\frac{\partial L}{\partial y'}(x,y(x),y'(x)),
\eta(x)
\biggr\rangle
\end{align*}
\end{block}
\end{column}
\end{columns}
\end{frame}
\egroup

%
% ritz.tex -- Das Verfahren von Ritz
%
% (c) 2024 Prof Dr Andreas Müller
%
\section{Das Verfahren von Ritz
\label{buch:direkt:section:ritz}}

\input{chapters/070-direkt/fem.tex}
