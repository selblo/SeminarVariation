%
% chapter.tex -- direkte Methoden
%
% (c) 2023 Prof Dr Andreas Müller
%
\chapter{Direkte Methoden
\label{buch:chapter:direkt}}
\kopflinks{Direkte Methoden}
Alle bisher dargestellten Lösungsverfahren haben das globale Extremalproblem
in ein lokales Problem in der Form einer aus der Idee der Richtungsableitung
entwickelten Differentialgleichung umgewandelt.
Das Extremum wurde dann als Lösung der Differentialgleichung gefunden.
Dieser Umweg ist vor allem dann problematisch, wenn die resultierenden
Differentialgleichungen sehr schwer zu lösen sind.

Endlichdimensionale Optimierungsprobleme können aber auch direkt gelöst
werden, indem man zum Beispiel dem Gradienten folgt, kann man schrittweise
zu immer grösseren Werten der Funktion folgen.
Dieses Verfahren wird zum Beispiel erfolgreich beim Trainieren
künstlicher neuronaler Netzwerke verwendet.
Es wird in Abschnitt~\ref{buch:direkt:section:gradientabstieg}
kurz gezeigt.

In Kapitel~\ref{buch:chapter:variation} wurde erkannt, dass die
Euler-Lagrange-Differentialgleichung in vielfacher Beziehung als 
der Gradient in einem Funktionenraum betrachtet werden kann.
Die in Abschnitt~\ref{buch:direkt:section:gradient} dargestellte
Methode des Gradient-Abstiegs ist damit potentiell auch auf Funktionen
anwendbar.
Die Schwierigkeit besteht aber darin, beliebige Funktionen darzustellen
und damit die von der Methode geforderten algebraischen Operationen
durchzuführen.
Denkbare Methoden für diesen Zweck könnten Polynome oder Fourier-Reihen
sein.
Eine Funktion wird damit durch einzelne Koeffizienten dargestellt, aus
praktischen Gründen aber nur durch endlich viele.
Die Methode kann also nur endlichdimensionale Probleme lösen.
Die Schwierigkeit ist daher vor allem, aus dem unendlichdimensionalen
Raum aller Funktionen einen genügend grossen endlichdimensionalen
Funktionenraum auszuscheiden, in dem die Aufgaben mit ausreichender
Genauigkeit gelöst werden können.
Das Verfahren von Ritz von Abschnitt~\ref{buch:direkt:section:ritz}
realisiert dies mit ad hoc gewählten Funktionen.
Der Nachteil dieses Verfahrens ist allerdings, dass sich Änderungen an
den Koeffizienten typischerweise auf alle Funktionswerte auswirken.
Die Finite-Element-Methode, kurz angedeutet in
Abschnitt~\ref{buch:direkt:section:fem}, verwendet eine alternative Basis
von Funktionen, die besser lokalisiert sind.

%
% gradient.tex
%
% (c) 2024 Prof Dr Andreas Müller
%
\section{Direkte Lösung endlichdimensionaler Optimierungsprobleme
\label{buch:direkt:section:gradient}}
Die direkten Methoden der Variationsrechnung beruhen darauf, dass
sie sich auf endlichdimensionale Extremalprobleme reduzieren oder
sich durch solche approximieren lassen, die sich viel einfacher
mit dem Computer lösen lassen.
Auch das Training von künstlichen neuronalen Netzwerken verlangt nach
der Lösung eines nichtlinearen, endlichdimensionalen Optimierungsproblems,
wenngleich die Dimension sehr hoch sein kann.
In diesem Abschnitt soll daher Methoden zur numerischen Lösung von
endlichdimensionalen Extremalproblemen gezeigt werden.

%
% Aufgabenstellung
%
\subsection{Aufgabenstellung
\label{buch:direkt:gradient:subsection:aufgabenstellung}}

\begin{definition}
Ein {\em relatives Maximum} einer Funktion $f\colon\Omega\to\mathbb{R}$
ist ein Punkt $x\in\Omega$ mit einer $\varepsilon$-Umgebung
$U$ von $x$ derart, $f(x)\ge f(x')$ für $x'\in U$.
Ein {\em absolutes Maximum} von $f$ ist ein Punkt $x\in\Omega$ derart,
dass $f(x)\ge f(x')$ für alle $x'\in\Omega$.
Sinngemäss werden {\em relatives Minimum} und {\em absolutes Minimum}
definiert.
\index{Maximum!relativ}%
\index{Maximum!absolut}%
\index{Minimum!relativ}%
\index{Minimum!absolut}%
\index{absolutes Maximum}%
\index{relatives Maximum}%
\index{absolutes Minimum}%
\index{relatives Minimum}%
\end{definition}

\begin{aufgabe}
\label{buch:direkt:gradient:aufgabe:extremal}
Sei $\Omega\subset\mathbb{R}^n$ ein Gebiet in $\mathbb{R}$ und
$f\colon\Omega\to\mathbb{R}$ eine stetig differenzierbare
Funktion.
\end{aufgabe}

%
% Gradientabstieg
%
\subsection{Gradientabstieg
\label{buch:direkt:gradient:subsection:gradientabstieg}}

%
% Quadratische Minimalproblem
%
\subsection{Quadratische Extremalprobleme
\label{buch:direkt:gradient:subsetion:quadratisch}}


%
% ritz.tex -- Das Verfahren von Ritz
%
% (c) 2024 Prof Dr Andreas Müller
%
\section{Das Verfahren von Ritz
\label{buch:direkt:section:ritz}}
Zu einem gegebenen Funktional
\begin{equation}
I(y)
=
\int_{x_1}^{x_2}
L(x,y(x),y'(x))
\,dx
\label{buch:direkt:ritz:eqn:funktional}
\end{equation}
soll eine Funktion $y(x)$ gefunden werden, die das Funktional
extremal macht.

%
% Idee der Methode
%
\subsection{Idee der Methode}

%
% Reduktion auf eine endlichdimensionales Problem
%
\subsubsection{Reduktion auf ein endlichdimensionales Problem}
In Abschnitt~\ref{buch:direkt:section:gradient} wurde gezeigt, wie 
Extermalprobleme für endlich viele Variablen zum Beispiel mit der
Gradientabstiegsmethode gelöst werden können.
Diese Methoden sind in dieser Form nicht auf das Variationsproblem
\eqref{buch:direkt:ritz:eqn:funktional} anwendbar, da die Funktion
$y(x)$ nicht nur die Information enthält, die in endlich vielen
Koordinaten $x_1,\dots,x_n$ stecken kann.
Es ist daher nötig, die in Frage kommenden Funktionen durch eine
endlich Zahl von Parametern zu beschreiben.
Man wählt daher Funktionen $\psi_1(x),\dots,\psi_n(x)$ und beschränkt
sich auf Funktion $y(x)$ der Form
\[
y(x)
=
\sum_{k=1}^n
a_k \psi_k(x).
\]
Der Funktionenraum 
\[
B_n
=
\biggl\{
y(x)
=
\sum_{k=1}^n a_k\psi_k(x)
\bigg|
a_1,\dots,a_k\in\mathbb{R}
\biggr\}
\]
ist endlichdimensional und kann durch die Koeffizienten $a_1,\dots,a_n$
parametrisiert werden.
Die Funktionen $\psi_k$ heissen auch die Koordinatenfunktionen und die
$a_k$ die Koordinaten.
Das Funktional $I(y)$ wird jetzt ebenfalls eine Funktion
\[
f(a_1,\dots,a_n)
=
\int_{x_1}^{x_2}
L\biggl(x,
\sum_{k=1}^n a_k\psi_k(x),
\sum_{k=1}^n a_k\psi_k'(x)
\biggr)
\,dx
\]
der Variablen $a_1,\dots,a_n$.
Die Koordinatenfunktionen ermöglichen also, das unendlichdimensionale
Variationsproblem auf ein endlichdimensionales Problem zu reduzieren.

%
% Konvergenz
%
\subsubsection{Konvergenz}
Da der Funktionenraum $B_n$ nur endlichdimensional ist, wird sich darin
im Allgemeinen nur eine Approximation der Lösung des ursprünglichen
Extremalproblems für das Funktional finden lassen.
Je genauer die Koordinatenfunktionen die Lösung zu approximieren gestatten,
desto genauer kann auch der Gradientabstieg die Lösung in $B_n$ zu finden.
Um die Genauigkeit zu steigern, müssen weitere Funktionen zu $B_n$ hinzukommen.
Wir gehen daher im Folgenden von einer Folge von Koordinatenfunktionen
$\psi_k(x)$ mit $k\in \mathbb{N}$ und den Funktionenräumen
\[
B_n
=
\langle \psi_1(x),\dots,\psi_n(x)\rangle
=
\biggl\{
\sum_{k=1}^n a_k^{(n)} \psi_k(x)
\;
\bigg|
\;
a_1^{(n)},\dots,a_n^{(n)}\in\mathbb{R}
\biggr\}
\]
für alle $n\in\mathbb{N}$ aus.
In jedem $B_n$ gibt es eine optimale Lösung
\[
y_n(x)
=
\sum_{k=1}^n a_k^{(n)} \psi_k(x)
\]
Im besten Fall konvergiert nicht nur die Folge $y_n(x)$ gegen die
Funktion $y(x)$ die das Funktional extremal macht, sondern auch
die Folge
\[
a_k^{(k)},
a_k^{(k+1)},
a_k^{(k+2)},
\dots,
a_k^{(k+n)},
\dots
\]
Koeffizienten $a_k^{(k)}$ gegen den Grenzwert $a_k$.
Durch Vergrössern von $n$ kann so eine beliebig genaue Lösung des
ursprünglichen Problems gewonnen werden.
Allerdings ist die Entscheidung, ob die genannten Folgen tatsächlich
konvergieren werden, nicht immer einfach.

%
% Refraktion mit der Methode von Ritz
%
\subsection{Refraktion mit der Methode von Ritz
\label{buch:direkt:ritz:subsection:refraktion}}
In der Übungsaufgabe~\ref{201} wurde das Problem der Refraktion eines
Lichtstrahls in der inhomogenen Atmosphäre untersucht und es wurde als
Euler-Lagrange-Differentialgleichung eine Gleichung gefunden, die nicht
direkt lösbar war.
In diesem Abschnitt versuchen wir das Problem mit dem Verfahren von Ritz
zu lösen.

\begin{aufgabe}
Man finde den Weg eines Lichtstrahls durch ein inhomogenes Medium
in der $x$-$y$-Ebene,
dessen Brechungsindex $n(y)=1+\nu y$ ist, der die Zeit zwischen
den Punkten $(0,0)$ und $(\pi,0)$ minimiert.
\end{aufgabe}

Die Lichtgeschwindigkeit ist $c/n(y)$.
Die Zeit $t$, die der Lichtstrahl entlang des Pfades $y(x)$ braucht, ist
daher
\[
t
=
\frac{1}{c}
\int_0^\pi n(y) \sqrt{1+y'(x)^2}.
\]
Dieses Funktional ist also zu minimieren.
Als approximierende Funktionen auf dem Intervall $[0,\pi]$ verwenden 
wir die Funktionen $\psi_n(x) = \sin nx$, die Funktion $y(x)$ wird also
in eine Fourier-Sinus-Reihe entwickelt.
Die Randbedingung $y(0)=y(\pi)=0$ ist automatisch immer erfüllt.

Für $n=1$ wird das Extremalproblem zu der Aufgabe, den Koeffizienten $a_1$
im Integral
\[
f(a_1)
=
\int_0^\pi (1+\nu a_1\sin x)\sqrt{1+a_1^2\sin^2 x}\,dx
\]
so zu wählen, dass das Minimum erreicht wird.
Leider ist das Integral nicht elementar auswertbar, so dass die Aufgabe
nur numerisch gelöst werden kann.
Dasselbe gilt für die Approximation $n$-ter Ordnung
\[
f_n(a_1,\dots,a_n)
=
\int_0^\pi
\biggl(1+\nu \sum_{k=1}^n a_k\sin kx\biggr)
\sqrt{1+\sum_{k=1}^n a_k^2\sin^2 x}
\,dx.
\]

%
% Die Methode von Galerkin
%
\subsection{Die Methode von Galerkin
\label{buch:direkt:ritz:subsection:galerkin}}




%
% fem.tex -- Das FEM-Verfahren
%
% (c) 2024 Prof Dr Andreas Müller
%
\section{Diskretisation nach Euler
\label{buch:direkt:section:euler}}
\kopfrechts{Diskretisation nach Euler}
Leonhard Euler
\index{Euler, Leonhard}%
\index{Leonhard Euler}%
hat die Euler-Lagrange auf eine Weise hergeleitet, die eher an die 
Vorgehensweise direkter Methoden erinnert.
In diesem Abschnitt soll das Integral
\begin{equation}
I(y)
=
\int_a^b
F(x,y(x),y'(x))
\,dx
\label{buch:direkt:euler:eqn:aufgabe}
\end{equation}
mit den Randbedingungen $y(a)=y_a$ und $y(b)=y_b$ minimiert werden.

%
% Polygon-Approximation
%
\subsection{Polygon-Approximation
\label{buch:direkt:euler:subsection:polygon}}
%
% euler.tex
%
% (c) 2024 Prof Dr Andreas Müller
%
\begin{figure}
\centering
\includegraphics{chapters/070-direkt/images/euler.pdf}
\caption{Diskretisation der Lösungsfunktion für das Funktional
$I(y)$ von \eqref{buch:direkt:euler:eqn:aufgabe}, mit der Leonhard
Euler die Euler-Lagrange-Differentialgleichung hergeleitet hat.
\label{buch:direkt:euler:fig:euler}}
\end{figure}

Die gesuchte Funktion $y\colon[a,b]\mathbb{R}$ wird durch eine stückweise
lineare Funktion approximiert (Abbildung~\ref{buch:direkt:euler:fig:euler}).
Sie ist definiert durch die Werte an den $n-1$ Zwischenpunkten
$x_k=a+kh$ mit $h=(b-a)/n$.
Die Endpunkte des Intervalls sind $x_0=a$ und $x_n=b$.
Zu bestimmen sind die Funktionswerte $y_k = y(x_k)$ für $k=1,\dots,n-1$.
Die Funktionswerte $y_0=y(x_0)=y(a)=y_a$ und $y_n=y(x_n)=y(b)=y_b$ 
sind durch die Randbedingung vorgegeben.

Das Integral $I(y)$ von \eqref{buch:direkt:euler:eqn:aufgabe} muss
jetzt durch die unbekannten Werte $y_1,\dots,y_{n-1}$ approximiert
werden.
Für die Ableitung an der Stelle $x_k$ kann der Differenzenquotient
\[
y'(x_k) 
\approx
y'_k
=
\frac{y_{k+1\mathstrut}-y_{k\mathstrut}}{h}
\]
verwendet werden.
Der Wert des Integrals 
\begin{equation}
I(y)
\approx
I(y_1,\dots,y_{n-1})
=
\sum_{k=1}^{n-1}
F(x_k, y_k, y'_k)\cdot h
\label{buch:direkt:euler:eqn:summe}
\end{equation}
Die Aufgabenstellung wird damit zum folgenden endlichdimensionalen
Problem.

\begin{aufgabe}
Finde die Werte $(y_1,\dots,y_{n-1})\in\mathbb{R}^{n-1}$, für die
die Summe~\eqref{buch:direkt:euler:eqn:summe} extremal wird.
\end{aufgabe}

%
% Notwendig Bedingung
%
\subsection{Euler-Lagrange-Differentialgleichung
\label{buch:direkt:euler:subsection:eldgl}}
Eine notwendige Bedingung dafür, dass die Funktion $I(y_1,\dots,y_{n-1})$
ein Extremum annimmt, ist das Verschwinden der partiellen Ableitungen
\begin{equation}
\frac{\partial I}{\partial y_i} (y_1,\dots,y_{n-1})
=
0
\end{equation}
für $i=1,\dots,n-1$.
In der Summe~\eqref{buch:direkt:euler:eqn:summe} kommt die Variable
$y_i$ zunächst im Term mit $k=i$ vor.
Da aber $y'_k$ aus $y_{k+1}$ und $y_k$ berechnet wird, kommt $y_i$
in der Summe auch im Term mit $k=i-1$ vor.
Die Ableitung der Summe nach $y_k$ ist daher
\begin{align*}
\frac{\partial I}{\partial y_i}(y_1,\dots,y_{n-1})
&=
\frac{\partial}{\partial y_i}
\sum_{k=1}^{n-1}
F(x_k, y_k, y'_k)\cdot h
\\
&=
\frac{\partial}{\partial y_i}
F(x_i, {\color{darkred}y_i}, {\color{darkred}y'_i})\cdot h
+
\frac{\partial}{\partial y_i}
F(x_{i-1}, y_{i-1}, {\color{darkred}y'_{i-1}})\cdot h
\\
&=
\frac{\partial}{\partial y_i}
F\biggl(x_i, {\color{darkred}y_i}, \frac{y_{i+1}-{\color{darkred}y_i}}{h}\biggr)\cdot h
+
\frac{\partial}{\partial y_i}
F\biggl(x_{i-1}, y_{i-1}, \frac{{\color{darkred}y_i}-y_{i-1}}{h}\biggr)\cdot h.
\intertext{Die Ableitung kann mit der Kettenregel berechnet werden
und ergibt}
&=
\frac{\partial F}{\partial y}\biggl(x_i, y_i,\frac{y_{i+1}-y_i}{h}\biggr)
\cdot h
\\
&\qquad\qquad
-
\frac{\partial F}{\partial y'}\biggl(x_i,y_i,\frac{y_{i+1}-y_i}{h}\biggr)
+
\frac{\partial F}{\partial y'}\biggl(x_{i-1},y_{i-1},\frac{y_i-y_{i-1}}h\biggr).
\intertext{Im zweiten und dritten Term sind die Faktoren $h$ verschwunden,
weil sie gegen einen von der inneren Ableitung herstammenden Faktor $1/h$ 
gekürzt wurden.
Die letzten beiden Terme können noch zusammengefasst werden:
}
&=
h\cdot \biggl(
\frac{\partial F}{\partial y}(x_i,y_i,y'_i)
-
\frac1h\biggl(
\frac{\partial F}{\partial y'}(x_i,y_i,y'_i)
-
\frac{\partial F}{\partial y'}(x_{i-1},y_{i-1},y'_{i-1})
\biggr)\biggr).
\intertext{Die innere Klammer ist ein Differenzenquotient, der die
Ableitung nach $x$ approximiert.
Für genügend feine Diskretisation ist daher }
\frac{\partial I}{\partial y_i}(y_1,\dots,y_{n-1})
&\approx
h\cdot\biggl(
\frac{\partial F}{\partial y}(x_i,y_i,y'_i)
-
\frac{d}{dx}\frac{\partial F}{\partial y'}(x_{i-1}, y_{i-1}, y'_{i-1})
\biggr)
=
0.
\end{align*}
Die Ableitungen $\partial I/\partial y_i$ verschwinden also, wenn 
die grosse Klammer den Wert $0$ annimmt.
Eine notwendige Bedingung dafür, dass die Lösungsfunktion $y(x)$, die in den
Stützstellen $x_i$ die Werte $y(x_i) = y_i$ annimmt, das
Integral $I(y)$ von \eqref{buch:direkt:euler:eqn:aufgabe} extremal macht,
ist also, dass $y(x)$ die Euler-Lagrange-Differentialgleichung
\[
\frac{\partial F}{\partial y}(x,y(x),y'(x))
-
\frac{d}{dx}
\frac{\partial F}{\partial y'}(x,y(x),y'(x))
=
0
\]
erfüllt.






