%
% 3-eulerlagrange.tex
%
% (c) 2023 Prof Dr Andreas Müller
%
\section{Die Euler-Lagrange Differentialgleichung
\label{buch:variation:section:eulerlagrange}}
\kopfrechts{Die Euler-Lagrange Differentialgleichung}
Das Neuartige an der Aufgabenstellung des Brachistochronenproblems
war, dass eine Funktion gesucht war, so dass ein damit gebildetes
Integral eine Minimaleigenschaft erfüllt.
Für die damalige Mathematik war die Aufgabe, eine Funktion zu finden,
nicht neu.
Die Theorie der Differentialgleichungen war bereits entwickelt,
Newton hat die Infinitesimalrechnung ja erfunden, um damit die
Bewegungsgleichungen der Physik zu formulieren und zu lösen.
In einer Differentialgleichung werden Werte und Ableitungen einer
Funktion an einer einzigen Stelle miteinander verbunden.
Etwas salop formuliert sagt die Differentialgleichung in jedem
Punkt, in welche Richtung und mit welcher Krümmung die Funktionskurve
weiter zu zeichnen ist.

Im Brachistochronenproblem tragen aber alle Werte der gesuchten
Funktion zum Integral bei, es scheint daher auf den ersten Blick
nicht möglich, das Problem durch schrittweise Konstruktion
``von Punkt zu Punkt'' der Lösungskurve zu konstruieren.

Bernoullis Lösung des Brachistochrononproblems beruht auf der
Beobachtung, dass sich die Bedinung für die schnellste Bahn
durch eine Bedingung ersetzen lässt, die in jedem einzelnen
Punkt ausgewertet werden kann.
Das von ihm verwendete Fermat-Prinzip wurde ursprünglich ebenfalls
als eine globale Eigenschaft eines Lichtstrahls formuliert.
Aus dem Fermat-Prinzip folgt aber das Brechungsgesetzt, welches
sagt, dass die Richtung eines Strahls in einem Punkt genau dann
ändert, wenn sich dort auch der Brechungsindex der beiden Medien
ändert.
Das Fermat-Prinzip ist also ein Beispiel dafür, wie eine globale
Bedingung erfüllt werden kann, indem einer lokalen Regel in jedem
Punkt gefolgt wird.

Es ist das Verdienst von Euler und Lagrange, zu erkennen, dass diese
Übersetzung eines globalen Variationsproblems in ein lokales 
Problem immer möglich ist.
Es entsteht dabei die Euler-Lagrange-Differentialgleichung, welche
die Problemstellung auf die Lösung einer Differentialgleichung
reduziert.
Damit ist ein allgemein anwendbares Lösungsverfahren gefunden.
Zu einem Variationsproblem lässt sich immer eine Differentialgleichung
finden, welche die gesuchte Funktion als Lösung hat.

In diesem Abschnitt soll dieser indirekte Weg der Lösung von
Variationsaufgaben dargestellt werden.
Wir werden später zeigen, dass diese Vorgehensweise nicht immer
erfolgreich sein kann.
Zum Beispiel werden wir in Kapitel~\ref{buch:chapter:nichtdiff}
Variationsprobleme kennenlernen, deren Lösungskurven nicht
differenzierbar sind und daher auch nicht von einer Differentialgleichung
gefunden werden können.
Im Kapitel~\ref{buch:chapter:direkt} werden daher die sogenannten
direkten Methodn vorgestellt, die den Umweg über eine
Differentialgleichung vermeiden.

%
% Die Lagrange-Funktion
%
\subsection{Die Lagrange-Funktion}

%
% Euler-Lagrange_Differentialgleichung
%
\subsection{Euler-Lagrange-Differentialgleichung}

%
% Freie Randbedingungen
%
\subsection{Freie Randbedingungen}


