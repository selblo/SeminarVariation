%
% chapter.tex
%
% (c) 2023 Prof Dr Andreas Müller
%
\chapter{Erste Variation
\label{buch:chapter:variation}}
\kopflinks{Erste Variation}
In Kapitel~\ref{buch:chapter:fuvar} wurde gezeigt, wie Extremalprobleme
für Funktionen mehrerer Variablen mit Hilfe der Richtungsableitung
gelöst werden können.
Die Unbekannte war ein Vektor in einem endlichdimensionalen $\mathbb{R}^n$.
Die Variationsrechnung verallgemeinert die Fragestellung auf das Problem,
eine Funktion zu finden, die eine von der Funktion abhängige Grösse
minimiert.
Die Menge der Funktion bilden wie die Vektoren in $\mathbb{R}^n$
einen Vektorraum, der aber keine endliche Basis mehr hat.
Die Idee der partiellen Ableitung nach ``Basisrichtungen'' ist
damit nicht mehr anwendbar, es muss ein alternativer Ansatz gefunden
werden.
Es wird sich zeigen, dass die Idee der Richtungsableitung anwendbar
bleibt.
Wie im endlichdimensionalen Fall entsteht eine Bedingung, die sich
mit dem Skalarprodukt formulieren lässt.
Das Fundamentallemma in
Abschnitt~\ref{buch:variation:section:fundamentallemma}
schliesst dann aus dem Verschwinden aller Richtungsableitungen 
auf das Verschwinden einer Funktion, was auf die
Euler-Lagrange-Differentialgleichung und damit auf eine allgemeine
Methode zur Lösung von Variationsproblemen führt.

%
% 1-problem.tex
%
% (c) 2023 Prof Dr Andreas Müller
%
\section{Problemstellung
\label{buch:variation:section:problemstellung}}
\kopfrechts{Problemstellung}
Das Brachistochronenproblem, welches Johann Bernoulli im Jahr 1696
gestellt hat, war nicht das erste Optimierungsproblem für eine 
gesuchte Funktion, welches Physiker betrachtet haben.
Aber es darf als Ausgangspunkt für die Entwicklung der Variationsrechnung
betrachtet werden.
Alle diese Probleme zeichnen sich dadurch aus, dass Extremwerte
einer Grösse, die von allen Werten einer unbekannten Funktion abhängt,
gefunden werden sollen.
Die von Euler und Lagrange entwickelte Theorie zeigt, dass eine
Lösung für diese globale Eigenschaft immer auf eine lokale Bedingungen,
genauer auf eine Differentialgleichung reduziert werden kann, für deren
Lösung bereits eine ausgefeilte Theorie existiert.

%
% Der Anfang: das Brachistochronenproblem von Bernoulli
%
\subsection{Der Anfang: Das Brachistochronenproblem von Bernoulli}
%
% brachistochronenproblem.tex -- Brachistochronenproblem
%
% (c) 2021 Prof Dr Andreas Müller, OST Ostschweizer Fachhochschule
%
\documentclass[tikz]{standalone}
\usepackage{amsmath}
\usepackage{times}
\usepackage{txfonts}
\usepackage{pgfplots}
\usepackage{csvsimple}
\usetikzlibrary{arrows,intersections,math}
\begin{document}
\def\skala{1}
\def\r{1.5}
\pgfmathparse{3.14159/180}
\xdef\m{\pgfmathresult}
\def\xwert#1{\r*((#1)*\m-sin(#1))}
\def\ywert#1{\r*(cos(#1)-1)}
\def\punkt#1{ ({\r*((#1)*\m-sin(#1))},{\r*(cos(#1)-1)}) }
\begin{tikzpicture}[>=latex,thick,scale=\skala]

\draw[color=gray!50] plot[domain=0:360,samples=360]
	({\r*((\x)*\m-sin(\x))},{\r*(cos(\x)-1)});

\draw[->] (-0.1,0) -- (10,0) coordinate[label={$x$}];
\draw[->] (0,0.1) -- (0,-3.5) coordinate[label={left:$y$}];

\draw ({\m*\r*180},0.05) -- ({\m*\r*180},-0.05);
\node at ({\m*\r*180},0) [above] {$\frac{\pi}2\mathstrut$};
\draw ({\m*\r*360},0.05) -- ({\m*\r*360},-0.05);
\node at ({\m*\r*360},0) [above] {$\pi\mathstrut$};

\draw[line width=0.2pt]  ({\xwert{60}},0) -- \punkt{60};
\node at ({\xwert{60}},0) [above] {$a\mathstrut$};
\draw ({\xwert{60}},0.05) -- ({\xwert{60}},-0.05);

\draw[line width=0.2pt]  ({\xwert{160}},0) -- \punkt{160};
\node at ({\xwert{160}},0) [above] {$b\mathstrut$};
\draw ({\xwert{160}},0.05) -- ({\xwert{160}},-0.05);

\draw[color=red,line width=1.2pt] plot[domain=60:160,samples=100]
	({\r*((\x)*\m-sin(\x))},{\r*(cos(\x)-1)});

\fill[color=red] \punkt{60} circle[radius=0.08];
\node at \punkt{60} [above right] {$A$};
\fill[color=red] \punkt{160} circle[radius=0.08];
\node at \punkt{160} [below] {$B$};
\fill[color=red] \punkt{100} circle[radius=0.08];
\node at \punkt{100} [below left] {$M$};

\end{tikzpicture}
\end{document}


Im Jahr 1696 publiziert der Basler Mathematiker Johann Bernoulli, damals
Professor für Mathematik in Groningen das folgende Problem in der
von Leibniz herausgegebenen Zeitschrift {\em Acta eruditorum}:
\begin{center}
\includegraphics[width=0.8\textwidth]{chapters/020-variation/images/latein.jpg}
\end{center}
Zu deutsch:
\begin{quote}
Neue Aufgabe, zu deren Lösung die Mathematiker eingeladen werden.
Gegeben zwei Punkte $A$ und $B$ in einer vertikalen Ebene, finde
die Bahn $AMB$ eines Punktes $M$, der unter der Wirkung seines
Gewichtes in kürzester Zeit vom Punkt $A$ zum anderen Punkt $B$ absteigt.
\end{quote}
Die Situation der Aufgabenstellung ist in
Abbildung~\ref{buch:variation:fig:brachistochronenproblem}
dargestellt.
Bernoulli hat als Lösung gefunden, dass die Kurve eine Ausschnitt
aus einer Zykloide (in der Abbildung grau) sein muss.
Seine Lösung beruhte auf der Beobachtung, dass sich das Problem analog
zu einem Lichtausbreitungsproblem ist, für welches Fermat bereits
eine Lösung gefunden hat.

Da die Reibung vernachlässigt wird, ist die Energie des Massepunktes
erhalten.
Sie setzt sich aus der potenziellen und der kinetischen Energie
zusammen.
Die potenzielle Energie ist $-mgy$, die kinetische Energie ist
$\frac12mv^2$.
Die Energieerhaltung wird daher zu
\[
E=\frac12mv^2-mgy
\qquad\Rightarrow\qquad
v
=
\sqrt{2g}\!\sqrt{\frac{E}{gm}+y}
=
\!\sqrt{2(C+y)}.
\]
Durch Wahl einer anderen Zeiteinheit kann die Gleichung noch weiter
vereinfacht zu
\(
v = \sqrt{C+y}
\)
vereinfacht werden.
Gesucht ist also die zeitlich kürzeste Bahn eines Teilchens, 
dessen Geschwindigkeit auf bekannte Art $v(y)$ von der vertikalen
Koordinate abhängt.

%
% Das Fermat-Problem
%
\subsection{Das Fermat-Prinzip}
Bereits Fermat hat erkannt, dass das Brechnungsgesetz von Snellius
als Lösung eines Extremalproblems verstanden werden kann.

\begin{satz}[Fermat]
Sie $c/n_i$ die Geschwindigkeit, mit der sich Licht im Medium $M_i$
ausbreitet.
Ein Lichtstrahl von $A_1$ nach $A_2$ geht durch denjenigen Punkt $B$ 
auf der Grenzfläche zwischen den Medien, für den sich die Sinus der
Winkel $\alpha_i$ zwischen den Strahlen und der Normalen zur Grenzfläche
umgekehrt wie die $n_i$ verhalten, wenn also das Brechungsgesetz
\[
\frac{\sin\alpha_1}{\sin\alpha_2}
=
\frac{n_2}{n_1}
\]
gilt.
\end{satz}

\begin{proof}
Ohne der Beschränkung der Allgmeinheit können wir auf die Betrachtung
einer Ebene beschränken, die die beiden Punkte $A_i$ enthält und senkrecht
auf der Grenzfläche steht.
Wir dürfen weiter annehmen, dass die $x$-Achse in der Grenzfläche liegt 
und die Punkte $A_i$ die Koordinaten $(x_i,y_i)$ und der Punkt $B$ die
Koordinaten $(x,0)$ hat.
Es ist derjenige Punkt $x$ zu bestimmen, für den die Lichtzeit entlang 
des Pfades $A_1BA_2$ minimal wird.
Diese Zeit ist
\begin{align*}
t
&=
\frac{\overline{A_1B}}{c/n_1}
+
\frac{\overline{BA_2}}{c/n_2}
\\
ct
&=
n_1\overline{A_1B}
+
n_2\overline{A_2B}
\\
&=
n_1\!\sqrt{(x-x_1)^2 + y_1^2}
+
n_2\!\sqrt{(x_2-x)^2 + y_2^2}
\end{align*}
Das Minimum wird bei einer Nullstelle der Ableitung nach $x$ gefunden,
also bei einer Lösung der Gleichung
\begin{align*}
0
&=
n_1\frac{2(x_1-x)x}{\sqrt{(x_1-x)^2+y_1^2}}
+
n_2\frac{-2(x-x_2)x}{\sqrt{(x_2-x)^2+y_2^2}}.
\intertext{Indem man den zweiten Term auf der rechten Seite auf die linke
Seite bringt und durch $x$ dividiert, erhält man}
n_1
\frac{x_1-x}{\sqrt{(x_1-x)^2+y_1^2}}
&=
n_2
\frac{x-x_2}{\sqrt{(x_2-x)^2+y_2^2}}.
\end{align*}
Der Nenner ist auf beiden Seiten die Hypothenuse eines rechtwinkligen
Dreiecks, welches als Ankathete die Normale zur Grenzfläche hat.
Der Zähler ist die Gegenkathete des Winkels $\alpha_i$ zwischen der
Hypothenuse und der Normalen.
Daher ist der Quotient der Sinus des Winkels oder
\begin{equation}
n_1 \sin\alpha_1 = n_2 \sin\alpha_2.
\label{buch:variation:problem:eqn:snelliusinvariante}
\end{equation}
Die Gleichung~\eqref{buch:variation:problem:eqn:snelliusinvariante}
ist gleichbedeutend mit dem Brechungsgesetz
\[
\frac{\sin\alpha_1}{\sin\alpha_2}
=
\frac{n_2}{n_1}
\]
von Snellius.
\end{proof}

Der Satz von Fermat etabliert das Brechungsgsetz also Lösung eines
Extremalproblems.
Die Natur wählt für einen Lichtstrahl den zeitlich kürzesten Weg.
Der Beweis des Satzes von Fermat zeigt, dass entlang des Lichtstrahls
an jeder Grenzfläche zwischen Medien die Bedingung
\eqref{buch:variation:problem:eqn:snelliusinvariante}
erfüllt.
Wenn die optische Dichte $n$ eine Funktion von $n(y)$ ist, dann
wird der Lichtstrahl nicht nur in diskreten Punkten geknickt, sondern
entlang des ganzen Strahles gekrümmt.
Folgt der Strahl der Kurve $x(y)$, die mit der vertikalen den Winkel
$x'(y) = \tan\alpha(y)$ einschliesst.
Damit lässt sich auch die Sinus-Funktion ausdrücken, es gilt
\[
\sin\alpha(y)
=
\frac{x'(y)}{\pm\!\sqrt{x'(y)^2+1}}.
\]
Aus der Form~\eqref{buch:variation:problem:eqn:snelliusinvariante}
des Brechungsgesetztes wird dann die Gleichung
\begin{equation}
n_1\sin\alpha(y)
=
\frac{n_1(y)x'(y)}{\pm\!\sqrt{x'(y)^2+1}}
=
\operatorname{const}
\qquad\Rightarrow\qquad
\frac{n_1(y)^2x'(y)^2}{x'(y)^2+1}=C.
\label{buch:variation:eqn:fermatdgl}
\end{equation}
Dies ist eine Differentialgleichung für die Funktion $x(y)$.
Sie kann auch in die Form
\[
x'(y)^2
=
\frac{C}{(n_1(y)^2-C)}
\]
gebracht werden.

%
% Das Brachistochronenproblem als Lichtausbreitungsproblem
%
\subsubsection{Das Brachistochronenproblem als Lichtausbreitungsproblem}
Das Fermat-Prinzip besagt, dass ein Lichtstrahl, der sich in einem Medium
mit der Geschwindigkeit $c/n(y)$ ausbreitet, die Gleichung 
\eqref{buch:variation:eqn:fermatdgl} erfüllt.
Beim Brachistochronenproblem ist die Geschwindigkeit $v(y)=\!\sqrt{C-y}$ und 
damit $n(y) = c/\!\sqrt{C-y}$.
Eine Brachistochrone ist also eine Kurve, die die aus
\eqref{buch:variation:eqn:fermatdgl} folgende Gleichung
\begin{equation}
\frac{x'(y)^2}{(1+x'(y)^2)(C-y)} = K
\label{buch:variation:problem:eqn:bernoullidgl}
\end{equation}
erfüllen.

%
% Die Bernoullische Lösung
%
\subsubsection{Die Bernoullische Lösung}
Bernoulli hat gefunden, dass die Brachistochrone ein Zykloidenbogen ist.
Dies lässt sich dadurch verifizieren, dass man die Parametrisierung
einer Zykloide in die
Gleichung~\eqref{buch:variation:problem:eqn:bernoullidgl}
einsetzt.
Die Zykloide hat die Parametrisierung
\[
\left.
\begin{aligned}
x &= r(\varphi - \sin\varphi) 
\\
y &= r(1-\cos\varphi)
\end{aligned}
\right\}
\quad
\text{mit der Ableitung}
\quad
\left\{
\begin{aligned}
\dot{x}(\varphi) &= r(1-\cos\varphi)\\
\dot{y}(\varphi) &= r\sin\varphi
\end{aligned}
\right.
\]
für $\varphi\in\mathbb{R}$.
Die Ableitung ist
\[
x'(y)
=
\frac{\dot{x}(\varphi)}{\dot{y}(\varphi)}
=
\frac{1-\cos\varphi}{\sin\varphi}.
\]
Eingesetzt in \eqref{buch:variation:problem:eqn:bernoullidgl}
wird daraus
\[
\frac{\dot{x}(\varphi)^2}{
(\dot{y}(\varphi)^2 +\dot{x}(\varphi)^2)
(C-r(1-\cos\varphi))
}
=
\frac{(1-\cos\varphi)^2}{
((1-\cos\varphi)^2+\sin^2\varphi)
(C-r+r\cos\varphi)
}
=
K.
\]
Ausmultiplizieren im Nenner ergibt
\[
\frac{(1-\cos\varphi)^2}{
(1-2\cos\varphi+\cos^2\varphi+\sin^2\varphi)
(C-r+r\cos\varphi)
}
=
\frac{1-\cos\varphi}{
2(C-r+r\cos\varphi)
}
\]

%
% Das Brachistochronenproblem als Variationsproblem
%
\subsection{Das Brachistochronenproblem als Variationsproblem}
Die Bernoullische Lösung des Brachistochronenproblems verwendet die
Analogie zum Fermat-Prinzip.
Eine solche Analogie ist nur selten möglich, daher soll das Problem
jetzt in eine Form gebracht werden, in die auch viele ähnliche
Optimierungsproblem gebracht werden können.

Wir erinnern daran, dass die Geschwindigkeit des Massepunktes durch
$v(y)=\sqrt{C-y}$ gegeben ist.
Damit lässt sich die Zeit berechnen, die der Massepunkt entlang der
Lösungskurve braucht, wenn man diese als Funktion $y(x)$ mit beschreibt.
Die Punkte $A$ und $B$ sollen die $x$-Koordinaten $a$ bzw.~$b$ haben.
Für das Kurvenstück zwischen den $x$-Koordinaten $x$ und $x+\Delta x$
braucht der Massepunkt die Zeit
\[
\frac{ \sqrt{\Delta x^2 + \Delta y^2} }{v(y)}
=
\frac{ \sqrt{1 + y'(x)^2} }{ v(y) } \Delta x.
\]
Die Zeit ist das Integral
\begin{equation}
t
=
\int_a^b \frac{\sqrt{1+y'(x)^2}}{v(y(x))}\,dx
=
\int_a^b \sqrt{\frac{1+y'(x)^2}{C-y(x)}}\,dx.
\label{buch:variation:problem:eqn:brachint}
\end{equation}
Der Integrand auf der rechten Seite hängt nur von den Funktion $y(x)$
und $y'(x)$ ab.
Dies kommt vor allem daher, dass die Geschwindigkeit nur von $y$ abhängt,
nicht auch noch von $x$.
Im Allgemeinen wird man also davon ausgehen müssen, dass der Integrand
auch noch von $x$ abhängt.
Die Variationsrechnung befasst sich mit Problemen, in denen Funktionen
gefunden werden müssen, die ein Integral wie das in
\eqref{buch:variation:problem:eqn:brachint}
minimiert oder maximiert werden müssen.

\begin{definition}[Lagrange-Funktion des Brachistochronenproblems]
Die Lagrange-Funk\-tion des Brachistochronenproblems ist der
Integrand des Integrals
\eqref{buch:variation:problem:eqn:brachint},
\index{Lagrange-Funktion}%
also die Funktion
\[
L(x,y,y')
=
\sqrt{\frac{1+y^{\prime 2}}{C-y}}.
\]
\end{definition}

%
% Funktionale
%
\subsection{Funktionale}



%
% 2-fundamtenallemma.tex
%
% (c) 2023 Prof Dr Andreas Müller
%
\section{Das Fundamentallemma
\label{buch:variation:section:fundamentallemma}}
\kopfrechts{Das Fundamentallemma}
Im Fall des endlichdimensionalen Extremalproblems ist aus der
Forderung, dass alle Richtungsableitung verschwinden müssen, 
die Bedingung geworden, dass
\[
v\cdot\grad f = 0
\]
sein muss für alle Vektoren $v\in\mathbb{R}^n$.
Wir haben daraus geschlossen, dass der Gradient $\grad f=0$
sein muss.
Wir hatten dies das endlichdimensionale Fundamentallemma genannt,
wegen $e_k\cdot \grad f = D_kf$ war es eine ziemliche Selbstverständlichkeit.
Bei der Lösung von Variationsproblemen, wo es nicht um endlichdimensionale
Vektoren und das Skalarprodukt, sondern um Funktionen und Integrale
geht, brauchen wir eine ähnliche Aussage für Funktionen.

%
% Positive glatte Funktionen mit kompaktem Träger
%
\subsection{Positive glatte Funktionen mit kompaktem Träger}
Die Aussage des Fundamentallemmas für endlichdimensionale Vektoren 
folgte sofort aus der Tatsache, dass es für jedes $k$ einen Vektor
$e_k$ gibt, der nur in der Koordinaten $k$ von $0$ verschieden ist.
Natürlich gibt es auch Funktionen, die nur in genau einem Punkt
von $0$ verschieden sind.
Eine solche Funktion ist aber im allgemeinen nicht differenzier-
oder integrierbar.
In diesem Abschnitt soll daher gezeigt werden, dass es unendlich
oft stetig differnzierbare Funktionen gibt, die nur in einem beliebig
kleinen vorgegebenen Intervall $\ge 0$ sind.

\begin{definition}[Träger]
Der {\em Träger} einer Funktion $f\colon X\to\mathbb{R}$ ist die Menge
\index{Träger}%
\[
\supp f = \{ x\in X\mid f(x)\ne \}.
\]
\end{definition}

Gesucht ist also eine beliebig oft stetig differenzierbare Funktion,
deren Träger in einem vorgegebenen Intervall $[a,b]$ enthalten ist.
Wir konstruieren so eine Funktion in zwei Schritten.

%
% f.tex -- Abbildung der Funktion f
%
% (c) 2023 Prof Dr Andreas Müller
%
\begin{figure}
\centering
\includegraphics{chapters/020-variation/images/f.pdf}
\caption{Die beliebig oft stetig differenzierbare Funktion von
Satz~\ref{buch:variation:fundamentallemma:satz:glatt}
\label{buch:variation:fundamentallemma:fig:glatt}}
\end{figure}


\begin{satz}
\label{buch:variation:fundamentallemma:satz:glatt}
Die Funktion
\[
f(x)
=
\begin{cases}
e^{-1/x}&\qquad x>0\\
0&\qquad x\le 0
\end{cases}
\]
(siehe auch Abbildung~\ref{buch:variation:fundamentallemma:fig:glatt})
ist beliebig oft stetig differenzierbar.
\end{satz}

\begin{proof}
Es ist klar, dass die Funktion $f$ beliebig oft stetig differenzierbar
ist in jedem Punkt $x\ne 0$.
Es ist also nur nachzuweisen, dass $f(x)$ im Punkt $0$ beliebig
oft stetig differenzierbar ist.

Die ersten drei Ableitungen von $f(x)$ sind
\begin{align}
f'(x) &= \frac{1}{x^2} f(x)
\label{buch:variation:fundamentallemma:eqn:f1}
\\
f''(x) &= \frac{1-2x}{x^4}f(x)
\notag
\\
f'''(x) &= \frac{6x^2-6x+1}{x^6}f(x).
\notag
\end{align}
Daraus lässt sich die Vermutung ableiten, dass
\begin{equation}
f^{(n)}(x)
=
\frac{p_{n-1}(x)}{x^{2n}} f(x)
\label{buch:variation:fundamentallemma:eqn:fabl}
\end{equation}
ist, wobei $p_k(x)$ ein Polynom vom Grad $k$ ist.
Wir beweisen diese Vermutung mit Hilfe von vollständiger Induktion.
Die Induktionsverankerung für die $0$-te Ableitung ist trivial.

Wir nehmen jetzt im Sinne der Induktionsannahme an, dass die $n$-te
Ableitung die Form \eqref{buch:variation:fundamentallemma:eqn:fabl}
hat.
Wir müssen zeigen, dass dann auch $f^{(n+1)}(x)$ diese Form hat.
Dazu berechnen wir
\begin{align}
f^{(n+1)}(x)
&=
\frac{d}{dx}
\frac{p_n(x)}{x^{2n}} f(x)
\notag
\\
&=
\frac{p_n'(x)}{x^{2n}} f(x)
-2n
\frac{p_n(x)}{x^{2n+1}} f(x)
+
\frac{p_n(x)}{x^{2n}} f'(x).
\notag
\intertext{Mit der ersten Ableitung
\eqref{buch:variation:fundamentallemma:eqn:f1} wird dies zu}
&=
\frac{p_n'(x)}{x^{2n}} f(x)
-2n
\frac{p_n(x)}{x^{2n+1}} f(x)
+
\frac{p_n(x)}{x^{2n}} \frac{1}{x^2}f(x)
\notag
\\
&=
\frac{x^2p_n'(x) -2nxp_n(x)+p_n(x)}{x^{2n+2}} f(x).
\label{buch:variation:fundamentallemma:eqn:induktionsschritt}
\end{align}
Die Ableitung $p_n'(x)$ ist ein Polynom vom Grad $n-1$ und damit
ist $x^2p_n'(x)$ ein Polynom vom Grad $n+1$.
Ebenso ist $xp_n(x)$ ein Polynom vom Grad $n+1$ während
$p_n(x)$ ein Polynom vom Grad $n$ ist.
Der Zähler von
\eqref{buch:variation:fundamentallemma:eqn:induktionsschritt}
ist
\[
p_{n+1}(x)
=
x^2p_n'(x)+(1 -2nx)p_n(x),
\]
ein Polynom vom Grad $n+1$.
Damit ist der Induktionsschritt erfolgreich und die Behauptung betreffend
die Form von $f^{(n)}(x)$ ist bewiesen.

Es ist jetzt nur noch zu zeigen, dass der Grenzwert von $f^{(n)}(x)$
für $x\to 0+$ verschwindet.
Da das Polynom $p_n(x)$ stetig ist, folgt
\[
\lim_{x\to 0}
f^{(n)}(x)
=
\lim_{x\to 0}\frac{p_n(x)}{x^{2n}}f(x)
=
p_n(0) \lim_{t\to\infty} t^{2n} e^{-t}
=
0.
\]
Damit ist die beliebige stetige Differenzierbarkeit an der Stelle
$x=0$ gezeigt.
\end{proof}

Die Funktion $f(x)$ von 
Satz~\ref{buch:variation:fundamentallemma:satz:glatt} 
erfüllt noch nicht die Forderung, dass sie nur in einem vorgegebenen
Intervall von $0$ verschieden ist.

%
% g.tex -- template for standalon tikz images
%
% (c) 2021 Prof Dr Andreas Müller, OST Ostschweizer Fachhochschule
%
\documentclass[tikz]{standalone}
\usepackage{amsmath}
\usepackage{times}
\usepackage{txfonts}
\usepackage{pgfplots}
\usepackage{csvsimple}
\usetikzlibrary{arrows,intersections,math}
\begin{document}
\def\skala{3}
\def\a{-0.5}
\def\b{2.5}
\pgfmathparse{2/(\b-\a)}
\xdef\l{\pgfmathresult}
\begin{tikzpicture}[>=latex,thick,scale=\skala]

\draw[->] (-1,0) -- (3,0) coordinate[label={$x$}];
\draw[->] (0,-0.05) -- (0,1.1) coordinate[label={left:$y$}];

\begin{scope}
\clip (-1,-0.1) rectangle (2.9,1);
\draw[color=red,line width=1.2pt]
	plot[domain=0.25:10,samples=100] ({\a+1/\x},{exp(-\x)})
	-- (\a,0) -- (-1,0);
\draw[color=red,line width=1.2pt]
	plot[domain=0.25:10,samples=100] ({\b-1/\x},{exp(-\x)})
	-- (\b,0) -- (3,0);

\draw[color=blue,line width=1.2pt]
	plot[domain=\l:10,samples=100]
		({\a+1/\x},{exp(-\x)*exp(-1/(\b-\a-1/\x))}) -- (\a,0) -- (-1,0);
\draw[color=blue,line width=1.2pt]
	plot[domain=\l:10,samples=100]
		({\b-1/\x},{exp(-\x)*exp(-1/(-\a+\b-1/\x))}) -- (\b,0) -- (3,0);
\end{scope}

\draw (\a,-0.02) -- (\a,0.02);
\node at (\a,0) [below] {$a\mathstrut$};
\draw (\b,-0.02) -- (\b,0.02);
\node at (\b,0) [below] {$b\mathstrut$};
\node[color=blue] at ({0.5*(\a+\b)},0.12) {$g_{a,b}(x)$};
\node[color=red] at (\a,0.8) {$f(x-a)\mathstrut$};
\node[color=red] at (\b,0.8) {$f(b-x)\mathstrut$};

\end{tikzpicture}
\end{document}



\begin{satz}
\label{buch:variation:fundamentallemma:satz:gab}
Sei $f(x)$ die Funktion von
Satz~\ref{buch:variation:fundamentallemma:satz:glatt}.
Dann ist
\[
g_{a,b}(x)
=
f(x-a) f(b-x)
\]
eine unendlich oft stetig differenzierbare, nichtnegative Funktion mit Träger
$\supp g_{a,b}=(a,b)$.
\end{satz}

Die Funktionen $g_{a,b}(x)$ sind beliebig oft differenzierbar und nur im
Intervall $[a,b]$ von $0$ verschieden und sogar positiv.
Weil sie stetig sind, sind sie auch integrierbar, man kann also das
Integral über $\mathbb{R}$ berechnen und die Funktion damit normieren.
Die neue Funktion
\[
\frac{1}{N}
\tilde{g}_{a,b}(x)
\qquad\text{mit}\;
N
=
\int_{-\infty}^{\infty}g_{a,b}(x)\,dx
=
\int_a^b g_{a,b}(x)\,dx
\]
ist immer noch beliebig oft stetig differenzierbar und hat zusätzlich die
Eigenschaft
\[
\int_{-\infty}^{\infty}
\tilde{g}_{a,b}(x)\,dx
=
\int_a^b
\tilde{g}_{a,b}(x)\,dx
=
1.
\]
Wir formulieren dieses Resultat als Satz.

\begin{satz}
\label{buch:variation:satz:gabeins}
Zu jedem Intervall $[a,b]$ gibt es eine beliebig oft stetig
differenzierbare Funktion $g(x)$, genau das Intervall $[a,b]$
als Träger hat und deren Integral über $[a,b]$ den Wert $1$ hat.
\end{satz}

%
% Das Fundamentallemma
%
\subsection{Das Fundamentallemma}
Mit der Funktion $g_{a,b}(x)$ von
Satz~\ref{buch:variation:fundamentallemma:satz:gab}
lässt sich jetzt das Fundamentallemma in der folgenden Form
leicht beweisen.

\begin{satz}[Fundamentallemma]
\label{buch:variation:fundamentallemma:satz:fundamentallemma}
Wenn für die stetige Funktion $f\colon[a,b]\to\mathbb{R}$ 
\begin{equation}
\int_a^b f(x)\varphi(x)\,dx = 0
\label{buch:variation:fundamentallemma:eqn:fundamentalbed}
\end{equation}
gilt für jede beliebig oft stetig differenzierbare Funktion $\varphi(x)$ 
dann ist $f(x)=0$.
Das Resultat gilt selbst dann, wenn
\eqref{buch:variation:fundamentallemma:eqn:fundamentalbed}
nur für beliebig oft stetig differenzierbare Funktionen $\varphi(x)$ 
gilt, die ausserdem an den Intervallenden verschwinden:
$\varphi(a)=\varphi(b)=0$.
\end{satz}

\begin{proof}
Wir zeigen mit Hilfe eines Widerspruchs, dass es keinen Punkt $x_0\in[a,b]$
geben kann, für den $f(x_0)\ne 0$ ist.
Dazu nehmen wir also an, dass $f(x_0)\ne 0$ ist.
Falls $f(x_0)<0$ ist, ersetzen wir $f$ durch $-f$, 
die Bedingung
\eqref{buch:variation:fundamentallemma:eqn:fundamentalbed}
ändert sich dadurch nicht.
%
% fundamentallemma.tex -- Beweis des Fundamentallemmas
%
% (c) 2024 Prof Dr Andreas Müller
%
\begin{figure}
\centering
\includegraphics{chapters/020-variation/images/fundamentallemma.pdf}
\caption{Beweis des Fundamentallemmas: Falls $f(x_0)>0$ ist, gibt es
eine Umgebung $[x_0-\delta,x_0+\delta]$, in der immer noch
$f(x)>\frac12f(x_0)$ gilt und eine Funktion $g(x)$ die genau in diesem
Intervall $\ge 0$ ist und ausserhalb verschwindet.
Es folgt, dass das Ingegral $\int_{x_1}^{x_2} f(x)g(x)\,dx>0$ ist
im Widerspruch zur Annahme des Fundamentallemmas.
\label{buch:variation:fundamentallemma:fig:beweis}}
\end{figure}

Wir dürfen daher annehmen, dass $f(x_0)>0$ ist
(Abbildung~\ref{buch:variation:fundamentallemma:fig:beweis}).
Da $f$ stetig ist, gibt es ein Intervall $[x_0-\varepsilon,x_0+\varepsilon]$
derart, dass $f(x)> \frac12 f(x_0)$ für
$x\in[x_0-\varepsilon,x_0+\varepsilon]$ gilt.
Dann gilt für das Integral
\[
\int_a^b
f(x)
g_{x_0-\varepsilon,x_0+\varepsilon} (x)
\,dx
>
\frac{f(x_0)}{2}
\int_a^b
g_{x_0-\varepsilon,x_0+\varepsilon} (x)
\,dx
>
0
\]
im Widerspruch zur Bedingung
\eqref{buch:variation:fundamentallemma:eqn:fundamentalbed}.
Der Widerspruch zeigt, dass $f(x)=0$ sein muss.
\end{proof}

%
% Skalarproduktformulierung des Fundamentallemmas
%
\subsection{Skalaproduktformulierung des Fundamentallemmas}
Die Richtungsableitung einer Funktion endlich vieler Variablen 
konnte als Skalarprodukt mit dem Gradienten geschrieben werden und
das Fundamentallemma hat besagt, dass der Gradient verschwindet,
wenn alle Richtungsableitungen verschwinden.
Diese Schlussweise ist auch für Funktionen möglich, wenn man Funktionen
ein Skalarprodukt definieren kann.

\begin{definition}[$L^2$-Skalarprodukt]
Das {\em Skalarprodukt} zweier quadratintegrierbarer Funktion $f$ und $g$
auf dem Intervall $[a,b]$ ist definiert durch
\[
\langle f,g\rangle
=
\int_a^b f(x)g(x)\,dx.
\]
\end{definition}

\begin{satz}[Fundamentallemma, Skalarproduktform]
Wenn für eine stetige Funktion $f\colon[a,b]\to\mathbb{R}$ das Skalarprodukt
\[
\langle f,\varphi\rangle = 0
\]
ist für jede unendlich oft differenzierbare Funktion $\varphi$ auf dem
Intervall $[a,b]$, dann ist $f=0$.
\end{satz}



%
% 3-eulerlagrange.tex
%
% (c) 2023 Prof Dr Andreas Müller
%
\section{Die Euler-Lagrange Differentialgleichung
\label{buch:variation:section:eulerlagrange}}
\kopfrechts{Die Euler-Lagrange Differentialgleichung}
Das Neuartige an der Aufgabenstellung des Brachistochronenproblems
war, dass eine Funktion gesucht war, so dass ein damit gebildetes
Integral eine Minimaleigenschaft erfüllt.
Für die damalige Mathematik war die Aufgabe, eine Funktion zu finden,
nicht neu.
Die Theorie der Differentialgleichungen war bereits entwickelt,
Newton hat die Infinitesimalrechnung ja erfunden, um damit die
Bewegungsgleichungen der Physik zu formulieren und zu lösen.
In einer Differentialgleichung werden Werte und Ableitungen einer
Funktion an einer einzigen Stelle miteinander verbunden.
Etwas salop formuliert sagt die Differentialgleichung in jedem
Punkt, in welche Richtung und mit welcher Krümmung die Funktionskurve
weiter zu zeichnen ist.

Im Brachistochronenproblem tragen aber alle Werte der gesuchten
Funktion zum Integral bei, es scheint daher auf den ersten Blick
nicht möglich, das Problem durch schrittweise Konstruktion
``von Punkt zu Punkt'' der Lösungskurve zu konstruieren.

Bernoullis Lösung des Brachistochrononproblems beruht auf der
Beobachtung, dass sich die Bedinung für die schnellste Bahn
durch eine Bedingung ersetzen lässt, die in jedem einzelnen
Punkt ausgewertet werden kann.
Das von ihm verwendete Fermat-Prinzip wurde ursprünglich ebenfalls
als eine globale Eigenschaft eines Lichtstrahls formuliert.
Aus dem Fermat-Prinzip folgt aber das Brechungsgesetzt, welches
sagt, dass die Richtung eines Strahls in einem Punkt genau dann
ändert, wenn sich dort auch der Brechungsindex der beiden Medien
ändert.
Das Fermat-Prinzip ist also ein Beispiel dafür, wie eine globale
Bedingung erfüllt werden kann, indem einer lokalen Regel in jedem
Punkt gefolgt wird.

Es ist das Verdienst von Euler und Lagrange, zu erkennen, dass diese
Übersetzung eines globalen Variationsproblems in ein lokales 
Problem immer möglich ist.
Es entsteht dabei die Euler-Lagrange-Differentialgleichung, welche
die Problemstellung auf die Lösung einer Differentialgleichung
reduziert.
Damit ist ein allgemein anwendbares Lösungsverfahren gefunden.
Zu einem Variationsproblem lässt sich immer eine Differentialgleichung
finden, welche die gesuchte Funktion als Lösung hat.

In diesem Abschnitt soll dieser indirekte Weg der Lösung von
Variationsaufgaben dargestellt werden.
Wir werden später zeigen, dass diese Vorgehensweise nicht immer
erfolgreich sein kann.
Zum Beispiel werden wir in Kapitel~\ref{buch:chapter:nichtdiff}
Variationsprobleme kennenlernen, deren Lösungskurven nicht
differenzierbar sind und daher auch nicht von einer Differentialgleichung
gefunden werden können.
Im Kapitel~\ref{buch:chapter:direkt} werden daher die sogenannten
direkten Methodn vorgestellt, die den Umweg über eine
Differentialgleichung vermeiden.

%
% Die Lagrange-Funktion
%
\subsection{Die Lagrange-Funktion}

%
% Euler-Lagrange_Differentialgleichung
%
\subsection{Euler-Lagrange-Differentialgleichung}

%
% Freie Randbedingungen
%
\subsection{Freie Randbedingungen}



\input{chapters/020-variation/4-hoehereableitungen.tex}
%
% 5-mehrerefunktionen.tex
%
% (c) 2023 Prof Dr Andreas Müller
%
\section{Varationsproblem für mehrere Funktionen
\label{buch:variation:section:mehrerefunktionen}}
\kopfrechts{Mehrere Funktionen}

\begin{verbatim}
- Lagrange-Funktion für mehrere Funktionen
- Die Euler-Lagrange-Differentialgleichungen 
\end{verbatim}

%
% 6-allgemein.tex
%
% (c) 2024 Prof Dr Andreas Müller
%
\section{Die allgemeine Theorie der ersten Variation
\label{buch:variation:section:allgemein}}
\kopfrechts{Die allgemeine Theorie der ersten Variation}
In Abschnitt~\ref{buch:variation:eulerlagrange:subsection:freierb}
wurde gezeigt, wie sich in einem Variationsproblem mit vorgegebenem
Randwert nur an einem Ende des Intervalls aus der ersten Variation
automatisch eine zusätzliche Randbedingung am anderen Ende des
Intervalls ergibt.
Die Endpunkte $x_0$ und $x_1$ des Intervalls waren aber immer noch 
fest.
In diesem Abschnitt soll gezeigt werden, wie sich diese Einschränkung
aufheben lässt.

%
% Allgemeine Variation einer Funktion
%
\subsection{Allgemeine Variation einer Funktion $y(x)$
\label{buch:variation:allgemein:subsection:vary}}
In Abschnitt~\ref{buch:variation:section:eulerlagrange} wurde die
erste Variation mit Hilfe von Funktionen der Form
\begin{equation}
x
\mapsto
y(x)+\varepsilon\eta(x)
\label{buch:variation:allgemein:eqn:ansatz}
\end{equation}
konstruiert.
Mit diesem Ansatz ist es nicht möglich, die Endpunkte $x_0$ und $x_1$
des Intervalls zu varieren.
Es war zwar möglich, die $y$-Werte zu varieren, was auf die zusätzliche
Randbedingung von
Satz~\ref{buch:variation:eulerlagrange:satz:zusaetzlicherb}
geführt hat.
Wir möchten aber sogar die Intervallgrenzen varieren können, dieser
Fall wird von den bisherigen Entwicklungen nicht abgedeckt.

%
% Verallgemeinerung des Variationsansatzes
%
\subsubsection{Verallgemeinerung des Variationsansatzes}
%
% variation.tex -- allgemeine Theorie der ersten Variation
%
% (c) 2021 Prof Dr Andreas Müller, OST Ostschweizer Fachhochschule
%
\documentclass[tikz]{standalone}
\usepackage{amsmath}
\usepackage{times}
\usepackage{txfonts}
\usepackage{pgfplots}
\usepackage{csvsimple}
\usetikzlibrary{arrows,intersections,math}
\definecolor{darkred}{rgb}{0.8,0,0}
\definecolor{hellblau}{rgb}{0.2,0.6,1}
\definecolor{darkgreen}{rgb}{0,0.6,0}
\begin{document}
\def\skala{1}
\begin{tikzpicture}[>=latex,thick,scale=\skala,
declare function = {
	x0(\a) = 1.5-\a;
	y0(\a) = 3*sqrt(2+\a)-1.9;
	x1(\a) = 8+cosh(1.2*\a+0.5)+0.7*\a;
	y1(\a) = 2.0+2.3*sinh(\a+0.5);
	x(\t,\a) = x0(\a)+\t*(x1(\a)-x0(\a));
	t(\x,\a) = (\x-x0(\a))/(x1(\a)-x0(\a));
	y(\x,\a) = 0.3*(\a+1)*t(\x,\a)*(1-t(\x,\a))+0.03*sin(180*t(\x,\a))-0.08*sin(3*180*t(\x,\a));
	X(\x,\a) = \x;
	Y(\x,\a) = y0(\a)*(1-t(\x,\a))+y1(\a)*t(\x,\a)+y(\x,\a)*(x1(\a)-x0(\a));
	f(\x) = 1;
	dx(\t,\a) = (X(x(\t,\a+0.01),\a+0.01)-X(x(\t,\a),\a));
	dy(\t,\a) = (Y(x(\t,\a+0.01),\a+0.01)-Y(x(\t,\a),\a));
	rl(\t,\a) = sqrt(dx(\t,\a)*dx(\t,\a)+dy(\t,\a)*dy(\t,\a));
	rx(\t,\a) = dx(\t,\a)/rl(\t,\a);
	ry(\t,\a) = dy(\t,\a)/rl(\t,\a);
	deltay(\x,\a) = (Y(\x+0.01,\a)-Y(\x,\a))/0.01;
}]

\draw[->] (-0.1,0) -- (12.3,0) coordinate[label={$x$}];
\draw[->] (0,-0.1) -- (0,8.4) coordinate[label={right:$y$}];

\draw[color=red] plot[domain=-1:1] ({x0(\x)},{y0(\x)});
\draw[color=blue] plot[domain=-1:1] ({x1(\x)},{y1(\x)});

\foreach \a in {-1,-0.8,...,1}{
	\draw[smooth,color=gray!40]
		plot[domain={x0(\a)}:{x1(\a)}]
			({X(\x,\a)},{Y(\x,\a)});
}

%\foreach \t in {0,0.1,...,1}{
%	\draw[smooth,color=gray!40] plot[domain=-1:1] ({x(\t,\x)},{y(x(\t,\x),\x)});
%}

\begin{scope}
	\clip[smooth]
	plot[domain={x0(-1)}:{x1(-1)}] (\x,{Y(\x,-1)})
	--
	plot[domain=-1:1] ({x(1,\x)},{Y(x(1,\x),\x)})
	--
	plot[domain={x1(1)}:{x0(1)}] (\x,{Y(\x,1)})
	--
	plot[domain=-1:1] ({x(0,-\x)},{Y(x(0,-\x),-\x)})
	--
	cycle;
	%\fill[color=orange!10] (0,0) rectangle (12,8);
	\foreach \x in {1,...,12}{
		\draw[color=gray!40] (\x,0) -- (\x,8);
	}
\end{scope}

\node[color=gray!80] at (5.51,6.38) {$y(x,\varepsilon)$};

\draw[color=darkred,line width=1.4pt,smooth]
	plot[domain={x0(0)}:{x1(0)}] ({X(\x,0)},{Y(\x,0)});
\node[color=darkred] at (5.5,4.45) {$y(x)$};

\draw[color=blue,line width=1.4pt,smooth]
	plot[domain=-1:1] ({X(x(0,\x),\x)},{Y(x(0,\x),\x)});
\draw[color=blue,line width=1.4pt,smooth]
	plot[domain=-1:1] ({X(x(1,\x),\x)},{Y(x(1,\x),\x)});

\node[color=blue] at (2.0,1.3)
	[rotate=-53] {$(x_0(\varepsilon),y_0(\varepsilon))$};
\node[color=blue] at (9.0,1.7)
	[rotate=80] {$(x_1(\varepsilon),y_1(\varepsilon))$};

% f1
\draw[->,color=darkgreen,line width=2pt]
	({X(x(1,0),0)},{Y(x(1,0),0)}) -- +({2.0*ry(1,0)},{-2.0*rx(1,0)});
%\fill[color=darkgreen] ({X(x(1,0),0)},{Y(x(1,0),0)}) circle[radius=0.08];
\node[color=darkgreen] at (10.7,2.0) {$\vec{f}_1$};

% f0
\draw[->,color=darkgreen,line width=2pt]
	({X(x(0,0),0)},{Y(x(0,0),0)}) -- +({-1.5*ry(0,0)},{1.5*rx(0,0)});
\fill[color=darkgreen] ({X(x(0,0),0)},{Y(x(0,0),0)}) circle[radius=0.08];
\node[color=darkgreen] at (0.8,1.3) {$\vec{f}_0$};

% Dreieck
\def\l{2}
\fill[color=darkred!20] ({x(1,0)},{Y(x(1,0),0)}) -- +(\l,{\l*deltay(x(1,0),0)})
	-- ({x(1,0)+\l},{Y(x(1,0),0)}) -- cycle;
\draw ({x(1,0)},{Y(x(1,0),0)}) -- +(\l,0);
\draw ({x(1,0)},{Y(x(1,0),0)}) -- +(\l,{\l*ry(1,0)/rx(1,0)})
	-- ({x(1,0)+\l},{Y(x(1,0),0)});
\node at ({x(1,0)+1.5},{Y(x(1,0),0)+0.06}) [below] {$\delta x_1$};
\draw[color=darkred,line width=1.2pt] ({x(1,0)},{Y(x(1,0),0)})
	-- +(\l,{\l*deltay(x(1,0),0)})
	-- ({x(1,0)+\l},{Y(x(1,0),0)});
\node[color=darkred] at ({x(1,0)+\l},{Y(x(1,0),0)+0.45})
	[right] {$\displaystyle\frac{\partial y}{\partial x}\cdot\delta x_1$};
\node at ({x(1,0)+\l},{Y(x(1,0),0)+2.2}) [right]
{$\displaystyle\frac{\partial y}{\partial\varepsilon}\cdot\delta\varepsilon$};

% r1(epsilon)
\def\L{4.5}
\draw[->,color=hellblau,line width=2pt]
	({X(x(1,0.0),0.0)},{Y(x(1,0.0),0.0)})
	--
	+({\L*rx(1,0.0)},{\L*ry(1,0.0)});
\fill[color=hellblau] ({X(x(1,0.0),0.0)},{Y(x(1,0.0),0.0)})
	circle[radius=0.08];
\node[color=hellblau] at 
	({X(x(1,0.0),0.0)+\L*rx(1,0.0)},{Y(x(1,0.0),0.0)+\L*ry(1,0.0)})
	[above] {$\vec{r}_1(\varepsilon)$};

% r0(epsilon)
\draw[->,color=hellblau,line width=2pt]
	({X(x(0,0.4),0.4)},{Y(x(0,0.4),0.4)}) -- +({rx(0,0.4)},{ry(0,0.4)});
\fill[color=hellblau] ({X(x(0,0.4),0.4)},{Y(x(0,0.4),0.4)}) circle[radius=0.08];
\node[color=hellblau] at (0.4,3.7) {$\vec{r}_0(\varepsilon)$};

\end{tikzpicture}
\end{document}

%
Statt des Ansatzes~\eqref{buch:variation:allgemein:eqn:ansatz}
verwenden wir jetzt eine Parametrisierung
\begin{equation}
y
\colon
[x_0(\varepsilon),x_1(\varepsilon)]
\to
\mathbb{R}
:
x\mapsto y(x,\varepsilon)
\label{buch:variation:allgemein:eqn:ansatz2}
\end{equation}
für $\varepsilon$-Werte in einer Umgebung von $0$
(Abbildung~\ref{buch:variation:fig:variation}).
Damit
\eqref{buch:variation:allgemein:eqn:ansatz2}
eine Variation der einer Funktion $y(x)$ beschreibt, soll
$y(x) = y(x,0)$ sein.
Die Funktionen $x_0(\varepsilon)$ und $x_1(\varepsilon)$ beschreiben
die Endpunkte des Definitionsintervalls der partiellen Funktion
$x\mapsto y(x,\varepsilon)$.

Der ursprüngliche Ansatz~\eqref{buch:variation:allgemein:eqn:ansatz}
ist in \eqref{buch:variation:allgemein:eqn:ansatz2} enthalten, wenn
man $x_0(\varepsilon)=x_0$, $x_1(\varepsilon)=x_1$ und
\[
y(x,\varepsilon) = y(x) + \varepsilon \eta(x)
\]
setzt.
In der Herleitung der Euler-Lagrange-Differentialgleichung wurde
entscheidend verwendet, dass man diese Funktion nach $\varepsilon$
ableiten kann.
Dass dies möglich ist, ist nach den bisherigen Forderungen an
$y(x,\varepsilon)$ nicht garantiert.
Es muss also zusätzlich Differenzierbarkeit nach $\varepsilon$
gefordert werden.

\begin{definition}[Allgemeine Variation einer Funktion]
\label{buch:variation:allgemein:def:variation}
Eine stetig nach beiden Variablen differenzierbare Funktion $y(x,\varepsilon)$
ist eine {\em allgemeine Variation} der Funktion $y(x)$, wenn es stetig
differenzierbare Funktionen
$x_0(\varepsilon)$ und $x_1(\varepsilon)$ gibt derart, dass
die partielle Funktion $x\mapsto y(x,\varepsilon)$ auf dem
Intervall $[x_0(\varepsilon),x_1(\varepsilon)]$ stetig differenzierbar
ist und für $\varepsilon=0$ mit $y(x)$ übereinstimmt: $y(x)=y(x,0)$.
Der Einfachheit halber wird die partielle Ableitung von $y(x,\varepsilon)$
nach $x$ auch
\[
y'(x,\varepsilon) = \frac{\partial y}{\partial x}(x,\varepsilon)
\]
geschrieben.
Für die Werte an den Endpunkten schreiben wir abkürzend
\begin{align*}
x_i &= x_i(0),
&
y_i &= y(x_i(0),0) = y(x_i)
&&\text{und}&
y_i'
=
y'(x_i(0),0)
=
\frac{\partial y'(x(\varepsilon),\varepsilon)}{\partial\varepsilon}
\bigg|_{\varepsilon=0}
\end{align*}
für $i=0,1$.
\end{definition}

%
% Kurven, auf denen sich die Endpunkte bewegen
%
\subsubsection{Kurven, auf denen sich die Endpunkte bewegen}
Der Ansatz~\eqref{buch:variation:allgemein:eqn:ansatz2} beschreibt
Kurven, die die beiden Punkte
\begin{align*}
P_0(\varepsilon)
&=
(x_0(\varepsilon),y(x_0(\varepsilon),\varepsilon))
&&\text{und}&
P_1
&=
(x_1(\varepsilon),y(x_1(\varepsilon),\varepsilon))
\end{align*}
miteinander verbinden.
Die Endpunkte der Kurve bewegen sich also auf den Kurven
\begin{align*}
\varepsilon
&\mapsto
P_0(\varepsilon)
=
(x_0(\varepsilon),y(x_0(\varepsilon),\varepsilon)
&&\text{und}&
\varepsilon
&\mapsto
P_1(\varepsilon)
=
(x_1(\varepsilon),y(x_1(\varepsilon),\varepsilon).
\end{align*}
Da alle beteiligten Funktionen als differenzierbar vorausgesetzt wurden,
lassen sich auch die Tangenten dieser Kurven bestimmen.
Der Richtungsvektor der Tangente im Punkt $x_i=x_i(0)$ ist
\begin{align}
\vec{r}_i(\varepsilon)
=
\frac{d}{d\varepsilon}
\begin{pmatrix}
x_i(\varepsilon)\\
y(x_i(\varepsilon),\varepsilon)
\end{pmatrix}
\bigg|_{\varepsilon=0}
\label{buch:variation:allgemein:eqn:tangential}
\end{align}
für $i=0,1$.
Die Kurve $\varepsilon\mapsto P_i(\varepsilon)$ lässt sich daher für
genügend kleine $\varepsilon$ beliebig genau durch
\[
x\mapsto
\begin{pmatrix}
x_0(0)\\
y(x,0) 
\end{pmatrix}
+
\varepsilon
\frac{d}{d\varepsilon}
\begin{pmatrix}
x_i(\varepsilon)\\
y(x_i(\varepsilon),\varepsilon)
\end{pmatrix}
\bigg|_{\varepsilon=0}
\]
approximieren.

%
% Erste Variation
%
\subsection{Erste Variation einer einzelnen Funktion
\label{buch:variation:allgemein:subsection:var1}}
Wir betrachten jetzt wieder das Funktional
\[
I(y)
=
\int_{x_0}^{x_1} F(x,y(x),y'(x)) \,dx
\]
für die Lagrange-Funktion $f(x,y,y')$.
Eine allgemeine Variation der Funktion $y(x)$ im Sinne der
Definition~\ref{buch:variation:allgemein:def:variation} macht aus
$I(y)$ eine Funktion 
\[
I(y,\varepsilon)
=
\int_{x_0(\varepsilon)}^{x_1(\varepsilon)}
F(x,y(x,\varepsilon),y'(x,\varepsilon))
\,dx
\]
von $\varepsilon$.
Davon ist wieder die Ableitung nach $\varepsilon$ an der Stelle
$\varepsilon=0$ zu ermitteln.
Sie ist
\begin{align*}
\frac{\partial}{\partial\varepsilon}I(y,\varepsilon)
&=
F(x_1(\varepsilon),y(x_1(\varepsilon),\varepsilon),
y'(x_1(\varepsilon),\varepsilon))
\frac{dx_1(\varepsilon)}{d\varepsilon}
-
F(x_0(\varepsilon),y(x_0(\varepsilon),\varepsilon),
y'(x_0(\varepsilon),\varepsilon))
\frac{dx_0(\varepsilon)}{d\varepsilon}
\\
&\qquad
+
\int_{x_0(\varepsilon)}^{x_1(\varepsilon)}
\frac{\partial F}{\partial y}(x, y(x,\varepsilon), y'(x,\varepsilon))
\frac{\partial y}{\partial \varepsilon}(x,\varepsilon)
\,dx
\\
&\qquad
+
\int_{x_0(\varepsilon)}^{x_1(\varepsilon)}
\frac{\partial F}{\partial y'}(x, y(x,\varepsilon), y'(x,\varepsilon))
\frac{\partial^2 y}{\partial x\,\partial\varepsilon}(x,\varepsilon)
\,dx
\end{align*}
An der Stelle $\varepsilon=0$ vereinfacht sich die Variation zu
\begin{align}
\delta I(y)
&=
F(x_1,y(x_1),y'(x_1))
\frac{dx_0(0)}{d\varepsilon}
-
F(x_0,y(x_0),y'(x_0))
\frac{dx_1(0)}{d\varepsilon}
\notag
\\
&\qquad
+
\int_{x_0}^{x_1}
\frac{\partial F}{\partial y}(x,y(x),y'(x))
\frac{\partial y}{\partial \varepsilon}(x,0)
\,dx
\notag
\\
&\qquad
+
\int_{x_0}^{x_1}
\frac{\partial F}{\partial y'}(x,y(x),y'(x))
\frac{\partial^2 y}{\partial x\,\partial\varepsilon}(x,0)
\,dx.
\label{buch:variation:allgemein:eqn:var1}
\end{align}
Die Ableitung $y'$ im zweiten Integral kann mit partieller Integration 
vereinfacht werden:
\begin{align*}
\int_{x_0}^{x_1}
\frac{\partial F}{\partial y'}(x,y(x),y'(x))
\frac{\partial y'}{\partial \varepsilon}(x,0)
\,dx
&=
\biggl[
\frac{\partial F}{\partial y'}(x,y(x),y'(x))
\frac{\partial y}{\partial\varepsilon}(x,0)
\biggr]_{x_0}^{x_1}
\\
&\qquad
-
\int_{x_0}^{x_1}
\frac{d}{dx}
\frac{\partial F}{\partial y'}(x,y(x),y'(x))
\frac{\partial y}{\partial\varepsilon}(x,0)
\,dx.
\end{align*}
Eingesetzt in 
\eqref{buch:variation:allgemein:eqn:var1}
ist die allgemeine Variation des Funktionals $I(y)$ daher
\begin{equation}
\begin{aligned}
\delta I(y)
&=
F(x_1,y(x_1),y'(x_1))
\frac{dx_0(0)}{d\varepsilon}
-
F(x_0,y(x_0),y'(x_0))
\frac{dx_1(0)}{d\varepsilon}
\\
&\qquad
+
\biggl[
\frac{\partial F}{\partial y'}(x,y(x),y'(x))
\frac{\partial y}{\partial\varepsilon}(x,0)
\biggr]_{x_0}^{x_1}
\\
&\qquad
+
\int_{x_0}^{x_1}
\biggl(
\frac{\partial F}{\partial y}(x,y(x),y'(x))
-
\frac{d}{dx}\frac{\partial F}{\partial y'}(x,y(x),y'(x))
\biggr)
\frac{\partial y}{\partial \varepsilon}(x,0)
\,dx.
\end{aligned}
\label{buch:variation:allgemein:eqn:variation}
\end{equation}

Sei jetzt $y(x)$ eine Funktion, die das Funktional $I(y)$ stationär macht,
für jede Wahl einer Variation von $y(x)$ ist also $\delta I(y)=0$.
Dies gilt insbesondere auch für alle Variationen, die Endpunkte der
Kurve unverändert lassen.
Für eine solche Variation verschwinden die ersten drei Terme, es bleibt
nur das Integral.
Wie früher folgt daher, dass der Ausdruck in der Klammer im Integral
verschwinden muss.
Eine Lösung des Variationsproblems muss also die
Euler-Lagrange-Differentialgleichung erfüllen.

Die verbleibenden Terme in \eqref{buch:variation:allgemein:eqn:variation}
drücken aus, wie sich das Funktional $I(y)$ ändert, wenn sich die
Endpunkte der Kurve verschieben.
Die ersten zwei Terme behandeln offensichtlich den Fall, dass die 
Intervalgrenzen verschoben werden.
Der dritte Term behandelt die Änderung, die ausschliesslich von der
Änderung des zweiten Parameters von $y(x,\varepsilon)$ ausgeht.
Der $y$-Wert kann sich auch dadurch ändern, dass $x$ variert wird.
\begin{align*}
\frac{dy(x_i(\varepsilon),\varepsilon)}{d\varepsilon}
&=
\frac{\partial y(x_i(\varepsilon),\varepsilon)}{\partial x}
\frac{dx_i(\varepsilon)}{d\varepsilon}
+
\frac{\partial y(x_i(\varepsilon), \varepsilon)}{\partial \varepsilon}
\intertext{oder an der Stelle $\varepsilon=0$}
\frac{dy(x_i(0),0)}{d\varepsilon}
&=
\frac{\partial y(x_i(0),0)}{\partial x}
\frac{dx_i(0)}{d\varepsilon}
+
\frac{\partial y(x_i(0), 0)}{\partial \varepsilon}
\intertext{Aufgelöst nach dem letzten Term ist dies}
\frac{\partial y(x_i(0), 0)}{\partial \varepsilon}
&=
\frac{dy(x_i(0),0)}{d\varepsilon}
-
\frac{\partial y(x_i(0),0)}{\partial x}
\frac{dx_i(0)}{d\varepsilon}
\end{align*}
Dies können wir in die Variation
\eqref{buch:variation:allgemein:eqn:variation}
einsetzen und erhalten
\begin{equation}
\begin{aligned}
\delta I(y)
&=
\biggl(
F(x_1,y(x_1),y'(x_1))
-
\frac{\partial y(x_1(0),0)}{\partial x}
\frac{\partial F}{\partial y'}(x_1,y(x_1),y'(x_1))
\biggr)
\frac{dx_1(0)}{d\varepsilon}
\\
&\qquad
+
\frac{\partial F}{\partial y'}(x_1,y(x_1),y'(x_1))
\frac{dy(x_1(0),0)}{d\varepsilon}
\\
&\qquad
-
\biggl(
F(x_0,y(x_0),y'(x_0))
-
\frac{\partial y(x_0(0),0)}{\partial x}
\frac{\partial F}{\partial y'}(x_0,y(x_0),y'(x_0))
\biggr)
\frac{dx_0(0)}{d\varepsilon}
\\
&\qquad
-
\frac{\partial F}{\partial y'}(x_0,y(x_0),y'(x_0))
\frac{dy(x_0(0),0)}{d\varepsilon}
\\
&\qquad
+
\int_{x_0}^{x_1}
\biggl(
\frac{\partial F}{\partial y}(x,y(x),y'(x))
-
\frac{d}{dx}\frac{\partial F}{\partial y'}(x,y(x),y'(x))
\biggr)
\frac{\partial y}{\partial \varepsilon}(x,0)
\,dx.
\end{aligned}
\label{buch:variation:allgemein:eqn:variation}
\end{equation}
Von den ersten vier Terme lassen sich jeweils zwei als ein Skalarprodukt
des Vektors
\[
\vec{f}_i
=
\begin{pmatrix}
\displaystyle
F(x_i,y(x_i),y'(x_i))
-
\frac{\partial F}{\partial y'}(x_i,y(x_i),y'(x_i))
\frac{\partial y(x_i(0),0)}{\partial\varepsilon}
\\[3pt]
\displaystyle
\frac{\partial F}{\partial y'}(x_i,y(x_i),y'(x_i))
\end{pmatrix}
\]
mit dem Richtungsvektor  $\vec{r}_i(0)$ von
\eqref{buch:variation:allgemein:eqn:tangential} schreiben.
Mit dieser Notation erhält die allgemeinste Form der ersten Variation
die Form
\begin{equation}
\delta I(y)
=
\vec{f}_1\cdot \vec{r}_i(0)
-
\vec{f}_0\cdot \vec{r}_0(0)
+
\int_{x_0}^{x_1}
\biggl(
\frac{\partial F}{\partial y}(x,y(x),y'(x))
-
\frac{d}{dx}\frac{\partial F}{\partial y'}(x,y(x),y'(x))
\biggr)
\frac{\partial y}{\partial \varepsilon}(x,0)
\,dx.
\end{equation}

\begin{satz}[Allgemeine Variationsformel]
\label{buch:variation:allgemein:satz:allgemeinvariation1}
Sei $y(x,\varepsilon)$ eine Variation der zweimal stetig differenzierbaren
Funktion $y(x)$ und sei
\[
\vec{r}_i(0)
=
\frac{d}{d\varepsilon}
\begin{pmatrix}
x_i(\varepsilon)\\
y(x_i(\varepsilon),\varepsilon)
\end{pmatrix}\bigg|_{\varepsilon=0}
\]
der Tangentialvektor an die Kurve
$\varepsilon\mapsto P_i(\varepsilon) = (x_i(\varepsilon),y(x_i(\varepsilon)))$,
auf der sich der Endpunkt $P_i$ der Kurve während der Variation
bewegt.
Dann ist die Variation
\begin{equation}
\delta I(y)
=
\vec{f}_1\cdot\vec{r}_1(0)
-
\vec{f}_0\cdot\vec{r}_0(0)
+
\int_{x_0}^{x_1}
\biggl(
\frac{\partial F}{\partial y}(x,y(x),y'(x))
-
\frac{d}{dx}\frac{\partial F}{\partial y'}(x,y(x),y'(x))
\biggr)
\frac{\partial y}{\partial \varepsilon}(x,0)
\,dx
\label{buch:variation:allgemein:eqn:allgemeinvariation1}
\end{equation}
mit
\[
\vec{f}_i
=
\begin{pmatrix}
\displaystyle
F(x_i,y(x_i),y'(x_i))
-
y'(x_i)
\frac{\partial F}{\partial y'}(x_i,y(x_i),y'(x_i))
\\[3pt]
\displaystyle
\frac{\partial F}{\partial y'}(x_i,y(x_i),y'(x_i))
\end{pmatrix}.
\]
\end{satz}

Beliebige Variationen der Endpunkte sind nicht sinnvoll, denn
wären die Endpunkte beliebig wählbar, dann lässt sich $I(y)$ kleiner
(oder grösser) machen, indem der Weg geeignet verkürzt wird.
Das einzig mögliche Minimum für das Funktional wird dann für eine
Funktion erreicht, bei der $x_0=x_1$ gilt.
Es sind daher nur Variationen zielführend, die tangential an eine
Kurve in der $x$-$y$-Ebene sind.
Die Skalarprodukte in 
\eqref{buch:variation:allgemein:eqn:allgemeinvariation1}
bedeuten liefern dann Randbedingungen für $y(x)$.

%
% Erste Variation für mehrere Funktionen 
%
\subsection{Erste Variation mehrere Funktionen
\label{buch:variation:allgemein:subsection:var1n}}
Die Entwicklung des letzten Abschnitts lässt sich auch für mehrere
Funktionen $y_1(x),\dots,y_n(x)$ durchführen.
Seien also $y_k(x,\varepsilon)$ Variationen der Funktionen $y_k(x)$
mit der Variation $x_0(\varepsilon)$ und $x_1(\varepsilon)$ der 
Intervallenden.
Die Endpunkte der Kurven
\[
x\mapsto
=
(x
y_1(x,\varepsilon),\dots
y_n(x,\varepsilon))
\in \mathbb{R}^{n+1}
\]
bewegen sich jetzt auf der Kurve
\[
\varepsilon
\mapsto
P_i(\varepsilon)
=
(x_i(\varepsilon),
y_1(x_i(\varepsilon),\varepsilon),
\dots
y_n(x_i(\varepsilon),\varepsilon))
\]
mit dem Tangentialvektor
\[
\vec{r}_i(\varepsilon)
=
\frac{d}{d\varepsilon}
\begin{pmatrix}
x_i(\varepsilon)\\
y_1(x_i(\varepsilon),\varepsilon)\\
\vdots\\
y_n(x_i(\varepsilon),\varepsilon)
\end{pmatrix}
\]
für $i=0,1$.

Dem Vektor $\vec{f}_i$ im Falle einer Funktion entspricht der Vektor
\[
\vec{f}_i
=
\begin{pmatrix}
F(x_i,y_1(x_i),y'_1(x_i),\dots,y_n(x_i),y_n'(x_i))
-
\sum_{k=1}^n y'_k(x_i) \frac{\partial F}{\partial y_k'}
(x_i,y_1(x_i),y'_1(x_i),\dots,y_n(x_i),y_n'(x_i))
\\
\frac{\partial F}{\partial y_1}
(x_i,y_1(x_i),y'_1(x_i),\dots,y_n(x_i),y_n'(x_i))
\\
\vdots
\\
\frac{\partial F}{\partial y_n}
(x_i,y_1(x_i),y'_1(x_i),\dots,y_n(x_i),y_n'(x_i))
\end{pmatrix}.
\]

Diese Notation ist nicht wirklich lesbar, wir greifen daher wieder
zurück auf die vektorielle Schreibweise, in der 
$y(x)$ und die Variation $y(x,\varepsilon)$ die $n$-dimensionalen
Vektoren
\[
y(x)
=
\begin{pmatrix}
y_1(x)\\
\vdots\\
y_n(x)
\end{pmatrix}
\qquad
\text{und}
\qquad
y(x,\varepsilon)
=
\begin{pmatrix}
y_1(x,\varepsilon)\\
\vdots\\
y_n(x,\varepsilon)
\end{pmatrix}
\]
sind.
Auch die Funktion $F$ können wir jetzt wie in Abschnitt
kompakter als
Funktion
\[
F\colon
\mathbb{R}\times \mathbb{R}^n\times\mathbb{R}^n
\to
\mathbb{R}
:
(x,y,y')
\mapsto
F(x,y,y')
\]
schreiben.
Die unteren $n$ Komponenten der Richtungsvektoren sind einfach
die Ableitung des Vektors
$y(x(\varepsilon),\varepsilon)$ nach $\varepsilon$.

Die unteren $n$ Komponenten von $\vec{r}_i$ und $\vec{f}_i$ bilden jeweils
einen Vektor.
Im Falle von $\vec{r}_i$ schreiben wir
\[
\vec{r}_i(\varepsilon)
=
\frac{d}{d\varepsilon}
\begin{pmatrix}
x_i(\varepsilon)\\
y(x_i(\varepsilon),\varepsilon)
\end{pmatrix}.
\]
Im Falle von $\vec{f}_i$ sind die unteren $n$ Komponenten
die Ableitungen von $F$ nach den $n$ Variablen $y_1,\dots,y_n$.
Dafür wurde in
Abschnitt~\ref{buch:variation:mehrerefunktionen:subsection:vektorableitung}
die Notation $\partial F/\partial y$ eingeführt.
Analog können wir den Vektor der Ableitungen von $F$ nach den
$y_k'$ als $\partial F/\partial y'$.
Mit dieser Notation bekommen wir die sehr kompakte Form
\[
\vec{f}_i
=
\begin{pmatrix}
\displaystyle
F(x_i,y(x_i),y'(x_i)) - \frac{\partial}{\partial y'}(x_i,y(x_i),y'(x_i))
\\[5pt]
\displaystyle
\frac{\partial}{\partial y} F(x_i,y(x_i),y'(x_i))
\end{pmatrix}
\]
für $i=0,1$.

Für die Variation finden wir dann
\begin{align*}
\delta I(y)
&=
\vec{f}_1\cdot \vec{r}_1(0)
-
\vec{f}_0\cdot \vec{r}_0(0)
\\
&\qquad
+
\int_{x_0}^{x_1}
\biggl(
\frac{\partial}{\partial y}F(x,y(x),y'(x))
-
\frac{d}{dx}
\frac{\partial}{\partial y'}F(x,y(x),y'(x))
\biggr)
\cdot
\frac{\partial y}{\partial\varepsilon}(x,0)
\,dx
\end{align*}
Damit die Variation verschwindet, müssen also die
Euler-Lagrange-Differentialgleichungen erfüllt sein.
Zusätzlich müssen aber die Skalarprodukte
$\vec{f}_1\cdot \vec{r}_1(0)$
und
$\vec{f}_0\cdot \vec{r}_0(0)$
verschwinden.
Dies bedeutet, dass die Vektoren $\vec{f}_i$ auf den erlaubten
Tangentialvektoren $\vec{r}_i(0)$ orthogonal sind.


%
% Erste Variation mit höheren Ableitungen
%
\subsection{Erste Variation mit höheren Ableitungen
\label{buch:variation:allgemein:subsection:var2h}}





\uebungsabschnitt

\aufgabetoplevel{chapters/020-variation/uebungsaufgaben}
\begin{uebungsaufgaben}
\uebungsaufgabe{201}
\end{uebungsaufgaben}
\enduebungsabschnitt

