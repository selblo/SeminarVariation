%
% brachistochronenproblem.tex -- Brachistochronenproblem
%
% (c) 2021 Prof Dr Andreas Müller, OST Ostschweizer Fachhochschule
%
\documentclass[tikz]{standalone}
\usepackage{amsmath}
\usepackage{times}
\usepackage{txfonts}
\usepackage{pgfplots}
\usepackage{csvsimple}
\usetikzlibrary{arrows,intersections,math}
\begin{document}
\def\skala{1}
\def\r{1.5}
\pgfmathparse{3.14159/180}
\xdef\m{\pgfmathresult}
\def\xwert#1{\r*((#1)*\m-sin(#1))}
\def\ywert#1{\r*(cos(#1)-1)}
\def\punkt#1{ ({\r*((#1)*\m-sin(#1))},{\r*(cos(#1)-1)}) }
\begin{tikzpicture}[>=latex,thick,scale=\skala]

\draw[color=gray!50] plot[domain=0:360,samples=360]
	({\r*((\x)*\m-sin(\x))},{\r*(cos(\x)-1)});

\draw[->] (-0.1,0) -- (10,0) coordinate[label={$x$}];
\draw[->] (0,0.1) -- (0,-3.5) coordinate[label={left:$y$}];

\draw ({\m*\r*180},0.05) -- ({\m*\r*180},-0.05);
\node at ({\m*\r*180},0) [above] {$\frac{\pi}2\mathstrut$};
\draw ({\m*\r*360},0.05) -- ({\m*\r*360},-0.05);
\node at ({\m*\r*360},0) [above] {$\pi\mathstrut$};

\draw[line width=0.2pt]  ({\xwert{60}},0) -- \punkt{60};
\node at ({\xwert{60}},0) [above] {$a\mathstrut$};
\draw ({\xwert{60}},0.05) -- ({\xwert{60}},-0.05);

\draw[line width=0.2pt]  ({\xwert{160}},0) -- \punkt{160};
\node at ({\xwert{160}},0) [above] {$b\mathstrut$};
\draw ({\xwert{160}},0.05) -- ({\xwert{160}},-0.05);

\draw[color=red,line width=1.2pt] plot[domain=60:160,samples=100]
	({\r*((\x)*\m-sin(\x))},{\r*(cos(\x)-1)});

\fill[color=red] \punkt{60} circle[radius=0.08];
\node at \punkt{60} [above right] {$A$};
\fill[color=red] \punkt{160} circle[radius=0.08];
\node at \punkt{160} [below] {$B$};
\fill[color=red] \punkt{100} circle[radius=0.08];
\node at \punkt{100} [below left] {$M$};

\end{tikzpicture}
\end{document}

