%
% nichtdiff.tex -- Lösung für das Variationsproblem \int y^2(1-y'^2)
%
% (c) 2021 Prof Dr Andreas Müller, OST Ostschweizer Fachhochschule
%
\documentclass[tikz]{standalone}
\usepackage{amsmath}
\usepackage{times}
\usepackage{txfonts}
\usepackage{pgfplots}
\usepackage{csvsimple}
\usetikzlibrary{arrows,intersections,math}
\definecolor{darkred}{rgb}{0.8,0,0}
\begin{document}
\def\skala{1}
\begin{tikzpicture}[>=latex,thick,scale=\skala]

\draw[->] (-5,0) -- (5.2,0) coordinate[label={$x$}];
\draw[->] (0,-0.3) -- (0,5) coordinate[label={left:$y$}];

\def\d{4.5}
\def\s{0}
%\foreach \a in {0.2,0.4,...,2}{
%	\draw[color=blue!20,line width=1.2pt]
%		plot[domain=-1:1,samples=100]
%			({\d*\x},{\d*(\a*(\x-\s)*(\x-\s)-1/(4*\a*\a))});
%}

\draw[color=darkred,line width=1.4pt] (-4.8,0) -- (0,0) -- (4.8,4.8);
\node at (-2.4,0) [above] {$F(x,0,0)=0^2\cdot(1-0^2)=0 $};
\node at (2.4,2.4) [above,rotate=45] {$F(x,x,1)=1^2\cdot\sqrt{1-1^2}=0$};

\coordinate (A) at (2,2);
\coordinate (B) at (3.5,1);

\draw[color=darkred,line width=0.2pt,shorten >= 0.3cm] (A) -- (B);
\node[color=darkred] at (B) {$y(x)$};

\end{tikzpicture}
\end{document}

