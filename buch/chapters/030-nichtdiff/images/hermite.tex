%
% hermite.tex -- Hermite Polynome
%
% (c) 2021 Prof Dr Andreas Müller, OST Ostschweizer Fachhochschule
%
\documentclass[tikz]{standalone}
\usepackage{amsmath}
\usepackage{times}
\usepackage{txfonts}
\usepackage{pgfplots}
\usepackage{csvsimple}
\usetikzlibrary{arrows,intersections,math}
\definecolor{darkred}{rgb}{0.8,0,0}
\begin{document}
\def\skala{1}
\def\achsen#1#2{
	\draw[->] (-0.1,0) -- (4.3,0) coordinate[label={$x$}];
	\draw (4,-0.05) -- (4,0.05);
	\draw[->] (0,{(#1)-0.1}) -- (0,{(#2)+0.3})
		coordinate[label={right:$y$}];
	\node at (0,0) [left] {$0$};
}
\begin{tikzpicture}[>=latex,thick,scale=\skala]

\begin{scope}
\draw[color=darkred,line width=1.4pt] plot[domain=0:1]
	({4*\x},{4*((2*\x-3)*\x*\x+1)});
\achsen{0}{4}
\node at (4,0) [below] {$1$};
\draw (-0.05,4) -- (0.05,4);
\node at (0,4) [left] {$1$};
\node at (3,3) {$H_0(x)$};
\end{scope}

\begin{scope}[xshift=6cm]
\draw[color=darkred,line width=1.4pt] plot[domain=0:1]
	({4*\x},{4*(-2*\x+3)*\x*\x});
\achsen{0}{4}
\node at (4,0) [below] {$1$};
\draw (-0.05,4) -- (0.05,4);
\node at (0,4) [left] {$1$};
\node at (1,3) {$H_1(x)$};
\end{scope}

\begin{scope}[yshift=-2cm]
\achsen{-0.6}{1}
\draw[color=darkred,line width=1.4pt] plot[domain=0:1]
	({4*\x},{4*((\x-2)*\x+1)*\x});
\node at (4,0) [below] {$1$};
\node at (3,1) {$H_0^1(x)$};
\end{scope}

\begin{scope}[yshift=-2cm,xshift=6cm]
\achsen{-0.6}{1}
\draw[color=darkred,line width=1.4pt] plot[domain=0:1]
	({4*\x},{4*((\x-1)*\x*\x)});
\node at (4,0) [above] {$1$};
\node at (3,1) {$H_1^1(x)$};
\end{scope}

\end{tikzpicture}
\end{document}

