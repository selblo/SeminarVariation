%
% 1-symmetrie.tex
%
% (c) 2023 Prof Dr Andreas Müller
%
\section{Symmetrien und Funktionale
\label{buch:symmetrien:section:symmetrie}}
\kopfrechts{Symmetrie und Funktionale}
In diesem Abschnitt betrachten wir immer eine Lagrange-Funktion
der Form $L(t,x,\dot{x})$ und damit ein Funktional der Form
\[
I(x)
=
\int_{t_0}^{t_1}
L(t,x(t),\dot{x}(t))\,dt,
\]
abhängig von einer Funktion $x\colon[t_0,t_1]\to\mathbb{R}$.
In diesem Abschnitt soll definiert werden, was unter einer
differenzierbare Symmetrie der Lagrange-Funktion $L$ zu verstehen
ist.

%
% Definition
%
\subsection{Definition}
Eine Symmetrie der Lagrange-Funktion soll ihren Wert nicht ändern.
Da sie aber auch von $\dot{x}$ abhängt, muss erklärt werden, wie
der Vektor $\dot{x}$ abgebildet werden muss.
Dazu beachtet man, dass $\dot{x}$ als die Geschwindigkeit eines
Punktes interpretiert werden, der sich entlang einer Kurve $x(t)$
bewegt.
Eine Symmetrieabbildung $\varphi\colon\mathbb{R}^n\to\mathbb{R}^n$ 
bildet die Kurve $t\mapsto x(t)$ auf die Kurve $t\mapsto \varphi(x(t))$
ab. 
Die Geschwindigkeit im Punkt $x(t)$ wird dabei nach der Kettenregel
auf 
\[
\frac{d}{dt}\varphi(x(t))
=
D_{x(t)}\varphi\cdot \dot{x}(t)
\]
abgebildet.
Die lineare Abbildung, die auf den Geschwindigkeitsvektor $\dot{x}(t)$
anzuwenden ist, hängt daher vom Punkt $x(t)$ ab, an dem sich der
Punkt gerade befindet.
Insbesondere ist es nur für differenzierbare Abbildungen $\varphi$
sinnvoll, von der Invarianz einer Lagrange-Funktion zu sprechen.

\begin{definition}[Invarianz einer Lagrange-Funktion]
Eine Lagrange-Funktion heisst {\em invariant} unter einer differenzierbaren
Symmetrieabbildung $\varphi\colon\mathbb{R}^n\to\mathbb{R}^n$, wenn
\[
L(t,\varphi(x), D_x\varphi(\dot{x}))
=
L(t,x,\dot{x})
\]
für alle $t$, $x$ und $\dot{x}$.
\end{definition}

%
% Extremalen
%
\subsection{Extremalen}
Die Invarianz der Lagrange-Funktion hat auch zur Folge, dass der
Wert des Funktionals nicht ändert, wenn man eine Funktion $t\mapsto x(t)$
der Transformation $\varphi$ unterwirft, es gilt also für alle
solchen Funktionen
\[
I(\varphi\circ x)
=
\int_{t_0}^{t_1}
L(t,\varphi(x(t)),D_{x(t)}\varphi(\dot{x}(t)))
\,dt
=
\int_{t_0}^{t_1}
L(t,x(t),\dot{x}(t))
\,dt.
\]
Daraus kann man allerdings noch nicht schliessen, dass $x(t)$ genau
dann eine extremale ist, wenn auch $\varphi(x(t))$ eine Extremale ist.
Dazu muss man zusätzlich verlangen, dass die Abbildung $\varphi$
invertierbar ist.

\begin{satz}
\label{buch:symmetrien:symmetrie:satz:invarianz}
Ist $\varphi$ eine invertierbare Symmetrieabbildung, unter der die
Lagrange-Funktion invariant ist.
Zudem sei auch die Umkehrabbildung $\varphi^{-1}$ differenzierbar.
Dann ist $\varphi(x(t))$ eine Extremale genau dann, wenn $x(t)$ eine
Extremale ist.
\end{satz}

\begin{proof}
Sei $x(t)$ eine Extremale, also eine Funktion, für die die Ableitung
\[
\frac{d}{d\varepsilon}I(x(t,\varepsilon))
\bigg|_{\varepsilon=0}
=
0
\]
für jede Variation $x(t,\varepsilon)$ verschwindet.
Da $\varphi$ invertierbar ist, lässt sich jede Variation von
$\varphi(x(t))$ in der Form $\varphi(x(t,\varepsilon))$ ausdrücken.
Ist nämlich $y(t,\varepsilon)$ ein Variation von $\varphi(x(t))$
mit $y(t,0) = \varphi(x(t))$, dann ist $\varphi^{-1}(y(t,\varepsilon))$
eine Variation der Funktion $x(t)$.
Die Abbildung $\varphi$ ändert aber den Wert des Funktionals nicht,
also ändert sich dadurch auch die Ableitung nach $\varepsilon$.
Verschwindet die Ableitung von $I(x(t,\varepsilon))$ nach $\varepsilon$
für alle Variationen $x(t,\varepsilon)$ von $x(t)$, dann verschwindet
auch die Ableitung von $I(y(t,\varepsilon))$ für alle Variationen
$y(t,\varepsilon)$ von $\varphi(x(t))$.
\end{proof}

Nach Satz~\ref{buch:symmetrien:symmetrie:satz:invarianz} gehen 
also bei einer unter $\varphi$ invarianten Lagrange-Funktion die
Extremalen durch $\varphi$ in Extremalen über.

%
% Stetige Symmetrien
%
\subsection{Stetige Symmetrien}
Drehungen oder Translationen sind invertierbare Symmetrieoperationen,
die zusätzlich stetig oder sogar differenzierbar von einem Parameter
abhängen.

\begin{definition}[Stetige Symmetrie]
Sei $h^s\colon \mathbb{R}^n\to\mathbb{R}^n$ für jeden Wert $s$
eine invertierbare und differenzierbare Abbildung.
Sie heisst eine {\em stetige Symmetrie} von $L$, wenn
\[
L(t, h^s(x), Dh^s(\dot{x}))
=
L(t,x,\dot{x})
\]
und $h^s$ hängt stetig von $s$ ab.
\end{definition}

\begin{definition}[Differenzierbare Symmetrie]
Eine stetige Symmetrie 
$h^s$ heisst {\em differenzierbar}, wenn $h^s(x)$
eine differenzierbare Funktion von $s$ ist.
\end{definition}

\begin{beispiel}
\label{buch:symmetrien:symmetrie:beispiel:homogen}
Wir betrachten die Transformation
\[
h^s
\colon
\mathbb{R}^3 \to \mathbb{R}^3
:
x \mapsto x + su
\]
mit einem Vektor $u\in\mathbb{R}$.
$h^s$ ist invertierbar, denn $(h^s)^{-1}=h^{-s}$.
Ausserdem ist sie differenzierbar, denn
\[
Dh^s(v)
=
sv.
\]
In der Lagrange-Funktion
\[
L
=
\frac12m\dot{x}^2 - mgx_3
\]
ist $D\varphi(\dot{x})=\dot{x}$, der erste Term ändert also nicht.
Der zweite Term dagegen wird bei der Symmetrieabbildung zu
$mgx_3\mapsto mg(x_3+u_3)$.
Die Lagrange-Funktion ist also nur für Translationen mit Vektoren $u$
invariant, die senkrecht auf $\vec{e}_3$ sind oder $u_3=0$ haben.
\end{beispiel}

\begin{beispiel}
\label{buch:symmetrien:symmetrie:beispiel:drehung}
Drehungen des dreidimensionalen Raumes $\mathbb{R}^3$ können durch
orthogonale Matrizen $O$ beschrieben werden.
In der Lagrange-Funktion $L=\frac12m\dot{x}^2-V(x)$ ist der erste
Term wegen
\[
\frac12m (O\dot{x})(O\dot{x})
=
\frac12m\dot{x}^2
\]
invariant.
Der zweite Term ist genau dann invariant, wenn das Potential $V(x)$
invariant ist, wenn also $V(O(x))=V(x)$ ist.
\end{beispiel}

%
% Zeitunabhängige Lagrange-Funktion
%
\subsection{Zeitinvarianz}
Die bisher betrachteten Symmetrien betrafen nur die abhängigen
Variablen $x$ und $\dot{x}$, nicht die unabhängige Variable $t$.
Die unabhängige Variable spielt eine spezielle Rolle, die nicht
direkt mit den abhängigen Variablen vergleichbar ist.
Eine Transformation der unabhängigen Variablen ändert auch den
Definitionsbereich, so dass eine erweiterte Theorie nötig wird, 
die in Abschnitt~\ref{buch:symmetrien:section:kontinuitaet}
besprochen wird.
Ein einfacher Spezialfall ist der Fall, wo $L$ nicht von der
Zeit abhängt, also $L(t,x,\dot{x})=L(x,\dot{x})$ geschrieben werden
kann.
In diesem Fall ergibt der Hamilton-Formalismus die Hamilton-Funktion
$H=L-pq$, die ebenfalls nicht explizit von der Zeit abhängt, also
auch in der Form $H(p,q)$ geschrieben werden kann.
Nach den hamiltonschen Differentialgleichungen ist dann
\begin{align*}
\frac{d}{dt}H(p,q)
=
\sum_{k=1}^n
\frac{\partial H}{\partial q_k}\dot{q}_k
+
\sum_{k=1}^n
\frac{\partial H}{\partial p_k}\dot{p}_k
=
\sum_{k=1}^n
\bigl(
-
\dot{p}_k\dot{q}_k
+
\dot{q}_k\dot{p}_k
\bigr)
=
0.
\end{align*}
Somit ist $H(p,q)$ eine Erhaltungsgrösse.
Im Falle einer zeitunabhängigen Lagrange-Funktion ist also die
Energie eine Erhaltungsgrösse.
Auch in diesem Fall ergibt sich also ein Erhaltungssatz, wie ihn
der Satz von Emmy Noether im nächsten Abschnitt für jede differenzierbare
Symmetrie einer Lagrange-Funktion liefern wird.

%
% Lie-Gruppen
%
\subsection{Lie-Gruppen}
Der allgemeine Rahmen der differenzierbaren Symmetrien ist die
Theorie der Lie-Gruppen.
\index{Lie-Gruppe}%
Sie ergeben sich auf natürliche Art und Weise als die Gruppen der
Matrizen, die gewisse Funktionen von Vektoren invariant lassen.
die Gruppe $\operatorname{SO}(3)$ der Drehungen des Raumes
zum Beispiel ist die Gruppe der Matrizen, die das Skalarprodukt
invariant lassen.
In der relativistischen Mechanik
\index{relativistische Mechanik}%
sind nur Transformationen zulässig, die die Grösse
\[
l^2
=
x^2+y^2+z^2-c^2t^2
\]
invariant lassen.
Sie bilden die Lie-Gruppe der Lorentz-Transformationen.

Die Transformationen in einer Lie-Gruppe können wie die Drehungen
durch die Matrixexponentialfunktion $\exp(sA)$ aus einer Matrix $A$
gefunden werden, die als die Ableitung der Funktion $s\to\exp(sA)$
erhalten werden können.
Sie sind damit automatisch differenzierbare Symmetrien.
Die Ableitungen haben die zusätzliche Struktur einer Lie-Algebra
\cite[p.~434]{buch:linalg}, die für die weiteren Untersuchungen
nützlich sein kann.

