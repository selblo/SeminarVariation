%
% 2-transversal.tex
%
% (c) 2023 Prof Dr Andreas Müller
%
\section{Freie Randbedingungen und Transversalität
\label{buch:nebenbedingungen:section:transversl}}
\kopfrechts{Freie Randbedingungen und Transversalität}
Bis jetzt wurden ausschliesslich Anfangspunkt-Endpunkt-Probleme gelöst.
Sie zeichnen sich dadurch aus, dass die Funktion $\eta(x)$, mit der
variiert wurde, in den Endpunkten des Definitionsintervalls
verschwinden.
Dies bedeutet, dass das Verhalten der Funktion in den Endpunkten des
Intervalls kaum einen Einfluss auf die Lösung haben.
Die einzige, durch die Lösungsfunktion zu erfüllende Bedingung ist die
Euler-Lagrange-Gleichung.
Die Anforderung, dass die Endpunkte vorgegeben sind, soll in diesem
Abschnitt aufgeweicht und die Lösung Anfangskurve-Endpunkt-,
Anfangspunkt-Endkurve- und Anfangskurve-Endkurve-Problemen dargestellt
werden.
Dazu ist nötig, zusätzliche Information über das Verhalten der Lösung
am Intervallende aus der allgemeinen Form der Variation zu gewinnen.

Ausserdem mussten bisher studierte Variationsprobleme nur ein 
Integral der Lagrange-Funktion minimieren.
In der Praxis treten jedoch auch Probleme auf, in denen das zu
minimierende Funktional nicht nur ein Integral über den Definitionsbereich
ist, sondern zusätzliche Terme in den Endpunktkoordinaten enthält.
Die Bildung eines Meniskus an einer Flüssigkeitsoberfläche in einer 
Kapillare ist ein solches Problem.
Im Abschnitt~\ref{buch:nebenbedingungen:transversal:subsection:randterme}
wird gezeigt, wie solche Probleme gelöst werden können.

%
% Transversalitätsbedingung
%
\subsection{Transversalitätsbedingung
\label{buch:nebenbedingungen:transversal:subsection:transversalitaetsbedingung}}
In diesem Abschnitt sollen Variationsprobleme gelöst werden, in denen
mindestens ein Endpunkt nicht fest ist sondern auf einer Kurve
varieren kann.

%
% Aufgabenstellung
%
\subsubsection{Aufgabenstellung}

%
% Bedingung an die Endpunkte
%
\subsubsection{Bedingung an die Endpunkte}

%
% Beispiel
%
\subsubsection{Beispiel}


%
% Randterme
%
\subsection{Randterme
\label{buch:nebenbedingungen:transversal:subsection:randterme}}
In diesem Abschnitt sollen Variationsproblem gelöst werden, in denen
nicht nur ein Integral minimiert werden soll, sondern ein Funktional,
welches zusätzlich von den Randpunkten abhängig ist.

%
% Aufgabenstellung
%
\subsubsection{Aufgabenstellung}

%
% Randbedingungen
%
\subsubsection{Randbedingungen}

%
% Beispiel: Meniskus
%
\subsubsection{Beispiel: Meniskus}

\begin{verbatim}
- Transversalitätsbedingung
- Randterme
\end{verbatim}
