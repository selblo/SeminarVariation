%
% 3-einseitig.tex
%
% (c) 2024 Prof Dr Andreas Müller
%
\section{Einseitige Bindungen
\label{buch:nebenbedingungen:section:einseitigebindungen}}
Die bisher untersuchten Nebenbedingungen waren alle von der
Form einer Gleichung, zum Beispiel 
\[
\int_{x_0}^{x_1} G(x,y(x),y'(x))\,dx = 0
\qquad\text{oder}\qquad
y'(x_2) = c \quad\text{mit $x_2\in (x_0,x_1)$.}
\]
Im ersten Fall liefert die Idee der Lagrange-Multiplikatoren
eine Lösungsmethode.
Das zweite Problem kann aufgeteilt werden in zwei Probleme auf
den Teilintervallen $[x_0,x_2]$ und $[x_2,x_1]$ mit einem zusätzlichen
Randterm an der Stelle $x=x_2$.
Es sind aber auch Aufgabenstellungen denkbar, in denen Nebenbedingungen
in der Form einer Ungleichung gegeben sind.

%
% Ungleichungen als Nebenbedingungen
%
\subsection{Ungleichungen als Nebenbedingungen
\label{buch:nebenbedingungen:einseitig:subsection:ungleichungen}}
Wir motivieren die Problemstellung mit zwei Anwendungsbeispielen.

%
% Ein Seil über eine Stütze leiten
%
\subsubsection{Ein Seil über eine Stütze leiten}
%
% seilbahn.tex -- Seilbahn
%
% (c) 2021 Prof Dr Andreas Müller, OST Ostschweizer Fachhochschule
%
\documentclass[tikz]{standalone}
\usepackage{amsmath}
\usepackage{times}
\usepackage{txfonts}
\usepackage{pgfplots}
\usepackage{csvsimple}
\usetikzlibrary{arrows,intersections,math}
\begin{document}
\def\skala{1}


\begin{tikzpicture}[>=latex,thick,scale=\skala]

\pgfmathparse{sqrt(9+sinh(-1.5)*sinh(-1.5)}
\xdef\l{\pgfmathresult}
\draw (-4.5,{cosh(-1.5)}) -- +({sinh(-1.5)/\l},{-3/\l});
\fill[color=gray!20] ({-4.5+sinh(-1.5)/\l},{cosh(-1.5)-3/\l}) circle[radius=1];

\pgfmathparse{sqrt(9+sinh(0.5)*sinh(0.5)}
\xdef\l{\pgfmathresult}
\draw (1.5,{cosh(0.5)}) -- +({sinh(0.5)/\l},{-3/\l});
\fill[color=gray!20] ({1.5+sinh(0.5)/\l},{cosh(0.5)-3/\l}) circle[radius=1];

\draw[color=red,line width=1.4pt] plot[domain=-1.5:0.5] ({3*\x},{cosh(\x)});

\draw[line width=0.3pt] (-7,-2) -- (-7,3);
\draw[line width=0.3pt] (4,-2) -- (4,3);

\draw[->] (-7.6,-2) -- (4.6,-2) coordinate[label={$x$}];
\draw[->] (-7.5,-2.1) -- (-7.5,3.3) coordinate[label={right:$y$}];

\draw (-7,-2.05) -- (-7,-1.95);
\node at (-7,-2) [below] {$x_0\mathstrut$};
\draw (4,-2.05) -- (4,-1.95);
\node at (4,-2) [below] {$x_1\mathstrut$};

\end{tikzpicture}
\end{document}


Das Tragseil einer Seilbahn trägt die Gondel, die vom Zugseil
hochgezogen wird.
Normalerweise wird das Trageseil nicht einfach zwischen Berg- und Talstation
gespannt sondern über mehrere Zwischstützen geleitet.
Das Seil liegt dort in einer gekrümmten Seilauflagerung, in der es sich
in vielen Fällen in Längsrichtung bewegen kann.
Die Gesamtlänge des Seils bleibt gleich, aber in der Seilauflagerung
folgt es mindestens ein Stück weit deren Krümmung
(Abbildung~\ref{buch:nebenbedingungen:fig:seilbahn}).

Gesucht ist die Funktion $y(x)$, definiert auf dem Intervall $[x_0,x_1]$,
welche das Seil beschreibt.
Die Gesamtlänge des Seils ist 
\[
l(y) = \int_{x_0}^{x_1} \sqrt{1+y'(x)^2}\,dx,
\]
wie wir in früheren Beispielen gesehen haben.
Im Kapitel~\ref{chapter:kettenlinie} wird dargelegt, dass das Seil
versucht, die potentielle Energie
\begin{equation}
E(y) = \mu g \int_{x_0}^{x_1} y\sqrt{1+y'(x)^2}\,dx
\label{buch:nebenbedingungen:einseitig:eqn:seilE}
\end{equation}
zu minimieren, wobei $g$ die Erdbeschleunigung und $\mu$ die
lineare Massendichte ist.
Der Faktor $\mu g$ ändert die Lösung des Extremalproblems nicht und 
kann daher auch weggelassen werden.
In Kapitel~\ref{chapter:kettenlinie} wird als Lösung für ein Seil
ohne Zwischenstüzen eine Kettenlinie gefunden.

Die Zwischenstützen erzwingen nun, dass sich das Seil oberhalb der 
Auflagerungen bewegen muss.
Es gibt daher eine zusätzliche Nebenbedingung der Form
\[
y(x) \ge \varphi(x),
\]
der die Funktion $y(x)$ genügen muss.

%
% Einen Schlauch über einen Schlauchanschluss ziehen
%
\subsubsection{Einen Schlauch über einen Schlauchanschluss ziehen}
Wenn ein Schlauch über einen Schlauchanschluss gezogen wird, spannt
er sich und versucht eine Minimalfläche zu bilden.
Der Schlauch bildet daher eine Rotationsfläche minimalen Inhalts.
In Kapitel~\ref{chapter:minimalflaechen} wird dieses Problem 
studiert.
Beschreibt man den Radius des Schlauchs in Abhängigkeit von der
Position $x$ entlang der Achse durch die Funktion $y(x)$, dann ist
der Flächeninhalt
\begin{equation}
F(y)
=
2\pi
\int_{x_0}^{x_1}
y(x)\sqrt{1+y'(x)^2}\,dx.
\label{buch:nebenbedingungen:einseitig:eqn:schlauchF}
\end{equation}
Der Faktor $2\pi$ hat keinen Einfluss auf das Minimum.
Das Funktional $F(y)$ von
\eqref{buch:nebenbedingungen:einseitig:eqn:seilE}
ist äquivalent zum Funktional $E(y)$ von
\eqref{buch:nebenbedingungen:einseitig:eqn:schlauchF}
im Seilbahnproblem,
die Lösungsfunktion $y(x)$ ist daher eine $\cosh$-Kurve,
die Fläche eine sogenannte {\em Katenoide}.
\index{Katenoide}

Der Schlauch muss aber auch auf Teilen der Oberfläche des
Schlauchanschlusses aufliegen.
Dies bedeutet, dass der Innenradius $y(x)$ des Schlauchs immer
mindestens so gross sein muss wie der Radius $r(x)$ des Schlauchanschlusses.
Die Funktion $y(x)$ muss daher eine Nebenbedingung der Form
\[
y(x) \ge r(x),\quad x\in[x_0,x_1]
\]
erfüllen.

%
% Einseitige Bindungen
%
\subsubsection{Einseitige Bindungen}
Eine Nebenbedingung in der Form einer Ungleichung der Art
\[
y(x) \ge \varphi(x)
\]
mit einer vorgegebenen Funktion $\varphi(x)$ heisst auch
ein {\em Problem mit einseitiger Bindung}
\index{einseitige Bindung}%
\index{Bindung, einseitig}%
oder ein Problem mit einer {\em Gebietseinschränkung}.
\index{Gebietseinschrankung@Gebietseinschränkung}%

%
% Euler-Lgrange-Differentialgleichung
%
\subsection{Euler-Lagrange-Differentialgleichung
\label{buch:nebenbedingungen:einseitig:subsection:eldgl}}



