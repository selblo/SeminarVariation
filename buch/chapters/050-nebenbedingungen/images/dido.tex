%
% dido.tex -- Problem der Dido
%
% (c) 2021 Prof Dr Andreas Müller, OST Ostschweizer Fachhochschule
%
\documentclass[tikz]{standalone}
\usepackage{amsmath}
\usepackage{times}
\usepackage{txfonts}
\usepackage{pgfplots}
\usepackage{csvsimple}
\definecolor{darkred}{rgb}{0.8,0,0}
\usetikzlibrary{arrows,intersections,math,calc}
\begin{document}
\def\a{4}
\def\r{5}
\pgfmathparse{sqrt(\r*\r-\a*\a)}
\xdef\b{\pgfmathresult}
\pgfmathparse{atan(\b/\a)}
\xdef\w{\pgfmathresult}
\def\skala{1}
\begin{tikzpicture}[>=latex,thick,scale=\skala]

\fill[color=darkred!20] (-\a,0) -- (\a,0) arc (\w:{180-\w}:\r) -- cycle;
\draw[color=darkred,line width=1.4pt] (\a,0) arc (\w:{180-\w}:\r);
\node[color=darkred] at ($(0,-\b)+(110:\r)$) [above] {$l$};

\draw[->] ({-\a-0.5},0) -- ({\a+0.5},0) coordinate[label={$x$}];
\draw[->] (0,{-\b-0.3}) -- (0,{\r-\b+0.4}) coordinate[label={right:$y$}];
\begin{scope}
\clip ({-\a-0.3},{-\b-0.3}) rectangle ({\a+0.3},{\r-\b+0.2});
\draw[color=darkred] (0,{-\b}) circle[radius=\r];
\end{scope}
\fill (0,-\b) circle[radius=0.05];
\fill[color=darkred] (-\a,0) circle[radius=0.05];
\fill[color=darkred] (\a,0) circle[radius=0.05];
\draw[->] (0,-\b) -- +(70:\r);
\node at ($(0,-\b)+0.5*(70:\r)$) [below right] {$r$};
\node at (-\a,0) [below right] {$-1$};
\node at (\a,0) [below left] {$1$};

\end{tikzpicture}
\end{document}

