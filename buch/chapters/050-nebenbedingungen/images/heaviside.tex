%
% heaviside.tex -- Heaviside-Funktion
%
% (c) 2021 Prof Dr Andreas Müller, OST Ostschweizer Fachhochschule
%
\documentclass[tikz]{standalone}
\usepackage{amsmath}
\usepackage{times}
\usepackage{txfonts}
\usepackage{pgfplots}
\usepackage{csvsimple}
\usetikzlibrary{arrows,intersections,math}
\definecolor{darkred}{rgb}{0.8,0,0}
\begin{document}
\def\skala{1}
\begin{tikzpicture}[>=latex,thick,scale=\skala]

\draw[->] (-2.8,0) -- (3.0,0) coordinate[label={$x$}];
\draw[->] (0,-0.1) -- (0,1.5) coordinate[label={left:$y$}];
\fill[color=darkred!10] (0,0) rectangle (2.8,1);
\draw[color=darkred,line width=1.4pt]
	(-2.7,0) -- (0,0) -- (0,1) -- (2.8,1);
\node[color=darkred] at (1.3,1) [above] {$\vartheta(x)$};

\begin{scope}[xshift=4.8cm]
\fill[color=darkred!10] (-1.4,0) rectangle (2.3,1);
\draw[->] (-1.5,0) -- (4.6,0) coordinate[label={$x$}];
\draw[->] (0,-0.1) -- (0,1.5) coordinate[label={left:$y$}];
\draw[color=darkred,line width=1.4pt]
	(-1.4,1) -- (2.3,1) -- (2.3,0) -- (4.3,0);
\node[color=darkred] at (1.2,1) [above] {$\vartheta(x_* -x)$};
\draw (2.3,-0.05) -- (2.3,0.05);
\node at (2.3,0.05) [below] {$x_*$};
\end{scope}

\end{tikzpicture}
\end{document}

