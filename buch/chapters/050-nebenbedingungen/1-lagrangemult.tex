%
% 1-lagrangemult.tex
%
% (c) 2023 Prof Dr Andreas Müller
%
\section{Lagrange-Multiplikatoren für Variationsprobleme
\label{buch:nebenbedingungen:section:lagrangemult}}
\kopfrechts{Lagrange-Multiplikatoren für Variationsprobleme}
In diesem Abschnitt gehen wir wieder von einem Funktional
\begin{equation}
I(y)
=
\int_{x_1}^{x_2}
F(x,y(x),y'(x))
\,dx,
\label{buch:nebenbedingungen:lagrangemult:eqn:}
\end{equation}
für das eine Funktion $y(x)$ mit Randbedingungen $y(x_1)=y_1$ und
$y(x_2)=y_2$ gesucht wird, deren Wert $I(y)$ extremal ist.
Für dieses Problem wurde als notwendige Bedingung für die Lösung die
Euler-Lagrange-Differentialgleichung gefunden.
In diesem Abschnitt sollen jetzt zusätzliche Bedingungen auferlegt 
werden, zum Beispiel vorgegebene Werte in vorgegebenen Punkten oder
vorgegebene Werte eines Funktionals.
In
Abschnitt~\ref{buch:nebenbedingungen:lagrangemult:subsection:nebenbedingungen}
soll gezeigt werden, wie sich sollche Nebenbedingungen formulieren
lassen.
Die Euler-Lagrange-Differentialgleichung entstand aus der Idee
der Richtungsableitung, die auch bei der Herleitung der Methode
der Lagrange-Multiplikatoren im Mittelpunkt stand.
Im
Abschnitt~\ref{buch:nebenbedingungen:lagrangemult:subsection:einzeln}
wird das Problem für eine einzelne Nebenbedingung gelöst, in
Abschnitt~\ref{buch:nebenbedingungen:lagrangemult:subsection:lagrangemult}
dann für eine endliche Anzahl von Nebenbedingungen, wobei ein der
Methode der Lagrange-Multiplikatoren ähnliches Lösungsverfahren entsteht.

%
% Nebenbedingungen für Variationsprobleme
%
\subsection{Nebenbedingungen für Variationsprobleme
\label{buch:nebenbedingungen:lagrangemult:subsection:nebenbedingungen}}
In diesem Abschnitt zeigen wir, dass sich verschiedene naheliegende
Nebenbedingungen an Funktionen in einem Variationsproblem allgemein
als Nebenbedingungsfunktionale der gleichen Art wie das zu minimierende
Funktional formulieren lassen.

%
% Integrale als Nebenbedingungen
%
\subsubsection{Integrale als Nebenbedingungen}
Der Sage nach soll Dido, Tochter des tyrischen Königs Mattan, auf der
Flucht vor ihrem Bruder Pygmalion über Zypern am Golf von Tunis 
gelandet sein.
\index{Dido}%
\index{Mattan}%
\index{Pygmalion}%
\index{Tunis}%
Der Numiderkönig Iarbas soll ihr so viel Land versprochen haben, wie
sie mit einer Kuhhaut umspannen konnte.
\index{Iarbas}%
Dido schnitt die Haut in einen dünnen Streifen und konnte damit ein
Gebiet umspannen, das das Gebiet der Byrsa, der Burg des späteren
Karthago umfasste.
\index{Byrsa}%
\index{Karthago}%

Dieser Mythos impliziert, dass Karthago als Lösung eines sogenannten
isoperimetrischen Problems gegründet wurde.
\index{isoperimetrisch}%
Im konkreten Fall geht es zum Beispiel darum, eine Funktion $y(x)$
auf dem Intervall $[x_1,x_2]$ zu finden, welche $y(x_1)=y(x_2)=0$,
die den Flächeninhalt
\[
I(y)
=
\int_{x_1}^{x_2} y(x)\,dx
\]
maximiert.
Die Länge der Kurve ist
\[
l(y)
=
\int_{x_1}^{x_2}
\sqrt{1+y'(x)^2}
\,dx
=
L
\]
dabei vorgeben.

%
% Vorgegebene Werte
%
\subsubsection{Vorgegebene Werte}
Wir betrachten zunächst den Fall einer Bedingung, dass an einer Stelle
$x_*$ der Werte $y_*=y(x_*)$ der Lösungsfunktion $y(x)$ vorgegeben ist.
Auch diese Art von Nebenbedingung kann als Integral geschrieben werden.
Dazu wird die Ableitung $y'(x)$ integriert und erhält
\[
J(y)
=
\int_{x_1}^{x_*} y'(x)\,dx
=
\biggl[y(x)\biggr]_{x_1}^{x_*}
=
y(x_*)-y(x_1).
\]
Das Integral kann mit der Heaviside-Funktion
\begin{equation}
\vartheta(x)
=
\begin{cases}
1&\qquad x\ge 0\\
0&\qquad\text{sonst}
\end{cases}
\label{buch:nebenbedingungen:lagrangemult:eqn:heaviside}
\end{equation}
geschrieben werden.
Die Funktion $x\mapsto\vartheta(x_*-x)$ verschwindet genau dann,
wenn $x_*\ge x$ ist.
Daher ist
\[
J(y)
=
\int_{x_1}^{x_*} y'(x)\,dx
=
\int_{x_1}^{x_2} y'(x)\vartheta(x_*-x)\,dx.
\]
Mit der Lagrange-Funktion
\begin{equation}
G(x,y,y')
=
\vartheta(x_*-x)
y'
\label{buch:nebenbedingungen:lagrangemult:eqn:heavilagrange}
\end{equation}
wird $J(y)$ zu einem Funktional
\[
J(y)
=
\int_{x_1}^{x_2}
G(x,y(x),y'(x))
\,dx.
\]
Die einzige Schwierigkeit ist, dass die Funktion $G(x,y,y')$ nicht
differenzierbar ist in der ersten Variablen.

%
% Ableitung der Heaviside-Funktion
%
\subsubsection{Ableitung der Heaviside-Funktion}
Mit der Theorie der Distributionen lässt sich die Schwierigkeit,
dass $G(x,y,y')$ nicht differenzierbar ist, umgehen.
Da es bei der Herleitung der Euler-Lagrange-Differentialgleichung nur
auf Integrale von Produkten mit beliebig oft stetig differenzierbaren
Funktionen ankommt, lässt sich die Ableitung der Heaviside-Funktion
als Dirac-\textdelta-Funktion
\[
\vartheta'(x-x_*) = \delta(x-x_*)
\]
an der Stelle $x_*$ schreiben.
Das Integral eines Produktes einer Funktion $f(x)$ mit der
Delta-Funktion an der Stelle $x_*$ liefert
\[
\int_{x_1}^{x_2} \delta(x-x_*) f(x)\,dx
=
f(x_*),
\]
den Wert des Faktors $f(x)$ an dieser Stelle.
Damit bleibt die Formel für die partielle Integration erhalten,
wie wir im folgenden nachrechnen wollen.
Dazu berechnen wir das Integral des Produktes $\vartheta(x-x_*)f'(x)$
direkt als
\begin{align}
\int_{x_1}^{x_2}
\vartheta(x_*-x) f'(x)
\,dx
&=
f(x_*)-f(x_1)
\label{buch:nebenbedingungen:lagrangemult:eqn:lhs}
\intertext{und andererseits die beiden Terme auf der rechten Seite der
Regel für das partielle Integrieren, die}
\biggl[\vartheta(x_*-x) f(x)\biggr]_{x_1}^{x_2}
&=
\underbrace{\vartheta(x_*-x_2)}_{\displaystyle = 0}
f(x_2)
-
\underbrace{\vartheta(x_*-x_1)}_{\displaystyle = 1}
f(x_1)
=
-f(x_1)
\notag
\\
\int_{x_1}^{x_2}
\delta(x-x_*)f(x)\,dx
&=
f(x_*)
\notag
\intertext{sind.
Die rechte Seite von~\ref{buch:nebenbedingungen:lagrangemult:eqn:lhs}
ist daher dasselbe wie
}
\int_{x_1}^{x_2}
\vartheta(x_*-x) f'(x)
\,dx
&=
\biggl[\vartheta(x_*-x) f(x)\biggr]_{x_1}^{x_2}
-
\int_{x_1}^{x_2}
\vartheta'(x-x_*)f(x)\,dx.
\end{align}
Dies zeigt, dass die Regel für das partielle Integrieren auch
für die Funktion $\vartheta$ gilt, sofern sie nur
als Faktor zusammen mit differenzierbare Funktionen verwendet wird.
Da genau diese Situation in der Herleitung der
Euler-Lagrange-Differentialgleichung vorliegt, lässt sich diese auf
Funktionale mit einer Lagrange-Funktion wie 
\eqref{buch:nebenbedingungen:lagrangemult:eqn:heavilagrange}
verallgemeinern.

%
% Allgemeine Form von Nebenbedingungen
%
\subsubsection{Nebenbedingungsfunktionale}
Die beiden Beispiele illustrieren, dass sich Nebenbedingungen für
Variationsprobleme meistens in der Form eines Nebenbedingungsfunktionls
schreiben kann.
Eine Nebenbedingung ist also gegeben durch eine Lagrange-Funktion
\[
G
\colon
\mathbb{R}
\times
\mathbb{R}
\times
\mathbb{R}
\to
\mathbb{R}
:
(x,y,y')
\mapsto
G(x,y,y')
\]
und den Wert, $g$, den das Funktional
\[
\int_{x_1}^{x_2}
G(x,y(x),y'(x))\,dx
=
g
\]
annehmen muss.

%
% Eine einzelne Nebenbedingung
%
\subsection{Eine einzelne Nebenbedingung
\label{buch:nebenbedingungen:lagrangemult:subsection:einzeln}}

%
% Laagrange-Multiplikatoren für Variationsprobleme
%
\subsection{Lagrange-Multiplikatoren für Variationsprobleme
\label{buch:nebenbedingungen:lagrangemult:subsection:lagrangemult}}

\begin{verbatim}
- Nebenbedingungen
- Richtungsableitung für Nebenbedingungs-Funktionale
- Lagrange-Multiplikatoren
\end{verbatim}
